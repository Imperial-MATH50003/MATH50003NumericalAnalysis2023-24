
\section{Integration}
One possible definition for an integral is the limit of a Riemann sum, for example:
\[
  \ensuremath{\int}_0^1 f(x) {\rm d}x = \lim_{n \ensuremath{\rightarrow} \ensuremath{\infty}} {1 \over n} \ensuremath{\sum}_{k=1}^n f(k/n).
\]
This suggests an algorithm known as the \emph{(right-sided) rectangular rule} for approximating an integral: choose $n$ large so that
\[
  \ensuremath{\int}_0^1 f(x) {\rm d}x \ensuremath{\approx} {1 \over n} \ensuremath{\sum}_{k=1}^n f(k/n).
\]
In the lab we explore practical implementation of this approximation, and observe that the error in approximation is bounded by $C/n$ for some constant $C$. This can be expressed using "Big-O" notation:
\[
\ensuremath{\int}_0^1 f(x) {\rm d}x = {1 \over n} \ensuremath{\sum}_{k=1}^n f(k/n) + O(1/n).
\]
In these notes we consider the "Analysis" part of "Numerical Analysis": we want to \emph{prove} the convergence rate of the approximation, including finding an explicit expression for the constant $C$.

To tackle this question we consider the error incurred on a single "rectangle", then sum up the errors on rectangles.

Now for a secret. There are only so many tools available in analysis (especially at this stage of your career), and  one can make a safe bet that the right tool in any analysis proof is either (1) integration-by-parts, (2) geometric series or (3) Taylor series. In this case we use (1):

\begin{lemma}[rect. rule on one panel] Assuming $f$ is differentiable we have
\[
\ensuremath{\int}_0^h f(x) {\rm d}x = h f(0) +  \ensuremath{\delta}_h
\]
where $|\ensuremath{\delta}_h| \ensuremath{\leq} M h^2$ for $M = \sup_{0 \ensuremath{\leq} x \ensuremath{\leq} h}|f'(x)|$.

\end{lemma}
\textbf{Proof} We write
\[
\ensuremath{\int}_0^h f(x) {\rm d}x = \ensuremath{\int}_0^h (x)' f(x)  {\rm d}x = [x f(x)]_0^h - \ensuremath{\int}_0^h x f'(x) {\rm d} x
= h f(h) + \underbrace{-\ensuremath{\int}_0^h x f'(x) {\rm d} x}_{\ensuremath{\delta}_h}.
\]
Recall that we can bound the absolute value of an integral by the sepremum of the integrand times the width of the integration interval:
\[
|\ensuremath{\int}_a^b g(x) {\rm d} x| \ensuremath{\leq} (b-a) \sup_{0 \ensuremath{\leq} x \ensuremath{\leq} h}|g(x)|.
\]
The lemma thus follows since
\[
|\ensuremath{\int}_0^h x f'(x) {\rm d} x| \ensuremath{\leq} h \sup_{0 \ensuremath{\leq} x \ensuremath{\leq} h}|x f'(x)| \ensuremath{\leq} M h^2.
\]
\ensuremath{\QED}



