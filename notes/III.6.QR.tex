
\section{QR Factorisation}
Let $A \ensuremath{\in} \ensuremath{\bbC}^{m \ensuremath{\times} n}$ be a rectangular or square matrix such that $m \ensuremath{\geq} n$ (i.e. more rows then columns). In this chapter we consider two closely related factorisations:

\begin{definition}[QR factorisation] The \emph{QR factorisation} is
\[
A = Q R = \underbrace{\begin{bmatrix} \ensuremath{\bm{\q}}_1 | \ensuremath{\cdots} | \ensuremath{\bm{\q}}_m \end{bmatrix}}_{Q \ensuremath{\in} U(m)} \underbrace{\begin{bmatrix} \ensuremath{\times} & \ensuremath{\cdots} & \ensuremath{\times} \\ & \ensuremath{\ddots} & \ensuremath{\vdots} \\ && \ensuremath{\times} \\ &&0 \\ &&\ensuremath{\vdots} \\ && 0 \end{bmatrix}}_{R \ensuremath{\in} \ensuremath{\bbC}^{m \ensuremath{\times} n}}
\]
where $Q$ is unitary (i.e., $Q \ensuremath{\in} U(m)$, satisfying $Q^\ensuremath{\star}Q = I$, with columns $\ensuremath{\bm{\q}}_j \ensuremath{\in} \ensuremath{\bbC}^m$) and $R$ is \emph{right triangular}, which means it  is only nonzero on or to the right of the diagonal ($r_{kj} = 0$ if $k > j$). \end{definition}

\begin{definition}[Reduced QR factorisation] The \emph{reduced QR factorisation}
\[
A = \hat Q \hat R = \underbrace{\begin{bmatrix} \ensuremath{\bm{\q}}_1 | \ensuremath{\cdots} | \ensuremath{\bm{\q}}_n \end{bmatrix}}_{ \hat Q \ensuremath{\in} \ensuremath{\bbC}^{m \ensuremath{\times} n}} \underbrace{\begin{bmatrix} \ensuremath{\times} & \ensuremath{\cdots} & \ensuremath{\times} \\ & \ensuremath{\ddots} & \ensuremath{\vdots} \\ && \ensuremath{\times}  \end{bmatrix}}_{\hat R \ensuremath{\in} \ensuremath{\bbC}^{n \ensuremath{\times} n}}
\]
where $\hat Q$ has orthonormal columns ($\hat Q^\ensuremath{\star} \hat Q = I$, $\ensuremath{\bm{\q}}_j \ensuremath{\in} \ensuremath{\bbC}^m$) and $\hat R$ is upper triangular. \end{definition}

Note for a square matrix the reduced QR factorisation is equivalent to the QR factorisation, in which case $R$ is \emph{upper triangular}. The importance of these factorisation for square matrices is that their component pieces are easy to invert:
\[
A = QR \qquad \ensuremath{\Rightarrow} \qquad A^{-1}\ensuremath{\bm{\b}} = R^{-1} Q^\ensuremath{\top} \ensuremath{\bm{\b}}
\]
and we saw previously that triangular and orthogonal matrices are easy to invert when applied to a vector $\ensuremath{\bm{\b}}$.

For rectangular matrices we will see that the QR factorisation leads to efficient solutions to the \emph{least squares problem}: find $\ensuremath{\bm{\x}}$ that minimizes the 2-norm $\| A \ensuremath{\bm{\x}} - \ensuremath{\bm{\b}} \|.$ Note in the rectangular case the QR factorisation contains within it the reduced QR factorisation:
\[
A = QR = \begin{bmatrix} \hat Q | \ensuremath{\bm{\q}}_{n+1} | \ensuremath{\cdots} | \ensuremath{\bm{\q}}_m \end{bmatrix} \begin{bmatrix} \hat R \\  \ensuremath{\bm{\zero}}_{m-n \ensuremath{\times} n} \end{bmatrix} = \hat Q \hat R.
\]
In this chapter we discuss the following:

\begin{itemize}
\item[1. ] Reduced QR and Gram\ensuremath{\endash}Schmidt: We discuss computation of the Reduced QR factorisation using Gram\ensuremath{\endash}Schmidt.


\item[2. ] Householder reflections and QR: We discuss computing the  QR factorisation using Householder reflections. This is a more accurate approach

\end{itemize}
for computing QR factorisations.

\begin{itemize}
\item[3. ] QR and least squares: We discuss the QR factorisation and its usage in solving least squares problems.

\end{itemize}
\subsection{Reduced QR and Gram\ensuremath{\endash}Schmidt}
How do we compute the QR factorisation? We begin with a method you may have seen before in another guise. Write
\[
A = \begin{bmatrix} \ensuremath{\bm{\a}}_1 | \ensuremath{\cdots} | \ensuremath{\bm{\a}}_n \end{bmatrix}
\]
where $\ensuremath{\bm{\a}}_k \ensuremath{\in}  \ensuremath{\bbC}^m$ and assume they are linearly independent ($A$ has full column rank).

\begin{proposition}[Column spaces match] Suppose $A = \hat Q  \hat R$ where $\hat Q = [\ensuremath{\bm{\q}}_1|\ensuremath{\ldots}|\ensuremath{\bm{\q}}_n]$ has orthonormal columns and $\hat R$ is upper-triangular, and $A$ has full rank. Then the first $j$ columns of $\hat Q$ span the same space as the first $j$ columns of $A$:
\[
\hbox{span}(\ensuremath{\bm{\a}}_1,\ensuremath{\ldots},\ensuremath{\bm{\a}}_j) = \hbox{span}(\ensuremath{\bm{\q}}_1,\ensuremath{\ldots},\ensuremath{\bm{\q}}_j).
\]
\end{proposition}
\textbf{Proof}

Because $A$ has full rank we know $\hat R$ is invertible, i.e. its diagonal entries do not vanish: $r_{jj} \ensuremath{\neq} 0$. If $\ensuremath{\bm{\v}} \ensuremath{\in} \hbox{span}(\ensuremath{\bm{\a}}_1,\ensuremath{\ldots},\ensuremath{\bm{\a}}_j)$ we have for $\ensuremath{\bm{\c}} \ensuremath{\in} \ensuremath{\bbC}^j$
\[
\ensuremath{\bm{\v}} = \begin{bmatrix} \ensuremath{\bm{\a}}_1 | \ensuremath{\cdots} | \ensuremath{\bm{\a}}_j \end{bmatrix} \ensuremath{\bm{\c}} = 
\begin{bmatrix} \ensuremath{\bm{\q}}_1 | \ensuremath{\cdots} | \ensuremath{\bm{\q}}_j \end{bmatrix}  \hat R[1:j,1:j] \ensuremath{\bm{\c}} \ensuremath{\in} \hbox{span}(\ensuremath{\bm{\q}}_1,\ensuremath{\ldots},\ensuremath{\bm{\q}}_j)
\]
while if $\ensuremath{\bm{\w}} \ensuremath{\in} \hbox{span}(\ensuremath{\bm{\q}}_1,\ensuremath{\ldots},\ensuremath{\bm{\q}}_j)$ we have for $\vc d \ensuremath{\in} \ensuremath{\bbR}^j$
\[
\ensuremath{\bm{\w}} = \begin{bmatrix} \ensuremath{\bm{\q}}_1 | \ensuremath{\cdots} | \ensuremath{\bm{\q}}_j \end{bmatrix} \vc d  =  \begin{bmatrix} \ensuremath{\bm{\a}}_1 | \ensuremath{\cdots} | \ensuremath{\bm{\a}}_j \end{bmatrix} \hat R[1:j,1:j]^{-1} \vc d \ensuremath{\in}  \hbox{span}(\ensuremath{\bm{\a}}_1,\ensuremath{\ldots},\ensuremath{\bm{\a}}_j).
\]
\ensuremath{\QED}

It is possible to find $\hat Q$ and $\hat R$ the  using the \emph{Gram\ensuremath{\endash}Schmidt algorithm}. We construct it column-by-column. For $j = 1, 2, \ensuremath{\ldots}, n$ define
\begin{align*}
\ensuremath{\bm{\v}}_j &:= \ensuremath{\bm{\a}}_j - \ensuremath{\sum}_{k=1}^{j-1} \underbrace{\ensuremath{\bm{\q}}_k^\ensuremath{\star} \ensuremath{\bm{\a}}_j}_{r_{kj}} \ensuremath{\bm{\q}}_k \\
r_{jj} &:= {\|\ensuremath{\bm{\v}}_j\|} \\
\ensuremath{\bm{\q}}_j &:= {\ensuremath{\bm{\v}}_j \over r_{jj}}
\end{align*}
\textbf{Theorem (Gram\ensuremath{\endash}Schmidt and reduced QR)} Define $\ensuremath{\bm{\q}}_j$ and $r_{kj}$ as above (with $r_{kj} = 0$ if $k > j$). Then a reduced QR factorisation is given by:
\[
A = \underbrace{\begin{bmatrix} \ensuremath{\bm{\q}}_1 | \ensuremath{\cdots} | \ensuremath{\bm{\q}}_n \end{bmatrix}}_{ \hat Q \ensuremath{\in} \ensuremath{\bbC}^{m \ensuremath{\times} n}} \underbrace{\begin{bmatrix} r_{11} & \ensuremath{\cdots} & r_{1n} \\ & \ensuremath{\ddots} & \ensuremath{\vdots} \\ && r_{nn}  \end{bmatrix}}_{\hat R \ensuremath{\in} \ensuremath{\bbC}^{n \ensuremath{\times} n}}
\]
\textbf{Proof}

We first show that $\hat Q$ has orthonormal columns. Assume that $\ensuremath{\bm{\q}}_\ensuremath{\ell}^\ensuremath{\star} \ensuremath{\bm{\q}}_k = \ensuremath{\delta}_{\ensuremath{\ell}k}$ for $k,\ensuremath{\ell} < j$.  For $\ensuremath{\ell} < j$ we then have
\[
\ensuremath{\bm{\q}}_\ensuremath{\ell}^\ensuremath{\star} \ensuremath{\bm{\v}}_j = \ensuremath{\bm{\q}}_\ensuremath{\ell}^\ensuremath{\star} \ensuremath{\bm{\a}}_j - \ensuremath{\sum}_{k=1}^{j-1}  \ensuremath{\bm{\q}}_\ensuremath{\ell}^\ensuremath{\star}\ensuremath{\bm{\q}}_k \ensuremath{\bm{\q}}_k^\ensuremath{\star} \ensuremath{\bm{\a}}_j = 0
\]
hence $\ensuremath{\bm{\q}}_\ensuremath{\ell}^\ensuremath{\star} \ensuremath{\bm{\q}}_j = 0$ and indeed $\hat Q$ has orthonormal columns. Further: from the definition of $\ensuremath{\bm{\v}}_j$ we find
\[
\ensuremath{\bm{\a}}_j = \ensuremath{\bm{\v}}_j + \ensuremath{\sum}_{k=1}^{j-1} r_{kj} \ensuremath{\bm{\q}}_k = \ensuremath{\sum}_{k=1}^j r_{kj} \ensuremath{\bm{\q}}_k  = \hat Q \hat R \ensuremath{\bm{\e}}_j
\]
\ensuremath{\QED}

\subsection{Householder reflections and QR}
As an alternative, we will consider using Householder reflections to introduce zeros below the diagonal. Thus, if Gram\ensuremath{\endash}Schmidt is a process of \emph{triangular orthogonalisation} (using triangular matrices to orthogonalise), Householder reflections is a process of \emph{orthogonal triangularisation}  (using orthogonal matrices to triangularise).

Consider multiplication by the Householder reflection corresponding to the first column, that is, for
\[
Q_1 := Q_{\ensuremath{\bm{\a}}_1}^{\rm H},
\]
consider
\[
Q_1 A = \begin{bmatrix} \ensuremath{\times} & \ensuremath{\times} & \ensuremath{\cdots} & \ensuremath{\times} \\
& \ensuremath{\times} & \ensuremath{\cdots} & \ensuremath{\times} \\
                    & \ensuremath{\vdots} & \ensuremath{\ddots} & \ensuremath{\vdots} \\
                    & \ensuremath{\times} & \ensuremath{\cdots} & \ensuremath{\times} \end{bmatrix} = 
\begin{bmatrix}  \ensuremath{\alpha}_1 & \ensuremath{\bm{\w}}_1^\ensuremath{\top} \\ 
& A_2   \end{bmatrix}
\]
where 
\[
\ensuremath{\alpha}_1 := -{\rm csign}(a_{11})  \|\ensuremath{\bm{\a}}_1\|, \ensuremath{\bm{\w}}_1 = (Q_1 A)[1, 2:n]  \qquad \hbox{and} \qquad A_2 = (Q_1 A)[2:m, 2:n],
\]
where as before ${\rm csign}(z) :=  {\rm e}^{{\rm i} \arg z}$. That is, we have made the first column triangular. In terms of an algorithm, we then introduce zeros into the first column of $A_2$, leaving an $A_3$, and so-on. But we can wrap this iterative algorithm into a simple proof by induction, reminisicent of our proof for the Cholesky factorisation:

\begin{theorem}[QR]  Every matrix $A \ensuremath{\in} \ensuremath{\bbC}^{m \ensuremath{\times} n}$ has a QR factorisation:
\[
A = QR
\]
where $Q \ensuremath{\in} U(m)$ and $R \ensuremath{\in} \ensuremath{\bbC}^{m \ensuremath{\times} n}$ is right triangular.

\end{theorem}
\textbf{Proof}

First assume $m \ensuremath{\geq} n$. If $A = [\ensuremath{\bm{\a}}_1] \ensuremath{\in} \ensuremath{\bbC}^{m \ensuremath{\times} 1}$ then we have for the Householder reflection $Q_1 = Q_{\ensuremath{\bm{\a}}_1}^{\rm H}$
\[
Q_1 A = \ensuremath{\alpha} \ensuremath{\bm{\e}}_1
\]
which is right triangular, where $\ensuremath{\alpha} = -{\rm csign}(a_{11}) \|\ensuremath{\bm{\a}}_1\|$.  In other words 
\[
A = \underbrace{Q_1}_Q \underbrace{\ensuremath{\alpha} \ensuremath{\bm{\e}}_1}_R.
\]
For $n > 1$, assume every matrix with less columns than $n$ has a QR factorisation. For $A = [\ensuremath{\bm{\a}}_1|\ensuremath{\ldots}|\ensuremath{\bm{\a}}_n] \ensuremath{\in} \ensuremath{\bbC}^{m \ensuremath{\times} n}$, let $Q_1 = Q_{\ensuremath{\bm{\a}}_1}^{\rm H}$ so that
\[
Q_1 A =  \begin{bmatrix} \ensuremath{\alpha} & \ensuremath{\bm{\w}}^\ensuremath{\top} \\ & A_2 \end{bmatrix}.
\]
By assumption $A_2 = Q_2 R_2$. Thus we have (recalling that $Q_1^{-1} = Q_1^\ensuremath{\star} = Q_1$):
\begin{align*}
A = Q_1 \begin{bmatrix} \ensuremath{\alpha} & \ensuremath{\bm{\w}}^\ensuremath{\top} \\ & Q_2 R_2 \end{bmatrix} \\
=\underbrace{Q_1 \begin{bmatrix} 1 \\ & Q_2 \end{bmatrix}}_Q  \underbrace{\begin{bmatrix} \ensuremath{\alpha} & \ensuremath{\bm{\w}}^\ensuremath{\top} \\ &  R_2 \end{bmatrix}}_R.
\end{align*}
If $m < n$, i.e., $A$ has more columns then rows, write 
\[
A = \begin{bmatrix} \At & B \end{bmatrix}
\]
where $\At \ensuremath{\in} \ensuremath{\bbC}^{m \ensuremath{\times} m}$. From above we know we can write $\At = Q \Rt$. We thus have
\[
A = Q \underbrace{\begin{bmatrix} \Rt & Q^\ensuremath{\star} B \end{bmatrix}}_R
\]
where $R$ is right triangular.

\ensuremath{\QED}

\begin{example}[QR by hand, non-examinable] We will now do an example by hand. Consider the $4 \ensuremath{\times} 3$ matrix
\[
A = \begin{bmatrix} 
2 & 3 & 0 \\ 
0 & 0 & 1 \\
-2 & -3 & 0 \\
-1 & -3 & -3
\end{bmatrix}
\]
For the first column we have
\[
\ensuremath{\bm{\y}}_1 := [-1,0,-2,-1]
\]
where $\| \ensuremath{\bm{\y}}_1 \|^2 = 6$. Hence
\[
Q_1 := I - {1 \over 3} \begin{bmatrix} -1 \\ 0 \\ -2 \\ -1 \end{bmatrix} \begin{bmatrix} -1 & 0 & -2 & -1 \end{bmatrix} =
 {1 \over 3} \begin{bmatrix}
2 & 0 & -2 & -1 \\
0 & 3 & 0 & 0 \\
-2 & 0 & -1 & -2 \\
-1 & 0 & -2 &  2
\end{bmatrix}
\]
so that
\[
Q_1 A = \begin{bmatrix} 3 &  5 & 1 \\
 & 0 & 1 \\
  & 1 & 2 \\
& -1 & -2
\end{bmatrix}
\]
For the second column we have
\[
\ensuremath{\bm{\y}}_2 :=  [-\sqrt{2},1,-1]
\]
where $\| \ensuremath{\bm{\y}}_2 \|^2 = 4$. Thus we have
\[
Q_2 := I - {1 \over 2}
 \begin{bmatrix} -\sqrt{2} \\1 \\ -1
\end{bmatrix} \begin{bmatrix} -\sqrt{2} & 1 & -1 \end{bmatrix}
= \begin{bmatrix}
0 & 1/\sqrt{2} & -1/\sqrt{2} \\
1/\sqrt{2} & 1/2 & 1/2 \\
-1/\sqrt{2} & 1/2 & 1/2
\end{bmatrix}
\]
so that
\[
\tilde Q_2 Q_1 A = \begin{bmatrix} 3 & 5 & 1 \\
 & \sqrt{2} & 2\sqrt{2} \\
  & 0 & 1/\sqrt{2} \\
& 0 & -1/\sqrt{2}
\end{bmatrix}
\]
The final vector is 
\[
\ensuremath{\bm{\y}}_3 := [1/\sqrt{2}-1,-1/\sqrt{2}]
\]
where $\| \ensuremath{\bm{\y}}_3 \|^2 = 2 - 2/\sqrt{2}$. Hence
\[
Q_3 := I - {\sqrt{2} \over \sqrt{2} - 1} \begin{bmatrix}
1/\sqrt{2}-1 \\
-1/\sqrt{2}
\end{bmatrix} \begin{bmatrix}
1/\sqrt{2}-1 &
-1/\sqrt{2}
\end{bmatrix} =
\begin{bmatrix}
\sqrt{2} & -\sqrt{2}\\
-\sqrt{2} & -\sqrt{2}
\end{bmatrix}
\]
so that 
\[
\tilde Q_3 \tilde Q_2 Q_1 A = \begin{bmatrix} 3 & 5 & 1 \\
 & \sqrt{2} & 2\sqrt{2} \\
  & 0 & 1 \\
& 0 & 0
\end{bmatrix} =: R
\]
and
\[
Q := Q_1 \tilde Q_2 \tilde Q_3 =  \begin{bmatrix}
2/3 & -1/(3\sqrt{2}) & 0 & 1/\sqrt{2} \\
0 &  0 & 1 & 0 \\
-2/3 & 1/(3\sqrt{2}) & 0 & 1/\sqrt{2} \\ 
-1/3 & - 4/(3\sqrt{2}) & 0 & 0
\end{bmatrix}.
\]
\subsection{QR and least squares}
We consider rectangular matrices with more rows than columns. Given $A \ensuremath{\in} \ensuremath{\bbC}^{m \ensuremath{\times} n}$ and $\ensuremath{\bm{\b}} \ensuremath{\in} \ensuremath{\bbC}^m$, least squares consists of finding a vector $\ensuremath{\bm{\x}} \ensuremath{\in} \ensuremath{\bbC}^n$ that minimises the 2-norm: $\| A \ensuremath{\bm{\x}} - \ensuremath{\bm{\b}} \|$.

\begin{theorem}[least squares via QR] Suppose $A \ensuremath{\in} \ensuremath{\bbC}^{m \ensuremath{\times} n}$ has full rank. Given a reduced QR factorisation $A = Q R$ then
\[
\ensuremath{\bm{\x}} = \hat R^{-1} \hat Q^\ensuremath{\star} \ensuremath{\bm{\b}}
\]
minimises $\| A \ensuremath{\bm{\x}} - \ensuremath{\bm{\b}} \|$. 

\end{theorem}
\textbf{Proof}

The norm-preserving property ($\|Q\ensuremath{\bm{\x}}\| = \|\ensuremath{\bm{\x}}\|$) of unitary matrices tells us
\[
\| A \ensuremath{\bm{\x}} - \ensuremath{\bm{\b}} \| = \| Q R \ensuremath{\bm{\x}} - \ensuremath{\bm{\b}} \| = \| Q (R \ensuremath{\bm{\x}} - Q^\ensuremath{\star} \ensuremath{\bm{\b}}) \| = \| R \ensuremath{\bm{\x}} - Q^\ensuremath{\star} \ensuremath{\bm{\b}} \| = \left \| 
\begin{bmatrix} \hat R \\ \ensuremath{\bm{\zero}}_{m-n \ensuremath{\times} n} \end{bmatrix} \ensuremath{\bm{\x}} - \begin{bmatrix} \hat Q^\ensuremath{\star} \\ \ensuremath{\bm{\q}}_{n+1}^\ensuremath{\star} \\ \ensuremath{\vdots} \\ \ensuremath{\bm{\q}}_m^\ensuremath{\star} \end{bmatrix}     \ensuremath{\bm{\b}} \right \|
\]
Now note that the rows $k > n$ are independent of $\ensuremath{\bm{\x}}$ and are a fixed contribution. Thus to minimise this norm it suffices to drop them and minimise:
\[
\| \hat R \ensuremath{\bm{\x}} - \hat Q^\ensuremath{\star} \ensuremath{\bm{\b}} \|
\]
This norm is minimised if it is zero. Provided the column rank of $A$ is full, $\hat R$ will be invertible.

\end{example}



