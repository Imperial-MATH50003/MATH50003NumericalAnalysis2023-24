
\section{Interval Arithmetic}
It is possible to choose the rounding mode (up/down/nearest/etc.) which we can use to do rigorous computation to compute bounds on, for example, computing the digits of $\E$. To do this we will use set/interval arithmetic. For sets $X,Y \ensuremath{\subseteq} \ensuremath{\bbR}$, the set arithmetic operations are defined as
\[
X + Y := \{x + y : x \ensuremath{\in} X, y \ensuremath{\in} Y\}, XY := \{xy : x \ensuremath{\in} X, y \ensuremath{\in} Y\}, X/Y := \{x/y : x \ensuremath{\in} X, y \ensuremath{\in} Y\}
\]
We will use floating point arithmetic to construct approximate set operations \ensuremath{\oplus}, \ensuremath{\otimes} so that
\[
  X + Y \ensuremath{\subseteq} X \ensuremath{\oplus} Y, XY \ensuremath{\subseteq} X \ensuremath{\otimes} Y, X/Y \ensuremath{\subseteq} X \ensuremath{\oslash} Y
\]
thereby a complicated algorithm can be run on sets and the true result is guaranteed to be a subset of the output.

When our sets are intervals we can deduce simple formulas for basic arithmetic operations. For simplicity we only consider the case where all values are positive.

\begin{proposition}[interval bounds] For intervals  $X = [a,b]$ and $Y = [c,d]$ satisfying $0 < a < b$ and $0 < c < d$, and $n > 0$, we have:
\meeq{
X + Y = [a+c, b+d] \ccr
X/n = [a/n,b/n] \ccr
XY = [ac, bd]
}
\end{proposition}
\textbf{Proof} We first show $X+Y \ensuremath{\subseteq} [a+c,b+d]$. If $z \ensuremath{\in} X + Y$ then $z = x+y$ such that $a \ensuremath{\leq} x \ensuremath{\leq} b$ and $c \ensuremath{\leq} y \ensuremath{\leq} d$ and therefore $a + c \ensuremath{\leq} z \ensuremath{\leq} c + d$ and $z \ensuremath{\in} [a+c,b+d]$. Equality follows from convexity. First note that $a+c, b+d \ensuremath{\in} X+Y$. Any point $z \ensuremath{\in}  [a+b,c+d]$ can be written  as a convex combination of the two endpoints: there exists $0 \ensuremath{\leq} t \ensuremath{\leq} 1$ such that
\[
z = t (a+c) + (1-t) (b+d) =  \underbrace{t a + (1-t) b}_x + \underbrace{t c + (1-t) d}_y
\]
Because intervals are convex we have $x \ensuremath{\in} X$ and $y \ensuremath{\in} Y$ and hence $z \ensuremath{\in} X+Y$. 

The remaining two proofs are left for the problem sheet. 

\ensuremath{\QED}

We want to  implement floating point variants of these operations that are guaranteed to contain the true set arithmetic operations. We do so as follows:

\begin{definition}[floating point interval arithmetic] For intervals  $A = [a,b]$ and $B = [c,d]$ satisfying $0 < a < b$ and $0 < c < d$, and $n > 0$, define:


\begin{align*}
[a,b] \ensuremath{\oplus} [c,d] &:= [{\rm fl}^{\rm down}(a+c), {\rm fl}^{\rm up}(b+d)] \\
[a,b] \ensuremath{\oslash} n &:= [{\rm fl}^{\rm down}(a/n), {\rm fl}^{\rm up}(b/n)] \\
[a,b] \ensuremath{\otimes} [c,d] &:= [{\rm fl}^{\rm down}(ac), {\rm fl}^{\rm up}(bd)]
\end{align*}
\end{definition}

\begin{example}[small sum] consider evaluating the first few terms in the Taylor series of the exponential at $x = 1$ using interval arithmetic with half-precision $F_{16}$ arithmetic.  The first three terms are exact since all numbers involved are exactly floats, in particular if we evaluate $1 + x + x^2/2$ with $x = 1$ we get
\[
1 + 1 + 1/2 \ensuremath{\in} 1 \ensuremath{\oplus} [1,1] \ensuremath{\oplus} ([1,1] \ensuremath{\otimes} [1,1]) \ensuremath{\oslash} 2 = [5/2, 5/2]
\]
Noting that 
\[
1/6 = (1/3)/2 = 2^{-3} (1.01010101\ensuremath{\ldots})_2
\]
we can extend the computation to another term:


\begin{align*}
1 + 1 + 1/2 + 1/6 &\ensuremath{\in} [5/2,5/2] \ensuremath{\oplus} ([1,1] \ensuremath{\oslash} 6) \ccr
= [2 (1.01)_2, 2 (1.01)_2] \ensuremath{\oplus} 2^{-3}[(1.0101010101)_2, (1.0101010110)_2] \ccr
= [{\rm fl}^{\rm down}(2 (1.0101010101\red{0101})_2), {\rm fl}^{\rm up}(2 (1.0101010101\red{011})_2)] \ccr
= [2(1.0101010101)_2, 2(1.0101010110)_2] \ccr 
= [2.666015625, 2.66796875]
\end{align*}
\end{example}

\begin{example}[exponential with intervals] Consider computing $\exp(x)$ for $0 \ensuremath{\leq} x \ensuremath{\leq} 1$ from the Taylor series approximation:
\[
\exp(x) = \sum_{k=0}^n {x^k \over k!} + \underbrace{\exp(t){x^{n+1} \over (n+1)!}}_{\ensuremath{\delta}_{x,n}}
\]
where we can bound the error by (using the fact that $\ensuremath{\euler} = 2.718\ensuremath{\ldots} \ensuremath{\leq} 3$)
\[
|\ensuremath{\delta}_{x,n}| \ensuremath{\leq} {\exp(1) \over (n+1)!} \ensuremath{\leq} {3 \over (n+1)!}.
\]
Put another way: $\ensuremath{\delta}_{x,n} \ensuremath{\in} \left[-{3 \over (n+1)!}, {3 \over (n+1)!}\right]$. We can use this to adjust the bounds derived from interval arithmetic for the interval arithmetic expression:
\[
\exp(X) \ensuremath{\subseteq} \left(\ensuremath{\bigoplus}_{k=0}^n {X^k \ensuremath{\oslash} k!}\right) \ensuremath{\oplus} \left[-{3 \over (n+1)!}, {3 \over (n+1)!}\right]
\]
For example, with $n = 3$ we have $|\ensuremath{\delta}_{1,2}| \ensuremath{\leq} 3/4! = 1/2^3$. Thus we can prove that:
\meeq{
\ensuremath{\euler} = 1 + 1 + 1/2 + 1/6 + \ensuremath{\delta}_x \ensuremath{\in} [2(1.0101010101)_2, 2(1.0101010110)_2] \ensuremath{\oplus} [-1/2^3, 1/2^3] \ccr
= [2(1.0100010101)_2, 2(1.0110010110)_2] = [2.541015625,2.79296875]
}
In the lab we get many more digits by using a computer to compute the bounds. \end{example}



