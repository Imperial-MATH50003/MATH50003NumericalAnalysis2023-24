
\section{Integers}
In this section we discuss the following:

\begin{itemize}
\item[1. ] Unsigned integers: how computers represent non-negative integers using only $p$-bits, via \href{https://en.wikipedia.org/wiki/Modular_arithmetic}{modular arithmetic}.


\item[2. ] Signed integers: how negative integers are handled using the \href{https://en.wikipedia.org/wiki/Two's_complement}{Two's-complement} format.

\end{itemize}
Mathematically, CPUs only act on $p$-bits at a time, with $2^p$ possible sequences. That is, essentially all functions $f$ are either of the form $f : \ensuremath{\bbZ}_{2^p} \ensuremath{\rightarrow} \ensuremath{\bbZ}_{2^p}$ or  $f : \ensuremath{\bbZ}_{2^p} \ensuremath{\times} \ensuremath{\bbZ}_{2^p} \ensuremath{\rightarrow} \ensuremath{\bbZ}_{2^p}$, where we use the following notation:

\begin{definition}[signed integers] Denote the
\[
\ensuremath{\bbZ}_m := \{0 , 1 , \ensuremath{\ldots}, m-1 \}
\]
\end{definition}

The limitations this imposes on representing integers is substantial. If we have an implementation of $+$, which we shall denote $\ensuremath{\oplus}_m$, how can we possibly represent $m + 1$ in this implementation when the result is above the largest possible integer?

The solution that is used is straightforward: \emph{modular arithmetic}. E.g., we have
\[
(m-1) \ensuremath{\oplus}_m 1 = m\ ({\rm mod}\ m) = 0.
\]
In this section we discuss the implications of this approach and how it works with negative numbers.

We will write integers in binary format, that is, as sequence of \texttt{0}s and \texttt{1}s:

\begin{definition}[binary format] For $B_0,\ldots,B_p \in \{0,1\}$ denote an integer in \emph{binary format} by:
\[
\ensuremath{\pm}(B_p\ldots B_1B_0)_2 := \ensuremath{\pm}\sum_{k=0}^p B_k 2^k
\]
\end{definition}

\begin{example}[integers in binary] A simple integer example is $5 = 2^2 + 2^0 = (101)_2$. On the other hand, we write $-5 = -(101)_2$. Another example is $258 = 2^8 + 2 = (1000000010)_2$. \end{example}

\subsection{Unsigned Integers}
Computers represent integers by a finite number of $p$ bits, with $2^p$ possible combinations of 0s and 1s. For \emph{unsigned integers} (non-negative integers) these bits dictate the first $p$ binary digits: $(B_{p-1}\ldots B_1B_0)_2$.

Integers on a computer follow \href{https://en.wikipedia.org/wiki/Modular_arithmetic}{modular arithmetic}: Integers represented with $p$-bits on a computer actually represent elements of ${\mathbb Z}_{2^p}$ and integer arithmetic on a computer is equivalent to arithmetic modulo $2^p$. We denote modular arithmetic with $m = 2^p$ as follows:


\begin{align*}
x \ensuremath{\oplus}_m y &:= (x+y)\ ({\rm mod}\ m) \\
x \ensuremath{\ominus}_m y &:= (x-y)\ ({\rm mod}\ m) \\
x \ensuremath{\otimes}_m y &:= (x*y)\ ({\rm mod}\ m)
\end{align*}
When $m$ is implied by context we just write $\ensuremath{\oplus}, \ensuremath{\ominus}, \ensuremath{\otimes}$.

\begin{example}[arithmetic with  8-bit unsigned integers] If  the result of an operation lies between $0$ and $m = 2^8 = 256$ then airthmetic works exactly like standard integer arithmetic. For example,


\begin{align*}
17 \ensuremath{\oplus}_{256} 3 = 20\ ({\rm mod}\ 256) = 20 \\
17 \ensuremath{\ominus}_{256} 3 = 14\ ({\rm mod}\ 256) = 14
\end{align*}
\end{example}

\begin{example}[overflow with 8-bit unsigned integers] If we go beyond the range the result \ensuremath{\ldq}wraps around". For example, with true integers we have
\[
255 + 1 = (11111111)_2 + (00000001)_2 = (100000000)_2 = 256
\]
However, the result is impossible to store in just 8-bits! So as mentioned instead it treats the integers as elements of ${\mathbb Z}_{256}$:
\[
255 \ensuremath{\oplus}_{256} 1 = 255 + 1 \ ({\rm mod}\ 256) = (00000000)_2 \ ({\rm mod}\ 256) = 0 \ ({\rm mod}\ 256)
\]
On the other hand, if we go below $0$ we wrap around from above:
\[
3 \ensuremath{\ominus}_{256} 5 = -2\ ({\rm mod}\ 256) = 254 = (11111110)_2
\]
\end{example}

\begin{example}[multiplication of 8-bit unsigned integers] Multiplication works similarly: for example,
\[
254 \ensuremath{\otimes}_{256} 2 = 254 * 2 \ ({\rm mod}\ 256) = 252 \ ({\rm mod}\ 256) = (11111100)_2 \ ({\rm mod}\ 256)
\]
\end{example}

\subsection{Signed integer}
Signed integers use the \href{https://epubs.siam.org/doi/abs/10.1137/1.9780898718072.ch3}{Two's complemement} convention. The convention is if the first bit is 1 then the number is negative: the number $2^p - y$ is interpreted as $-y$. Thus for $p = 8$ we are interpreting $2^7$ through $2^8-1$ as negative numbers. More precisely:

\textbf{Definition ($\ensuremath{\bbZ}_{2^p}^{\rm s}$, signed integers)}
\[
\ensuremath{\bbZ}_{2^p}^{\rm s} := \{-2^{p-1} ,\ensuremath{\ldots}, -1 ,0,1, \ensuremath{\ldots}, 2^{p-1}-1 \}
\]
\ensuremath{\QED}

\begin{definition}[Shifted mod] Define for $y = x\ ({\rm mod}\ 2^p)$
\[
x\ ({\rm mod}^{\rm s}\ 2^p) := \begin{cases} y & 0 \ensuremath{\leq} y \ensuremath{\leq} 2^{p-1}-1 \\
                             y - 2^p & 2^{p-1} \ensuremath{\leq} y \ensuremath{\leq} 2^p - 1
                             \end{cases}
\]
\end{definition}

Note that if $R_p(x) = x\ ({\rm mod}^{\rm s}\ 2^p)$ then it can be viewed as a map $R_p : \ensuremath{\bbZ} \ensuremath{\rightarrow} \ensuremath{\bbZ}_{2^p}^{\rm s}$ or a one-to-one map $R_p : \ensuremath{\bbZ}_{2^p} \ensuremath{\rightarrow} \ensuremath{\bbZ}_{2^p}^{\rm s}$ whose inverse is $R_p^{-1}(x) = x (\mod 2^p)$. It can also be viewed as the identity map on signed integers $R_p : \ensuremath{\bbZ}_{2^p}^{\rm s} \ensuremath{\rightarrow} \ensuremath{\bbZ}_{2^p}^{\rm s}$, that is,  $R_p(x) = x$ if $x \in \ensuremath{\bbZ}_{2^p}^{\rm s}$.

Arithmetic works precisely the same for signed and unsigned integers up to the mapping $R_p$, e.g. we have
\[
x \ensuremath{\oplus}_{2^p}^{\rm s} y := x + y\ ({\rm mod}^{\rm s}\ 2^p)
\]
\begin{example}[addition of 8-bit signed integers] Consider \texttt{(-1) + 1} in 8-bit arithmetic. The number $-1$ has the same bits as $2^8 - 1 = 255$. Thus this is equivalent to the previous question and we get the correct result of \texttt{0}. In other words:
\[
-1 \ensuremath{\oplus}_{256}^{\rm s} 1 = -1 + 1 \ ({\rm mod}^{\rm s}\ 256) = 0
\]
But on the bit level this computation is exactly the same as unsigned integers. We represent the number $-1$ using the bits \texttt{11111111} (i.e., we store it equivalently to  $(11111111)_2 = 255$) and the  number $1$ is stored using the bits \texttt{00000001}. When we add this with true integer arithmetic we have


\begin{align*}
(0 11111111)_2 &\ + \\
(0 00000001)_2 &\ = \\
(1 00000000)_2&
\end{align*}
The modular arithmetic drops the leading $1$ and we are left with all zeros.

\end{example}

\begin{example}[signed overflow with 8-bit signed integers] If we go above $2^{p-1}-1 = 2^7 - 1 = 127$  we have perhaps unexpected results:
\[
127 \ensuremath{\oplus}_{256}^{\rm s} 1 = 128\  ({\rm mod}^{\rm s}\ 256) = 128 - 256 = -128.
\]
Again on the bit level this computation is exactly the same as unsigned integers. We represent the number $127$ using the bits \texttt{01111111} and the  number $1$ is stored using the bits \texttt{00000001}. When we add this with true integer arithmetic we have


\begin{align*}
(01111111)_2 &\ + \\
(00000001)_2 &\ = \\
(10000000)_2&
\end{align*}
Because the first bit is \texttt{1} we interpret this as a negative number using the formula:
\[
(10000000)_2\ ({\rm mod}^{\rm s}\ 256) = 128   ({\rm mod}^{\rm s}\ 256) = -128.
\]
\end{example}

\begin{example}[multiplication of 8-bit signed integers] Consider \texttt{(-2) * 2}. $-2$ has the same bits as $2^{256} - 2 = 254$ and $-4$ has the same bits as $2^{256}-4 = 252$, and hence from the previous example we get the correct result of \texttt{-4}. In other words:
\[
(-2) \ensuremath{\otimes}_{2^p}^{\rm s} 2 = -4 \ ({\rm mod}^{\rm s}\ 2^p) = -4
\]
On the bit level, the bits of $-2$ (which is one less than $-1$) are \texttt{11111110}. Multiplying by 2 is like multiplying by 10 in base-10, that is, we shift the bits. Hence in true arithmetic we have


\begin{align*}
(0 11111110)_2 & * 2 = \\
(1 11111100)_2&
\end{align*}
We drop the leading 1 due to modular arithmetic. We still have a leading $1$ hence the number is viewed as negative. In particular we have
\meeq{
(1 11111100)_2 \ ({\rm mod}^{\rm s}\ 256) = (11111100)_2 \ ({\rm mod}^{\rm s}\ 256) = 
2^7+2^6+2^5+2^4+2^3+2^2 \ ({\rm mod}^{\rm s}\ 256) \ccr
 = 252  \ ({\rm mod}^{\rm s}\ 256) = -4.
}
\end{example}

\subsection{Hexadecimal format}
In coding it is often convenient to use base-16 as it is a power of $2$ but uses less characters than binary. The digits used are $0$ through $9$ followed by $a = 10$, $b = 11$, $c = 12$, $d = 13$, $e = 14$, and $f = 15$. 

\begin{example}[Hexadecimal number] We can interpret a number in format as follows:
\[
(a5f2)_{16} = a*16^3 + 5*16^2 + f*16 + 2 = 
10*16^3 + 5*16^2 + 15*16 + 2 = 42,482
\]
\end{example}

We will see in the labs that unsigned integers are displayed in base-16.



