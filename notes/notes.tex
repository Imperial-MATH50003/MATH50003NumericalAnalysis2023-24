\documentclass[12pt,a4paper]{book}

\usepackage[a4paper,text={16.5cm,25.2cm},centering]{geometry}
\usepackage{lmodern}
\usepackage{amssymb,amsmath}
\usepackage{bm}
\usepackage{graphicx}
\usepackage{microtype}
\usepackage{hyperref}
\usepackage{amsthm}
\usepackage{upquote}
\usepackage{listings}
\usepackage{appendix}
\usepackage[usename®s,dvipsnames]{xcolor}
\setlength{\parindent}{0pt}
\setlength{\parskip}{1.2ex}

\lstset{
    basicstyle=\ttfamily\footnotesize,
    upquote=true,
    breaklines=true,
    breakindent=0pt,
    keepspaces=true,
    showspaces=false,
    columns=fullflexible,
    showtabs=false,
    showstringspaces=false,
    escapeinside={(*@}{@*)},
    extendedchars=true,
}

\newcommand{\HLJLt}[1]{#1}
\newcommand{\HLJLw}[1]{#1}
\newcommand{\HLJLe}[1]{#1}
\newcommand{\HLJLeB}[1]{#1}
\newcommand{\HLJLo}[1]{#1}
\newcommand{\HLJLk}[1]{\textcolor[RGB]{148,91,176}{\textbf{#1}}}
\newcommand{\HLJLkc}[1]{\textcolor[RGB]{59,151,46}{\textit{#1}}}
\newcommand{\HLJLkd}[1]{\textcolor[RGB]{214,102,97}{\textit{#1}}}
\newcommand{\HLJLkn}[1]{\textcolor[RGB]{148,91,176}{\textbf{#1}}}
\newcommand{\HLJLkp}[1]{\textcolor[RGB]{148,91,176}{\textbf{#1}}}
\newcommand{\HLJLkr}[1]{\textcolor[RGB]{148,91,176}{\textbf{#1}}}
\newcommand{\HLJLkt}[1]{\textcolor[RGB]{148,91,176}{\textbf{#1}}}
\newcommand{\HLJLn}[1]{#1}
\newcommand{\HLJLna}[1]{#1}
\newcommand{\HLJLnb}[1]{#1}
\newcommand{\HLJLnbp}[1]{#1}
\newcommand{\HLJLnc}[1]{#1}
\newcommand{\HLJLncB}[1]{#1}
\newcommand{\HLJLnd}[1]{\textcolor[RGB]{214,102,97}{#1}}
\newcommand{\HLJLne}[1]{#1}
\newcommand{\HLJLneB}[1]{#1}
\newcommand{\HLJLnf}[1]{\textcolor[RGB]{66,102,213}{#1}}
\newcommand{\HLJLnfm}[1]{\textcolor[RGB]{66,102,213}{#1}}
\newcommand{\HLJLnp}[1]{#1}
\newcommand{\HLJLnl}[1]{#1}
\newcommand{\HLJLnn}[1]{#1}
\newcommand{\HLJLno}[1]{#1}
\newcommand{\HLJLnt}[1]{#1}
\newcommand{\HLJLnv}[1]{#1}
\newcommand{\HLJLnvc}[1]{#1}
\newcommand{\HLJLnvg}[1]{#1}
\newcommand{\HLJLnvi}[1]{#1}
\newcommand{\HLJLnvm}[1]{#1}
\newcommand{\HLJLl}[1]{#1}
\newcommand{\HLJLld}[1]{\textcolor[RGB]{148,91,176}{\textit{#1}}}
\newcommand{\HLJLs}[1]{\textcolor[RGB]{201,61,57}{#1}}
\newcommand{\HLJLsa}[1]{\textcolor[RGB]{201,61,57}{#1}}
\newcommand{\HLJLsb}[1]{\textcolor[RGB]{201,61,57}{#1}}
\newcommand{\HLJLsc}[1]{\textcolor[RGB]{201,61,57}{#1}}
\newcommand{\HLJLsd}[1]{\textcolor[RGB]{201,61,57}{#1}}
\newcommand{\HLJLsdB}[1]{\textcolor[RGB]{201,61,57}{#1}}
\newcommand{\HLJLsdC}[1]{\textcolor[RGB]{201,61,57}{#1}}
\newcommand{\HLJLse}[1]{\textcolor[RGB]{59,151,46}{#1}}
\newcommand{\HLJLsh}[1]{\textcolor[RGB]{201,61,57}{#1}}
\newcommand{\HLJLsi}[1]{#1}
\newcommand{\HLJLso}[1]{\textcolor[RGB]{201,61,57}{#1}}
\newcommand{\HLJLsr}[1]{\textcolor[RGB]{201,61,57}{#1}}
\newcommand{\HLJLss}[1]{\textcolor[RGB]{201,61,57}{#1}}
\newcommand{\HLJLssB}[1]{\textcolor[RGB]{201,61,57}{#1}}
\newcommand{\HLJLnB}[1]{\textcolor[RGB]{59,151,46}{#1}}
\newcommand{\HLJLnbB}[1]{\textcolor[RGB]{59,151,46}{#1}}
\newcommand{\HLJLnfB}[1]{\textcolor[RGB]{59,151,46}{#1}}
\newcommand{\HLJLnh}[1]{\textcolor[RGB]{59,151,46}{#1}}
\newcommand{\HLJLni}[1]{\textcolor[RGB]{59,151,46}{#1}}
\newcommand{\HLJLnil}[1]{\textcolor[RGB]{59,151,46}{#1}}
\newcommand{\HLJLnoB}[1]{\textcolor[RGB]{59,151,46}{#1}}
\newcommand{\HLJLoB}[1]{\textcolor[RGB]{102,102,102}{\textbf{#1}}}
\newcommand{\HLJLow}[1]{\textcolor[RGB]{102,102,102}{\textbf{#1}}}
\newcommand{\HLJLp}[1]{#1}
\newcommand{\HLJLc}[1]{\textcolor[RGB]{153,153,119}{\textit{#1}}}
\newcommand{\HLJLch}[1]{\textcolor[RGB]{153,153,119}{\textit{#1}}}
\newcommand{\HLJLcm}[1]{\textcolor[RGB]{153,153,119}{\textit{#1}}}
\newcommand{\HLJLcp}[1]{\textcolor[RGB]{153,153,119}{\textit{#1}}}
\newcommand{\HLJLcpB}[1]{\textcolor[RGB]{153,153,119}{\textit{#1}}}
\newcommand{\HLJLcs}[1]{\textcolor[RGB]{153,153,119}{\textit{#1}}}
\newcommand{\HLJLcsB}[1]{\textcolor[RGB]{153,153,119}{\textit{#1}}}
\newcommand{\HLJLg}[1]{#1}
\newcommand{\HLJLgd}[1]{#1}
\newcommand{\HLJLge}[1]{#1}
\newcommand{\HLJLgeB}[1]{#1}
\newcommand{\HLJLgh}[1]{#1}
\newcommand{\HLJLgi}[1]{#1}
\newcommand{\HLJLgo}[1]{#1}
\newcommand{\HLJLgp}[1]{#1}
\newcommand{\HLJLgs}[1]{#1}
\newcommand{\HLJLgsB}[1]{#1}
\newcommand{\HLJLgt}[1]{#1}


\let\QED=\blacksquare
\def\bbD{{\mathbb D}}
\def\bbZ{{\mathbb Z}}
\def\bbN{{\mathbb N}}
\def\bbF{{\mathbb F}}
\def\bbR{{\mathbb R}}
\def\bbT{{\mathbb T}}
\def\bbC{{\mathbb C}}
\def\emdash{\hbox{---}}
\def\endash{\hbox{--}}
\def\nsubset{\not\subset}
\def\ldq{``}
\def\x{{\vc x}}
\def\a{{\vc a}}
\def\b{{\vc b}}
\def\q{{\vc q}}
\def\c{{\vc c}}
\def\e{{\vc e}}
\def\f{{\vc f}}
\def\u{{\vc u}}
\def\w{{\vc w}}
\def\v{{\vc v}}
\def\y{{\vc y}}
\def\z{{\vc z}}
\def\k{{\vc k}}
\def\vchatf{{\vc {\hat f}}}
\def\zero{{\vc 0}}
\def\Lt{{\tilde L}}
\def\Pt{{\tilde P}}
\def\pt{{\tilde p}}
\def\Ut{{\tilde U}}
\def\baralpha{\bar\alpha}
\def\At{\tilde A}
\def\Rt{\tilde R}
\def\red#1{{\color{red} #1}}
\def\blue#1{{\color{blue} #1}}
\def\green#1{{\color{ForestGreen} #1}}
\def\euler{\E}
\def\ocaret{\wedge\mkern-19mu \bigcirc\,}

\def\fldown{{\rm fl}^{\rm down}}
\def\flup{{\rm fl}^{\rm up}}

\hypersetup
       {   pdfauthor = { {{Sheehan Olver}} },
           pdftitle={ {{MATH50003 Numerical Analysis}} },
           colorlinks=TRUE,
           linkcolor=black,
           citecolor=blue,
           urlcolor=blue
       }

\title{ MATH50003 Numerical Analysis }


\newtheorem{lemma}{Lemma}
\newtheorem{theorem}{Theorem}
\newtheorem{proposition}{Proposition}
\newtheorem{corollary}{Corollary}

\theoremstyle{definition}
\newtheorem{definition}{Definition}
\newtheorem{example}{Example}

\author{ Sheehan Olver }
\renewcommand{\thechapter}{\Roman{chapter}}


\def\addtab#1={#1\;&=}

\def\meeq#1{\def\ccr{\\\addtab}
%\tabskip=\@centering
 \begin{align*}
 \addtab#1
 \end{align*}
  }  
  
  \def\leqaddtab#1\leq{#1\;&\leq}
  \def\mleeq#1{\def\ccr{\\\addtab}
%\tabskip=\@centering
 \begin{align*}
 \leqaddtab#1
 \end{align*}
  }  


\def\vc#1{\mbox{\boldmath$#1$\unboldmath}}

\def\vcsmall#1{\mbox{\boldmath$\scriptstyle #1$\unboldmath}}

\def\vczero{{\mathbf 0}}


%\def\beginlist{\begin{itemize}}
%
%\def\endlist{\end{itemize}}


\def\pr(#1){\left({#1}\right)}
\def\br[#1]{\left[{#1}\right]}
\def\fbr[#1]{\!\left[{#1}\right]}
\def\set#1{\left\{{#1}\right\}}
\def\ip<#1>{\left\langle{#1}\right\rangle}
\def\iip<#1>{\left\langle\!\langle{#1}\right\rangle\!\rangle}

\def\norm#1{\left\| #1 \right\|}

\def\abs#1{\left|{#1}\right|}
\def\fpr(#1){\!\pr({#1})}

\def\Re{{\rm Re}\,}
\def\Im{{\rm Im}\,}

\def\floor#1{\left\lfloor#1\right\rfloor}
\def\ceil#1{\left\lceil#1\right\rceil}


\def\mapengine#1,#2.{\mapfunction{#1}\ifx\void#2\else\mapengine #2.\fi }

\def\map[#1]{\mapengine #1,\void.}

\def\mapenginesep_#1#2,#3.{\mapfunction{#2}\ifx\void#3\else#1\mapengine #3.\fi }

\def\mapsep_#1[#2]{\mapenginesep_{#1}#2,\void.}


\def\vcbr{\br}


\def\bvect[#1,#2]{
{
\def\dots{\cdots}
\def\mapfunction##1{\ | \  ##1}
\begin{pmatrix}
		 \,#1\map[#2]\,
\end{pmatrix}
}
}

\def\vect[#1]{
{\def\dots{\ldots}
	\vcbr[{#1}]
}}

\def\vectt[#1]{
{\def\dots{\ldots}
	\vect[{#1}]^{\top}
}}

\def\Vectt[#1]{
{
\def\mapfunction##1{##1 \cr} 
\def\dots{\vdots}
	\begin{pmatrix}
		\map[#1]
	\end{pmatrix}
}}



\def\thetaB{\mbox{\boldmath$\theta$}}
\def\zetaB{\mbox{\boldmath$\zeta$}}


\def\newterm#1{{\it #1}\index{#1}}


\def\TT{{\mathbb T}}
\def\C{{\mathbb C}}
\def\R{{\mathbb R}}
\def\II{{\mathbb I}}
\def\F{{\mathcal F}}
\def\E{{\rm e}}
\def\I{{\rm i}}
\def\D{{\rm d}}
\def\dx{\D x}
\def\ds{\D s}
\def\dt{\D t}
\def\CC{{\cal C}}
\def\DD{{\cal D}}
\def\U{{\mathbb U}}
\def\A{{\cal A}}
\def\K{{\cal K}}
\def\DTU{{\cal D}_{{\rm T} \rightarrow {\rm U}}}
\def\LL{{\cal L}}
\def\B{{\cal B}}
\def\T{{\cal T}}
\def\W{{\cal W}}


\def\tF_#1{{\tt F}_{#1}}
\def\Fm{\tF_m}
\def\Fab{\tF_{\alpha,\beta}}
\def\FC{\T}
\def\FCpmz{\FC^{\pm {\rm z}}}
\def\FCz{\FC^{\rm z}}

\def\tFC_#1{{\tt T}_{#1}}
\def\FCn{\tFC_n}

\def\rmz{{\rm z}}

\def\chapref#1{Chapter~\ref{Chapter:#1}}
\def\secref#1{Section~\ref{Section:#1}}
\def\exref#1{Exercise~\ref{Exercise:#1}}
\def\lmref#1{Lemma~\ref{Lemma:#1}}
\def\propref#1{Proposition~\ref{Proposition:#1}}
\def\warnref#1{Warning~\ref{Warning:#1}}
\def\thref#1{Theorem~\ref{Theorem:#1}}
\def\defref#1{Definition~\ref{Definition:#1}}
\def\probref#1{Problem~\ref{Problem:#1}}
\def\corref#1{Corollary~\ref{Corollary:#1}}

\def\sgn{{\rm sgn}\,}
\def\Ai{{\rm Ai}\,}
\def\Bi{{\rm Bi}\,}
\def\wind{{\rm wind}\,}
\def\erf{{\rm erf}\,}
\def\erfc{{\rm erfc}\,}
\def\qqquad{\qquad\quad}
\def\qqqquad{\qquad\qquad}


\def\spand{\hbox{ and }}
\def\spodd{\hbox{ odd}}
\def\speven{\hbox{ even}}
\def\qand{\quad\hbox{and}\quad}
\def\qqand{\qquad\hbox{and}\qquad}
\def\qfor{\quad\hbox{for}\quad}
\def\qqfor{\qquad\hbox{for}\qquad}
\def\qas{\quad\hbox{as}\quad}
\def\qqas{\qquad\hbox{as}\qquad}
\def\qor{\quad\hbox{or}\quad}
\def\qqor{\qquad\hbox{or}\qquad}
\def\qqwhere{\qquad\hbox{where}\qquad}



%%% Words

\def\naive{na\"\i ve\xspace}
\def\Jmap{Joukowsky map\xspace}
\def\Mobius{M\"obius\xspace}
\def\Holder{H\"older\xspace}
\def\Mathematica{{\sc Mathematica}\xspace}
\def\apriori{apriori\xspace}
\def\WHf{Weiner--Hopf factorization\xspace}
\def\WHfs{Weiner--Hopf factorizations\xspace}

\def\Jup{J_\uparrow^{-1}}
\def\Jdown{J_\downarrow^{-1}}
\def\Jin{J_+^{-1}}
\def\Jout{J_-^{-1}}



\def\bD{\D\!\!\!^-}




\def\questionequals{= \!\!\!\!\!\!{\scriptstyle ? \atop }\,\,\,}

\def\elll#1{\ell^{\lambda,#1}}
\def\elllp{\ell^{\lambda,p}}
\def\elllRp{\ell^{(\lambda,R),p}}


\def\elllRpz_#1{\ell_{#1{\rm z}}^{(\lambda,R),p}}


\def\sopmatrix#1{\begin{pmatrix}#1\end{pmatrix}}


\def\bbR{{\mathbb R}}
\def\bbC{{\mathbb C}}


\begin{document}

\maketitle

\tableofcontents

\chapter{Calculus on a Computer}

In this first chapter we explore the basics of mathematical computing and numerical analysis.
In particular we investigate the following mathematical problems which can not in general be solved exactly:

\begin{enumerate}
\item Integration. General integrals have no closed form expressions. Can we use a computer to approximate the values of definite integrals?
\item Differentiation. Differentiating a formula as in calculus is usually algorithmic, however, it is often needed to compute derivatives without access to an underlying formula, eg,  a function defined only in code. Can we use a computer to approximate derivatives?  A very important application is in Machine Learning, where there is a need to compute gradients to determine the ``right" weights in a neural network. 
\item Root finding. There is no general formula for finding roots (zeros) of arbitrary functions, or even polynomials that are of degree 5 (quintics) or higher. Can we compute roots of general functions using a computer?
\end{enumerate}

In this chapter we discuss:

\begin{enumerate}
\item I.1 Rectangular rule: we review the rectangular rule for integration and deduce the {\it converge rate} of the approximation. In the lab/problem sheet  we investigate its implementation as well as extensions to the Trapezium rule. 
\item I.2 Divided differences: we investigate approximating derivatives by a divided difference and again deduce the convergence rates. In the lab/problem sheet we extend the approach to the central differences formula and computing second derivatives. We also observe a mystery: the approximations may have significant errors in practice, and there is a limit to the accuracy.
\item I.3 Dual numbers: we introduce the algebraic notion of a {\it dual number} which allows the implemention of {\it forward-mode automatic differentiation}, a high accuracy alternative to divided differences for computing derivatives.
\item I.4 Newton's method: Newton's method is a basic approach for computing roots/zeros of a function. We use dual numbers to implement this algorithm.
\end{enumerate}




\section{Rectangular rule}
One possible definition for an integral is the limit of a Riemann sum, for example:
\[
  \ensuremath{\int}_a^b f(x) {\rm d}x = \lim_{n \ensuremath{\rightarrow} \ensuremath{\infty}} h \ensuremath{\sum}_{j=1}^n f(x_j)
\]
where $x_j = a+jh$ are evenly spaced points dividing up the interval $[a,b]$, that is  with the \emph{step size} $h = (b-a)/n$. This suggests an algorithm known as the \emph{(right-sided) rectangular rule} for approximating an integral: choose $n$ large so that
\[
  \ensuremath{\int}_a^b f(x) {\rm d}x \ensuremath{\approx} h \ensuremath{\sum}_{j=1}^n f(x_j).
\]
In the lab we explore practical implementation of this approximation, and observe that the error in approximation is bounded by $C/n$ for some constant $C$. This can be expressed using "Big-O" notation:
\[
\ensuremath{\int}_a^b f(x) {\rm d}x = h \ensuremath{\sum}_{j=1}^n f(x_j) + O(1/n).
\]
In these notes we consider the $``$Analysis" part of $``$Numerical Analysis": we want to \emph{prove} the convergence rate of the approximation, including finding an explicit expression for the constant $C$.

To tackle this question we consider the error incurred on a single $``$rectangle", then sum up the errors on rectangles.

Now for a secret. There are only so many tools available in analysis (especially at this stage of your career), and one can make a safe bet that the right tool in any analysis proof is either (1) integration-by-parts, (2) geometric series or (3) Taylor series. In this case we use (1):

\begin{lemma}[(Right-sided) Rectangular Rule error on one panel] Assuming $f$ is differentiable we have
\[
\ensuremath{\int}_a^b f(x) {\rm d}x = (b-a) f(b) + \ensuremath{\delta}
\]
where $|\ensuremath{\delta}| \ensuremath{\leq} M (b-a)^2$ for $M = \sup_{a \ensuremath{\leq} x \ensuremath{\leq} b}|f'(x)|$.

\end{lemma}
\textbf{Proof} We write
\meeq{
\ensuremath{\int}_a^b f(x) {\rm d}x = \ensuremath{\int}_a^b (x-a)' f(x)  {\rm d}x = [(x-a) f(x)]_a^b - \ensuremath{\int}_a^b (x-a) f'(x) {\rm d} x \ccr
= (b-a) f(b) + \underbrace{\left(-\ensuremath{\int}_a^b (x-a) f'(x) {\rm d} x \right)}_\ensuremath{\delta}.
}
Recall that we can bound the absolute value of an integral by the sepremum of the integrand times the width of the integration interval:
\[
\abs{\ensuremath{\int}_a^b g(x) {\rm d} x} \ensuremath{\leq} (b-a) \sup_{a \ensuremath{\leq} x \ensuremath{\leq} b}|g(x)|.
\]
The lemma thus follows since
\[
\abs{\ensuremath{\int}_a^b (x-a) f'(x) {\rm d} x} \ensuremath{\leq} (b-a) \sup_{a \ensuremath{\leq} x \ensuremath{\leq} b}|(x-a) f'(x)| \ensuremath{\leq} M (b-a)^2.
\]
\ensuremath{\QED}

Now summing up the errors in each panel gives us the error of using the Rectangular rule:

\begin{theorem}[Rectangular Rule error] Assuming $f$ is differentiable we have
\[
\ensuremath{\int}_a^b f(x) {\rm d}x =  h \ensuremath{\sum}_{j=1}^n f(x_j) +  \ensuremath{\delta}
\]
where $|\ensuremath{\delta}| \ensuremath{\leq} M (b-a) h$ for $M = \sup_{a \ensuremath{\leq} x \ensuremath{\leq} b}|f'(x)|$, $h = (b-a)/n$ and $x_j = a + jh$.

\end{theorem}
\textbf{Proof} We split the integral into a sum of smaller integrals:
\[
\ensuremath{\int}_a^b f(x) {\rm d}x = \ensuremath{\sum}_{j=1}^n  \ensuremath{\int}_{x_{j-1}}^{x_j} f(x) {\rm d}x =
\ensuremath{\sum}_{j=1}^n  \br[(x_j - x_{j-1}) f(x_j) + \ensuremath{\delta}_j] =  h \ensuremath{\sum}_{j=1}^n f(x_j) +  \underbrace{\ensuremath{\sum}_{j=1}^n \ensuremath{\delta}_j}_\ensuremath{\delta}
\]
where $\ensuremath{\delta}_j$, the error on each panel as in the preceding lemma, satisfies
\[
|\ensuremath{\delta}_j| \ensuremath{\leq} (x_j-x_{j-1})^2 \sup_{x_{j-1} \ensuremath{\leq} x \ensuremath{\leq} x_j}|f'(x)| \ensuremath{\leq} M h^2.
\]
Thus using the triangular inequality we have
\[
|\ensuremath{\delta}| = \abs{ \ensuremath{\sum}_{j=1}^n \ensuremath{\delta}_j} \ensuremath{\leq} \ensuremath{\sum}_{j=1}^n |\ensuremath{\delta}_j| \ensuremath{\leq} M n h^2 = M(b-a)h.
\]
\ensuremath{\QED}

Note a consequence of this lemma is that the approximation converges as $n \ensuremath{\rightarrow} \ensuremath{\infty}$ (i.e. $h \ensuremath{\rightarrow} 0$). In the labs and problem sheets we will consider the left-sided rule:
\[
\ensuremath{\int}_a^b f(x) {\rm d}x \ensuremath{\approx}  h \ensuremath{\sum}_{j=0}^{n-1} f(x_j).
\]
We also consider the \emph{Trapezium rule}. Here we approximate an integral  by an affine function:
\[
\ensuremath{\int}_a^b f(x) {\rm d} x \ensuremath{\approx} \ensuremath{\int}_a^b {(b-x)f(a) + (x-a)f(b) \over b-a} \dx
= {b-a \over 2} \br[f(a) + f(b)].
\]
Subdividing an interval $a = x_0 < x_1 < \ensuremath{\ldots} < x_n = b$ and applying this approximation separately on each subinterval $[x_{j-1},x_j]$, where $h = (b-a)/n$ and $x_j = a + jh$, leads to the approximation
\[
\ensuremath{\int}_a^b f(x) {\rm d}x \ensuremath{\approx}  {h \over 2} f(a) + h \ensuremath{\sum}_{j=1}^{n-1} f(x_j) + {h \over 2} f(b)
\]
We shall see both experimentally and provably that this approximation converges faster than the rectangular rule.





\section{Divided Differences}
Given a function, how can we approximate its derivative at a point? We consider an intuitive approach to this problem using \emph{(Right-sided) Divided Differences}: 
\[
f'(x) \ensuremath{\approx} {f(x+h) - f(x) \over h}
\]
Note by the definition of the derivative we know that this approximation will converge to the true derivative as $h \ensuremath{\rightarrow} 0$. But in numerical approimxations we also need to consider the rate of convergence. 

Now in the previous section I mentioned there are three basic tools in analysis:  (1) integration-by-parts, (2) geometric series or (3) Taylor series. In this case we use (3):

\begin{proposition}[divided differences error] Suppose that $f$ is twice-differentiable on the interval $[x,x+h]$. The error in approximating the derivative using divided differences is
\[
f'(x) = {f(x+h) - f(x) \over h} + \ensuremath{\delta}
\]
where $|\ensuremath{\delta}| \ensuremath{\leq} Mh/2$ for  $M = \sup_{x \ensuremath{\leq} t \ensuremath{\leq} x+h} |f''(t)|$.

\end{proposition}
\textbf{Proof} Follows immediately from Taylor's theorem:
\[
f(x+h) = f(x) + f'(x) h + \underbrace{{f''(t) \over 2} h^2}_{h \ensuremath{\delta}}
\]
for some $x \ensuremath{\leq} t \ensuremath{\leq} x+h$, by bounding:
\[
|\ensuremath{\delta}| \ensuremath{\leq} \abs{{f''(t) \over 2} h} \ensuremath{\leq} {M  h \over 2}.
\]
\ensuremath{\QED}

Unlike the rectangular rule, the computational cost of computing the divided difference is independent of $h$! We only need to evaluate a function $f$ twice and do a single division. Here we are assuming that the computational cost of evaluating $f$ is independent of the point of evaluation. Later we will investigate the details of how computers work with numbers via floating point,  and confirm that this is a sensible assumption.

So why not just set $h$ ridiculously small? In the lab we explore this question and observe that there are significant errors introduced in the numerical realisation of this algorithm. We will return to the question of understanding these errors after learning floating point numbers. 

There are alternative versions of divided differences. Left-side divided differences evaluates to the left of the point where wish to know the derivative:
\[
f'(x) \ensuremath{\approx} {f(x) - f(x-h) \over h}
\]
and central differences:
\[
f'(x) \ensuremath{\approx} {f(x + h) - f(x - h) \over 2h}
\]
We can further arrive at an approximation to the second derivative by composing a left- and right-sided finite difference:
\[
f''(x) \ensuremath{\approx} {f'(x+h) - f'(x) \over h} \ensuremath{\approx} {{f(x+h) - f(x) \over h} - {f(x) - f(x-h) \over h} \over h}
= {f(x+h) - 2f(x)  + f(x-h) \over h^2}
\]
In the lab we investigate the convergence rate of these approximations (in particular, that  central differences is more accurate than standard divided differences) and observe that they too suffer from unexplained (for now) loss of accuracy as $h \ensuremath{\rightarrow} 0$. In the problem sheet we prove the theoretical converge rate, which is never realised because of these errors.





\section{Dual Numbers}
In this section we introduce a mathematically beautiful  alternative to divided differences for computing derivatives: \emph{dual numbers}. These are a commutative ring that \emph{exactly} compute derivatives, which when implemented on a computer gives very high-accuracy approximations to derivatives. They underpin forward-mode \href{https://en.wikipedia.org/wiki/Automatic_differentiation}{automatic differentation}. Automatic differentiation  is a basic tool in Machine Learning for computing gradients necessary for training neural networks.

\begin{definition}[Dual numbers] Dual numbers $\ensuremath{\bbD}$ are a commutative ring (over $\ensuremath{\bbR}$) generated by $1$ and $\ensuremath{\epsilon}$ such that $\ensuremath{\epsilon}^2 = 0$. Dual numbers are typically written as $a + b \ensuremath{\epsilon}$ where $a$ and $b$ are real. \end{definition}

This is very much analoguous to complex numbers, which are a field generated by $1$ and $\I$ such that $\I^2 = -1$. Compare multiplication of each number type:
\meeq{
(a + b \I) (c + d \I) = ac + (bc + ad) \I + bd \I^2 = ac -bd + (bc + ad) \I \ccr
(a + b \ensuremath{\epsilon}) (c + d \ensuremath{\epsilon}) = ac + (bc + ad) \ensuremath{\epsilon} + bd \ensuremath{\epsilon}^2 = ac  + (bc + ad) \ensuremath{\epsilon} 
}
And just as we view $\ensuremath{\bbR} \ensuremath{\subset} \ensuremath{\bbC}$ by equating $a \ensuremath{\in} \ensuremath{\bbR}$ with $a + 0\I \ensuremath{\in} \ensuremath{\bbC}$, we can view $\ensuremath{\bbR} \ensuremath{\subset} \ensuremath{\bbD}$ by equating $a \ensuremath{\in} \ensuremath{\bbR}$ with $a + 0{\rm \ensuremath{\epsilon}} \ensuremath{\in} \ensuremath{\bbD}$.

\subsection{Differentiating polynomials}
Polynomials evaluated on dual numbers are well-defined as they depend only on the operations $+$ and $*$. From the formula for multiplication of dual numbers we deduce that evaluating a polynomial at a dual number $a + b \ensuremath{\epsilon}$ tells us the derivative of the polynomial at $a$:

\begin{theorem}[polynomials on dual numbers] Suppose $p$ is a polynomial. Then
\[
p(a + b \ensuremath{\epsilon}) = p(a) + b p'(a) \ensuremath{\epsilon}
\]
\end{theorem}
\textbf{Proof}

First consider $p(x) = x^n$ for $n \ensuremath{\geq} 0$.  The cases $n = 0$ and $n = 1$ are immediate. For $n > 1$ we have by induction:
\[
(a + b \ensuremath{\epsilon})^n = (a + b \ensuremath{\epsilon}) (a + b \ensuremath{\epsilon})^{n-1} = (a + b \ensuremath{\epsilon}) (a^{n-1} + (n-1) b a^{n-2} \ensuremath{\epsilon}) = a^n + b n a^{n-1} \ensuremath{\epsilon}.
\]
For a more general polynomial
\[
p(x) = \ensuremath{\sum}_{k=0}^n c_k x^k
\]
the result follows from linearity:
\[
p(a + b \ensuremath{\varepsilon}) = \ensuremath{\sum}_{k=0}^n c_k (a+b\ensuremath{\epsilon})^k = c_0 + \ensuremath{\sum}_{k=1}^n c_k (a^k +k b a^{k-1}\ensuremath{\epsilon})
= \ensuremath{\sum}_{k=0}^n c_k a^k + b \ensuremath{\sum}_{k=1}^n c_k k a^{k-1}\ensuremath{\epsilon} = p(a) + b p'(a) \ensuremath{\epsilon}.
\]
\ensuremath{\QED}

\begin{example}[differentiating polynomial] Consider computing $p'(2)$ where
\[
p(x) = (x-1)(x-2) + x^2.
\]
We can use dual numbers to differentiate, avoiding expanding in monomials or applying rules of differentiating:
\[
p(2+\ensuremath{\epsilon}) = (1+\ensuremath{\epsilon})\ensuremath{\epsilon} + (2+\ensuremath{\epsilon})^2 = \ensuremath{\epsilon} + 4 + 4\ensuremath{\epsilon} = 4 + \underbrace{5}_{p'(2)}\ensuremath{\epsilon}
\]
\end{example}

\subsection{Differentiating other functions}
We can extend real-valued differentiable functions to dual numbers in a similar manner. First, consider a standard function with a Taylor series (e.g. ${\rm cos}$, ${\rm sin}$, ${\rm exp}$, etc.)
\[
f(x) = \ensuremath{\sum}_{k=0}^\ensuremath{\infty} f_k x^k
\]
so that $a$ is inside the radius of convergence. This leads naturally to a definition on dual numbers:
\meeq{
f(a + b \ensuremath{\epsilon}) = \ensuremath{\sum}_{k=0}^\ensuremath{\infty} f_k (a + b \ensuremath{\epsilon})^k = f_0 + \ensuremath{\sum}_{k=1}^\ensuremath{\infty} f_k (a^k + k a^{k-1} b \ensuremath{\epsilon}) = \ensuremath{\sum}_{k=0}^\ensuremath{\infty} f_k a^k +  \ensuremath{\sum}_{k=1}^\ensuremath{\infty} f_k k a^{k-1} b \ensuremath{\epsilon}  \ccr
  = f(a) + b f'(a) \ensuremath{\epsilon}
}
More generally, given a differentiable function we can extend it to dual numbers:

\begin{definition}[dual extension] Suppose a real-valued function $f$ is differentiable at $a$. If
\[
f(a + b \ensuremath{\epsilon}) = f(a) + b f'(a) \ensuremath{\epsilon}
\]
then we say that it is a \emph{dual extension at} $a$.

Thus, for basic functions we have natural extensions:


\begin{align*}
\exp(a + b \ensuremath{\epsilon}) &:= \exp(a) + b \exp(a) \ensuremath{\epsilon} \\
\sin(a + b \ensuremath{\epsilon}) &:= \sin(a) + b \cos(a) \ensuremath{\epsilon} \\
\cos(a + b \ensuremath{\epsilon}) &:= \cos(a) - b \sin(a) \ensuremath{\epsilon} \\
\log(a + b \ensuremath{\epsilon}) &:= \log(a) + {b \over a} \ensuremath{\epsilon} \\
\sqrt{a+b \ensuremath{\epsilon}} &:= \sqrt{a} + {b \over 2 \sqrt{a}} \ensuremath{\epsilon} \\
|a + b \ensuremath{\epsilon}| &:= |a| + b\, {\rm sign} a\, \ensuremath{\epsilon}
\end{align*}
provided the function is differentiable at $a$. Note the last example does not have a convergent Taylor series (at 0) but we can still extend it where it is differentiable.

Going further, we can add, multiply, and compose such functions:

\begin{lemma}[product and chain rule] If $f$ is a dual extension at $g(a)$ and $g$ is a dual extension at $a$, then $q(x) := f(g(x))$ is a dual extension at $a$. If $f$ and $g$ are dual extensions at $a$ then  $r(x) := f(x) g(x)$ is also dual extensions at $a$. In other words:
\meeq{
q(a+b \ensuremath{\epsilon}) = q(a) + b q'(a) \ensuremath{\epsilon} \ccr
r(a+b \ensuremath{\epsilon}) = r(a) + b r'(a) \ensuremath{\epsilon}
}
\end{lemma}
\textbf{Proof} For $q$ it follows immediately:
\meeq{
q(a + b \ensuremath{\epsilon}) = f(g(a + b \ensuremath{\epsilon})) = f(g(a) + b g'(a) \ensuremath{\epsilon}) \ccr
= f(g(a)) + b g'(a) f'(g(a))\ensuremath{\epsilon} = q(a) + b q'(a) \ensuremath{\epsilon}.
}
For $r$ we have
\meeq{
r(a + b \ensuremath{\epsilon}) = f(a+b \ensuremath{\epsilon} )g(a+b \ensuremath{\epsilon} )= (f(a) + b f'(a) \ensuremath{\epsilon})(g(a) + b g'(a) \ensuremath{\epsilon}) \ccr
= f(a)g(a) + b (f'(a)g(a) + f(a)g'(a)) \ensuremath{\epsilon} = r(a) +b r'(a) \ensuremath{\epsilon}.
}
\end{definition}

A simple corollary is that any function defined in terms of addition, multiplication, composition, etc. of functions that are dual with differentiation will be differentiable via dual numbers.

\begin{example}[differentiating non-polynomial]

Consider differentiating $f(x) =  \exp(x^2 + \E^x)$ at the point $a = 1$ by evaluating on the duals:
\[
f(1 + \ensuremath{\epsilon}) = \exp(1 + 2\ensuremath{\epsilon} + \E + \E \ensuremath{\epsilon}) =  \exp(1 + \E) + \exp(1 + \E) (2 + \E) \ensuremath{\epsilon}.
\]
Therefore we deduce that
\[
f'(1) = \exp(1 + \E) (2 + \E).
\]
\end{example}





\section{Newton's method}
In school you may recall learning Newton's method: a way of approximating zeros/roots to a function by using a local approximation by an affine function. That is, approximate a function $f(x)$ locally around an initial guess $x_0$ by its first order Taylor series:
\[
f(x) \ensuremath{\approx} f(x_0) + f'(x_0) (x-x_0)
\]
and then find the root of the right-hand side which is
\[
 f(x_0) + f'(x_0) (x-x_0) = 0 \ensuremath{\Leftrightarrow} x = x_0 - {f(x_0) \over f'(x_0)}.
\]
We can then repeat using this root as the new initial guess. In other words we have a sequence of \emph{hopefully} more accurate approximations:
\[
x_{k+1} = x_k - {f(x_k) \over f'(x_k)}.
\]
The convergence theory of Newton's method is rich and beautiful but outside the scope of this module. But provided $f$ is smooth, if $x_0$ is sufficiently close to a root this iteration will converge. 

Thus \emph{if} we can compute derivatives, we can (sometimes) compute roots. The lab will explore using dual numbers to accomplish this task. This is in some sense a baby version of how Machine Learning algorithms train neural networks.






\chapter{Representing Numbers}

In this chapter we aim to answer the question: when can we rely on computations done on a computer?  Why are some computations (differentiation via divided differences), extremely inaccurate whilst others (integration via rectangular rule) accurate up to about 16 digits?  In order to address these questions we need to dig deeper and understand at a basic level what a computer is actually doing when manipulating numbers. 

Before we begin it is important to have a basic model of how a computer works. Our simplified model of a computer will consist of a \href{https://en.wikipedia.org/wiki/Central_processing_unit}{Central Processing Unit (CPU)}\ensuremath{\emdash}the  brains of the computer\ensuremath{\emdash}and \href{https://en.wikipedia.org/wiki/Computer_data_storage#Primary_storage}{Memory}\ensuremath{\emdash}where  data is stored. Inside the CPU there are \href{https://en.wikipedia.org/wiki/Processor_register}{registers}, where data is temporarily stored after being loaded from memory, manipulated by the CPU, then stored back to memory.  Memory is a sequence of bits: \texttt{1}s and \texttt{0}s, essentially ``on/off" switches, and memory is {\it finite}.  Finally, if one has a $p$-bit CPU (eg a 32-bit or 64-bit CPU), each register consists of exactly $p$-bits. Most likely $p = 64$ on your machine. 


Thus representing numbers on a computer must overcome three fundamental limitations:
\begin{enumerate}
\item CPUs can only manipulate data $p$-bits at a time.
\item Memory is finite (in particular at most $2^p$ bytes).
\item There is no such thing as an ``error'': if anything goes wrong in the computation we must use some of the $p$-bits to indicate this.
\end{enumerate}

This is clearly problematic: there are an infinite number of integers and an uncountable number of reals! Each of which we need to store in precisely $p$-bits. Moreover, some operations are simply undefined, like division by 0.  This chapter discusses the solution used to this problem, alongside the mathematical analysis that is needed to understand the implications, in particular, that computations have {\it error}.

In particular we discuss:

\begin{enumerate}
\item II.1 Integers: unsigned (non-negative) and signed integers are representable using exactly $p$-bits by using modular arithmetic in all operations.
\item II.2 Reals:  real numbers are approximated by floating point numbers, which are a computers version of scientific notation.
\item II.3 Floating Point Arithmetic:  arithmetic with floating point numbers is exact up-to-rounding, which introduces small-but-understandable errors in the computations. We explain how these errors can be analysed mathematically to get rigorous bounds. 
\item II.4 Interval Arithmetic: rounding can be controlled in order to implement {\it interval arithmetic}, a way to compute rigorous bounds for computations. In the lab, we use this to compute up to 15 digits of ${\rm e} \equiv \exp 1$ rigorously with precise bounds on the error.
\end{enumerate}



\section{Integers}
In this section we discuss the following:

\begin{itemize}
\item[1. ] Unsigned integers: how computers represent non-negative integers using only $p$-bits, via \href{https://en.wikipedia.org/wiki/Modular_arithmetic}{modular arithmetic}.


\item[2. ] Signed integers: how negative integers are handled using the \href{https://en.wikipedia.org/wiki/Two's_complement}{Two's-complement} format.

\end{itemize}
Mathematically, CPUs only act on $p$-bits at a time, with $2^p$ possible sequences. That is, essentially all functions $f$ are either of the form $f : \ensuremath{\bbZ}_{2^p} \ensuremath{\rightarrow} \ensuremath{\bbZ}_{2^p}$ or  $f : \ensuremath{\bbZ}_{2^p} \ensuremath{\times} \ensuremath{\bbZ}_{2^p} \ensuremath{\rightarrow} \ensuremath{\bbZ}_{2^p}$, where we use the following notation:

\begin{definition}[finite integers] Denote the set of the first $m$ non-negative integers as $\ensuremath{\bbZ}_m := \{0 , 1 , \ensuremath{\ldots}, m-1 \}$. \end{definition}

To translate between integers and bits we will need to write integers in binary format.  That is, as sequence of \texttt{0}s and \texttt{1}s:

\begin{definition}[binary format] For $B_0,\ldots,B_p \in \{0,1\}$ denote an integer in \emph{binary format} by:
\[
\ensuremath{\pm}(B_p\ldots B_1B_0)_2 := \ensuremath{\pm}\sum_{k=0}^p B_k 2^k
\]
\end{definition}

\begin{example}[integers in binary] A simple integer example is $5 = 2^2 + 2^0 = (101)_2$. On the other hand, we write $-5 = -(101)_2$. Another example is $258 = 2^8 + 2 = (100000010)_2$. \end{example}

\subsection{Unsigned Integers}
Computers represent integers by a finite number of $p$-bits, with $2^p$ possible combinations of 0s and 1s. Denote these $p$-bits as $B_{p-1}\ensuremath{\ldots}B_1B_0$ where $B_k \ensuremath{\in} \{0,1\}$. For \emph{unsigned integers} (non-negative integers) these bits dictate the first $p$ binary digits: $(B_{p-1}\ldots B_1B_0)_2$. Integers represented with $p$-bits on a computer are interpreted as  representing elements of ${\mathbb Z}_{2^p}$ and integer arithmetic on a computer is equivalent to arithmetic modulo $2^p$. We denote modular arithmetic with $m = 2^p$ as follows:
\begin{align*}
x \ensuremath{\oplus}_m y &:= (x+y)\ ({\rm mod}\ m) \\
x \ensuremath{\ominus}_m y &:= (x-y)\ ({\rm mod}\ m) \\
x \ensuremath{\otimes}_m y &:= (x*y)\ ({\rm mod}\ m)
\end{align*}
When $m$ is implied by context we just write $\ensuremath{\oplus}, \ensuremath{\ominus}, \ensuremath{\otimes}$. Note that  the $({\rm mod}\ m)$ function simply drops all bits except for the first $p$-bits when writing a number in binary.

\begin{example}[arithmetic with  8-bit unsigned integers] If  the result of an operation lies between $0$ and $m = 2^8 = 256$ then airthmetic works exactly like standard integer arithmetic. For example,
\begin{align*}
17 \ensuremath{\oplus}_{256} 3 = 20\ ({\rm mod}\ 256) = 20 \\
17 \ensuremath{\ominus}_{256} 3 = 14\ ({\rm mod}\ 256) = 14
\end{align*}
\end{example}

\begin{example}[overflow with 8-bit unsigned integers] If we go beyond the range the result \ensuremath{\ldq}wraps around". For example, with true integers we have
\[
255 + 1 = (11111111)_2 + (00000001)_2 = (100000000)_2 = 256
\]
However, the result is impossible to store in just 8-bits! So as mentioned instead it treats the integers as elements of ${\mathbb Z}_{256}$ by dropping any extra digits:
\[
255 \ensuremath{\oplus}_{256} 1 = 255 + 1 \ ({\rm mod}\ 256) = (100000000)_2 \ ({\rm mod}\ 256) = 0.
\]
On the other hand, if we go below $0$ we wrap around from above:
\[
3 \ensuremath{\ominus}_{256} 5 = -2\ ({\rm mod}\ 256) = 254 = (11111110)_2
\]
\end{example}

\begin{example}[multiplication of 8-bit unsigned integers] Multiplication works similarly: for example,
\[
254 \ensuremath{\otimes}_{256} 2 = 254 * 2 \ ({\rm mod}\ 256) = (11111110)_2 * 2  \ ({\rm mod}\ 256)
= (111111100)_2  \ ({\rm mod}\ 256) = 252.
\]
Note that multiplication by $2$ is the same as shifting the binary digits left by one, just as multiplication by $10$ shifts base-10 digits left by 1. \end{example}

\subsection{Signed integer}
Signed integers use the \href{https://epubs.siam.org/doi/abs/10.1137/1.9780898718072.ch3}{Two's complemement} convention. The convention is if the first bit is 1 then the number is negative: in this case if the bits had represented the unsigned integer $2^p - y$ then the represent the signed integer $-y$. Thus for $p = 8$ we are interpreting $2^7$ through $2^8-1$ as negative numbers. More precisely:

\begin{definition}[signed integers] Denote the finite signed integers as
\[
\ensuremath{\bbZ}_{2^p}^{\rm s} := \{-2^{p-1} ,\ensuremath{\ldots}, -1 ,0,1, \ensuremath{\ldots}, 2^{p-1}-1\}.
\]
\end{definition}

\begin{definition}[Shifted mod] Define for $y = x\ ({\rm mod}\ 2^p)$
\[
x\ ({\rm mod}^{\rm s}\ 2^p) := \begin{cases} y & 0 \ensuremath{\leq} y \ensuremath{\leq} 2^{p-1}-1 \\
                             y - 2^p & 2^{p-1} \ensuremath{\leq} y \ensuremath{\leq} 2^p - 1
                             \end{cases}
\]
\end{definition}

Note that if $R_p(x) = x\ ({\rm mod}^{\rm s}\ 2^p)$ then it can be viewed as a map $R_p : \ensuremath{\bbZ} \ensuremath{\rightarrow} \ensuremath{\bbZ}_{2^p}^{\rm s}$ or a one-to-one map $R_p : \ensuremath{\bbZ}_{2^p} \ensuremath{\rightarrow} \ensuremath{\bbZ}_{2^p}^{\rm s}$ whose inverse is $R_p^{-1}(x) = x\ ({\rm mod}\ 2^p)$. It can also be viewed as the identity map on signed integers $R_p : \ensuremath{\bbZ}_{2^p}^{\rm s} \ensuremath{\rightarrow} \ensuremath{\bbZ}_{2^p}^{\rm s}$, that is,  $R_p(x) = x$ if $x \in \ensuremath{\bbZ}_{2^p}^{\rm s}$.

Arithmetic works precisely the same for signed and unsigned integers up to the mapping $R_p$, e.g. we have for $m = 2^p$
\begin{align*}
x \ensuremath{\oplus}_{m}^{\rm s} y &:= (x+y)\ ({\rm mod}^{\rm s}\ m) \\
x \ensuremath{\ominus}_{m}^{\rm s} y &:= (x-y)\ ({\rm mod}^{\rm s}\ m) \\
x \ensuremath{\otimes}_{m}^{\rm s} y &:= (x*y)\ ({\rm mod}^{\rm s}\ m)
\end{align*}
\begin{example}[addition of 8-bit signed integers] Consider \texttt{(-1) + 1} in 8-bit arithmetic:
\[
-1 \ensuremath{\oplus}_{256}^{\rm s} 1 = -1 + 1 \ ({\rm mod}^{\rm s}\ 256) = 0
\]
On the bit level this computation is exactly the same as unsigned integers. We represent the number $-1$ using the same bits as the unsigned integer $2^8 - 1 = 255$, that is  using the bits \texttt{11111111} (i.e., we store it equivalently to  $(11111111)_2 = 255$) and the  number $1$ is stored using the bits \texttt{00000001}. When we add this with true integer arithmetic we have
\begin{align*}
(0 11111111)_2 &\ + \\
(0 00000001)_2 &\ = \\
(1 00000000)_2&
\end{align*}
Modular arithmetic drops the leading $1$ and we are left with all zeros. \end{example}

\begin{example}[signed overflow with 8-bit signed integers] If we go above $2^{p-1}-1 = 2^7 - 1 = 127$  we have perhaps unexpected results:
\[
127 \ensuremath{\oplus}_{256}^{\rm s} 1 = 128\  ({\rm mod}^{\rm s}\ 256) = 128 - 256 = -128.
\]
Again on the bit level this computation is exactly the same as unsigned integers. We represent the number $127$ using the bits \texttt{01111111} and the  number $1$ is stored using the bits \texttt{00000001}. When we add this with true integer arithmetic we have
\begin{align*}
(01111111)_2 &\ + \\
(00000001)_2 &\ = \\
(10000000)_2&
\end{align*}
Because the first bit is \texttt{1} we interpret this as a negative number using the formula:
\[
(10000000)_2\ ({\rm mod}^{\rm s}\ 256) = 128   ({\rm mod}^{\rm s}\ 256) = -128.
\]
\end{example}

\begin{example}[multiplication of 8-bit signed integers] Consider computation of \texttt{(-2) * 2}:
\[
(-2) \ensuremath{\otimes}_{2^p}^{\rm s} 2 = -4 \ ({\rm mod}^{\rm s}\ 2^p) = -4
\]
On the bit level, the bits of $-2$ (which is one less than $-1$) are \texttt{11111110}. Multiplying by 2 is like multiplying by 10 in base-10, that is, we shift the bits. Hence in true arithmetic we have
\begin{align*}
(0 11111110)_2 & * 2 = \\
(1 11111100)_2&
\end{align*}
We drop the leading 1 due to modular arithmetic. We still have a leading $1$ hence the number is viewed as negative. In particular we have
\meeq{
(1 11111100)_2 \ ({\rm mod}^{\rm s}\ 256) = (11111100)_2 \ ({\rm mod}^{\rm s}\ 256) = 
2^7+2^6+2^5+2^4+2^3+2^2 \ ({\rm mod}^{\rm s}\ 256) \ccr
 = 252  \ ({\rm mod}^{\rm s}\ 256) = -4.
}
\end{example}

\subsection{Hexadecimal format}
In coding it is often convenient to use base-16 as it is a power of $2$ but uses less characters than binary. The digits used are $0$ through $9$ followed by $a = 10$, $b = 11$, $c = 12$, $d = 13$, $e = 14$, and $f = 15$. 

\begin{example}[Hexadecimal number] We can interpret a number in format as follows:
\[
(a5f2)_{16} = a*16^3 + 5*16^2 + f*16 + 2 = 
10*16^3 + 5*16^2 + 15*16 + 2 = 42,482
\]
\end{example}

We will see in the labs that unsigned integers are displayed in base-16.





\section{Reals}
In this chapter, we introduce  the  \href{https://en.wikipedia.org/wiki/IEEE_754}{IEEE Standard for Floating-Point Arithmetic}. There are multiplies ways of representing real numbers on a computer, as well as  the precise behaviour of operations such as addition, multiplication, etc.: one can use

\begin{itemize}
\item[1. ] \href{https://en.wikipedia.org/wiki/Fixed-point_arithmetic}{Fixed-point arithmetic}: essentially representing a real number as an integer where a decimal point is inserted at a fixed position. This turns out to be impractical in most applications, e.g., due to loss of relative accuracy for small numbers.


\item[2. ] \href{https://en.wikipedia.org/wiki/Floating-point_arithmetic}{Floating-point arithmetic}: essentially scientific notation where an exponent is stored alongside a fixed number of digits. This is what is used in practice.


\item[3. ] \href{https://en.wikipedia.org/wiki/Symmetric_level-index_arithmetic}{Level-index arithmetic}: stores numbers as iterated exponents. This is the most beautiful mathematically but unfortunately is not as useful for most applications and is not implemented in hardware.

\end{itemize}
Before the 1980s each processor had potentially a different representation for  floating-point numbers, as well as different behaviour for operations.  IEEE introduced in 1985 was a means to standardise this across processors so that algorithms would produce consistent and reliable results.

This chapter may seem very low level for a mathematics course but there are two important reasons to understand the behaviour of floating-point numbers in details:

\begin{itemize}
\item[1. ] Floating-point arithmetic is very precisely defined, and can even be used in rigorous computations as we shall see in the labs. But it is not exact and its important to understand how errors in computations can accumulate.


\item[2. ] Failure to understand floating-point arithmetic can cause catastrophic issues in practice, with the extreme example being the  \href{https://youtu.be/N6PWATvLQCY?t=86}{explosion of the Ariane 5 rocket}.

\end{itemize}
\subsection{Real numbers in binary}
Reals can also be presented in binary format, that is, a sequence of \texttt{0}s and \texttt{1}s alongside a decimal point:

\begin{definition}[real binary format] For $b_1,b_2,\ensuremath{\ldots}\in \{0,1\}$, Denote a non-negative real number in \emph{binary format} by:
\[
(B_p \ensuremath{\ldots}B_0.b_1b_2b_3\ensuremath{\ldots})_2 := (B_p \ensuremath{\ldots}B_0)_2 +  \sum_{k=1}^\ensuremath{\infty} {b_k \over 2^k}.
\]
\end{definition}

\begin{example}[rational in binary] Consider the number \texttt{1/3}.  In decimal recall that:
\[
1/3 = 0.3333\ensuremath{\ldots}=  \sum_{k=1}^\ensuremath{\infty} {3 \over 10^k}
\]
We will see that in binary
\[
1/3 = (0.010101\ensuremath{\ldots})_2 = \sum_{k=1}^\ensuremath{\infty} {1 \over 2^{2k}}
\]
Both results can be proven using the geometric series:
\[
\sum_{k=0}^\ensuremath{\infty} z^k = {1 \over 1 - z}
\]
provided $|z| < 1$. That is, with $z = {1 \over 4}$ we verify the binary expansion:
\[
\sum_{k=1}^\ensuremath{\infty} {1 \over 4^k} = {1 \over 1 - 1/4} - 1 = {1 \over 3}
\]
A similar argument with $z = 1/10$ shows the decimal case. \end{example}

\subsection{Floating-point numbers}
Floating-point numbers are a subset of real numbers that are representable using a fixed number of bits.

\begin{definition}[floating-point numbers] Given integers $\ensuremath{\sigma}$ (the \emph{exponential shift}), $Q$ (the number of \emph{exponent bits}) and  $S$ (the \emph{precision}), define the set of \emph{Floating-point numbers} by dividing into \emph{normal}, \emph{sub-normal}, and \emph{special number} subsets:
\[
F_{\ensuremath{\sigma},Q,S} := F^{\rm normal}_{\ensuremath{\sigma},Q,S} \cup F^{\rm sub}_{\ensuremath{\sigma},Q,S} \cup F^{\rm special}.
\]
The \emph{normal numbers} $F^{\rm normal}_{\ensuremath{\sigma},Q,S} \ensuremath{\subset} \ensuremath{\bbR}$ are
\[
F^{\rm normal}_{\ensuremath{\sigma},Q,S} := \{\ensuremath{\pm} 2^{q-\ensuremath{\sigma}} \ensuremath{\times} (1.b_1b_2b_3\ensuremath{\ldots}b_S)_2 : 1 \ensuremath{\leq} q < 2^Q-1 \}.
\]
The \emph{sub-normal numbers} $F^{\rm sub}_{\ensuremath{\sigma},Q,S} \ensuremath{\subset} \ensuremath{\bbR}$ are
\[
F^{\rm sub}_{\ensuremath{\sigma},Q,S} := \{\ensuremath{\pm} 2^{1-\ensuremath{\sigma}} \ensuremath{\times} (0.b_1b_2b_3\ensuremath{\ldots}b_S)_2\}.
\]
The \emph{special numbers} $F^{\rm special} \ensuremath{\nsubset} \ensuremath{\bbR}$ are 
\[
F^{\rm special} :=  \{\ensuremath{\infty}, -\ensuremath{\infty}, {\rm NaN}\}
\]
where ${\rm NaN}$ is a special symbol representing \ensuremath{\ldq}not a number", essentially an error flag. \end{definition}

Note this set of real numbers has no nice \emph{algebraic structure}: it is not closed under addition, subtraction, etc. On the other hand, we can control errors effectively hence it is extremely useful for analysis.

Floating-point numbers are stored in $1 + Q + S$ total number of bits, in the format
\[
sq_{Q-1}\ensuremath{\ldots}q_0 b_1 \ensuremath{\ldots}b_S
\]
The first bit ($s$) is the \emph{sign bit}: 0 means positive and 1 means negative. The bits $q_{Q-1}\ensuremath{\ldots}q_0$ are the \emph{exponent bits}: they are the binary digits of the unsigned integer $q$: 
\[
q = (q_{Q-1}\ensuremath{\ldots}q_0)_2.
\]
Finally, the bits $b_1\ensuremath{\ldots}b_S$ are the \emph{significand bits}. If $1 \ensuremath{\leq} q < 2^Q-1$ then the bits represent the normal number
\[
x = \ensuremath{\pm} 2^{q-\ensuremath{\sigma}} \ensuremath{\times} (1.b_1b_2b_3\ensuremath{\ldots}b_S)_2.
\]
If $q = 0$ (i.e. all bits are 0) then the bits represent the sub-normal number
\[
x = \ensuremath{\pm} 2^{1-\ensuremath{\sigma}} \ensuremath{\times} (0.b_1b_2b_3\ensuremath{\ldots}b_S)_2.
\]
If $q = 2^Q-1$  (i.e. all bits are 1) then the bits represent a special number, discussed later.

\subsection{IEEE floating-point numbers}
\begin{definition}[IEEE floating-point numbers]  IEEE has 3 standard floating-point formats: 16-bit (half precision), 32-bit (single precision) and 64-bit (double precision) defined by (you \emph{do not} need to memorise these):


\begin{align*}
F_{16} &:= F_{15,5,10} \\
F_{32} &:= F_{127,8,23} \\
F_{64} &:= F_{1023,11,52}
\end{align*}
\end{definition}

\begin{example}[a real number in 16-bits] Consider the number with bits

\begin{verbatim}
0 10000 1010000000
\end{verbatim}
assuming it is a half-prevision float ($F_{16}$).  Since the sign bit is \texttt{0} it is positive. The exponent is $2^4 - Q = 16 - 15 = 1$ Hence this number is:
\[
2^1 (1.1010000000)_2 = 2 (1 + 1/2 + 1/8) = 3+1/4 = 3.25.
\]
\end{example}

\begin{example}[rational in 16-bits] How is the number $1/3$ stored in $F_{16}$? Recall that
\[
1/3 = (0.010101\ensuremath{\ldots})_2 = 2^{-2} (1.0101\ensuremath{\ldots})_2 = 2^{13-15} (1.0101\ensuremath{\ldots})_2
\]
and since $13 = (1101)_2$  the exponent bits are \texttt{01101}. For the significand we round the last bit to the nearest element of $F_{16}$,  (the exact rule for rounding is explained in detail later), so we have
\[
1.010101010101010101010101\ensuremath{\ldots}\approx 1.0101010101 \in F_{16} 
\]
and the significand bits are \texttt{0101010101}. Thus the stored bits for $1/3$ are:

\begin{verbatim}
0 01101 0101010101
\end{verbatim}
\end{example}

\subsubsection{Sub-normal and special numbers}
For sub-normal numbers, the simplest example is zero, which has $q=0$ and all significand bits zero: \texttt{0 00000 0000000000}. Unlike integers, we also have a negative zero, which has bits: \texttt{1 00000 0000000000}. This is treated as identical to positive \texttt{0} (except for degenerate operations as explained in special numbers).

\begin{example}[subnormal in 16-bits] Consider the number with bits

\begin{verbatim}
1 00000 1100000000
\end{verbatim}
assuming it is a half-prevision float ($F_{16}$).  Since all exponent bits are zero it is sub-normal. Since the sign bit is \texttt{1} it is negative.  Hence this number is:
\[
-2^{1-\ensuremath{\sigma}} (0.1100000000)_2 = -2^{-14} (2^{-1} + 2^{-2}) = -3 \ensuremath{\times} 2^{-16}
\]
\end{example}

The special numbers extend the real line by adding $\ensuremath{\pm}\ensuremath{\infty}$ but also a notion of ``not-a-number" ${\rm NaN}$. Whenever the bits of $q$ of a floating-point number are all 1 then they represent an element of $F^{\rm special}$. If all $b_k=0$, then the number represents either $\ensuremath{\pm}\ensuremath{\infty}$. All other special floating-point numbers represent ${\rm NaN}$. 

\begin{example}[special in 16-bits] The number with bits

\begin{verbatim}
1 11111 0000000000
\end{verbatim}
has all exponent bits equal to $1$, and significand bits $0$ and sign bit $1$, hence represents $-\ensuremath{\infty}$. On the other hand, the number with bits

\begin{verbatim}
1 11111 0000000001
\end{verbatim}
has all exponent bits equal to $1$ but does not have all significand bits equal to $0$, hence is one of many representations for  ${\rm NaN}$. \end{example}





\section{Floating Point Arithmetic}
Arithmetic operations on floating-point numbers are  \emph{exact up to rounding}. There are three basic rounding strategies: round up/down/nearest. Mathematically we introduce a function to capture the notion of rounding:

\begin{definition}[rounding] ${\rm fl}^{\rm up}_{\ensuremath{\sigma},Q,S} : \mathbb R \rightarrow F_{\ensuremath{\sigma},Q,S}$ denotes the function that rounds a real number up to the nearest floating-point number that is greater or equal. ${\rm fl}^{\rm down}_{\ensuremath{\sigma},Q,S} : \mathbb R \rightarrow F_{\ensuremath{\sigma},Q,S}$ denotes the function that rounds a real number down to the nearest floating-point number that is greater or equal. ${\rm fl}^{\rm nearest}_{\ensuremath{\sigma},Q,S} : \mathbb R \rightarrow F_{\ensuremath{\sigma},Q,S}$ denotes the function that rounds a real number to the nearest floating-point number. In case of a tie, it returns the floating-point number whose least significant bit is equal to zero. We use the notation ${\rm fl}$ when $\ensuremath{\sigma},Q,S$ and the rounding mode are implied by context, with ${\rm fl}^{\rm nearest}$ being the default rounding mode. \end{definition}

In more detail on the behaviour of nearest mode, if a positive number $x$ is between two normal floats $x_- \ensuremath{\leq} x \ensuremath{\leq} x_+$ we can write its expansion as
\[
x = 2^{\green{q}-\ensuremath{\sigma}} (1.\blue{b_1b_2\ensuremath{\ldots}b_S}\red{b_{S+1}\ensuremath{\ldots}})_2
\]
where
\begin{align*}
x_- &:= {\rm fl}^{\rm down}(x) = 2^{\green{q}-\ensuremath{\sigma}} (1.\blue{b_1b_2\ensuremath{\ldots}b_S})_2 \\
x_+ &:= {\rm fl}^{\rm up}(x) = x_- + 2^{\green{q}-S}
\end{align*}
Write the half-way point as:
\[
x_{\rm h} := {x_+ + x_- \over 2} = x_- + 2^{\green{q}-S-1} = 2^{\green{q}-\ensuremath{\sigma}} (1.\blue{b_1b_2\ensuremath{\ldots}b_S}\red{1})_2
\]
If $x_- \ensuremath{\leq} x < x_{\rm h}$ then ${\rm fl}(x) = x_-$ and if $x_{\rm h} < x \ensuremath{\leq} x_+$ then ${\rm fl}(x) = x_{\rm h}$. If $x = x_{\rm h}$ then it is exactly half-way between $x_-$ and $x_+$. The rule is if $b_S = 0$  then ${\rm fl}(x) = x_-$ and otherwise ${\rm fl}(x) = x_+$.

In IEEE arithmetic, the arithmetic operations \texttt{+}, \texttt{-}, \texttt{*}, \texttt{/} are defined by the property that they are exact up to rounding.  Mathematically we denote these operations as $\ensuremath{\oplus}, \ensuremath{\ominus}, \ensuremath{\otimes}, \ensuremath{\oslash} : F_{\ensuremath{\sigma},Q,S} \ensuremath{\otimes} F_{\ensuremath{\sigma},Q,S} \ensuremath{\rightarrow} F_{\ensuremath{\sigma},Q,S}$ as follows:
\begin{align*}
x \ensuremath{\oplus} y &:= {\rm fl}(x+y) \\
x \ensuremath{\ominus} y &:= {\rm fl}(x - y) \\
x \ensuremath{\otimes} y &:= {\rm fl}(x * y) \\
x \ensuremath{\oslash} y &:= {\rm fl}(x / y)
\end{align*}
Note also that  \texttt{\^{}} and \texttt{sqrt} are similarly exact up to rounding. Also, note that when we convert a Julia command with constants specified by decimal expansions we first round the constants to floats, e.g., \texttt{1.1 + 0.1} is actually reduced to
\[
{\rm fl}(1.1) \ensuremath{\oplus} {\rm fl}(0.1)
\]
This includes the case where the constants are integers (which are normally exactly floats but may be rounded if extremely large).

\begin{example}[decimal is not exact] On a computer \texttt{1.1+0.1} is close to but not exactly the same thing as \texttt{1.2}. This is because ${\rm fl}(1.1) \ensuremath{\neq} 1+1/10$ and ${\rm fl}(0.1) \ensuremath{\neq} 1/10$ since their expansion in \emph{binary} is not finite. For $F_{16}$ we have:
\begin{align*}
{\rm fl}(1.1) &= {\rm fl}((1.0001100110\red{011\ensuremath{\ldots}})_2) =  (1.0001100110)_2 \\
{\rm fl}(0.1) &= {\rm fl}(2^{-4}(1.1001100110\red{011\ensuremath{\ldots}})_2) =  2^{-4} * (1.1001100110)_2 = (0.00011001100110)_2
\end{align*}
Thus when we add them we get
\[
{\rm fl}(1.1) + {\rm fl}(0.1) = (1.0011001100\red{011})_2
\]
where the red digits indicate those beyond the 10 significant digits representable in $F_{16}$. In this case we round down and get
\[
{\rm fl}(1.1) \ensuremath{\oplus} {\rm fl}(0.1) = (1.0011001100)_2
\]
On the other hand,
\[
{\rm fl}(1.2) = {\rm fl}((1.0011001100\red{11001100\ensuremath{\ldots}})_2) = (1.0011001101)_2
\]
which differs by 1 bit. \end{example}

\textbf{WARNING (non-associative)} These operations are not associative! E.g. $(x \ensuremath{\oplus} y) \ensuremath{\oplus} z$ is not necessarily equal to $x \ensuremath{\oplus} (y \ensuremath{\oplus} z)$. Commutativity is preserved, at least.

\subsection{Bounding errors in floating point arithmetic}
When dealing with normal numbers there are some important constants that we will use to bound errors.

\begin{definition}[machine epsilon/smallest positive normal number/largest normal number] \emph{Machine epsilon} is denoted
\[
\ensuremath{\epsilon}_{{\rm m},S} := 2^{-S}.
\]
When $S$ is implied by context we use the notation $\ensuremath{\epsilon}_{\rm m}$. The \emph{smallest positive normal number} is $q = 1$ and $b_k$ all zero:
\[
\min |F_{\ensuremath{\sigma},Q,S}^{\rm normal}| = 2^{1-\ensuremath{\sigma}}
\]
where $|A| := \{|x| : x \in A \}$. The \emph{largest (positive) normal number} is
\[
\max F_{\ensuremath{\sigma},Q,S}^{\rm normal} = 2^{2^Q-2-\ensuremath{\sigma}} (1.11\ensuremath{\ldots})_2 = 2^{2^Q-2-\ensuremath{\sigma}} (2-\ensuremath{\epsilon}_{\rm m})
\]
\end{definition}

We can bound the error of basic arithmetic operations in terms of machine epsilon, provided a real number is close to a normal number:

\begin{definition}[normalised range] The \emph{normalised range} ${\cal N}_{\ensuremath{\sigma},Q,S} \ensuremath{\subset} \ensuremath{\bbR}$ is the subset of real numbers that lies between the smallest and largest normal floating-point number:
\[
{\cal N}_{\ensuremath{\sigma},Q,S} := \{x : \min |F_{\ensuremath{\sigma},Q,S}^{\rm normal}| \ensuremath{\leq} |x| \ensuremath{\leq} \max F_{\ensuremath{\sigma},Q,S}^{\rm normal} \}
\]
When $\ensuremath{\sigma},Q,S$ are implied by context we use the notation ${\cal N}$. \end{definition}

We can use machine epsilon to determine bounds on rounding:

\begin{proposition}[round bound] If $x \in {\cal N}$ then
\[
{\rm fl}^{\rm mode}(x) = x (1 + \ensuremath{\delta}_x^{\rm mode})
\]
where the \emph{relative error} is bounded by:
\begin{align*}
|\ensuremath{\delta}_x^{\rm nearest}| &\ensuremath{\leq} {\ensuremath{\epsilon}_{\rm m} \over 2} \\
|\ensuremath{\delta}_x^{\rm up/down}| &< {\ensuremath{\epsilon}_{\rm m}}.
\end{align*}
\end{proposition}
\textbf{Proof}

We will show this result for the nearest rounding mode. Note first that
\[
{\rm fl}(-x) = -{\rm fl}(x)
\]
and hence it suffices to prove the result for positive $x$. Write
\[
x = 2^{\green{q}-\ensuremath{\sigma}} (1.b_1b_2\ensuremath{\ldots}b_S\red{b_{S+1}\ensuremath{\ldots}})_2.
\]
Define
\begin{align*}
x_- &:= {\rm fl}^{\rm down}(x) = 2^{\green{q}-\ensuremath{\sigma}} (1.b_1b_2\ensuremath{\ldots}b_S)_2 \\
x_+ &:= {\rm fl}^{\rm up}(x) = x_- + 2^{j-S} \\
x_{\rm h} &:= {x_+ + x_- \over 2} = x_- + 2^{j-S-1} = 2^{\green{q}-\ensuremath{\sigma}} (1.b_1b_2\ensuremath{\ldots}b_S\red{1})_2
\end{align*}
so that $x_- \ensuremath{\leq} x \ensuremath{\leq} x_+$. We consider two cases separately.

(\textbf{Round Down}) First consider the case where $x$ is such that we round down: ${\rm fl}(x) = x_-$. Since $2^{\green{q}-\ensuremath{\sigma}} \ensuremath{\leq} x_- \ensuremath{\leq} x \ensuremath{\leq} x_{\rm h}$ we have
\[
|\ensuremath{\delta}_x| = {x - x_- \over x} \ensuremath{\leq} {x_{\rm h} - x_- \over x_-} = {2^{j-S-1} \over 2^{\green{q}-\ensuremath{\sigma}}} = 2^{-S-1} = {\ensuremath{\epsilon}_{\rm m} \over 2}.
\]
(\textbf{Round Up}) If ${\rm fl}(x) = x_+$ then $2^{\green{q}-\ensuremath{\sigma}} \ensuremath{\leq} x_- < x_{\rm h} \ensuremath{\leq} x \ensuremath{\leq} x_+$ and hence
\[
|\ensuremath{\delta}_x| = {x_+ - x \over x} \ensuremath{\leq} {x_+ - x_{\rm h} \over x_-} = {2^{j-S-1} \over 2^{\green{q}-\ensuremath{\sigma}}} = 2^{-S-1} = {\ensuremath{\epsilon}_{\rm m} \over 2}.
\]
\ensuremath{\QED}

This immediately implies relative error bounds on all IEEE arithmetic operations, e.g., if $x+y \in {\cal N}$ then we have
\[
x \ensuremath{\oplus} y = (x+y) (1 + \ensuremath{\delta}_1)
\]
where (assuming the default nearest rounding) $|\ensuremath{\delta}_1| \ensuremath{\leq} {\ensuremath{\epsilon}_{\rm m} \over 2}.$

\subsection{Idealised floating point}
With a complicated formula it is mathematically inelegant to work with normalised ranges: one cannot guarantee apriori that a computation always results in a normal float. Extending the bounds to subnormal numbers is tedious, rarely relevant, and beyond the scope of this module. Thus to avoid this issue we will work with an alternative mathematical model:

\begin{definition}[idealised floating point] An idealised mathematical model of floating point numbers for which the only subnormal number is zero can be defined as:
\[
F_{\ensuremath{\infty},S} := \{\ensuremath{\pm} 2^q \ensuremath{\times} (1.b_1b_2b_3\ensuremath{\ldots}b_S)_2 :  q \ensuremath{\in} \ensuremath{\bbZ} \} \ensuremath{\cup} \{0\}
\]
\end{definition}

Note that $F^{\rm normal}_{\ensuremath{\sigma},Q,S} \ensuremath{\subset} F_{\ensuremath{\infty},S}$ for all $\ensuremath{\sigma},Q \ensuremath{\in} \ensuremath{\bbN}$. The definition of rounding ${\rm fl}_{\ensuremath{\infty},S}^{mode} : \ensuremath{\bbR} \ensuremath{\rightarrow} F_{\ensuremath{\infty},S}$ naturally extend to $F_{\ensuremath{\infty},S}$ and hence we can consider bounds for floating point operations such as $\ensuremath{\oplus}$, $\ensuremath{\ominus}$, etc. And in this model the round bound is valid for all real numbers (including $x = 0$).

\begin{example}[bounding a simple computation] We show how to bound the error in computing $(1.1 + 1.2) * 1.3 = 2.99$ and we may assume idealised floating-point arithmetic $F_{\ensuremath{\infty},S}$. First note that \texttt{1.1} on a computer is in fact ${\rm fl}(1.1)$, and we will always assume nearest rounding unless otherwise stated. Thus this computation becomes
\[
({\rm fl}(1.1) \ensuremath{\oplus} {\rm fl}(1.2)) \ensuremath{\otimes} {\rm fl}(1.3)
\]
We will show the \emph{absolute error} is given by
\[
({\rm fl}(1.1) \ensuremath{\oplus} {\rm fl}(1.2)) \ensuremath{\otimes} {\rm fl}(1.3) = 2.99 + \ensuremath{\delta}
\]
where $|\ensuremath{\delta}| \ensuremath{\leq}  11 \ensuremath{\epsilon}_{\rm m}.$ First we find
\meeq{
{\rm fl}(1.1) \ensuremath{\oplus} {\rm fl}(1.2) = (1.1(1 + \ensuremath{\delta}_1) + 1.2 (1+\ensuremath{\delta}_2))(1 + \ensuremath{\delta}_3) \ccr
 = 2.3 + \underbrace{1.1 \ensuremath{\delta}_1 + 1.2 \ensuremath{\delta}_2 + 2.3 \ensuremath{\delta}_3 + 1.1 \ensuremath{\delta}_1 \ensuremath{\delta}_3 + 1.2 \ensuremath{\delta}_2 \ensuremath{\delta}_3}_{\ensuremath{\delta}_4}.
}
While $\ensuremath{\delta}_1 \ensuremath{\delta}_3$ and $\ensuremath{\delta}_2 \ensuremath{\delta}_3$ are absolutely tiny in practice we will bound them rather naïvely by eg.
\[
|\ensuremath{\delta}_1 \ensuremath{\delta}_3| \ensuremath{\leq} |\ensuremath{\epsilon}_{\rm m}^2/4| \ensuremath{\leq} |\ensuremath{\epsilon}_{\rm m}/4|.
\]
Further we round up constants to integers in the bounds for simplicity. We thus have the bound
\[
|\ensuremath{\delta}_4| \ensuremath{\leq} (1+1+2+1+1) \ensuremath{\epsilon}_{\rm m} = 6\ensuremath{\epsilon}_{\rm m}
\]
Thus the computation becomes
\[
(2.3 + \ensuremath{\delta}_4) 1.3 (1 + \ensuremath{\delta}_5) (1 + \ensuremath{\delta}_6) = 2.99 + \underbrace{1.3( \ensuremath{\delta}_4 + 2.3\ensuremath{\delta}_5 + 2.3\ensuremath{\delta}_6 + 
\ensuremath{\delta}_4 \ensuremath{\delta}_5 + \ensuremath{\delta}_4 \ensuremath{\delta}_6 + 2.3 \ensuremath{\delta}_5 \ensuremath{\delta}_6 + \ensuremath{\delta}_4\ensuremath{\delta}_5\ensuremath{\delta}_6)}_{\ensuremath{\delta}_7}
\]
where the \emph{absolute error} is bounded by
\[
|\ensuremath{\delta}_7| \ensuremath{\leq} 2 (6 +  2 + 2 + 1/4 + 1/4 + 3/4 + 1/8) \ensuremath{\epsilon}_{\rm m} \ensuremath{\leq} 24 \ensuremath{\epsilon}_{\rm m}
\]
\end{example}

\subsection{Divided differences floating point error bound}
We can use the bound on floating point arithmetic to deduce a bound on divided differences that captures the phenomena we observed where the error of divided differences became large as $h \ensuremath{\rightarrow} 0$. We assume that the function we are attempting to differentiate is computed using floating point arithmetic in a way that has a small absolute error.

\begin{theorem}[divided difference error bound] Assume we are working in idealised floating-point arithmetic $F_{\ensuremath{\infty},S}$. Let $f$ be twice-differentiable in a neighbourhood of $x \ensuremath{\in} F_{\ensuremath{\infty},S}$ and assume that
\[
 f(x) = f^{\rm FP}(x) + \ensuremath{\delta}_x^f
\]
where $f^{\rm FP} : F_{S,\ensuremath{\infty}} \ensuremath{\rightarrow} F_{S,\ensuremath{\infty}}$ has uniform absolute accuracy in that neighbourhood, that is:
\[
|\ensuremath{\delta}_x^f| \ensuremath{\leq} c \ensuremath{\epsilon}_{\rm m}
\]
for a fixed constant $c \ensuremath{\geq} 0$. For simplicity assume that $h = 2^{-n}$ where $n \ensuremath{\leq} S$ and $|x| \ensuremath{\leq} 1$. The divided difference approximation satisfies
\[
(f^{\rm FP}(x + h) \ensuremath{\ominus} f^{\rm FP}(x)) \ensuremath{\oslash} h = f'(x) + \ensuremath{\delta}_{x,h}^{\rm FD}
\]
where
\[
|\ensuremath{\delta}_{x,h}^{\rm FD}| \ensuremath{\leq} {|f'(x)| \over 2} \ensuremath{\epsilon}_{\rm m} + M h +  {4c \ensuremath{\epsilon}_{\rm m} \over h}
\]
for $M = \sup_{x \ensuremath{\leq} t \ensuremath{\leq} x+h} |f''(t)|$.

\end{theorem}
\textbf{Proof}

We have (noting by our assumptions $x \ensuremath{\oplus} h = x + h$ and that dividing by $h$ will only change the exponent so is exact)
\begin{align*}
(f^{\rm FP}(x + h) \ensuremath{\ominus} f^{\rm FP}(x)) \ensuremath{\oslash} h &= {f(x + h) +  \ensuremath{\delta}^f_{x+h} - f(x) - \ensuremath{\delta}^f_x \over h} (1 + \ensuremath{\delta}_1) \\
&= {f(x+h) - f(x) \over h} (1 + \ensuremath{\delta}_1) + {\ensuremath{\delta}^f_{x+h}- \ensuremath{\delta}^f_x \over h} (1 + \ensuremath{\delta}_1)
\end{align*}
where $|\ensuremath{\delta}_1| \ensuremath{\leq} {\ensuremath{\epsilon}_{\rm m} / 2}$. Applying Taylor's theorem we get
\[
(f^{\rm FP}(x + h) \ensuremath{\ominus} f^{\rm FP}(x)) \ensuremath{\oslash} h = f'(x) + \underbrace{f'(x) \ensuremath{\delta}_1 + {f''(t) \over 2} h (1 + \delta_1) + {\ensuremath{\delta}^f_{x+h}- \ensuremath{\delta}^f_x \over h} (1 + \ensuremath{\delta}_1)}_{\ensuremath{\delta}_{x,h}^{\rm FD}}
\]
The bound then follows, using the very pessimistic bound $|1 + \ensuremath{\delta}_1| \ensuremath{\leq} 2$.

\ensuremath{\QED}

The three-terms of this bound tell us a story: the first term is a fixed (small) error, the second term tends to zero as $h \rightarrow 0$, while the last term grows like $\ensuremath{\epsilon}_{\rm m}/h$ as $h \rightarrow 0$.  Thus we observe convergence while the second term dominates, until the last term takes over. Of course, a bad upper bound is not the same as a proof that something grows, but it is a good indication of what happens \emph{in general} and suffices to choose $h$ so that these errors are balanced (and thus minimised). Since in general we do not have access to the constants $c$ and $M$   we employ the following heuristic to balance the two sources of errors:

\textbf{Heuristic (divided difference with floating-point step)} Choose $h$ proportional to $\sqrt{\ensuremath{\epsilon}_{\rm m}}$ in divided differences  so that $M h$ and ${4c \ensuremath{\epsilon}_{\rm m} \over h}$ are (roughly) the same magnitude.

In the case of double precision $\sqrt{\ensuremath{\epsilon}_{\rm m}} \ensuremath{\approx} 1.5\ensuremath{\times} 10^{-8}$, which is close to when the observed error begins to increase in the examples we saw before.

\textbf{Remark} While divided differences is of debatable utility for computing derivatives, it is extremely effective in building methods for solving differential equations, as we shall see later. It is also very useful as a \ensuremath{\ldq}sanity check" if one wants something to compare with other numerical methods for differentiation.

\textbf{Remark} It is also possible to deduce an error bound for the rectangular rule showing that the error caused by round-off is on the order of $n \ensuremath{\epsilon}_{\rm m}$, that is it does in fact grow but the error without round-off which was bounded by $M/n$ will be substantially greater for all reasonable values of $n$.





\section{Interval Arithmetic}
It is possible to use rounding modes (up/down)  to do rigorous computation to compute bounds on the error in, for example, the digits of $\E$. To do this we will use set/interval arithmetic. For sets $X,Y \ensuremath{\subseteq} \ensuremath{\bbR}$, the set arithmetic operations are defined as
\begin{align*}
X + Y &:= \{x + y : x \ensuremath{\in} X, y \ensuremath{\in} Y\}, \\
XY &:= \{xy : x \ensuremath{\in} X, y \ensuremath{\in} Y\}, \\
X/Y &:= \{x/y : x \ensuremath{\in} X, y \ensuremath{\in} Y\}
\end{align*}
We will use floating point arithmetic to construct approximate set operations $\ensuremath{\oplus}$, $\ensuremath{\otimes}$ so that
\begin{align*}
  X + Y &\ensuremath{\subseteq} X \ensuremath{\oplus} Y, \\
   XY &\ensuremath{\subseteq} X \ensuremath{\otimes} Y,\\
    X/Y &\ensuremath{\subseteq} X \ensuremath{\oslash} Y
    \end{align*}
thereby a complicated algorithm can be run on sets and the true result is guaranteed to be a subset of the output.

When our sets are intervals we can deduce simple formulas for basic arithmetic operations. For simplicity we only consider the case where all values are positive.

\begin{proposition}[interval bounds] For intervals  $X = [a,b]$ and $Y = [c,d]$ satisfying $0 < a \ensuremath{\leq} b$ and $0 < c \ensuremath{\leq} d$, and $n > 0$, we have:
\meeq{
X + Y = [a+c, b+d] \ccr
X/n = [a/n,b/n] \ccr
XY = [ac, bd]
}
\end{proposition}
\textbf{Proof} We first show $X+Y \ensuremath{\subseteq} [a+c,b+d]$. If $z \ensuremath{\in} X + Y$ then $z = x+y$ such that $a \ensuremath{\leq} x \ensuremath{\leq} b$ and $c \ensuremath{\leq} y \ensuremath{\leq} d$ and therefore $a + c \ensuremath{\leq} z \ensuremath{\leq} c + d$ and $z \ensuremath{\in} [a+c,b+d]$. Equality follows from convexity. First note that $a+c, b+d \ensuremath{\in} X+Y$. Any point $z \ensuremath{\in}  [a+b,c+d]$ can be written  as a convex combination of the two endpoints: there exists $0 \ensuremath{\leq} t \ensuremath{\leq} 1$ such that
\[
z = (1-t) (a+c) + t (b+d) =  \underbrace{(1-t) a + t b}_x + \underbrace{(1-t) c + t d}_y
\]
Because intervals are convex we have $x \ensuremath{\in} X$ and $y \ensuremath{\in} Y$ and hence $z \ensuremath{\in} X+Y$. 

The remaining two proofs are left for the problem sheet. 

\ensuremath{\QED}

We want to  implement floating point variants of these operations that are guaranteed to contain the true set arithmetic operations. We do so as follows:

\begin{definition}[floating point interval arithmetic] For intervals  $A = [a,b]$ and $B = [c,d]$ satisfying $0 < a \ensuremath{\leq} b$ and $0 < c \ensuremath{\leq} d$, and $n > 0$, define:
\begin{align*}
[a,b] \ensuremath{\oplus} [c,d] &:= [{\rm fl}^{\rm down}(a+c), {\rm fl}^{\rm up}(b+d)] \\
[a,b] \ensuremath{\ominus} [c,d] &:= [{\rm fl}^{\rm down}(a-d), {\rm fl}^{\rm up}(b-c)] \\
[a,b] \ensuremath{\oslash} n &:= [{\rm fl}^{\rm down}(a/n), {\rm fl}^{\rm up}(b/n)] \\
[a,b] \ensuremath{\otimes} [c,d] &:= [{\rm fl}^{\rm down}(ac), {\rm fl}^{\rm up}(bd)]
\end{align*}
\end{definition}

\begin{example}[small sum] consider evaluating the first few terms in the Taylor series of the exponential at $x = 1$ using interval arithmetic with half-precision $F_{16}$ arithmetic.  The first three terms are exact since all numbers involved are exactly floats, in particular if we evaluate $1 + x + x^2/2$ with $x = 1$ we get
\[
1 + 1 + 1/2 \ensuremath{\in} 1 \ensuremath{\oplus} [1,1] \ensuremath{\oplus} ([1,1] \ensuremath{\otimes} [1,1]) \ensuremath{\oslash} 2 = [5/2, 5/2]
\]
Noting that 
\[
1/6 = (1/3)/2 = 2^{-3} (1.01010101\ensuremath{\ldots})_2
\]
we can extend the computation to another term:
\begin{align*}
1 + 1 + 1/2 + 1/6 &\ensuremath{\in} [5/2,5/2] \ensuremath{\oplus} ([1,1] \ensuremath{\oslash} 6) \ccr
= [2 (1.01)_2, 2 (1.01)_2] \ensuremath{\oplus} 2^{-3}[(1.0101010101)_2, (1.0101010110)_2] \ccr
= [{\rm fl}^{\rm down}(2 (1.0101010101\red{0101})_2), {\rm fl}^{\rm up}(2 (1.0101010101\red{011})_2)] \ccr
= [2(1.0101010101)_2, 2(1.0101010110)_2] \ccr 
= [2.666015625, 2.66796875]
\end{align*}
\end{example}

\begin{example}[exponential with intervals] Consider computing $\exp(x)$ for $0 \ensuremath{\leq} x \ensuremath{\leq} 1$ from the Taylor series approximation:
\[
\exp(x) = \sum_{k=0}^n {x^k \over k!} + \underbrace{\exp(t){x^{n+1} \over (n+1)!}}_{\ensuremath{\delta}_{x,n}}
\]
where we can bound the error by (using the fact that $\ensuremath{\euler} = 2.718\ensuremath{\ldots} \ensuremath{\leq} 3$)
\[
|\ensuremath{\delta}_{x,n}| \ensuremath{\leq} {\exp(1) \over (n+1)!} \ensuremath{\leq} {3 \over (n+1)!}.
\]
Put another way: $\ensuremath{\delta}_{x,n} \ensuremath{\in} \left[-{3 \over (n+1)!}, {3 \over (n+1)!}\right]$. We can use this to adjust the bounds derived from interval arithmetic for the interval arithmetic expression:
\[
\exp(X) \ensuremath{\subseteq} \left(\ensuremath{\bigoplus}_{k=0}^n {X^k \ensuremath{\oslash} k!}\right) \ensuremath{\oplus} \left[-{3 \over (n+1)!}, {3 \over (n+1)!}\right]
\]
For example, with $n = 3$ we have $|\ensuremath{\delta}_{1,2}| \ensuremath{\leq} 3/4! = 1/2^3$. Thus we can prove that:
\meeq{
\ensuremath{\euler} = 1 + 1 + 1/2 + 1/6 + \ensuremath{\delta}_x \ensuremath{\in} [2(1.0101010101)_2, 2(1.0101010110)_2] \ensuremath{\oplus} [-1/2^3, 1/2^3] \ccr
= [2(1.0100010101)_2, 2(1.0110010110)_2] = [2.541015625,2.79296875]
}
In the lab we get many more digits by using a computer to compute the bounds. \end{example}






\chapter{Numerical Linear Algebra}

Many problems in mathematics are linear: for example, polynomial regression and
differential equations. Numerical methods for such applications invariably result
in (finite-dimensional) linear systems that must be solved numerically on a computer: 
the dimensions of the problems are often in the 1000s, millions, or even billions.
One would certainly not want to tackle that with Gaussian elimination by hand!
In this chapter we discuss algorithms, and in particular matrix factorisations, that are
computed using floating point operations. We also introduce some basic applications.



In particular we discuss:

\begin{enumerate}
    \item III.1 Structured Matrices: we discuss special structured matrices such as triangular and tridiagonal.
    \item III.2 Differential Equations: using divided differences we can reduce differential equations
    to linear systems. This motivates the investigation of numerical algorithms for solving linear systems.
    \item III.3 LU and Cholesky Factorisations: we look at computing a factorisation of a square matrix as a product of a lower and upper triangular matrix, including the special case where the matrix is symmetric positive
    definite. Hidden in this is an algorithm to prove positive definiteness.
\item III.4 Polynomial Regression: often in data science one needs to approximate data by a polynomial.
We discuss how to reduce this problem to solving a rectangular least squares problem.
\item III.5 Orthogonal Matrices: we discuss different types of orthogonal matrices, which will be used to simplify rectangular least squares problems.
\item III.6 QR Factorisation: we introduce an algorithm to compute a factorisation of a rectangular matrix as a product of an orthogonal and upper triangular matrix, thereby solving least squares problems.
\end{enumerate}


\section{Structured Matrices}
We have seen how algebraic operations (\texttt{+}, \texttt{-}, \texttt{*}, \texttt{/}) are defined exactly in terms of rounding ($\ensuremath{\oplus}$, $\ensuremath{\ominus}$, $\ensuremath{\otimes}$, $\ensuremath{\oslash}$)  for floating point numbers. Now we see how this allows us to do (approximate) linear algebra operations on matrices. 

A matrix can be stored in different formats, in particular we can take advantage of \emph{sparsity}: if we know a matrix has entries that are guaranteed to be zero we can implement faster algorithms. We shall see that this comes up naturally in numerical methods for solving differential equations.  In particular, we will discuss some basic matrices:

\begin{itemize}
\item[1. ] \emph{Dense}: This can be considered unstructured, where we need to store all entries in a vector or matrix. Matrix multiplication reduces directly to standard algebraic operations. Solving linear systems with dense matrices will be discussed later.


\item[2. ] \emph{Triangular}: If a matrix is upper or lower triangular, we can immediately invert using back-substitution. In practice we store a dense matrix and ignore the upper/lower entries.


\item[3. ] \emph{Banded}: If a matrix is zero apart from entries a fixed distance from  the diagonal it is called banded and this allows for more efficient algorithms. We discuss diagonal, tridiagonal and bidiagonal matrices.

\end{itemize}
\textbf{Remark} For those who took the first half of the module, there was an important emphasis on working with \emph{linear operators} rather than \emph{matrices}. That is, there was an emphasis on basis-independent mathematical techniques. However, in terms of practical computation we need to work with some representation of an operator and the  most natural is a matrix, a table of numbers. Hence we will only consider the case where we use the canonical basis. (This is also important as some of the students have not taken the first half of the module.)

\subsection{Dense matrices}
A basic operation is matrix-vector multiplication. For a field $\ensuremath{\bbF}$ (typically $\ensuremath{\bbR}$ or $\ensuremath{\bbC}$, or this can be relaxed to be a ring), $A \ensuremath{\in} \ensuremath{\bbF}^{m \ensuremath{\times} n}$ and $\ensuremath{\bm{\x}} \ensuremath{\in} \ensuremath{\bbF}^n$ we have the usual definition of matrix multiplication:
\[
A\ensuremath{\bm{\x}} := \begin{bmatrix} \ensuremath{\sum}_{j=1}^n a_{1,j} x_j \\ \ensuremath{\vdots} \\ \ensuremath{\sum}_{j=1}^n a_{m,j} x_j \end{bmatrix}.
\]
When we are working with floating point numbers $A \ensuremath{\in} F^{m \ensuremath{\times} n}$ we obtain an approximation:
\[
A\ensuremath{\bm{\x}} \ensuremath{\approx} \begin{bmatrix} \ensuremath{\bigoplus}_{j=1}^n (a_{1,j}  \ensuremath{\otimes} x_j) \\ \ensuremath{\vdots} \\  \ensuremath{\bigoplus}_{j=1}^n (a_{m,j}  \ensuremath{\otimes} x_j) \end{bmatrix}.
\]
This actually encodes an algorithm for computing the entries. And this algorithm uses $O(m n)$ floating point operations (FLOPs): each of the $m$ entry consists of $n$ multiplications and $n-1$ additions, hence we $2n-1 = O(n)$ per row for a total of $m(2n-1) = O(mn)$ operations. In the problem sheet we see how the floating point error can be bounded in terms of norms, thus reducing the problem to a purely mathematical concept.

Sometimes there are multiple ways of implementing numerical algorithms. We have an alternative formula where we multiple by columns: if we write
\[
A = \begin{bmatrix} \ensuremath{\bm{\a}}_1 | \ensuremath{\cdots} | \ensuremath{\bm{\a}}_n \end{bmatrix}
\]
where $\ensuremath{\bm{\a}}_j = A \ensuremath{\bm{\e}}_j \ensuremath{\in} \ensuremath{\bbF}^m$ being the columns of $A$ we also have
\[
A \ensuremath{\bm{\x}} = x_1 \ensuremath{\bm{\a}}_1  + \ensuremath{\cdots} + x_n \ensuremath{\bm{\a}}_n.
\]
The floating point formula for this is exactly the same as the previous algorithm and the number of operations is the same. Just the order of operations has changed.  Suprisingly, this latter version is significantly faster.

\subsection{Triangular matrices}
The simplest sparsity case is being triangular: where all entries above or below the diagonal are zero.  Matrix multiplication can be modified to take advantage of this. Eg., if $U \ensuremath{\in} \ensuremath{\bbF}^{n \ensuremath{\times} n}$ is upper triangular we have:
\[
U\ensuremath{\bm{\x}} = \begin{bmatrix} \ensuremath{\sum}_{j=1}^n a_{1,j} x_j \\ \ensuremath{\sum}_{j=2}^n a_{2,j} x_j  \\ \ensuremath{\vdots} \\ \ensuremath{\sum}_{j=n}^n a_{m,j} x_j \end{bmatrix}.
\]
When implemented in floating point this uses roughly half the number of multiplications: $n + (n-1) + \ensuremath{\ldots} + 1 = n(n+1)/2$ multiplications. The complexity is still $O(n^2)$ however. 

Triangularity allows us to also invert systems using forward- or back-substitution. In particular if $\ensuremath{\bm{\x}}$ solves $U \ensuremath{\bm{\x}} = \ensuremath{\bm{\b}}$ then we have:
\[
x_k = {b_k - \ensuremath{\sum}_{j=k+1}^n u_{kj} x_j \over u_{kk}}
\]
Thus we can compute $x_n, x_{n-1},\ensuremath{\ldots},x_1$ in sequence (using floating point operations).

\subsection{Banded matrices}
A \emph{banded matrix} is zero off a prescribed number of diagonals.  We call the number of (potentially) non-zero diagonals the \emph{bandwidths}:

\textbf{Definition (bandwidths)} A matrix $A$ has \emph{lower-bandwidth} $l$ if  $A[k,j] = 0$ for all $k-j > l$ and \emph{upper-bandwidth} $u$ if $A[k,j] = 0$ for all $j-k > u$. We say that it has \emph{strictly lower-bandwidth} $l$ if it has lower-bandwidth $l$ and there exists a $j$ such that $A[j+l,j] \neq 0$. We say that it has \emph{strictly upper-bandwidth} $u$ if it has upper-bandwidth $u$ and there exists a $k$ such that $A[k,k+u] \neq 0$.

A banded matrix has better complexity for matrix multiplication and solving linear systems:  we can multiple in $O(\min(m,n))$ operations. We consider two cases in particular (in addition to diagonal): bidiagonal and tridiagonal. 

\textbf{Definition (Bidiagonal)} If a square matrix has bandwidths $(l,u) = (1,0)$ it is \emph{lower-bidiagonal} and if it has bandwidths $(l,u) = (0,1)$ it is \emph{upper-bidiagonal}. 

For example, if
\[
U = \begin{bmatrix} u_{1,1} & u_{1,2} \\ & \ensuremath{\ddots} & \ensuremath{\ddots} \\
&&                                 u_{n-1,n-1} & u_{n-1,n} \\
&&& u_{n,n}
\end{bmatrix}
\]
is upper-bidiagonal multiplication becomes
\[
U\ensuremath{\bm{\x}} = \begin{bmatrix} u_{1,1} x_1 + u_{1,2} x_2 \\ u_{2,2} x_2 + u_{2,3} x_3  \\ \ensuremath{\vdots} \\ u_{n-1,n-1} x_{n-1} + u_{n-1,n} x_n \\u_{n,n} x_n \end{bmatrix}.
\]
This requires only $O(n)$ operations. A bidiagonal matrix is always triangular and we can also invert in $O(n)$ operations: if $U \ensuremath{\bm{\x}} = \ensuremath{\bm{\b}}$ then $x_n = b_n/u{n,n}$ and for $k = n-1,\ensuremath{\ldots},1$
\[
x_k = {b_k - u_{k,k+1} x_{k+1} \over u_{k,k}}.
\]
\textbf{Definition (Tridiagonal)} If a square matrix has bandwidths $l = u = 1$ it is \emph{tridiagonal}.

For example, 
\[
A = \begin{bmatrix} a_{1,1} & a_{1,2} \\
a_{2,1} & a_{2,2} & a_{2,3} \\
 & \ensuremath{\ddots} & \ensuremath{\ddots} & \ensuremath{\ddots} \\
&& a_{n-1,n-2} &                                 a_{n-1,n-1} & a_{n-1,n} \\
&&&a_{n,n-1} & a_{n,n}
\end{bmatrix}
\]
is tridiagonal. Matrix multiplication is clearly $O(n)$ operations. But so is solving linear systems.  We will see why later. 





\section{Differential Equations via Finite Differences}
Linear algebra is a powerful tool for solving linear equations, including \ensuremath{\infty}-dimensional ones like differential equations. In this section we discuss \emph{finite differences}: an algorithmic way of reducing ODEs to linear systems by replacing derivatives with divided differences approximations. 

We will focus on the following differential equations:

\begin{itemize}
\item[1. ] Indefinite integration for $a \ensuremath{\leq} x \ensuremath{\leq} b$:

\end{itemize}

\begin{align*}
u(a) = c, u'(x) = f(x)
\end{align*}
\begin{itemize}
\item[2. ] First-order linear ODEs:
\end{itemize}

\begin{align*}
u(a) = c, u'(x) - \ensuremath{\omega}(x) u(x) = f(x)
\end{align*}
\begin{itemize}
\item[3. ] Poisson equation with Dirichlet conditions:

\end{itemize}

\begin{align*}
u(a) &= c_0, u(b) = c_1, \\
u''(x) &= f(x)
\end{align*}
Briefly, the basic idea of finite differences is as follows:

\begin{itemize}
\item[1. ] Discretise $[a,b]$ by a grid of points $x_0,\ensuremath{\ldots},x_n$.


\item[2. ] Replace derivatives with divided difference approximations.


\item[3. ] Use this to deduce a linear system whose solution approximates $u(x_j)$. 

\end{itemize}
\subsection{Indefinite integration}
We begin with the simplest differential equation on an interval $[a,b]$:
\begin{align*}
u(a) &= c \\
u'(x) &= f(x)
\end{align*}
As in integration we will use an evenly spaced grid $a = x_0 , x_1 , \ensuremath{\ldots}, x_n = b$ defined by $x_j :=  a + h j$ where $h := (b-a)/n$. The solution is of course $u(x) = c + \ensuremath{\int}_a^x f(x) {\rm d}x$. We could use Rectangular or Trapezium rules to to obtain approximations to $u(x_j)$ for each $j$, however, we shall take another (equivalent) approach that will generalise to other differential equations. 

Finite differences consists of three stages to reduce the problem to a linear system. Consider a divided difference approximation like right-sided divided differences: 
\[
u'(x) \ensuremath{\approx} {u(x+h) - u(x)\over h}.
\]
When applied to a grid point $x_0,\ensuremath{\ldots},x_{n-1}$ this becomes:
\[
u'(x_j) \ensuremath{\approx} {u(x_j+h) - u(x_j)\over h} = {u(x_{j+1}) - u(x_j)\over h}
\]
Note that $x_n$ is not permitted since that would go past the interval. We use this approximation as follows:

\begin{itemize}
\item[1. ] Write the ODE and initial conditions on the grid. Since right-sided differences will depend on $x_j$ and $x_{j+1}$ we stop at $x_{n-1}$ to avoid going past our grid: 

\end{itemize}
\[
\Vectt[u(x_0) , 
     u'(x_0) ,
u'(x_1) ,
\ensuremath{\vdots} ,
u'(x_{n-1})] = \underbrace{\Vectt[c, f(x_0), f(x_1), \ensuremath{\vdots} , f(x_{n-1})]}_{\ensuremath{\bm{\b}}}
\]
\begin{itemize}
\item[2. ] Replace derivatives with divided differences:

\end{itemize}
\[
\Vectt[u(x_0) \\ 
(u(x_1) - u(x_0))/h \\
(u(x_2) - u(x_1))/h \\
\ensuremath{\vdots} \\
(u(x_n) - u(x_{n-1})/h] \ensuremath{\approx} \ensuremath{\bm{\b}}
\]
\begin{itemize}
\item[3. ] We do not know $u(x_j)$ hence we replace it with other unknowns $u_j$,

\end{itemize}
but where the approximation becomes equality:
\[
\Vectt[u_0 \\ 
(u_1 - u_0)/h \\
(u_2 - u_1)/h \\
\ensuremath{\vdots} \\
(u_n - u_{n-1})/h] = \ensuremath{\bm{\b}}
\]
\begin{itemize}
\item[4. ] This is actually a lower bidiagonal linear system:

\end{itemize}
\[
\underbrace{\begin{bmatrix}
    1 \\ 
    -1/h & 1/h \\
    & \ensuremath{\ddots} & \ensuremath{\ddots} \\
    && -1/h & 1/h \end{bmatrix}}_L \underbrace{\Vectt[u_0,u_1,\ensuremath{\vdots},u_n]}_{\ensuremath{\bm{\u}}} = \ensuremath{\bm{\b}}
\]
We can solve $L \ensuremath{\bm{\u}} = \ensuremath{\bm{\b}}$ using forward-substitution.

\subsection{Forward Euler}
We can extend this to more general first-order linear differential equations with a variable coefficient:
\begin{align*}
u(a) &= c \\
u'(x) - \ensuremath{\omega}(x) u(x) &= f(x)
\end{align*}
The steps proceed very similar to before:

\begin{itemize}
\item[1. ] Write the ODE and initial conditions on the grid:

\end{itemize}
\[
\Vectt[u(x_0) \\ 
u'(x_0) + \ensuremath{\omega}(x_0) u(x_0) \\
u'(x_1) + \ensuremath{\omega}(x_1) u(x_1) \\
\ensuremath{\vdots} \\
u'(x_{n-1})+ \ensuremath{\omega}(x_{n-1}) u(x_{n-1})] = \underbrace{\Vectt[c, f(x_0), f(x_1), \ensuremath{\vdots} , f(x_{n-1})]}_{\ensuremath{\bm{\b}}}
\]
\begin{itemize}
\item[2. ] Replace derivatives with divided differences:

\end{itemize}
\[
\Vectt[u(x_0) \\ 
(u(x_1) - u(x_0))/h + \ensuremath{\omega}(x_0)u(x_0) \\
(u(x_2) - u(x_1))/h + \ensuremath{\omega}(x_1)u(x_1) \\
\ensuremath{\vdots} \\
(u(x_n) - u(x_{n-1})/h + \ensuremath{\omega}(x_{n-1})u(x_{n-1})] \ensuremath{\approx} \ensuremath{\bm{\b}}
\]
\begin{itemize}
\item[3. ] Replace $u(x_j)$  by its approximation $u_j$:

\end{itemize}
\[
\Vectt[u_0 \\ 
(u_1 - u_0)/h + \ensuremath{\omega}(x_0) u_0 \\
(u_2 - u_1)/h + \ensuremath{\omega}(x_1) u_1 \\
\ensuremath{\vdots} \\
(u_n - u_{n-1})/h  + \ensuremath{\omega}(x_{n-1}) u_{n-1}] = \ensuremath{\bm{\b}}
\]
\begin{itemize}
\item[4. ] We now get the linear system:

\end{itemize}
\[
\underbrace{\begin{bmatrix}
    1 \\ 
    \ensuremath{\omega}(x_0)-1/h & 1/h \\
    & \ensuremath{\ddots} & \ensuremath{\ddots} \\
    && \ensuremath{\omega}(x_{n-1})-1/h & 1/h \end{bmatrix}}_L \underbrace{\Vectt[u_0,u_1,\ensuremath{\vdots},u_n]}_{\ensuremath{\bm{\u}}} = \ensuremath{\bm{\b}}
\]
We can solve $L \ensuremath{\bm{\u}} = \ensuremath{\bm{\b}}$ using forward-substitution.

\subsection{Poisson equation}
Consider the Poisson equation with Dirichlet conditions (a two-point boundary value problem):
\begin{align*}
u(0) &= c \\
u''(x) &= f(x) \\
u(1) &= d
\end{align*}
We shall adapt the procedure using the second-order divided difference approximation from the first probem sheet:
\[
u''(x) \ensuremath{\approx} {u(x-h) - 2u(x) + u(x+h)\over h^2}
\]
When applied to a grid point $x_1,\ensuremath{\ldots},x_{n-1}$ this becomes:
\[
u'(x_j) \ensuremath{\approx} {u(x_j-h) - 2u(x_j) + u(x_j+h)\over h^2} = {u(x_{j-1}) - 2u(x_j) + u(x_{j+1})\over h^2}
\]
Note that $x_0$ and $x_n$ is not permitted since that would go past the interval. We use this approximation as follows:

\begin{itemize}
\item[1. ] Write the ODE and boundary conditions on the grid:

\end{itemize}
\[
\Vectt[u(x_0) \\ 
u''(x_1) \\
u''(x_1) \\
\ensuremath{\vdots} \\
u''(x_{n-1}) \\
u(x_n)] = \underbrace{\Vectt[c, f(x_0), f(x_1), \ensuremath{\vdots} , f(x_{n-1}), d]}_{\ensuremath{\bm{\b}}}
\]
\begin{itemize}
\item[2. ] Replace derivatives with divided differences:

\end{itemize}
\[
\Vectt[u(x_0) \\ 
{u(x_0) - 2u(x_1) + u(x_2)\over h^2} \\
{u(x_1) - 2u(x_2) + u(x_3)\over h^2} \\
\ensuremath{\vdots} \\
{u(x_{n-2}) - 2u(x_{n-1}) + u(x_n)\over h^2} \\
u(x_n)] \ensuremath{\approx} \ensuremath{\bm{\b}}
\]
\begin{itemize}
\item[3. ] Replace $u(x_j)$  by its approximation $u_j$:

\end{itemize}
\[
\Vectt[u_0 \\ 
{u_0 - 2u_1 + u_2\over h^2} \\
{u_1 - 2u_2 + u_3\over h^2} \\
\ensuremath{\vdots} \\
{u_{n-2} - 2u_{n-1} + u_n\over h^2} \\
u_n] = \ensuremath{\bm{\b}}
\]
\begin{itemize}
\item[4. ] We now get a tridiagonal linear system:

\end{itemize}
\[
\underbrace{\begin{bmatrix}
    1 \\ 
    1/h^2 & -2/h^2 & 1/h \\
    & \ensuremath{\ddots} & \ensuremath{\ddots} & \ensuremath{\ddots} \\
   && 1/h^2 & -2/h^2 & 1/h \\ 
   &&&& 1 \end{bmatrix}}_A \underbrace{\Vectt[u_0,u_1,\ensuremath{\vdots},u_n]}_{\ensuremath{\bm{\u}}} = \ensuremath{\bm{\b}}
\]
But how do we solve a tridiagonal linear system $A \ensuremath{\bm{\u}} = \ensuremath{\bm{\b}}$?





\section{LU and Cholesky factorisations}
In this section we consider the following factorisations for square invertible  matrices $A$:

\begin{itemize}
\item[1. ] The \emph{LU factorisation}: $A = LU$ where $L$ is lower triangular and $U$ is upper triangular. This is equivalent to Gaussian elimination without pivoting, so may not exist (e.g. if $a_{11} = 0$).


\item[2. ] The \emph{PLU factorisation}: $A = P^\ensuremath{\top} LU$ where $P$ is a permutation matrix (a matrix when multiplying a vector is equivalent to permuting its rows), $L$ is lower triangular and $U$ is upper triangular. This is equivalent to Gaussian elimination with pivoting. It always exists but may be unstable in extremely rare cases. We won't discuss the details of computing the PLU factorisation but will explore practical usage in the lab.


\item[3. ] For a real square \emph{symmetric positive definite} ($A \ensuremath{\in} \ensuremath{\bbR}^{n \ensuremath{\times} n}$ such that $A^\ensuremath{\top} = A$ and $\ensuremath{\bm{\x}}^\ensuremath{\top} A \ensuremath{\bm{\x}} > 0$ for all $\ensuremath{\bm{\x}} \ensuremath{\in} \ensuremath{\bbR}^n$, $\ensuremath{\bm{\x}} \ensuremath{\neq} 0$)  matrix the LU decomposition has a special form called the \emph{Cholesky factorisation}: $A = L L^\ensuremath{\top}$. This provides an algorithmic way to \emph{prove} that a matrix is symmetric positive definite, and is roughly twice as fast as the LU factorisation to compute.

\end{itemize}
\subsection{Outer products}
In what follows we will use outer products extensively:

\begin{definition}[outer product] Given $\ensuremath{\bm{\x}} \ensuremath{\in} \ensuremath{\bbF}^m$ and $\ensuremath{\bm{\y}} \ensuremath{\in} \ensuremath{\bbF}^n$ the \emph{outer product} is:
\[
\ensuremath{\bm{\x}} \ensuremath{\bm{\y}}^\ensuremath{\top} := [\ensuremath{\bm{\x}} y_1 | \ensuremath{\cdots} | \ensuremath{\bm{\x}} y_n] = \begin{bmatrix} x_1 y_1 & \ensuremath{\cdots} & x_1 y_n \\
                        \ensuremath{\vdots} & \ensuremath{\ddots} & \ensuremath{\vdots} \\
                        x_m y_1 & \ensuremath{\cdots} & x_m y_n \end{bmatrix} \ensuremath{\in} \ensuremath{\bbF}^{m \ensuremath{\times} n}.
\]
Note this is equivalent to matrix-matrix multiplication if we view $\ensuremath{\bm{\x}}$ as a $m \ensuremath{\times} 1$ matrix and $\ensuremath{\bm{\y}}^\ensuremath{\top}$ as a $1 \ensuremath{\times} n$ matrix. \end{definition}

\begin{proposition}[rank-1] A matrix $A \ensuremath{\in} \ensuremath{\bbF}^{m\ensuremath{\times}n}$ has rank 1 if and only if there exists $\ensuremath{\bm{\x}} \ensuremath{\in} \ensuremath{\bbF}^m$ and $\ensuremath{\bm{\y}} \ensuremath{\in} \ensuremath{\bbF}^n$ such that
\[
A = \ensuremath{\bm{\x}} \ensuremath{\bm{\y}}^\ensuremath{\top}.
\]
\end{proposition}
\textbf{Proof} This follows immediately as if $A = \ensuremath{\bm{\x}} \ensuremath{\bm{\y}}^\ensuremath{\top}$ then all columns are multiples of $\ensuremath{\bm{\x}}$. On the other hand, if $A$ has rank-1 there exists a nonzero column $\ensuremath{\bm{\x}} := \ensuremath{\bm{\a}}_j$ that all other columns are multiples of. \ensuremath{\QED}

\subsection{LU factorisation}
Gaussian elimination  can be interpreted as an LU factorisation. Write a matrix $A \ensuremath{\in} \ensuremath{\bbF}^{n \ensuremath{\times} n}$ as follows:
\[
A =  \begin{bmatrix} \ensuremath{\alpha}_1 & \ensuremath{\bm{\w}}_1^\ensuremath{\top} \\ \ensuremath{\bm{\v}}_1 & K_1 \end{bmatrix}
\]
where $\ensuremath{\alpha}_1 = a_{11}$, $\ensuremath{\bm{\v}}_1 = A[2:n, 1]$ and $\ensuremath{\bm{\w}}_1 = A[1, 2:n]$ (that is, $\ensuremath{\bm{\v}}_1 \ensuremath{\in} \ensuremath{\bbF}^{n-1}$ is a vector whose entries are the 2nd through last row of the first column of $A$ whilst $\ensuremath{\bm{\w}}_1 \ensuremath{\in} \ensuremath{\bbF}^{n-1}$ is a vector containing the 2nd through last entries in the last column of $A$). Gaussian elimination consists of taking the first row, dividing by $\ensuremath{\alpha}_1$ and subtracting from all other rows. That is equivalent to multiplying by a lower triangular matrix:
\[
\begin{bmatrix}
1 \\
-\ensuremath{\bm{\v}}_1/\ensuremath{\alpha}_1 & I \end{bmatrix} A = \begin{bmatrix} \ensuremath{\alpha}_1 & \ensuremath{\bm{\w}}_1^\ensuremath{\top} \\  & K_1 -\ensuremath{\bm{\v}}_1\ensuremath{\bm{\w}}_1^\ensuremath{\top} /\ensuremath{\alpha}_1 \end{bmatrix}
\]
where $A_2 := K_1 -\ensuremath{\bm{\v}}_1\ensuremath{\bm{\w}}_1^\ensuremath{\top} /\ensuremath{\alpha}_1$  happens to be a rank-1 perturbation of $K_1$. We can write this another way:
\[
A = \underbrace{\begin{bmatrix}
1 \\
\ensuremath{\bm{\v}}_1/\ensuremath{\alpha}_1 & I \end{bmatrix}}_{L_1}  \begin{bmatrix} \ensuremath{\alpha}_1 & \ensuremath{\bm{\w}}_1^\ensuremath{\top} \\  & A_2 \end{bmatrix}
\]
Now assume we continue this process and manage to deduce an LU factorisation $A_2 = L_2 U_2$. Then
\[
A = L_1 \begin{bmatrix} \ensuremath{\alpha}_1 & \ensuremath{\bm{\w}}_1^\ensuremath{\top} \\  & L_2U_2 \end{bmatrix}
= \underbrace{L_1 \begin{bmatrix}
1 \\
 & L_2 \end{bmatrix}}_L  \underbrace{\begin{bmatrix} \ensuremath{\alpha}_1 & \ensuremath{\bm{\w}}_1^\ensuremath{\top} \\  & U_2 \end{bmatrix}}_U
\]
Note we can multiply through to find
\[
L = \begin{bmatrix}
1 \\
\ensuremath{\bm{\v}}_1/\ensuremath{\alpha}_1 & L_2 \end{bmatrix}.
\]
Noting that if $A \ensuremath{\in} \ensuremath{\bbF}^{1 \ensuremath{\times} 1}$ then it has a trivial LU factorisation we can use the above construction to proceed recursively until we arrive at the trivial case.

\begin{example}[LU by-hand] Consider the matrix
\[
A = \begin{bmatrix} 1 & 1 & 1 \\
                    2 & 4 & 8 \\
                    1 & 4 & 9
                    \end{bmatrix} = \underbrace{\begin{bmatrix} 1  \\
                    2 & 1 &  \\
                    1 &  & 1
                    \end{bmatrix}}_{L_1} \begin{bmatrix} 1 & 1 & 1 \\
                    0 & 2 & 6 \\
                    0 & 3 & 8
                    \end{bmatrix}
\]
In more detail, for $\ensuremath{\alpha}_1 := a_{11} = 1$, $\ensuremath{\bm{\v}}_1 := A[2:3,1] = \vectt[2,1]$, $\ensuremath{\bm{\w}}_1 = A[1,2:3] = \vectt[1,1]$ and
\[
K_1 := A[2:3,2:3] = \begin{bmatrix} 4 & 8 \\ 4 & 9 \end{bmatrix}
\]
we have
\[
A_2 := K_1 -\ensuremath{\bm{\v}}_1\ensuremath{\bm{\w}}_1^\ensuremath{\top} /\ensuremath{\alpha}_1 = \begin{bmatrix} 4 & 8 \\ 4 & 9 \end{bmatrix} - \begin{bmatrix} 2 & 2 \\ 1 & 1 \end{bmatrix} = \begin{bmatrix} 2 & 6 \\ 3 & 8 \end{bmatrix}.
\]
We then repeat the process and determine (with $\ensuremath{\alpha}_2 := A_2[1,1] = 2$, $\ensuremath{\bm{\v}}_2 := A_2[2:2,1] = [3]$, $\ensuremath{\bm{\w}}_2 := A_2[1,2:2] = [6]$ and $K_2 := A_2[2:2,2:2] = [8]$):
\[
A_2 =  \begin{bmatrix}2 & 6 \\ 3 & 8 \end{bmatrix} =
\underbrace{\begin{bmatrix}
1 \\
3/2 & 1
\end{bmatrix}}_{L_2} \begin{bmatrix} 2 & 6 \\
            & -1 \end{bmatrix}
\]
The last \ensuremath{\ldq}matrix" is 1 x 1 so we get the trivial decomposition:
\[
A_3 := K_2 - \ensuremath{\bm{\v}}_2 \ensuremath{\bm{\w}}_2^\ensuremath{\top} /\ensuremath{\alpha}_2 =  [-1] = \underbrace{[1]}_{L_3} [-1]
\]
Putting everything together and placing the $j$-th column of $L_j$ inside the $j$-th column of $L$ we have
\[
A = \underbrace{\begin{bmatrix} 1  \\
                    2 & 1 &  \\
                    1 & 3/2 & 1
                    \end{bmatrix}}_{L} \underbrace{\begin{bmatrix} 1 & 1 & 1 \\
                     & 2 & 6 \\
                     &  & -1
                    \end{bmatrix}}_U
\]
\end{example}

\subsection{PLU factorisation}
We learned in first year linear algebra that if a diagonal entry is zero when doing Gaussian elimination one has to \emph{row pivot}. For stability, in implementation one may wish to pivot even if the diagonal entry is nonzero: swap the largest in magnitude entry for the entry on the diagonal turns out to be significantly more stable than standard LU.

This is equivalent to a PLU decomposition. Here we use a \emph{permutation matrix}, whose action on a vector permutes its entries, as discussed in the appendix. That is, consider a permutation which we identify with a vector ${\mathbf \ensuremath{\sigma}} = [\ensuremath{\sigma}_1,\ensuremath{\ldots},\ensuremath{\sigma}_n]$ containing the integers $1,\ensuremath{\ldots},n$ exactly once. The permutation operator represents the action of permuting the entries in a vector:
\[
P_\ensuremath{\sigma}(\ensuremath{\bm{\v}}) := \ensuremath{\bm{\v}}[{\mathbf \ensuremath{\sigma}}] = \Vectt[v_{\ensuremath{\sigma}_1},\ensuremath{\vdots},v_{\ensuremath{\sigma}_n}]
\]
This is a linear operator, and hence we can identify it with a \emph{permutation matrix} $P_\ensuremath{\sigma} \ensuremath{\in} \ensuremath{\bbR}^{n \ensuremath{\times} n}$ (more precisely the entries  of $P_\ensuremath{\sigma}$ are either 1 or 0). Importantly, products of permutation matrices are also permutation matrices and permutation matrices are orthogonal, that is, $P_\ensuremath{\sigma}^{-1} = P_\ensuremath{\sigma}^\top$.

\begin{theorem}[PLU] A matrix $A \ensuremath{\in} \ensuremath{\bbC}^{n \ensuremath{\times} n}$ is invertible if and only if it has a PLU decomposition:
\[
A = P^\ensuremath{\top} L U
\]
where the diagonal of $L$ are all equal to 1 and the diagonal of $U$ are all non-zero, and $P$ is a permutation matrix.

\end{theorem}
\textbf{Proof}

If we have a PLU decomposition of this form then $L$ and $U$ are invertible and hence the inverse is simply $A^{-1} = U^{-1} L^{-1} P$. Hence we consider the orther direction.

If $A \ensuremath{\in} \ensuremath{\bbC}^{1 \ensuremath{\times} 1}$ we trivially have an LU decomposition $A = [1] * [a_{11}]$ as all $1 \ensuremath{\times} 1$ matrices are triangular. We now proceed by induction: assume all invertible matrices of lower dimension have a PLU factorisation. As $A$ is invertible not all entries in the first column are zero. Therefore there exists a permutation $P_1$ so that $\ensuremath{\alpha} := (P_1 A)[1,1] \ensuremath{\neq} 0$. Hence we write
\[
P_1 A = \begin{bmatrix} \ensuremath{\alpha} & \ensuremath{\bm{\w}}^\ensuremath{\top} \\
                        \ensuremath{\bm{\v}} & K
                        \end{bmatrix} = \underbrace{\begin{bmatrix}
1 \\
\ensuremath{\bm{\v}}/\ensuremath{\alpha} & I \end{bmatrix}}_{L_1}  \begin{bmatrix} \ensuremath{\alpha} & \ensuremath{\bm{\w}}^\ensuremath{\top} \\  & K - \ensuremath{\bm{\v}} \ensuremath{\bm{\w}}^\ensuremath{\top}/\ensuremath{\alpha} \end{bmatrix}
\]
We deduce that $A_2 := K - \ensuremath{\bm{\v}} \ensuremath{\bm{\w}}^\ensuremath{\top}/\ensuremath{\alpha}$ is invertible because $A$ and $L_1$ are invertible (Exercise).

By assumption we can write $A_2 = P_2^\ensuremath{\top} L_2 U_2$. Thus we have:
\begin{align*}
\underbrace{\begin{bmatrix} 1 \\
            & P_2 \end{bmatrix} P_1}_P A &= \begin{bmatrix} 1 \\
            & P_2 \end{bmatrix}  \begin{bmatrix} \ensuremath{\alpha} & \ensuremath{\bm{\w}}^\ensuremath{\top} \\
                        \ensuremath{\bm{\v}} & A_2
                        \end{bmatrix}  =
            \begin{bmatrix} 1 \\ & P_2 \end{bmatrix} L_1  \begin{bmatrix} \ensuremath{\alpha} & \ensuremath{\bm{\w}}^\ensuremath{\top} \\  & P_2^\ensuremath{\top} L_2  U_2 \end{bmatrix} \\
            &= \begin{bmatrix}
1 \\
P_2 \ensuremath{\bm{\v}}/\ensuremath{\alpha} & P_2 \end{bmatrix} \begin{bmatrix} 1 &  \\  &  P_2^\ensuremath{\top} L_2  \end{bmatrix}  \begin{bmatrix} \ensuremath{\alpha} & \ensuremath{\bm{\w}}^\ensuremath{\top} \\  &  U_2 \end{bmatrix} \\
&= \underbrace{\begin{bmatrix}
1 \\
P_2 \ensuremath{\bm{\v}}/\ensuremath{\alpha} & L_2  \end{bmatrix}}_L \underbrace{\begin{bmatrix} \ensuremath{\alpha} & \ensuremath{\bm{\w}}^\ensuremath{\top} \\  &  U_2 \end{bmatrix}}_U. \\
\end{align*}
\ensuremath{\QED}

We don't discuss the practical implementation of this factorisation (though an algorithm is hidden in the above proof). We also note that for stability one uses the permutation that always puts the largest in magnitude entry in the top row. In the lab we explore the practical usage of this factorisation.

\subsection{Cholesky factorisation}
A \emph{Cholesky factorisation} is a form of Gaussian elimination (without pivoting) that exploits symmetry in the problem, resulting in a substantial speedup. It is only applicable for \emph{symmetric positive definite} (SPD) matrices, or rather, the algorithm for computing it succeeds if and only if the matrix is SPD. In other words, it gives an algorithmic way to prove whether or not a matrix is SPD.

\begin{definition}[positive definite] A square matrix $A \ensuremath{\in} \ensuremath{\bbR}^{n \ensuremath{\times} n}$ is \emph{positive definite} if for all $\ensuremath{\bm{\x}} \ensuremath{\in} \ensuremath{\bbR}^n, x \ensuremath{\neq} 0$ we have
\[
\ensuremath{\bm{\x}}^\ensuremath{\top} A \ensuremath{\bm{\x}} > 0
\]
\end{definition}

First we establish some basic properties of positive definite matrices:

\begin{proposition}[conj. pos. def.] If  $A \ensuremath{\in} \ensuremath{\bbR}^{n \ensuremath{\times} n}$ is positive definite and $V \ensuremath{\in} \ensuremath{\bbR}^{n \ensuremath{\times} n}$ is non-singular then
\[
V^\ensuremath{\top} A V
\]
is positive definite. \end{proposition}
\textbf{Proof}

For all  $\ensuremath{\bm{\x}} \ensuremath{\in} \ensuremath{\bbR}^n, \ensuremath{\bm{\x}} \ensuremath{\neq} 0$, define $\ensuremath{\bm{\y}} = V \ensuremath{\bm{\x}} \ensuremath{\neq} 0$ (since $V$ is non-singular). Thus we have
\[
\ensuremath{\bm{\x}}^\ensuremath{\top} V^\ensuremath{\top} A V \ensuremath{\bm{\x}} = \ensuremath{\bm{\y}}^\ensuremath{\top} A \ensuremath{\bm{\y}} > 0.\ensuremath{\aa}
\]
\ensuremath{\QED}

\begin{proposition}[diag positivity] If $A \ensuremath{\in} \ensuremath{\bbR}^{n \ensuremath{\times} n}$ is positive definite then its diagonal entries are positive: $a_{kk} > 0$. \end{proposition}
\textbf{Proof}
\[
a_{kk} = \ensuremath{\bm{\e}}_k^\ensuremath{\top} A \ensuremath{\bm{\e}}_k > 0.
\]
\ensuremath{\QED}

\begin{theorem}[subslice pos. def.] If $A \ensuremath{\in} \ensuremath{\bbR}^{n \ensuremath{\times} n}$ is positive definite and $\ensuremath{\bm{\k}} = [k_1,\ensuremath{\ldots},k_m]^\ensuremath{\top} \ensuremath{\in} \{1,\ensuremath{\ldots},n\}^m$ is a vector of $m$ integers where any integer appears only once,  then $A[\ensuremath{\bm{\k}},\ensuremath{\bm{\k}}] \ensuremath{\in} \ensuremath{\bbR}^{m \ensuremath{\times} m}$ is also positive definite. \end{theorem}
\textbf{Proof} For all $\ensuremath{\bm{\x}} \ensuremath{\in} \ensuremath{\bbR}^m, \ensuremath{\bm{\x}} \ensuremath{\neq} 0$, consider $\ensuremath{\bm{\y}} \ensuremath{\in} \ensuremath{\bbR}^n$ such that $y_j = x_{k_j}$ and zero otherwise. Then we have
\[
\ensuremath{\bm{\x}}^\ensuremath{\top} A[\ensuremath{\bm{\k}},\ensuremath{\bm{\k}}] \ensuremath{\bm{\x}} = \ensuremath{\sum}_{\ensuremath{\ell}=1}^m \ensuremath{\sum}_{j=1}^m x_\ensuremath{\ell} x_j a_{k_\ensuremath{\ell},k_j}  = \ensuremath{\sum}_{\ensuremath{\ell}=1}^n \ensuremath{\sum}_{j=1}^n y_\ensuremath{\ell} y_j a_{\ensuremath{\ell},j} = \ensuremath{\bm{\y}}^\ensuremath{\top} A \ensuremath{\bm{\y}} > 0.
\]
\ensuremath{\QED}

Here is the key result:

\begin{theorem}[Cholesky and SPD] A matrix $A$ is symmetric positive definite if and only if it has a Cholesky factorisation
\[
A = L L^\ensuremath{\top}
\]
where $L$ is lower triangular with positive diagonal entries.

\end{theorem}
\textbf{Proof} If $A$ has a Cholesky factorisation it is symmetric ($A^\ensuremath{\top} = (L L^\ensuremath{\top})^\ensuremath{\top} = A$) and for $\ensuremath{\bm{\x}} \ensuremath{\neq} 0$ we have
\[
\ensuremath{\bm{\x}}^\ensuremath{\top} A \ensuremath{\bm{\x}} = (L^\ensuremath{\top}\ensuremath{\bm{\x}})^\ensuremath{\top} L^\ensuremath{\top} \ensuremath{\bm{\x}} = \|L^\ensuremath{\top}\ensuremath{\bm{\x}}\|^2 > 0
\]
where we use the fact that $L$ is non-singular.

For the other direction we will prove it by induction, with the $1 \ensuremath{\times} 1$ case being trivial. Assume all lower dimensional symmetric positive definite matrices have Cholesky decompositions. Write
\[
A = \begin{bmatrix} \ensuremath{\alpha} & \ensuremath{\bm{\v}}^\ensuremath{\top} \\
                    \ensuremath{\bm{\v}}   & K
                    \end{bmatrix} = \underbrace{\begin{bmatrix} \sqrt{\ensuremath{\alpha}} \\
                                    {\ensuremath{\bm{\v}} \over \sqrt{\ensuremath{\alpha}}} & I \end{bmatrix}}_{L_1}
                                    \begin{bmatrix} 1  \\ & K - {\ensuremath{\bm{\v}} \ensuremath{\bm{\v}}^\ensuremath{\top} \over \ensuremath{\alpha}} \end{bmatrix}
                                    \underbrace{\begin{bmatrix} \sqrt{\ensuremath{\alpha}} & {\ensuremath{\bm{\v}}^\ensuremath{\top} \over \sqrt{\ensuremath{\alpha}}} \\
                                     & I \end{bmatrix}}_{L_1^\ensuremath{\top}}.
\]
Note that $A_2 := K - {\ensuremath{\bm{\v}} \ensuremath{\bm{\v}}^\ensuremath{\top} \over \ensuremath{\alpha}}$ is a subslice of $L_1^{-1} A L_1^{-\ensuremath{\top}}$, hence by combining the previous propositions is itself SPD. Thus we can write
\[
A_2 = K - {\ensuremath{\bm{\v}} \ensuremath{\bm{\v}}^\ensuremath{\top} \over \ensuremath{\alpha}} = L_2 L_2^\ensuremath{\top}
\]
and hence $A = L L^\ensuremath{\top}$ for
\[
L= L_1 \begin{bmatrix}1 \\ & L_2 \end{bmatrix} = \begin{bmatrix} \sqrt{\ensuremath{\alpha}} \\ {\ensuremath{\bm{\v}} \over \sqrt{\ensuremath{\alpha}}} & L_2 \end{bmatrix}
\]
satisfies $A = L L^\ensuremath{\top}$. \ensuremath{\QED}

\begin{example}[Cholesky by hand] Consider the matrix
\[
A = \begin{bmatrix}
2 &1 &1 &1 \\
1 & 2 & 1 & 1 \\
1 & 1 & 2 & 1 \\
1 & 1 & 1 & 2
\end{bmatrix}
\]
Then $\ensuremath{\alpha}_1 = 2$, $\ensuremath{\bm{\v}}_1 = [1,1,1]$, and
\[
A_2 = \begin{bmatrix}
2 &1 &1 \\
1 & 2 & 1 \\
1 & 1 & 2
\end{bmatrix} - {1 \over 2} \begin{bmatrix} 1 \\ 1 \\ 1 \end{bmatrix} \begin{bmatrix} 1 & 1 & 1 \end{bmatrix}
={1 \over 2} \begin{bmatrix}
3 & 1 & 1 \\
1 & 3 & 1 \\
1 & 1 & 3
\end{bmatrix}.
\]
Continuing, we have $\ensuremath{\alpha}_2 = 3/2$, $\ensuremath{\bm{\v}}_2 = [1/2,1/2]$, and
\[
A_3 = {1 \over 2} \left( \begin{bmatrix}
3 & 1 \\ 1 & 3
\end{bmatrix} - {1 \over 3} \begin{bmatrix} 1 \\ 1  \end{bmatrix} \begin{bmatrix} 1 & 1  \end{bmatrix}
\right)
= {1 \over 3} \begin{bmatrix} 4 & 1 \\ 1 & 4 \end{bmatrix}
\]
Next, $\ensuremath{\alpha}_3 = 4/3$, $\ensuremath{\bm{\v}}_3 = [1]$, and
\[
A_4 = [4/3 - 3/4 * (1/3)^2] = [5/4]
\]
i.e. $\ensuremath{\alpha}_4 = 5/4$.

Thus we get
\[
L= \begin{bmatrix}
\sqrt{\ensuremath{\alpha}_1} \\
{\ensuremath{\bm{\v}}_1[1] \over \sqrt{\ensuremath{\alpha}_1}} & \sqrt{\ensuremath{\alpha}_2} \\
{\ensuremath{\bm{\v}}_1[2] \over \sqrt{\ensuremath{\alpha}_1}} & {\ensuremath{\bm{\v}}_2[1] \over \sqrt{\ensuremath{\alpha}_2}}  & \sqrt{\ensuremath{\alpha}_3} \\
{\ensuremath{\bm{\v}}_1[3] \over \sqrt{\ensuremath{\alpha}_1}} & {\ensuremath{\bm{\v}}_2[2] \over \sqrt{\ensuremath{\alpha}_2}}  & {\ensuremath{\bm{\v}}_3[1] \over \sqrt{\ensuremath{\alpha}_3}}  & \sqrt{\ensuremath{\alpha}_4}
\end{bmatrix}
 = \begin{bmatrix} \sqrt{2} \\ {1 \over \sqrt{2}} & \sqrt{3 \over 2} \\
{1 \over \sqrt{2}} & {1 \over \sqrt 6} & {2 \over \sqrt{3}} \\
{1 \over \sqrt{2}} & {1 \over \sqrt 6} & {1 \over \sqrt{12}} & {\sqrt{5} \over 2}
\end{bmatrix}
\]
\end{example}





\section{Polynomial Interpolation and Regression}
\emph{Polynomial interpolation} is the process of finding a polynomial that equals data at a precise set of points. A more robust scheme is \emph{polynomial regression} where we use more data than the degrees of freedom in the polynomial. We therefore determine the polynomial using \emph{least squares}: find the polynomial whose samples at the points are as close as possible to the data, as measured in the $2$-norm. This least squares problem is done numerically will be discussed in the next few sections.

\subsection{Polynomial interpolation}
Our primary goal is given a set of points $x_j$ and data $f_j$, usually samples of a function $f_j = f(x_j)$, find a polynomial that interpolates the data at the points:

\begin{definition}[interpolatory polynomial] Given \emph{distinct} points $\ensuremath{\bm{\x}} = \vectt[x_1,\ensuremath{\ldots},x_n] \ensuremath{\in} \ensuremath{\bbF}^n$ and \emph{data} $\ensuremath{\bm{\f}} = \vectt[f_1,\ensuremath{\ldots},f_n] \ensuremath{\in} \ensuremath{\bbF}^n$, a degree $n-1$ \emph{interpolatory polynomial} $p(x)$ satisfies
\[
p(x_j) = f_j
\]
\end{definition}

The easiest way to solve this problem is to invert the Vandermonde system:

\begin{definition}[Vandermonde] The \emph{Vandermonde matrix} associated with $\ensuremath{\bm{\x}} \ensuremath{\in} \ensuremath{\bbF}^m$ is the matrix
\[
V_{\ensuremath{\bm{\x}},n} := \begin{bmatrix} 1 & x_1 & \ensuremath{\cdots} & x_1^{n-1} \\
                    \ensuremath{\vdots} & \ensuremath{\vdots} & \ensuremath{\ddots} & \ensuremath{\vdots} \\
                    1 & x_m & \ensuremath{\cdots} & x_m^{n-1}
                    \end{bmatrix} \ensuremath{\in} \ensuremath{\bbF}^{m \ensuremath{\times} n}.
\]
When it is clear from context we omit the subscripts $\ensuremath{\bm{\x}},n$. \end{definition}

Writing the coefficients of a polynomial
\[
p(x) = \ensuremath{\sum}_{k=0}^{n-1} c_k x^k
\]
as a vector  $\ensuremath{\bm{\c}} = \vectt[c_0,\ensuremath{\ldots},c_{n-1}] \ensuremath{\in} \ensuremath{\bbF}^n$, we note that $V$ encodes the linear map from coefficients to values at a grid, that is,
\[
V\ensuremath{\bm{\c}} = \Vectt[c_0 + c_1 x_1 + \ensuremath{\cdots} + c_{n-1} x_1^{n-1}, \ensuremath{\vdots}, c_0 + c_1 x_m + \ensuremath{\cdots} + c_{n-1} x_m^{n-1}] = \Vectt[p(x_1),\ensuremath{\vdots},p(x_m)].
\]
In the square case (where $m=n$), the coefficients of an interpolatory polynomial are given by $\ensuremath{\bm{\c}} = V^{-1} \ensuremath{\bm{\f}}$, so that
\[
\Vectt[p(x_1),\ensuremath{\vdots},p(x_n)] = V \ensuremath{\bm{\c}} = V V^{-1} \ensuremath{\bm{\f}}.
\]
This inversion is justified by the following:

\begin{proposition}[interpolatory polynomial uniqueness] Interpolatory polynomials are unique and therefore square Vandermonde matrices are invertible.

\end{proposition}
\textbf{Proof} Suppose $p$ and $\pt$ are both interpolatory polynomials of the same function. Then $p(x) - \pt(x)$ vanishes at $n$ distinct points $x_j$. By the fundamental theorem of algebra it must be zero, i.e., $p = \pt$.

For the second part, if $V \ensuremath{\bm{\c}} = 0$ for $\ensuremath{\bm{\c}} = \vectt[c_0,\ensuremath{\ldots},c_{n-1}] \ensuremath{\in} \ensuremath{\bbF}^n$ then for $q(x) = c_0 + \ensuremath{\cdots} + c_{n-1} x^{n-1}$ we have
\[
q(x_j) = \ensuremath{\bm{\e}}_j^\ensuremath{\top} V \ensuremath{\bm{\c}} = 0
\]
hence $q$ vanishes at $n$ distinct points and is therefore 0, i.e., $\ensuremath{\bm{\c}} = 0$.

\ensuremath{\QED}

We can invert square Vandermonde matrix numerically in $O(n^3)$ operations using the PLU factorisation. But it turns out we can also construct the interpolatory polynomial directly, and evaluate the polynomial in only $O(n^2)$ operations. We will use the following polynomials which equal $1$ at one grid point and zero at the others:

\begin{definition}[Lagrange basis polynomial] The \emph{Lagrange basis polynomial} is defined as
\[
\ensuremath{\ell}_k(x) := \ensuremath{\prod}_{j \ensuremath{\neq} k} {x-x_j \over x_k - x_j} =  {(x-x_1) \ensuremath{\cdots}(x-x_{k-1})(x-x_{k+1}) \ensuremath{\cdots} (x-x_n) \over (x_k - x_1) \ensuremath{\cdots} (x_k - x_{k-1}) (x_k - x_{k+1}) \ensuremath{\cdots} (x_k - x_n)}
\]
\end{definition}

Plugging in the grid points verifies that: $\ensuremath{\ell}_k(x_j) = \ensuremath{\delta}_{kj}$.

We can use these to construct the interpolatory polynomial:

\begin{theorem}[Lagrange interpolation] The unique interpolation polynomial is:
\[
p(x) = f_1 \ensuremath{\ell}_1(x) + \ensuremath{\cdots} + f_n \ensuremath{\ell}_n(x)
\]
\end{theorem}
\textbf{Proof} Note that
\[
p(x_j) = \ensuremath{\sum}_{j=1}^n f_j \ensuremath{\ell}_k(x_j) = f_j.
\]
\ensuremath{\QED}

\begin{example}[interpolating an exponential] We can interpolate $\exp(x)$ at the points $0,1,2$. That is, our data is $\ensuremath{\bm{\f}} = \vectt[{\rm e}, {\rm e},{\rm e}^2]$ and the interpolatory polynomial is
\begin{align*}
p(x) &= \ensuremath{\ell}_1(x) + {\rm e} \ensuremath{\ell}_2(x) + {\rm e}^2 \ensuremath{\ell}_3(x) =
{(x - 1) (x-2) \over (-1)(-2)} + {\rm e} {x (x-2) \over (-1)} +
{\rm e}^2 {x (x-1) \over 2} \\
&= (1/2 - {\rm e} +{\rm e}^2/2)x^2 + (-3/2 + 2 {\rm e}  - {\rm e}^2 /2) x + 1
\end{align*}
\end{example}

\textbf{Remark} Interpolating at evenly spaced points is a really \emph{bad} idea: interpolation is inheritely ill-conditioned. The labs will explore this issue experimentally. Another serious issue is that monomials are a horrible basis for interpolation. This is intuitive: when $n$ is large $x^n$ is basically zero near the origin and hence $x_j^n$ in numerically lose linear independence. We will discuss alternative bases in Part IV.

\subsection{Polynomial regression}
Often interpolation is not sufficient. Data is often on an evenly spaced grid in which case (as seen in the labs) interpolation breaks down catastrophically. Or the data is noisy and one ends up over resolving: approximating the noise rather than the signal. A simple solution is \emph{polynomial regression} use more sample points than than the degrees of freedom in the polynomial. The special case of an affine polynomial is called \emph{linear regression}.

More precisely, for $\ensuremath{\bm{\x}} \ensuremath{\in} \ensuremath{\bbF}^m$ and for $n < m$ we want to find a degree $n-1$ polynomial
\[
p(x) = \ensuremath{\sum}_{k=0}^{n-1} c_k x^k
\]
such that
\[
\Vectt[p(x_1), \ensuremath{\vdots}, p(x_m)] \ensuremath{\approx} \underbrace{\Vectt[f_1,\ensuremath{\vdots},f_m]}_{\ensuremath{\bm{\f}}}.
\]
Mapping between coefficients $\ensuremath{\bm{\c}} \ensuremath{\in} \ensuremath{\bbF}^n$ to polynomial values on a grid can be accomplished  via rectangular Vandermonde matrices. In particular, our goal is to choose $\ensuremath{\bm{\c}} \ensuremath{\in} \ensuremath{\bbF}^n$ so that
\[
V \ensuremath{\bm{\c}}  = \Vectt[p(x_1), \ensuremath{\vdots}, p(x_m)] \ensuremath{\approx} \ensuremath{\bm{\f}}.
\]
We do so by solving the \emph{least squares} system: given $V \ensuremath{\in} \ensuremath{\bbF}^{m \ensuremath{\times} n}$ and $\ensuremath{\bm{\f}} \ensuremath{\in} \ensuremath{\bbF}^m$ we want to find $\ensuremath{\bm{\c}} \ensuremath{\in} \ensuremath{\bbF}^n$ such that
\[
\| V \ensuremath{\bm{\c}} - \ensuremath{\bm{\f}} \|
\]
is minimal. Note interpolation is a special case where this norm is precisely zero (which is indeed minimal), but in general this norm may be rather large.   We will discuss the numerical solution of least squares problems in the next few sections.

\textbf{Remark} Using regression instead of interpolation can overcome the issues with evenly spaced grids. However, the monomial basis is still very problematic.





\section{Orthogonal and Unitary Matrices}
To solve least squares problems, we will  factorise a matrix $A$ as $A = QR$ where $R$ is a \emph{right-triangular matrix} and $Q$ is either an \emph{orthogonal} or \emph{unitary} matrix.

\begin{definition}[orthogonal/unitary matrix] A square real matrix is \emph{orthogonal} if its inverse is its transpose:
\[
O(n) = \{Q \ensuremath{\in} \ensuremath{\bbR}^{n \ensuremath{\times} n} : Q^\ensuremath{\top}Q = I \}
\]
A square complex matrix is \emph{unitary} if its inverse is its adjoint:
\[
U(n) = \{Q \ensuremath{\in} \ensuremath{\bbC}^{n \ensuremath{\times} n} : Q^\ensuremath{\star}Q = I \}.
\]
Here the adjoint is the same as the conjugate-transpose: $Q^\ensuremath{\star} := \bar Q^\ensuremath{\top}$.  \end{definition}

Note that $O(n) \ensuremath{\subset} U(n)$ as for real matrices $Q^\ensuremath{\star} = Q^\ensuremath{\top}$. Because in either case $Q^{-1} = Q^\ensuremath{\star}$ we also have $Q Q^\ensuremath{\star} = I$ (which for real matrices is $Q Q^\ensuremath{\top} = I$). These matrices are particularly important for numerical linear algebra for a number of reasons (we'll explore these properties in the problem sheets):

\begin{itemize}
\item[1. ] They are norm-preserving: for any vector $\ensuremath{\bm{\x}} \ensuremath{\in} \ensuremath{\bbC}^n$ and $Q \ensuremath{\in} U(n)$    we have $\|Q \ensuremath{\bm{\x}} \| = \| \ensuremath{\bm{\x}}\|$ where $\| \ensuremath{\bm{\x}} \|^2 := \ensuremath{\sum}_{k=1}^n x_k^2$ (i.e. the 2-norm).


\item[2. ] All eigenvalues have absolute value equal to $1$.


\item[3. ] For $Q \ensuremath{\in} O(n)$,  $\det Q = \ensuremath{\pm}1$.


\item[4. ] They are trivially invertible (just take the adjoint).


\item[5. ] They are generally \ensuremath{\ldq}stable": errors due to rounding when multiplying a vector by $Q$ are controlled.


\item[6. ] They are \emph{normal matrices}: they commute with their adjoint ($Q Q^\ensuremath{\star} = Q Q^\ensuremath{\star}$). 


\item[7. ] Both $O(n)$ and $U(n)$ are groups, in particular, they are closed under multiplication.

\end{itemize}
On a computer there are multiple ways of representing orthogonal/unitary matrices, and it is almost never to store a dense matrix, that is, we do not want to store all the entries. In the appendices we have seen permutation matrices, which are a special type of orthogonal matrices where we can store only the order the entries are permuted as a vector. 

More generally, we will use the group structure: represent general orthogonal/unitary matrices as products of simpler elements of the group. In partular we will use two building blocks:

\begin{itemize}
\item[1. ] \emph{Rotations}: Rotations are equivalent to special orthogonal matrices $SO(2)$  and correspond to rotations in 2D.


\item[2. ] \emph{Reflections}:  Reflections are elements of $U(n)$ that are defined in terms of a single unit vector $\ensuremath{\bm{\v}} \ensuremath{\in} \ensuremath{\bbC}^n$ which is reflected.

\end{itemize}
We remark a related concept to orthogonal/unitary matrices are rectangular matrices with orthonormal columns, e.g.
\[
U = [\ensuremath{\bm{\u}}_1 | \ensuremath{\cdots} | \ensuremath{\bm{\u}}_n] \ensuremath{\in} \ensuremath{\bbC}^{m \ensuremath{\times} n}
\]
where $m \ensuremath{\geq} n$ such that $U^\ensuremath{\star} U =  I_n$ (the $n \ensuremath{\times} n$ identity matrix). In this case we must have $UU^\ensuremath{\star} \ensuremath{\neq} I_m$ as the rank of $U$ is $n < m$. 

\subsection{Rotations}
We begin with a general definition:

\begin{definition}[Special Orthogonal and Rotations] \emph{Special Orthogonal Matrices} are
\[
SO(n) := \{Q \ensuremath{\in} O(n) | \det Q = 1 \}
\]
And (simple) \emph{rotations} are $SO(2)$. \end{definition}

In what follows we use the following for writing the angle of a vector:

\begin{definition}[two-arg arctan] The two-argument arctan function gives the angle \texttt{\ensuremath{\theta}} through the point $[a,b]^\ensuremath{\top}$, i.e., 
\[
\sqrt{a^2 + b^2} \begin{bmatrix} \cos \ensuremath{\theta} \\ \sin \ensuremath{\theta} \end{bmatrix} =  \begin{bmatrix} a \\ b \end{bmatrix}.
\]
It can be defined in terms of the standard arctan as follows:
\[
{\rm atan}(b,a) := \begin{cases} {\rm atan}{b \over a} & a > 0 \\
                            {\rm atan}{b \over a} + \ensuremath{\pi} & a < 0\hbox{ and }b >0 \\
                            {\rm atan}{b \over a} - \ensuremath{\pi} & a < 0\hbox{ and }b < 0 \\
                            \ensuremath{\pi}/2 & a = 0\hbox{ and }b >0 \\
                            -\ensuremath{\pi}/2 & a = 0\hbox{ and }b < 0 
                            \end{cases}
\]
\end{definition}

We show $SO(2)$ are exactly equivalent to standard rotations:

\begin{proposition}[simple rotation] A 2\ensuremath{\times}2 \emph{rotation matrix} through angle $\ensuremath{\theta}$ is
\[
Q_\ensuremath{\theta} := \begin{bmatrix} \cos \ensuremath{\theta} & -\sin \ensuremath{\theta} \cr \sin \ensuremath{\theta} & \cos \ensuremath{\theta} \end{bmatrix}.
\]
We have $Q \ensuremath{\in} SO(2)$ if and only if $Q = Q_\ensuremath{\theta}$ for some $\ensuremath{\theta} \ensuremath{\in} \ensuremath{\bbR}$.

\end{proposition}
\textbf{Proof}

We will write $c = \cos \ensuremath{\theta}$ and $s = \sin \ensuremath{\theta}$. Then we have
\[
Q_\ensuremath{\theta}^\ensuremath{\top}Q_\ensuremath{\theta} = \begin{pmatrix} c & s \\ -s & c \end{pmatrix} \begin{pmatrix} c & -s \\ s & c \end{pmatrix} = 
\begin{pmatrix} c^2 + s^2 & 0 \\ 0 & c^2 + s^2 \end{pmatrix} = I
\]
and $\det Q_\ensuremath{\theta} = c^2 + s^2 = 1$ hence $Q_\ensuremath{\theta} \ensuremath{\in} SO(2)$. 

Now suppose $Q = [\ensuremath{\bm{\q}}_1, \ensuremath{\bm{\q}}_2] \ensuremath{\in} SO(2)$ where we know its columns have norm 1, i.e. $\|\ensuremath{\bm{\q}}_k\| = 1$, and are orthogonal. Write $\ensuremath{\bm{\q}}_1 = [c,s]$ where we know $c = \cos \ensuremath{\theta}$ and $s = \sin \ensuremath{\theta}$ for $\ensuremath{\theta} = {\rm atan}(s, c)$.  Since $\ensuremath{\bm{\q}}_1\cdot \ensuremath{\bm{\q}}_2 = 0$ we can deduce $\ensuremath{\bm{\q}}_2 = \ensuremath{\pm} [-s,c]$. The sign is positive as $\det Q = \ensuremath{\pm}(c^2 + s^2) = \ensuremath{\pm}1$.

\ensuremath{\QED}

We can rotate an arbitrary vector in $\ensuremath{\bbR}^2$ to the unit axis using rotations, which are useful in linear algebra decompositions. Interestingly it only requires basic algebraic functions (no trigonometric functions):

\begin{proposition}[rotation of a vector]  The matrix
\[
Q = {1 \over \sqrt{a^2 + b^2}}
\begin{bmatrix}
 a & b \cr -b & a
\end{bmatrix}
\]
is a rotation matrix ($Q \ensuremath{\in} SO(2)$) satisfying
\[
Q \begin{bmatrix} a \\ b \end{bmatrix} = \sqrt{a^2 + b^2} \begin{bmatrix} 1 \\ 0 \end{bmatrix}
\]
\end{proposition}
\textbf{Proof} 

The last equation is trivial so the only question is that it is a rotation matrix. This follows immediately:
\[
Q^\ensuremath{\top} Q = {1 \over a^2 + b^2}  \begin{bmatrix}
 a^2 + b^2 & 0 \cr 0 & a^2 + b^2
\end{bmatrix} = I
\]
and $\det Q = 1$.

\ensuremath{\QED}

\begin{example}[rotating a vector] Consider the vector
\[
\ensuremath{\bm{\x}} = \Vectt[-1,-\sqrt{3}].
\]
We can use the proposition above to deduce the rotation matrix that rotates this vector to the positive real axis is:
\[
{1 \over \sqrt{1+3}} \begin{bmatrix} -1 & -\sqrt{3} \\ \sqrt{3} & -1 \end{bmatrix} = 
{1 \over 2} \begin{bmatrix} -1 & -\sqrt{3} \\ \sqrt{3} & -1 \end{bmatrix}.
\]
Alternatively, we could determine the matrix by computing the angle of the vector via:
\[
\ensuremath{\theta} =  {\rm atan}(-\sqrt{3}, -1) = {\rm atan}(\sqrt{3}) - \ensuremath{\pi} = -{2\ensuremath{\pi} \over 3}.
\]
We thus compute:
\[
Q_{-\ensuremath{\theta}} = \begin{bmatrix}
\cos(2\ensuremath{\pi}/3) & -\sin(2\ensuremath{\pi}/3) \\
\sin(2\ensuremath{\pi}/3) & \cos(2\ensuremath{\pi}/3)
\end{bmatrix} = {1 \over 2} \begin{bmatrix} -1 & -\sqrt{3} \\ \sqrt{3} & -1 \end{bmatrix}.
\]
\end{example}

More generally, we can consider rotations that operate on two entries of a vector at a time. This will be explored in the problem sheet/lab.

\subsection{Reflections}
In addition to rotations, another type of orthogonal/unitary matrix are reflections. These are specified by a single vector which is reflected, with everything orthogonal to the vector left fixed. 

\begin{definition}[reflection matrix]  Given a unit vector $\ensuremath{\bm{\v}} \ensuremath{\in} \ensuremath{\bbC}^n$ (satisfying $\|\ensuremath{\bm{\v}}\|=1$), define the corresponding \emph{reflection matrix} as:
\[
Q_{\ensuremath{\bm{\v}}} := I - 2 \ensuremath{\bm{\v}} \ensuremath{\bm{\v}}^\ensuremath{\star}
\]
\end{definition}

These are indeed reflections in the direction of $\ensuremath{\bm{\v}}$. We can show this as follows:

\begin{proposition}[Householder properties] $Q_{\ensuremath{\bm{\v}}}$ satisfies:

\begin{itemize}
\item[1. ] Symmetry: $Q_{\ensuremath{\bm{\v}}} = Q_{\ensuremath{\bm{\v}}}^\ensuremath{\star}$


\item[2. ] Orthogonality: $Q_{\ensuremath{\bm{\v}}} \ensuremath{\in} U(n)$


\item[3. ] The vector $\ensuremath{\bm{\v}}$ is an eigenvector of $Q_{\ensuremath{\bm{\v}}}$ with eigenvalue $-1$


\item[4. ] For the dimension $n-1$ space $W := \{\ensuremath{\bm{\w}} : \ensuremath{\bm{\w}}^\ensuremath{\star} \ensuremath{\bm{\v}} = 0 \}$, all vectors $\ensuremath{\bm{\w}} \ensuremath{\in} W$ satisfy $Q_{\ensuremath{\bm{\v}}}\ensuremath{\bm{\w}} = \ensuremath{\bm{\w}}$.


\item[5. ] Not a rotation: $\det Q_{\ensuremath{\bm{\v}}} = -1$

\end{itemize}
\end{proposition}
\textbf{Proof}

Property 1 follows immediately. Property 2 follows from
\[
Q_{\ensuremath{\bm{\v}}}^\ensuremath{\star} Q_{\ensuremath{\bm{\v}}} = Q_{\ensuremath{\bm{\v}}}^2 = I - 4 \ensuremath{\bm{\v}} \ensuremath{\bm{\v}}^\ensuremath{\star} + 4 \ensuremath{\bm{\v}} \ensuremath{\bm{\v}}^\ensuremath{\star} \ensuremath{\bm{\v}} \ensuremath{\bm{\v}}^\ensuremath{\star} = I.
\]
Property 3 follows since
\[
Q_{\ensuremath{\bm{\v}}} \ensuremath{\bm{\v}} = \ensuremath{\bm{\v}} - 2\ensuremath{\bm{\v}} (\ensuremath{\bm{\v}}^\ensuremath{\star}\ensuremath{\bm{\v}}) = -\ensuremath{\bm{\v}}.
\]
Property 4 follows from:
\[
Q_{\ensuremath{\bm{\v}}} \ensuremath{\bm{\w}} = \ensuremath{\bm{\w}} - 2 \ensuremath{\bm{\v}} (\ensuremath{\bm{\w}}^\ensuremath{\star} \ensuremath{\bm{\v}}) =  \ensuremath{\bm{\w}}
\]
Property 5 then follows: Property 4 tells us that $1$ is an eigenvalue with multiplicity $n-1$. Since $-1$ is an eigenvalue with multiplicity 1,  the determinant, which is product of the eigenvalues, is $-1$.

\ensuremath{\QED}

\begin{example}[reflection through 2-vector] Consider reflection through $\ensuremath{\bm{\x}} = [1,2]^\ensuremath{\top}$.  We first need to normalise $\ensuremath{\bm{\x}}$:
\[
\ensuremath{\bm{\v}} = {\ensuremath{\bm{\x}} \over \|\ensuremath{\bm{\x}}\|} = \begin{bmatrix} {1 \over \sqrt{5}} \\ {2 \over \sqrt{5}} \end{bmatrix}
\]
The reflection matrix is:
\[
Q_{\ensuremath{\bm{\v}}} = I - 2 \ensuremath{\bm{\v}} \ensuremath{\bm{\v}}^\ensuremath{\top} = \begin{bmatrix}1 \\ & 1 \end{bmatrix} - {2 \over 5} \begin{bmatrix} 1 & 2 \\ 2 & 4 \end{bmatrix}
 =  {1 \over 5} \begin{bmatrix} 3 & -4 \\ -4 & -3 \end{bmatrix}
\]
Indeed it is symmetric, and orthogonal. It sends $\ensuremath{\bm{\x}}$ to $-\ensuremath{\bm{\x}}$:
\[
Q_{\ensuremath{\bm{\v}}} \ensuremath{\bm{\x}} = {1 \over 5} \begin{bmatrix}3 - 8 \\ -4 - 6 \end{bmatrix} = -\ensuremath{\bm{\x}}
\]
Any vector orthogonal to $\ensuremath{\bm{\x}}$, like $\ensuremath{\bm{\y}} = [-2,1]^\ensuremath{\top}$, is left fixed:
\[
Q_{\ensuremath{\bm{\v}}} \ensuremath{\bm{\y}} = {1 \over 5} \begin{bmatrix}-6 -4 \\ 8 - 3 \end{bmatrix} = \ensuremath{\bm{\y}}
\]
\end{example}

Note that \emph{building} the matrix $Q_{\ensuremath{\bm{\v}}}$ will be expensive ($O(n^2)$ operations), but we can \emph{apply} $Q_{\ensuremath{\bm{\v}}}$ to a vector in $O(n)$ operations using the expression:
\[
Q_{\ensuremath{\bm{\v}}} \ensuremath{\bm{\x}} = \ensuremath{\bm{\x}} - 2 \ensuremath{\bm{\v}} (\ensuremath{\bm{\v}}^\ensuremath{\star} \ensuremath{\bm{\x}}) = \ensuremath{\bm{\x}} - 2 \ensuremath{\bm{\v}} (\ensuremath{\bm{\v}} \ensuremath{\cdot} \ensuremath{\bm{\x}}).
\]
\subsubsection{Householder reflections}
Just as rotations can be used to rotate vectors to be aligned with coordinate axis, so can reflections, but in this case it works for vectors in $\ensuremath{\bbC}^n$, not just $\ensuremath{\bbR}^2$. We begin with the real case:

\begin{definition}[Householder reflection, real case] For a given vector $\ensuremath{\bm{\x}} \ensuremath{\in} \ensuremath{\bbR}^n$, define the Householder reflection
\[
Q_{\ensuremath{\bm{\x}}}^{\ensuremath{\pm},\rm H} := Q_{\ensuremath{\bm{\w}}}
\]
for $\ensuremath{\bm{\y}} = \ensuremath{\mp} \|\ensuremath{\bm{\x}}\| \ensuremath{\bm{\e}}_1 + \ensuremath{\bm{\x}}$ and $\ensuremath{\bm{\w}} = {\ensuremath{\bm{\y}} \over \|\ensuremath{\bm{\y}}\|}$. The default choice in sign is:
\[
Q_{\ensuremath{\bm{\x}}}^{\rm H} := Q_{\ensuremath{\bm{\x}}}^{-\hbox{sign}(x_1),\rm H}.
\]
\end{definition}

\begin{lemma}[Householder reflection maps to axis] For $\ensuremath{\bm{\x}} \ensuremath{\in} \ensuremath{\bbR}^n$,
\[
Q_{\ensuremath{\bm{\x}}}^{\ensuremath{\pm},\rm H} \ensuremath{\bm{\x}} = \ensuremath{\pm}\|\ensuremath{\bm{\x}}\| \ensuremath{\bm{\e}}_1
\]
\end{lemma}
\textbf{Proof} Note that
\begin{align*}
\| \ensuremath{\bm{\y}} \|^2 &= 2\|\ensuremath{\bm{\x}}\|^2 \ensuremath{\mp} 2 \|\ensuremath{\bm{\x}}\| x_1, \\
\ensuremath{\bm{\y}}^\ensuremath{\top} \ensuremath{\bm{\x}} &= \|\ensuremath{\bm{\x}}\|^2 \ensuremath{\mp}  \|\ensuremath{\bm{\x}}\| x_1
\end{align*}
where $x_1 = \ensuremath{\bm{\e}}_1^\ensuremath{\top} \ensuremath{\bm{\x}}$. Therefore:
\[
Q_{\ensuremath{\bm{\x}}}^{\ensuremath{\pm},\rm H} \ensuremath{\bm{\x}}  =  (I - 2 \ensuremath{\bm{\w}} \ensuremath{\bm{\w}}^\ensuremath{\top}) \ensuremath{\bm{\x}} = \ensuremath{\bm{\x}} - 2 {\ensuremath{\bm{\y}}  \|\ensuremath{\bm{\x}}\|  \over \|\ensuremath{\bm{\y}}\|^2} (\|\ensuremath{\bm{\x}}\|\ensuremath{\mp}x_1) = \ensuremath{\bm{\x}} - \ensuremath{\bm{\y}} =  \ensuremath{\pm}\|\ensuremath{\bm{\x}}\| \ensuremath{\bm{\e}}_1.
\]
\ensuremath{\QED}

\textbf{Remark} Why do we choose the the opposite sign of $x_1$ for the default reflection? For stability, but we won't discuss this in more detail.

We can extend this definition for complexes:

\begin{definition}[Householder reflection, complex case] For a given vector $\ensuremath{\bm{\x}} \ensuremath{\in} \ensuremath{\bbC}^n$, define the Householder reflection as
\[
Q_{\ensuremath{\bm{\x}}}^{\rm H} := Q_{\ensuremath{\bm{\w}}}
\]
for $\ensuremath{\bm{\y}} = {\rm csign}(x_1) \|\ensuremath{\bm{\x}}\| \ensuremath{\bm{\e}}_1 + \ensuremath{\bm{\x}}$ and $\ensuremath{\bm{\w}} = {\ensuremath{\bm{\y}} \over \|\ensuremath{\bm{\y}}\|}$, for ${\rm csign}(z) = {\rm e}^{{\rm i} \arg z}$.  \end{definition}

\begin{lemma}[Householder reflection maps to axis, complex case] For $\ensuremath{\bm{\x}} \ensuremath{\in} \ensuremath{\bbC}^n$,
\[
Q_{\ensuremath{\bm{\x}}}^{\rm H} \ensuremath{\bm{\x}} = -{\rm csign}(x_1) \|\ensuremath{\bm{\x}}\| \ensuremath{\bm{\e}}_1
\]
\end{lemma}
\textbf{Proof} Denote $\ensuremath{\alpha} := {\rm csign}(x_1)$.  Note that $\baralpha x_1 = {\rm e}^{-{\rm i} \arg x_1} x_1 = |x_1|$.  Now we have
\begin{align*}
\| \ensuremath{\bm{\y}} \|^2 &= (\ensuremath{\alpha} \|\ensuremath{\bm{\x}}\| \ensuremath{\bm{\e}}_1 + \ensuremath{\bm{\x}})^\ensuremath{\star}(\ensuremath{\alpha} \|\ensuremath{\bm{\x}}\| \ensuremath{\bm{\e}}_1 + \ensuremath{\bm{\x}}) = |\ensuremath{\alpha}|\| \ensuremath{\bm{\x}} \|^2 + \| \ensuremath{\bm{\x}} \|  \ensuremath{\alpha} \bar x_1 + \baralpha x_1 \| \ensuremath{\bm{\x}} \| + \| \ensuremath{\bm{\x}} \|^2 \\
&= 2\| \ensuremath{\bm{\x}} \|^2 + 2|x_1| \| \ensuremath{\bm{\x}} \| \\
\ensuremath{\bm{\y}}^\ensuremath{\star} \ensuremath{\bm{\x}} &= \baralpha x_1 \| \ensuremath{\bm{\x}} \| + \|\ensuremath{\bm{\x}} \|^2 = \|\ensuremath{\bm{\x}} \|^2 + |x_1| \| \ensuremath{\bm{\x}} \|
\end{align*}
Therefore:
\[
Q_{\ensuremath{\bm{\x}}}^{\rm H} \ensuremath{\bm{\x}}  =  (I - 2 \ensuremath{\bm{\w}} \ensuremath{\bm{\w}}^\ensuremath{\star}) \ensuremath{\bm{\x}} = \ensuremath{\bm{\x}} - 2 {\ensuremath{\bm{\y}}    \over \|\ensuremath{\bm{\y}}\|^2} (\|\ensuremath{\bm{\x}} \|^2 + |x_1| \|\ensuremath{\bm{\x}} \|) = \ensuremath{\bm{\x}} - \ensuremath{\bm{\y}} =  -\ensuremath{\alpha} \|\ensuremath{\bm{\x}}\| \ensuremath{\bm{\e}}_1.
\]
\ensuremath{\QED}





\section{QR Factorisation}
Let $A \ensuremath{\in} \ensuremath{\bbC}^{m \ensuremath{\times} n}$ be a rectangular or square matrix such that $m \ensuremath{\geq} n$ (i.e. more rows then columns). In this chapter we consider two closely related factorisations:

\begin{definition}[QR factorisation] The \emph{QR factorisation} is
\[
A = Q R = \underbrace{\begin{bmatrix} \ensuremath{\bm{\q}}_1 | \ensuremath{\cdots} | \ensuremath{\bm{\q}}_m \end{bmatrix}}_{Q \ensuremath{\in} U(m)} \underbrace{\begin{bmatrix} \ensuremath{\times} & \ensuremath{\cdots} & \ensuremath{\times} \\ & \ensuremath{\ddots} & \ensuremath{\vdots} \\ && \ensuremath{\times} \\ &&0 \\ &&\ensuremath{\vdots} \\ && 0 \end{bmatrix}}_{R \ensuremath{\in} \ensuremath{\bbC}^{m \ensuremath{\times} n}}
\]
where $Q$ is unitary (i.e., $Q \ensuremath{\in} U(m)$, satisfying $Q^\ensuremath{\star}Q = I$, with columns $\ensuremath{\bm{\q}}_j \ensuremath{\in} \ensuremath{\bbC}^m$) and $R$ is \emph{right triangular}, which means it  is only nonzero on or to the right of the diagonal ($r_{kj} = 0$ if $k > j$). \end{definition}

\begin{definition}[Reduced QR factorisation] The \emph{reduced QR factorisation}
\[
A = \hat Q \hat R = \underbrace{\begin{bmatrix} \ensuremath{\bm{\q}}_1 | \ensuremath{\cdots} | \ensuremath{\bm{\q}}_n \end{bmatrix}}_{ \hat Q \ensuremath{\in} \ensuremath{\bbC}^{m \ensuremath{\times} n}} \underbrace{\begin{bmatrix} \ensuremath{\times} & \ensuremath{\cdots} & \ensuremath{\times} \\ & \ensuremath{\ddots} & \ensuremath{\vdots} \\ && \ensuremath{\times}  \end{bmatrix}}_{\hat R \ensuremath{\in} \ensuremath{\bbC}^{n \ensuremath{\times} n}}
\]
where $\hat Q$ has orthonormal columns ($\hat Q^\ensuremath{\star} \hat Q = I$, $\ensuremath{\bm{\q}}_j \ensuremath{\in} \ensuremath{\bbC}^m$) and $\hat R$ is upper triangular. \end{definition}

Note for a square matrix the reduced QR factorisation is equivalent to the QR factorisation, in which case $R$ is \emph{upper triangular}. The importance of these factorisation for square matrices is that their component pieces are easy to invert:
\[
A = QR \qquad \ensuremath{\Rightarrow} \qquad A^{-1}\ensuremath{\bm{\b}} = R^{-1} Q^\ensuremath{\top} \ensuremath{\bm{\b}}
\]
and we saw previously that triangular and orthogonal matrices are easy to invert when applied to a vector $\ensuremath{\bm{\b}}$.

For rectangular matrices we will see that the QR factorisation leads to efficient solutions to the \emph{least squares problem}: find $\ensuremath{\bm{\x}}$ that minimizes the 2-norm $\| A \ensuremath{\bm{\x}} - \ensuremath{\bm{\b}} \|.$ Note in the rectangular case the QR factorisation contains within it the reduced QR factorisation:
\[
A = QR = \begin{bmatrix} \hat Q | \ensuremath{\bm{\q}}_{n+1} | \ensuremath{\cdots} | \ensuremath{\bm{\q}}_m \end{bmatrix} \begin{bmatrix} \hat R \\  \ensuremath{\bm{\zero}}_{m-n \ensuremath{\times} n} \end{bmatrix} = \hat Q \hat R.
\]
In this chapter we discuss the following:

\begin{itemize}
\item[1. ] Reduced QR and Gram\ensuremath{\endash}Schmidt: We discuss computation of the Reduced QR factorisation using Gram\ensuremath{\endash}Schmidt.


\item[2. ] Householder reflections and QR: We discuss computing the  QR factorisation using Householder reflections. This is a more accurate approach

\end{itemize}
for computing QR factorisations.

\begin{itemize}
\item[3. ] QR and least squares: We discuss the QR factorisation and its usage in solving least squares problems.

\end{itemize}
\subsection{Reduced QR and Gram\ensuremath{\endash}Schmidt}
How do we compute the QR factorisation? We begin with a method you may have seen before in another guise. Write
\[
A = \begin{bmatrix} \ensuremath{\bm{\a}}_1 | \ensuremath{\cdots} | \ensuremath{\bm{\a}}_n \end{bmatrix}
\]
where $\ensuremath{\bm{\a}}_k \ensuremath{\in}  \ensuremath{\bbC}^m$ and assume they are linearly independent ($A$ has full column rank).

\begin{proposition}[Column spaces match] Suppose $A = \hat Q  \hat R$ where $\hat Q = [\ensuremath{\bm{\q}}_1|\ensuremath{\ldots}|\ensuremath{\bm{\q}}_n]$ has orthonormal columns and $\hat R$ is upper-triangular, and $A$ has full rank. Then the first $j$ columns of $\hat Q$ span the same space as the first $j$ columns of $A$:
\[
\hbox{span}(\ensuremath{\bm{\a}}_1,\ensuremath{\ldots},\ensuremath{\bm{\a}}_j) = \hbox{span}(\ensuremath{\bm{\q}}_1,\ensuremath{\ldots},\ensuremath{\bm{\q}}_j).
\]
\end{proposition}
\textbf{Proof}

Because $A$ has full rank we know $\hat R$ is invertible, i.e. its diagonal entries do not vanish: $r_{jj} \ensuremath{\neq} 0$. If $\ensuremath{\bm{\v}} \ensuremath{\in} \hbox{span}(\ensuremath{\bm{\a}}_1,\ensuremath{\ldots},\ensuremath{\bm{\a}}_j)$ we have for $\ensuremath{\bm{\c}} \ensuremath{\in} \ensuremath{\bbC}^j$
\[
\ensuremath{\bm{\v}} = \begin{bmatrix} \ensuremath{\bm{\a}}_1 | \ensuremath{\cdots} | \ensuremath{\bm{\a}}_j \end{bmatrix} \ensuremath{\bm{\c}} = 
\begin{bmatrix} \ensuremath{\bm{\q}}_1 | \ensuremath{\cdots} | \ensuremath{\bm{\q}}_j \end{bmatrix}  \hat R[1:j,1:j] \ensuremath{\bm{\c}} \ensuremath{\in} \hbox{span}(\ensuremath{\bm{\q}}_1,\ensuremath{\ldots},\ensuremath{\bm{\q}}_j)
\]
while if $\ensuremath{\bm{\w}} \ensuremath{\in} \hbox{span}(\ensuremath{\bm{\q}}_1,\ensuremath{\ldots},\ensuremath{\bm{\q}}_j)$ we have for $\vc d \ensuremath{\in} \ensuremath{\bbR}^j$
\[
\ensuremath{\bm{\w}} = \begin{bmatrix} \ensuremath{\bm{\q}}_1 | \ensuremath{\cdots} | \ensuremath{\bm{\q}}_j \end{bmatrix} \vc d  =  \begin{bmatrix} \ensuremath{\bm{\a}}_1 | \ensuremath{\cdots} | \ensuremath{\bm{\a}}_j \end{bmatrix} \hat R[1:j,1:j]^{-1} \vc d \ensuremath{\in}  \hbox{span}(\ensuremath{\bm{\a}}_1,\ensuremath{\ldots},\ensuremath{\bm{\a}}_j).
\]
\ensuremath{\QED}

It is possible to find $\hat Q$ and $\hat R$ the  using the \emph{Gram\ensuremath{\endash}Schmidt algorithm}. We construct it column-by-column. For $j = 1, 2, \ensuremath{\ldots}, n$ define
\begin{align*}
\ensuremath{\bm{\v}}_j &:= \ensuremath{\bm{\a}}_j - \ensuremath{\sum}_{k=1}^{j-1} \underbrace{\ensuremath{\bm{\q}}_k^\ensuremath{\star} \ensuremath{\bm{\a}}_j}_{r_{kj}} \ensuremath{\bm{\q}}_k \\
r_{jj} &:= {\|\ensuremath{\bm{\v}}_j\|} \\
\ensuremath{\bm{\q}}_j &:= {\ensuremath{\bm{\v}}_j \over r_{jj}}
\end{align*}
\textbf{Theorem (Gram\ensuremath{\endash}Schmidt and reduced QR)} Define $\ensuremath{\bm{\q}}_j$ and $r_{kj}$ as above (with $r_{kj} = 0$ if $k > j$). Then a reduced QR factorisation is given by:
\[
A = \underbrace{\begin{bmatrix} \ensuremath{\bm{\q}}_1 | \ensuremath{\cdots} | \ensuremath{\bm{\q}}_n \end{bmatrix}}_{ \hat Q \ensuremath{\in} \ensuremath{\bbC}^{m \ensuremath{\times} n}} \underbrace{\begin{bmatrix} r_{11} & \ensuremath{\cdots} & r_{1n} \\ & \ensuremath{\ddots} & \ensuremath{\vdots} \\ && r_{nn}  \end{bmatrix}}_{\hat R \ensuremath{\in} \ensuremath{\bbC}^{n \ensuremath{\times} n}}
\]
\textbf{Proof}

We first show that $\hat Q$ has orthonormal columns. Assume that $\ensuremath{\bm{\q}}_\ensuremath{\ell}^\ensuremath{\star} \ensuremath{\bm{\q}}_k = \ensuremath{\delta}_{\ensuremath{\ell}k}$ for $k,\ensuremath{\ell} < j$.  For $\ensuremath{\ell} < j$ we then have
\[
\ensuremath{\bm{\q}}_\ensuremath{\ell}^\ensuremath{\star} \ensuremath{\bm{\v}}_j = \ensuremath{\bm{\q}}_\ensuremath{\ell}^\ensuremath{\star} \ensuremath{\bm{\a}}_j - \ensuremath{\sum}_{k=1}^{j-1}  \ensuremath{\bm{\q}}_\ensuremath{\ell}^\ensuremath{\star}\ensuremath{\bm{\q}}_k \ensuremath{\bm{\q}}_k^\ensuremath{\star} \ensuremath{\bm{\a}}_j = 0
\]
hence $\ensuremath{\bm{\q}}_\ensuremath{\ell}^\ensuremath{\star} \ensuremath{\bm{\q}}_j = 0$ and indeed $\hat Q$ has orthonormal columns. Further: from the definition of $\ensuremath{\bm{\v}}_j$ we find
\[
\ensuremath{\bm{\a}}_j = \ensuremath{\bm{\v}}_j + \ensuremath{\sum}_{k=1}^{j-1} r_{kj} \ensuremath{\bm{\q}}_k = \ensuremath{\sum}_{k=1}^j r_{kj} \ensuremath{\bm{\q}}_k  = \hat Q \hat R \ensuremath{\bm{\e}}_j
\]
\ensuremath{\QED}

\subsection{Householder reflections and QR}
As an alternative, we will consider using Householder reflections to introduce zeros below the diagonal. Thus, if Gram\ensuremath{\endash}Schmidt is a process of \emph{triangular orthogonalisation} (using triangular matrices to orthogonalise), Householder reflections is a process of \emph{orthogonal triangularisation}  (using orthogonal matrices to triangularise).

Consider multiplication by the Householder reflection corresponding to the first column, that is, for
\[
Q_1 := Q_{\ensuremath{\bm{\a}}_1}^{\rm H},
\]
consider
\[
Q_1 A = \begin{bmatrix} \ensuremath{\times} & \ensuremath{\times} & \ensuremath{\cdots} & \ensuremath{\times} \\
& \ensuremath{\times} & \ensuremath{\cdots} & \ensuremath{\times} \\
                    & \ensuremath{\vdots} & \ensuremath{\ddots} & \ensuremath{\vdots} \\
                    & \ensuremath{\times} & \ensuremath{\cdots} & \ensuremath{\times} \end{bmatrix} = 
\begin{bmatrix}  \ensuremath{\alpha}_1 & \ensuremath{\bm{\w}}_1^\ensuremath{\top} \\ 
& A_2   \end{bmatrix}
\]
where 
\[
\ensuremath{\alpha}_1 := -{\rm csign}(a_{11})  \|\ensuremath{\bm{\a}}_1\|, \ensuremath{\bm{\w}}_1 = (Q_1 A)[1, 2:n]  \qquad \hbox{and} \qquad A_2 = (Q_1 A)[2:m, 2:n],
\]
where as before ${\rm csign}(z) :=  {\rm e}^{{\rm i} \arg z}$. That is, we have made the first column triangular. In terms of an algorithm, we then introduce zeros into the first column of $A_2$, leaving an $A_3$, and so-on. But we can wrap this iterative algorithm into a simple proof by induction, reminisicent of our proof for the Cholesky factorisation:

\begin{theorem}[QR]  Every matrix $A \ensuremath{\in} \ensuremath{\bbC}^{m \ensuremath{\times} n}$ has a QR factorisation:
\[
A = QR
\]
where $Q \ensuremath{\in} U(m)$ and $R \ensuremath{\in} \ensuremath{\bbC}^{m \ensuremath{\times} n}$ is right triangular.

\end{theorem}
\textbf{Proof}

First assume $m \ensuremath{\geq} n$. If $A = [\ensuremath{\bm{\a}}_1] \ensuremath{\in} \ensuremath{\bbC}^{m \ensuremath{\times} 1}$ then we have for the Householder reflection $Q_1 = Q_{\ensuremath{\bm{\a}}_1}^{\rm H}$
\[
Q_1 A = \ensuremath{\alpha} \ensuremath{\bm{\e}}_1
\]
which is right triangular, where $\ensuremath{\alpha} = -{\rm csign}(a_{11}) \|\ensuremath{\bm{\a}}_1\|$.  In other words 
\[
A = \underbrace{Q_1}_Q \underbrace{\ensuremath{\alpha} \ensuremath{\bm{\e}}_1}_R.
\]
For $n > 1$, assume every matrix with less columns than $n$ has a QR factorisation. For $A = [\ensuremath{\bm{\a}}_1|\ensuremath{\ldots}|\ensuremath{\bm{\a}}_n] \ensuremath{\in} \ensuremath{\bbC}^{m \ensuremath{\times} n}$, let $Q_1 = Q_{\ensuremath{\bm{\a}}_1}^{\rm H}$ so that
\[
Q_1 A =  \begin{bmatrix} \ensuremath{\alpha} & \ensuremath{\bm{\w}}^\ensuremath{\top} \\ & A_2 \end{bmatrix}.
\]
By assumption $A_2 = Q_2 R_2$. Thus we have (recalling that $Q_1^{-1} = Q_1^\ensuremath{\star} = Q_1$):
\begin{align*}
A = Q_1 \begin{bmatrix} \ensuremath{\alpha} & \ensuremath{\bm{\w}}^\ensuremath{\top} \\ & Q_2 R_2 \end{bmatrix} \\
=\underbrace{Q_1 \begin{bmatrix} 1 \\ & Q_2 \end{bmatrix}}_Q  \underbrace{\begin{bmatrix} \ensuremath{\alpha} & \ensuremath{\bm{\w}}^\ensuremath{\top} \\ &  R_2 \end{bmatrix}}_R.
\end{align*}
If $m < n$, i.e., $A$ has more columns then rows, write 
\[
A = \begin{bmatrix} \At & B \end{bmatrix}
\]
where $\At \ensuremath{\in} \ensuremath{\bbC}^{m \ensuremath{\times} m}$. From above we know we can write $\At = Q \Rt$. We thus have
\[
A = Q \underbrace{\begin{bmatrix} \Rt & Q^\ensuremath{\star} B \end{bmatrix}}_R
\]
where $R$ is right triangular.

\ensuremath{\QED}

\begin{example}[QR by hand, non-examinable] We will now do an example by hand. Consider the $4 \ensuremath{\times} 3$ matrix
\[
A = \begin{bmatrix} 
2 & 3 & 0 \\ 
0 & 0 & 1 \\
-2 & -3 & 0 \\
-1 & -3 & -3
\end{bmatrix}
\]
For the first column we have
\[
\ensuremath{\bm{\y}}_1 := [-1,0,-2,-1]
\]
where $\| \ensuremath{\bm{\y}}_1 \|^2 = 6$. Hence
\[
Q_1 := I - {1 \over 3} \begin{bmatrix} -1 \\ 0 \\ -2 \\ -1 \end{bmatrix} \begin{bmatrix} -1 & 0 & -2 & -1 \end{bmatrix} =
 {1 \over 3} \begin{bmatrix}
2 & 0 & -2 & -1 \\
0 & 3 & 0 & 0 \\
-2 & 0 & -1 & -2 \\
-1 & 0 & -2 &  2
\end{bmatrix}
\]
so that
\[
Q_1 A = \begin{bmatrix} 3 &  5 & 1 \\
 & 0 & 1 \\
  & 1 & 2 \\
& -1 & -2
\end{bmatrix}
\]
For the second column we have
\[
\ensuremath{\bm{\y}}_2 :=  [-\sqrt{2},1,-1]
\]
where $\| \ensuremath{\bm{\y}}_2 \|^2 = 4$. Thus we have
\[
Q_2 := I - {1 \over 2}
 \begin{bmatrix} -\sqrt{2} \\1 \\ -1
\end{bmatrix} \begin{bmatrix} -\sqrt{2} & 1 & -1 \end{bmatrix}
= \begin{bmatrix}
0 & 1/\sqrt{2} & -1/\sqrt{2} \\
1/\sqrt{2} & 1/2 & 1/2 \\
-1/\sqrt{2} & 1/2 & 1/2
\end{bmatrix}
\]
so that
\[
\tilde Q_2 Q_1 A = \begin{bmatrix} 3 & 5 & 1 \\
 & \sqrt{2} & 2\sqrt{2} \\
  & 0 & 1/\sqrt{2} \\
& 0 & -1/\sqrt{2}
\end{bmatrix}
\]
The final vector is 
\[
\ensuremath{\bm{\y}}_3 := [1/\sqrt{2}-1,-1/\sqrt{2}]
\]
where $\| \ensuremath{\bm{\y}}_3 \|^2 = 2 - 2/\sqrt{2}$. Hence
\[
Q_3 := I - {\sqrt{2} \over \sqrt{2} - 1} \begin{bmatrix}
1/\sqrt{2}-1 \\
-1/\sqrt{2}
\end{bmatrix} \begin{bmatrix}
1/\sqrt{2}-1 &
-1/\sqrt{2}
\end{bmatrix} =
\begin{bmatrix}
\sqrt{2} & -\sqrt{2}\\
-\sqrt{2} & -\sqrt{2}
\end{bmatrix}
\]
so that 
\[
\tilde Q_3 \tilde Q_2 Q_1 A = \begin{bmatrix} 3 & 5 & 1 \\
 & \sqrt{2} & 2\sqrt{2} \\
  & 0 & 1 \\
& 0 & 0
\end{bmatrix} =: R
\]
and
\[
Q := Q_1 \tilde Q_2 \tilde Q_3 =  \begin{bmatrix}
2/3 & -1/(3\sqrt{2}) & 0 & 1/\sqrt{2} \\
0 &  0 & 1 & 0 \\
-2/3 & 1/(3\sqrt{2}) & 0 & 1/\sqrt{2} \\ 
-1/3 & - 4/(3\sqrt{2}) & 0 & 0
\end{bmatrix}.
\]
\subsection{QR and least squares}
We consider rectangular matrices with more rows than columns. Given $A \ensuremath{\in} \ensuremath{\bbC}^{m \ensuremath{\times} n}$ and $\ensuremath{\bm{\b}} \ensuremath{\in} \ensuremath{\bbC}^m$, least squares consists of finding a vector $\ensuremath{\bm{\x}} \ensuremath{\in} \ensuremath{\bbC}^n$ that minimises the 2-norm: $\| A \ensuremath{\bm{\x}} - \ensuremath{\bm{\b}} \|$.

\begin{theorem}[least squares via QR] Suppose $A \ensuremath{\in} \ensuremath{\bbC}^{m \ensuremath{\times} n}$ has full rank. Given a reduced QR factorisation $A = Q R$ then
\[
\ensuremath{\bm{\x}} = \hat R^{-1} \hat Q^\ensuremath{\star} \ensuremath{\bm{\b}}
\]
minimises $\| A \ensuremath{\bm{\x}} - \ensuremath{\bm{\b}} \|$. 

\end{theorem}
\textbf{Proof}

The norm-preserving property ($\|Q\ensuremath{\bm{\x}}\| = \|\ensuremath{\bm{\x}}\|$) of unitary matrices tells us
\[
\| A \ensuremath{\bm{\x}} - \ensuremath{\bm{\b}} \| = \| Q R \ensuremath{\bm{\x}} - \ensuremath{\bm{\b}} \| = \| Q (R \ensuremath{\bm{\x}} - Q^\ensuremath{\star} \ensuremath{\bm{\b}}) \| = \| R \ensuremath{\bm{\x}} - Q^\ensuremath{\star} \ensuremath{\bm{\b}} \| = \left \| 
\begin{bmatrix} \hat R \\ \ensuremath{\bm{\zero}}_{m-n \ensuremath{\times} n} \end{bmatrix} \ensuremath{\bm{\x}} - \begin{bmatrix} \hat Q^\ensuremath{\star} \\ \ensuremath{\bm{\q}}_{n+1}^\ensuremath{\star} \\ \ensuremath{\vdots} \\ \ensuremath{\bm{\q}}_m^\ensuremath{\star} \end{bmatrix}     \ensuremath{\bm{\b}} \right \|
\]
Now note that the rows $k > n$ are independent of $\ensuremath{\bm{\x}}$ and are a fixed contribution. Thus to minimise this norm it suffices to drop them and minimise:
\[
\| \hat R \ensuremath{\bm{\x}} - \hat Q^\ensuremath{\star} \ensuremath{\bm{\b}} \|
\]
This norm is minimised if it is zero. Provided the column rank of $A$ is full, $\hat R$ will be invertible.

\end{example}






\chapter{Approximation Theory}

So far, we have seen intuitive numerical methods for computing derivatives, integrals, and solving
differential equations, primarily based on representing functions by their values at a grid of points.
But by using more sophisticated mathematical tools, we can achieve much more accurate and reliable
numerical methods. In particular, we can effectively use Fourier series for computing very accurately with periodic functions,
and orthogonal polynomials for non-periodic functions that are smooth within an interval.
Here we introduce these fundamental tools and explore applications to quadrature (computing integrals) where they
produce incredibly accurate approximations, ones that converge exponentially (or faster) for analytic functions.

\begin{enumerate}
    \item IV.1 Fourier Expansions: we discuss Fourier series and their usage in approximating periodic functions, using the Trapezium rule to compute the Fourier coefficients.
    \item IV.2 Discrete Fourier Transform: The Trapezium rule approximation can be recast as a unitary matrix, known as the Discrete Fourier Transform (DFT). This is used to prove interpolation properties.
    \item IV.3 Orthogonal Polynomials: For non-periodic functions we consider orthogonal polynomials, and discuss their basic properties.
    \item IV.4 Classical Orthogonal Polynomials: For certain weights, orthogonal polynomials are classical and have addition structure that are useful for computations.
    \item IV.5 Gaussian Quadrature: Finally, we revisit the  problem of computing integrals, and see that using orthogonal polynomials we can derive much more accurate methods.
    \end{enumerate}

We stop at integration, but Fourier and orthogonal polynomial expansions also lead to very effective scheme for solving differential equations
and many other applications.


\section{Fourier expansions}
Fourier series are a powerful tool in wide areas of mathematics, including solving partial differential equations, signal processing, and elsewhere. They are also very useful in computational methods, particularly for problems that have periodicity. Periodicity arises naturally when solving problems in radial coordinates, or when approximating a problem on the real line by a periodic problem with a large problem. Fourier series are also related to orthogonal polynomials, which can be used for non-periodic problems.

\subsection{Basics of Fourier series}
The most fundamental basis is (complex) Fourier: we have ${\rm e}^{{\rm i} k \ensuremath{\theta}}$ are orthogonal with respect to the inner product
\[
\ensuremath{\langle}f, g \ensuremath{\rangle} := {1 \over 2\ensuremath{\pi}} \ensuremath{\int}_0^{2\ensuremath{\pi}} \bar f(\ensuremath{\theta}) g(\ensuremath{\theta}) {\rm d}\ensuremath{\theta},
\]
where we conjugate the first argument to be consistent with the vector inner product $\ensuremath{\bm{\x}}^\ensuremath{\star} \ensuremath{\bm{\y}}$. We will use the notation $\ensuremath{\bbT} := [0,2\ensuremath{\pi})$ (typically this has the topology of a circle attached but we do not need to worry about that here). We can (typically) expand functions in this basis:

\begin{definition}[Fourier] A function $f$ has a Fourier expansion if
\[
f(\ensuremath{\theta}) = \ensuremath{\sum}_{k = -\ensuremath{\infty}}^\ensuremath{\infty} \hat f_k {\rm e}^{{\rm i} k \ensuremath{\theta}}
\]
where
\[
\hat f_k := \ensuremath{\langle}{\rm e}^{{\rm i} k \ensuremath{\theta}}, f\ensuremath{\rangle} = {1 \over 2\ensuremath{\pi}} \ensuremath{\int}_0^{2\ensuremath{\pi}}  {\rm e}^{-{\rm i} k \ensuremath{\theta}} f(\ensuremath{\theta}) {\rm d}\ensuremath{\theta}
\]
\end{definition}

A basic observation is if a Fourier expansion has no negative terms it is equivalent to a Taylor series if we write $z = {\rm e}^{{\rm i} \ensuremath{\theta}}$:

\begin{definition}[Fourier-Taylor] A function $f$ has a Fourier\ensuremath{\endash}Taylor expansion if
\[
f(\ensuremath{\theta}) = \ensuremath{\sum}_{k = 0}^\ensuremath{\infty} \hat f_k {\rm e}^{{\rm i} k \ensuremath{\theta}}
\]
where $\hat f_k := \ensuremath{\langle}{\rm e}^{{\rm i} k \ensuremath{\theta}}, f\ensuremath{\rangle}$. \end{definition}

In numerical analysis we try to build on the analogy with linear algebra as much as possible. Therefore we  can write this this as:
\[
f(\ensuremath{\theta}) = \underbrace{[1 | {\rm e}^{{\rm i}\ensuremath{\theta}} | {\rm e}^{2{\rm i}\ensuremath{\theta}} | \ensuremath{\cdots}]}_{T(\ensuremath{\theta})}
\underbrace{\begin{bmatrix} \hat f_0 \\ \hat f_1 \\ \hat f_2 \\ \ensuremath{\vdots} \end{bmatrix}}_{\vchatf}.
\]
Essentially, expansions in bases are viewed as a way of turning \emph{functions} into (infinite) \emph{vectors}. And (differential) \emph{operators} into \emph{matrices}.

In analysis one typically works with continuous functions and relates results to continuity. In numerical analysis we inheritely have to work with \emph{vectors}, so it is more natural to  focus on the case where the \emph{Fourier coefficients} $\hat f_k$ are \emph{absolutely convergent}:

\begin{definition}[absolute convergent] We write $\vchatf \ensuremath{\in} \ensuremath{\ell}^1$ if it is absolutely convergent, or in otherwords, the $1$-norm of $\vchatf$ is bounded:
\[
\|\vchatf\|_1 := \ensuremath{\sum}_{k=-\ensuremath{\infty}}^\ensuremath{\infty} |\hat f_k| < \ensuremath{\infty}
\]
\end{definition}

We first state a  basic results (whose proof is beyond the scope of this module):

\begin{theorem}[2-norm convergence] If $f, g : \ensuremath{\bbT} \ensuremath{\rightarrow} \ensuremath{\bbC}$ are continuous and $\hat f_k = \hat g_k$ for all $k \ensuremath{\in} \ensuremath{\bbZ}$ then $f = g$.

\end{theorem}
\textbf{Proof} See \href{https://www.cambridge.org/core/books/fourier-analysis/5FD8F0FD69DDB139019655D7F8440D2F}{Körner 2022 (Theorem 2.4)}. \ensuremath{\QED}

This allows us to prove the following:

\begin{theorem}[Absolute convergence] If $\vchatf \ensuremath{\in} \ensuremath{\ell}^1$ then
\[
f(\ensuremath{\theta}) = \ensuremath{\sum}_{k = -\ensuremath{\infty}}^\ensuremath{\infty} \hat f_k {\rm e}^{{\rm i} k \ensuremath{\theta}},
\]
which converges uniformly. \end{theorem}
\textbf{Proof}

Note that
\[
g_n(\ensuremath{\theta}) := \ensuremath{\sum}_{k = -n}^n \hat f_k {\rm e}^{{\rm i} k \ensuremath{\theta}}
\]
is uniformly-absolutely convergent as $n \ensuremath{\rightarrow} \ensuremath{\infty}$, that is,
\[
\ensuremath{\sum}_{k = -n}^n |\hat f_k {\rm e}^{{\rm i} k \ensuremath{\theta}}| = \ensuremath{\sum}_{k = -n}^n |\hat f_k| \ensuremath{\rightarrow} \|\vchatf\|_1.
\]
This guarantees that $g_n(\ensuremath{\theta})$ converges uniformly to a continuous function $g(\ensuremath{\theta})$. We have for $n > k$, that the $k$-th Fourier coefficient of $g_n(\ensuremath{\theta})$ equals $\hat f_k$. Thus, by the properties of uniform convergence,
\[
\hat f_k = \lim_{n \ensuremath{\rightarrow} \ensuremath{\infty}} \hat f_k =  \lim_{n \ensuremath{\rightarrow} \ensuremath{\infty}} {1 \over 2\ensuremath{\pi}} \ensuremath{\int}_0^{2\ensuremath{\pi}}  {\rm e}^{-{\rm i} k \ensuremath{\theta}} g_n(\ensuremath{\theta}) {\rm d}\ensuremath{\theta} =
 {1 \over 2\ensuremath{\pi}} \ensuremath{\int}_0^{2\ensuremath{\pi}}  {\rm e}^{-{\rm i} k \ensuremath{\theta}} \lim_{n \ensuremath{\rightarrow} \ensuremath{\infty}} g_n(\ensuremath{\theta}) {\rm d}\ensuremath{\theta} = \hat g_k.
\]
Since $f$ and $g$ are continuous and share the same Fourier coefficients, they are equal.

\ensuremath{\QED}

When does a function have absolutely convergent Fourier coefficients? We can deduce it from periodic differentiability of the function:

\begin{lemma}[differentiability and absolutely convergence] If $f : \ensuremath{\bbR} \ensuremath{\rightarrow} \ensuremath{\bbC}$ and $f'$ are periodic  and $f''$ is uniformly bounded, then $\vchatf \ensuremath{\in} \ensuremath{\ell}^1$.

\end{lemma}
\textbf{Proof} Integrate by parts twice using the fact that $f(0) = f(2\ensuremath{\pi})$, $f'(0) = f'(2\ensuremath{\pi})$:
\begin{align*}
2\ensuremath{\pi}\hat f_k &= \ensuremath{\int}_0^{2\ensuremath{\pi}} f(\ensuremath{\theta}) {\rm e}^{-{\rm i} k \ensuremath{\theta}} {\rm d}\ensuremath{\theta} =
[f(\ensuremath{\theta}) {\rm e}^{-{\rm i} k \ensuremath{\theta}}]_0^{2\ensuremath{\pi}} + {1 \over {\rm i} k} \ensuremath{\int}_0^{2\ensuremath{\pi}} f'(\ensuremath{\theta}) {\rm e}^{-{\rm i} k \ensuremath{\theta}} {\rm d}\ensuremath{\theta} \\
&= {1 \over {\rm i} k} [f'(\ensuremath{\theta}) {\rm e}^{-{\rm i} k \ensuremath{\theta}}]_0^{2\ensuremath{\pi}} - {1 \over k^2} \ensuremath{\int}_0^{2\ensuremath{\pi}} f''(\ensuremath{\theta}) {\rm e}^{-{\rm i} k \ensuremath{\theta}} {\rm d}\ensuremath{\theta} \\
&= - {1 \over k^2} \ensuremath{\int}_0^{2\ensuremath{\pi}} f''(\ensuremath{\theta}) {\rm e}^{-{\rm i} k \ensuremath{\theta}} {\rm d}\ensuremath{\theta}
\end{align*}
thus uniform boundedness of $f''$ guarantees $|\hat f_k| \ensuremath{\leq} M |k|^{-2}$ for some $M$, and we have
\[
\ensuremath{\sum}_{k = -\ensuremath{\infty}}^\ensuremath{\infty} |\hat f_k| \ensuremath{\leq} |\hat f_0|  + 2M \ensuremath{\sum}_{k = 1}^\ensuremath{\infty} |k|^{-2}  < \ensuremath{\infty}
\]
using the dominant convergence test.

\ensuremath{\QED}

This condition can be weakened to Lipschitz continuity but the proof is  beyond the scope of this module. Of more practical importance is the other direction: the more times differentiable a function the faster the coefficients decay, and thence the faster Fourier expansions converge. In fact, if a function is smooth and 2\ensuremath{\pi}-periodic its Fourier coefficients decay faster than algebraically: they decay like $O(k^{-\ensuremath{\lambda}})$ for any $\ensuremath{\lambda}$. This will be explored in the problem sheet.

\textbf{Remark} Going further, if we let $z = {\rm e}^{{\rm i} \ensuremath{\theta}}$ then if $f(z)$ is \emph{analytic} in a neighbourhood of the unit circle the Fourier coefficients decay \emph{exponentially fast}. And if $f(z)$ is entire they decay even faster than exponentially.

\subsection{Trapezium rule and discrete Fourier coefficients}
\begin{definition}[Trapezium Rule] Let $\ensuremath{\theta}_j = 2\ensuremath{\pi}j/n$ for $j = 0,1,\ensuremath{\ldots},n$ denote $n+1$ evenly spaced points over $[0,2\ensuremath{\pi}]$. Recall that the \emph{Trapezium rule} over $[0,2\ensuremath{\pi}]$ is the approximation:
\[
\ensuremath{\int}_0^{2\ensuremath{\pi}} f(\ensuremath{\theta}) {\rm d}\ensuremath{\theta} \ensuremath{\approx} {2 \ensuremath{\pi} \over n} \left[{f(0) \over 2} + \ensuremath{\sum}_{j=1}^{n-1} f(\ensuremath{\theta}_j) + {f(2 \ensuremath{\pi}) \over 2} \right]
\]
But if $f$ is periodic we have $f(0) = f(2\ensuremath{\pi})$ and we get the \emph{periodic Trapezium rule}:
\[
{1 \over 2\ensuremath{\pi}} \ensuremath{\int}_0^{2\ensuremath{\pi}} f(\ensuremath{\theta}) {\rm d}\ensuremath{\theta} \ensuremath{\approx} \underbrace{{1 \over n} \ensuremath{\sum}_{j=0}^{n-1} f(\ensuremath{\theta}_j)}_{\ensuremath{\Sigma}_n[f]}
\]
\end{definition}

We know that ${\rm e}^{{\rm i} k \ensuremath{\theta}}$ are orthogonal with respect to the continuous inner product. The following says that this property is maintained (up to \ensuremath{\ldq}aliasing") when we replace the continuous integral with a trapezium rule approximation:

\begin{lemma}[Discrete orthogonality] We have:
\[
\ensuremath{\sum}_{j=0}^{n-1} {\rm e}^{{\rm i} k \ensuremath{\theta}_j} =
\begin{cases} n & k = \ldots,-2n,-n,0,n,2n,\ldots  \cr
              0 & \hbox{otherwise}
\end{cases}
\]
In other words,
\[
\ensuremath{\Sigma}_n[{\rm e}^{{\rm i} (k-\ensuremath{\ell}) \ensuremath{\theta}}] =
\begin{cases} 1 & k-\ensuremath{\ell} = \ldots,-2n,-n,0,n,2n,\ldots  \cr
              0 & \hbox{otherwise}
\end{cases}.
\]
\end{lemma}
\textbf{Proof}

Consider $\ensuremath{\omega} := {\rm e}^{{\rm i} \ensuremath{\theta}_1} = {\rm e}^{2 \ensuremath{\pi} {\rm i} \over n}$. This is an $n$ th root of unity: $\ensuremath{\omega}^n = 1$. Note that ${\rm e}^{{\rm i} \ensuremath{\theta}_j} ={\rm e}^{2 \ensuremath{\pi} {\rm i} j \over n}= \ensuremath{\omega}^j$.

(Case 1: $k = pn$ for an integer $p$) We have
\[
\ensuremath{\sum}_{j=0}^{n-1} {\rm e}^{{\rm i} k \ensuremath{\theta}_j} = \ensuremath{\sum}_{j=0}^{n-1} \ensuremath{\omega}^{kj} = \ensuremath{\sum}_{j=0}^{n-1} ({\ensuremath{\omega}^{pn}})^j =   \ensuremath{\sum}_{j=0}^{n-1} 1 = n
\]
(Case 2 $k \ensuremath{\neq} pn$ for an integer $p$)  Recall that
\[
\ensuremath{\sum}_{j=0}^{n-1} z^j = {z^n-1 \over z-1}.
\]
Then we have
\[
\ensuremath{\sum}_{j=0}^{n-1} {\rm e}^{{\rm i} k \ensuremath{\theta}_j} = \ensuremath{\sum}_{j=0}^{n-1} (\ensuremath{\omega}^k)^j = {\ensuremath{\omega}^{kn} -1 \over \ensuremath{\omega}^k -1} = 0.
\]
where we use the fact that $k$ is not a multiple of $n$ to guarantee that $\ensuremath{\omega}^k \ensuremath{\neq} 1$.

\ensuremath{\QED}

\subsection{Convergence of Approximate Fourier expansions}
We will now use the Trapezium rule to approximate Fourier coefficients and expansions:

\begin{definition}[Discrete Fourier coefficients] Define the Trapezium rule approximation to the Fourier coefficients by:
\[
\hat f_k^n := \ensuremath{\Sigma}_n[{\rm e}^{-i k \ensuremath{\theta}} f(\ensuremath{\theta})]  = {1 \over n} \ensuremath{\sum}_{j=0}^{n-1} {\rm e}^{-i k \ensuremath{\theta}_j} f(\ensuremath{\theta}_j)
\]
\end{definition}

A remarkable fact is that the discete Fourier coefficients can be expressed as a sum of the true Fourier coefficients:

\begin{theorem}[discrete Fourier coefficients] If $\vchatf \ensuremath{\in} \ensuremath{\ell}^1$ (absolutely convergent Fourier coefficients) then
\[
\hat f_k^n = \ensuremath{\cdots} + \hat f_{k-2n} + \hat f_{k-n} + \hat f_k + \hat f_{k+n} + \hat f_{k+2n} + \ensuremath{\cdots}
\]
\end{theorem}
\textbf{Proof}
\begin{align*}
\hat f_k^n &= \ensuremath{\Sigma}_n[f(\ensuremath{\theta}) {\rm e}^{-{\rm i} k \ensuremath{\theta}}] = \ensuremath{\sum}_{\ensuremath{\ell}=-\ensuremath{\infty}}^\ensuremath{\infty} \hat f_\ensuremath{\ell} \ensuremath{\Sigma}_n[{\rm e}^{{\rm i} (\ensuremath{\ell}-k) \ensuremath{\theta}}] \\
&= \ensuremath{\sum}_{\ensuremath{\ell}=-\ensuremath{\infty}}^\ensuremath{\infty} \hat f_\ensuremath{\ell} \begin{cases} 1 & \ensuremath{\ell}-k = \ldots,-2n,-n,0,n,2n,\ldots  \cr
0 & \hbox{otherwise}
\end{cases}
\end{align*}
\ensuremath{\QED}

Note that there is redundancy:

\begin{corollary}[aliasing] For all $p \ensuremath{\in} \ensuremath{\bbZ}$, $\hat f_k^n = \hat f_{k+pn}^n$.

\end{corollary}
\textbf{Proof} Follows immediately:
\[
\hat f_{k+pn}^n = \sum_{j=-\ensuremath{\infty}}^\ensuremath{\infty} \hat f_{k+(p+j)n}= \sum_{j=-\ensuremath{\infty}}^\ensuremath{\infty} \hat f_{k+j n} = \hat f_k^n.
\]
\ensuremath{\QED}

In other words if we know $\hat f_0^n, \ensuremath{\ldots}, \hat f_{n-1}^n$, we know $\hat f_k^n$ for all $k$ via a permutation, for example if $n = 2m+1$ we have
\[
\begin{bmatrix}
\hat f_{-m}^n \\
\ensuremath{\vdots}\\
\hat f_m^n
\end{bmatrix} = \underbrace{\begin{bmatrix} &&& 1 \\ &&&& \ensuremath{\ddots} \\ &&&&& 1 \\
    1 \\ & \ensuremath{\ddots} \\ && 1 \end{bmatrix}}_{P_\ensuremath{\sigma}}
\begin{bmatrix}
\hat f_0^n \\
\ensuremath{\vdots}\\
\hat f_{n-1}^n
\end{bmatrix}
\]
where $\ensuremath{\sigma}$ has Cauchy notation (\emph{Careful}: we are using 1-based indexing here):
\[
\begin{pmatrix}
1 & 2 & \ensuremath{\cdots} & m & m+1 & m+2 & \ensuremath{\cdots} & n  \\
m+2 & m+3 & \ensuremath{\cdots} & n & 1 & 2 & \ensuremath{\cdots} & m+1
\end{pmatrix}.
\]
\begin{example}[Taylor coefficients via Geometric series] Consider the function
\[
f(\ensuremath{\theta}) = {2 \over 2 - {\rm e}^{{\rm i} \ensuremath{\theta}}}
\]
Under the change of variables $z = {\rm e}^{{\rm i} \ensuremath{\theta}}$ we know for $z$ on the unit circle this becomes (using the geometric series with $z/2$)
\[
{2 \over 2-z} = \ensuremath{\sum}_{k=0}^\ensuremath{\infty} {z^k \over 2^k}
\]
i.e., $\hat f_k = 1/2^k$ which is absolutely summable:
\[
\ensuremath{\sum}_{k=0}^\ensuremath{\infty} |\hat f_k| = f(0) = 2.
\]
If we use an $n$ point discretisation we get (using the geoemtric series with $2^{-n}$)
\[
\hat f_k^n = \hat f_k + \hat f_{k+n} + \hat f_{k+n} + \ensuremath{\cdots} = \ensuremath{\sum}_{p=0}^\ensuremath{\infty} {1 \over 2^{k+pn}} = {2^{n-k} \over 2^n - 1}
\]
Note that as $n \rightarrow \ensuremath{\infty}$, we have $\hat f_k^n \rightarrow \hat f_k$. \end{example}

We can  prove \emph{convergence} whenever $f$ has absolutely summable coefficients. We will prove the result here in the special case where the negative coefficients are zero. That is, $\hat f_0^n, \ensuremath{\ldots}, \hat f_{n-1}^n$ are approximations of the Fourier\ensuremath{\endash}Taylor coefficients.

\begin{theorem}[Approximate Fourier-Taylor expansions converge] If $0 = \hat f_{-1} = \hat f_{-2} = \ensuremath{\cdots}$ and $\vchatf$ is absolutely convergent then
\[
f_n(\ensuremath{\theta}) = \ensuremath{\sum}_{k=0}^{n-1} \hat f_k^n {\rm e}^{{\rm i} k \ensuremath{\theta}}
\]
converges uniformly to $f(\ensuremath{\theta})$.

\end{theorem}
\textbf{Proof}
\begin{align*}
|f(\ensuremath{\theta}) - f_n(\ensuremath{\theta})| = |\ensuremath{\sum}_{k=0}^{n-1} (\hat f_k - \hat f_k^n) {\rm e}^{{\rm i} k \ensuremath{\theta}} + \ensuremath{\sum}_{k=n}^\ensuremath{\infty} \hat f_k {\rm e}^{{\rm i} k \ensuremath{\theta}}|
= |\ensuremath{\sum}_{k=n}^\ensuremath{\infty} \hat f_k ({\rm e}^{{\rm i} k \ensuremath{\theta}} - {\rm e}^{{\rm i} {\rm mod}(k,n) \ensuremath{\theta}})|
\ensuremath{\leq} 2 \ensuremath{\sum}_{k=n}^\ensuremath{\infty} |\hat f_k|
\end{align*}
which goes to zero as $n \ensuremath{\rightarrow} \ensuremath{\infty}$. \ensuremath{\QED}

For the general case we need to choose a range of coefficients that includes roughly an equal number of negative and positive coefficients (preferring negative over positive in a tie as a convention):
\[
f_n(\ensuremath{\theta}) = \ensuremath{\sum}_{k=-\ensuremath{\lceil}n/2\ensuremath{\rceil}}^{\ensuremath{\lfloor}n/2\ensuremath{\rfloor}} \hat f_k {\rm e}^{{\rm i} k \ensuremath{\theta}}
\]
In the problem sheet we will prove this converges provided the coefficients are absolutely convergent.





\section{Discrete Fourier Transform}
In the last section we explored using the trapezium rule for approximating Fourier coefficients. This is a linear map from function values to coefficients and thus can be reinterpreted as a matrix-vector product, called the the Discrete Fourier Transform. It turns out the matrix is unitary which leads to important properties including interpolation. Finally, we discuss how a clever way of decomposing the DFT leads to a fast way of applying and inverting it, which is one of the most influencial algorithms of the 20th century: the Fast Fourier Transform.

\subsection{The Discrete Fourier transform}
\begin{definition}[DFT] The \emph{Discrete Fourier Transform (DFT)} is defined as:
\begin{align*}
Q_n &:= {1 \over \sqrt{n}} \begin{bmatrix} 1 & 1 & 1&  \ensuremath{\cdots} & 1 \\
                                    1 & {\rm e}^{-\I \ensuremath{\theta}_1} & {\rm e}^{-\I \ensuremath{\theta}_2} & \ensuremath{\cdots} & {\rm e}^{-\I \ensuremath{\theta}_{n-1}} \\
                                    1 & {\rm e}^{-\I 2 \ensuremath{\theta}_1} & {\rm e}^{-\I 2 \ensuremath{\theta}_2} & \ensuremath{\cdots} & {\rm e}^{-\I 2\ensuremath{\theta}_{n-1}} \\
                                    \ensuremath{\vdots} & \ensuremath{\vdots} & \ensuremath{\vdots} & \ensuremath{\ddots} & \ensuremath{\vdots} \\
                                    1 & {\rm e}^{-\I (n-1) \ensuremath{\theta}_1} & {\rm e}^{-\I (n-1) \ensuremath{\theta}_2} & \ensuremath{\cdots} & {\rm e}^{-\I (n-1) \ensuremath{\theta}_{n-1}}
\end{bmatrix} \\
&= {1 \over \sqrt{n}} \begin{bmatrix} 1 & 1 & 1&  \ensuremath{\cdots} & 1 \\
                                    1 & \ensuremath{\omega}^{-1} & \ensuremath{\omega}^{-2} & \ensuremath{\cdots} & \ensuremath{\omega}^{-(n-1)}\\
                                    1 & \ensuremath{\omega}^{-2} & \ensuremath{\omega}^{-4} & \ensuremath{\cdots} & \ensuremath{\omega}^{-2(n-1)}\\
                                    \ensuremath{\vdots} & \ensuremath{\vdots} & \ensuremath{\vdots} & \ensuremath{\ddots} & \ensuremath{\vdots} \\
                                    1 & \ensuremath{\omega}^{-(n-1)} & \ensuremath{\omega}^{-2(n-1)} & \ensuremath{\cdots} & \ensuremath{\omega}^{-(n-1)^2}
\end{bmatrix}
\end{align*}
for the $n$-th root of unity $\ensuremath{\omega} = {\rm e}^{2\ensuremath{\pi}\I/n}$.  \end{definition}

Note that
\begin{align*}
Q_n^\ensuremath{\star} &= {1 \over \sqrt{n}} \begin{bmatrix}
1 & 1 & 1&  \ensuremath{\cdots} & 1 \\
1 & {\rm e}^{\I \ensuremath{\theta}_1} & {\rm e}^{\I 2 \ensuremath{\theta}_1} & \ensuremath{\cdots} & {\rm e}^{\I (n-1) \ensuremath{\theta}_1} \\
1 &  {\rm e}^{\I \ensuremath{\theta}_2}  & {\rm e}^{\I 2 \ensuremath{\theta}_2} & \ensuremath{\cdots} & {\rm e}^{\I (n-1)\ensuremath{\theta}_2} \\
\ensuremath{\vdots} & \ensuremath{\vdots} & \ensuremath{\vdots} & \ensuremath{\ddots} & \ensuremath{\vdots} \\
1 & {\rm e}^{\I \ensuremath{\theta}_{n-1}} & {\rm e}^{\I 2 \ensuremath{\theta}_{n-1}} & \ensuremath{\cdots} & {\rm e}^{\I (n-1) \ensuremath{\theta}_{n-1}}
\end{bmatrix} \\
&= {1 \over \sqrt{n}} \begin{bmatrix}
1 & 1 & 1&  \ensuremath{\cdots} & 1 \\
1 & \ensuremath{\omega}^{1} & \ensuremath{\omega}^{2} & \ensuremath{\cdots} & \ensuremath{\omega}^{(n-1)}\\
1 & \ensuremath{\omega}^{2} & \ensuremath{\omega}^{4} & \ensuremath{\cdots} & \ensuremath{\omega}^{2(n-1)}\\
\ensuremath{\vdots} & \ensuremath{\vdots} & \ensuremath{\vdots} & \ensuremath{\ddots} & \ensuremath{\vdots} \\
1 & \ensuremath{\omega}^{(n-1)} & \ensuremath{\omega}^{2(n-1)} & \ensuremath{\cdots} & \ensuremath{\omega}^{(n-1)^2}
\end{bmatrix}
\end{align*}
Hence we have
\[
\underbrace{\begin{bmatrix} \hat f_0^n \\ \ensuremath{\vdots} \\ \hat f_{n-1}^n \end{bmatrix}}_{\vchatf^n} =
{1 \over \sqrt{n}} Q_n \underbrace{\begin{bmatrix} f(\ensuremath{\theta}_0) \\ \ensuremath{\vdots} \\ f(\ensuremath{\theta}_{n-1}) \end{bmatrix}}_{\ensuremath{\bm{\f}}^n}
\]
The choice of normalisation constant is motivated by the following:

\textbf{Proposition 1 (DFT is Unitary)} $Q_n \ensuremath{\in} U(n)$, that is, $Q_n^\ensuremath{\star} Q_n = Q_n Q_n^\ensuremath{\star} = I$.

\textbf{Proof}
\[
Q_n Q_n^\ensuremath{\star}  = \begin{bmatrix} \ensuremath{\Sigma}_n[1] & \ensuremath{\Sigma}_n[{\rm e}^{\I \ensuremath{\theta}}] & \ensuremath{\cdots} & \ensuremath{\Sigma}_n[{\rm e}^{\I (n-1) \ensuremath{\theta}}] \\
                            \ensuremath{\Sigma}_n[{\rm e}^{-\I \ensuremath{\theta}}] & \ensuremath{\Sigma}_n[1] & \ensuremath{\cdots} & \ensuremath{\Sigma}_n[{\rm e}^{\I (n-2) \ensuremath{\theta}}] \\
                            \ensuremath{\vdots} & \ensuremath{\vdots} & \ensuremath{\ddots} & \ensuremath{\vdots} \\
                            \ensuremath{\Sigma}_n[{\rm e}^{-\I(n-1) \ensuremath{\theta}}] & \ensuremath{\Sigma}_n[{\rm e}^{-\I(n-2) \ensuremath{\theta}}] & \ensuremath{\cdots} & \ensuremath{\Sigma}_n[1]
                            \end{bmatrix} = I
\]
\ensuremath{\QED}

In other words, $Q_n$ is easily inverted and we also have a map from discrete Fourier coefficients back to values:
\[
\sqrt{n} Q_n^\ensuremath{\star} \vchatf^n = \ensuremath{\bm{\f}}^n
\]
\begin{example}[Computing Sum] Define the following infinite sum (which has no name apparently, according to Mathematica):
\[
S_n(k) := \ensuremath{\sum}_{p=0}^\ensuremath{\infty} {1 \over (k+pn)!}
\]
We can use the DFT to compute $S_n(0), \ensuremath{\ldots}, S_n(n-1)$. Consider
\[
f(\ensuremath{\theta}) = \exp({\rm e}^{\I \ensuremath{\theta}}) = \ensuremath{\sum}_{k=0}^\ensuremath{\infty} {{\rm e}^{\I k \ensuremath{\theta}} \over k!}
\]
where we know the Fourier coefficients from the Taylor series of ${\rm e}^z$. The discrete Fourier coefficients satisfy for $0 \ensuremath{\leq} k \ensuremath{\leq} n-1$:
\[
\hat f_k^n = \hat f_k + \hat f_{k+n} + \hat f_{k+2n} + \ensuremath{\cdots} = S_n(k)
\]
Thus we have
\[
\begin{bmatrix}
S_n(0) \\
\ensuremath{\vdots} \\
S_n(n-1)
\end{bmatrix} = {1 \over \sqrt{n}} Q_n \begin{bmatrix} 1 \\
                                \exp({\rm e}^{2\I \ensuremath{\pi}/n}) \\
                                \ensuremath{\vdots} \\
                                \exp({\rm e}^{2\I (n-1) \ensuremath{\pi}/n}) \end{bmatrix}
\]
\end{example}

\subsection{Interpolation}
We investigated  interpolation and least squares using polynomials at evenly spaced points, observing that there were issues with stability. We now show that the DFT actually gives coefficients that interpolate using Fourier expansions. As the DFT is a unitary matrix multiplication is \ensuremath{\ldq}stable", i.e. it preserves norms and hence we know it cannot cause the same huge blow-up we saw for polynomials. That is: whilst polynomials are bad for interpolation at evenly spaced points, trigonometric polynomials are great. 

The following guarantees that our approximate Fourier series actually interpolates the data:

\begin{corollary}[Interpolation]
\[
f_n(\ensuremath{\theta}) := \ensuremath{\sum}_{k=0}^{n-1} \hat f_k^n {\rm e}^{\I k \ensuremath{\theta}}
\]
interpolates $f$ at $\ensuremath{\theta}_j$:
\[
f_n(\ensuremath{\theta}_j) = f(\ensuremath{\theta}_j)
\]
\end{corollary}
\textbf{Proof} We have
\[
f_n(\ensuremath{\theta}_j) = \ensuremath{\sum}_{k=0}^{n-1} \hat f_k^n {\rm e}^{\I k \ensuremath{\theta}_j} = \sqrt n \ensuremath{\bm{\e}}_j^\ensuremath{\top} Q_n^\ensuremath{\star} \vchatf^n = \ensuremath{\bm{\e}}_j^\ensuremath{\top} Q_n^\ensuremath{\star} Q_n \ensuremath{\bm{\f}}^n = f(\ensuremath{\theta}_j).
\]
\ensuremath{\QED}

\begin{example}[DFT versus Lagrange] Consider interpolating $f(z) = \exp z$ by a polynomial at the points $1, \I, -1, -\I$. We can use Lagrange polynomials:
\meeq{
\ensuremath{\ell}_1(z) ={ (z - \I)(z + 1)(z + \I) \over 2(1 - \I)(1 + \I)} = { z^3 + z^2 + z + 1 \over 4} \ccr
\ensuremath{\ell}_2(z) ={ (z - 1)(z + 1)(z + \I) \over (\I - 1) (\I + 1) 2\I} = { \I z^3 - z^2 - \I z + 1 \over 4} \ccr
\ensuremath{\ell}_3(z) ={ (z - 1)(z - \I)(z + \I) \over -2 (-1-\I)(-1+\I)} = {-z^3 + z^2 - z + 1 \over 4} \ccr
\ensuremath{\ell}_4(z) ={ (z - 1)(z - \I)(z+1) \over (-\I-1)(-2\I)(-\I+1)} = {- \I z^3 -z^2 + \I z + 1 \over 4}
}
So we get the interpolant:
\begin{align*}
\E & \ensuremath{\ell}_1(z) + \E^\I \ensuremath{\ell}_2(z) + \E^{-1} \ensuremath{\ell}_3(z) + \E^{-\I} \ensuremath{\ell}_4(z) \\
 &= 
{\E + \E^\I + \E^{-1} + \E^{-\I} \over 4} +
{\E - \I \E^\I - \E^{-1} + \I \E^{-\I} \over 4} z +
 {\E - \E^\I + \E^{-1} - \I \E^{-\I} \over 4} z^2 +
 {\E + \I \E^\I - \E^{-1} - \I \E^{-\I} \over 4} z^3 
\end{align*}
Alternatively we could have deduced this directly from the DCT. In particular, we know the coefficients of the interpolating polynomial must be, for $\ensuremath{\omega} = \I$,
\[
\Vectt[\hat f_0^4, \hat f_1^4, \hat f_2^4, \hat f_3^4] = 
{1 \over 4} \begin{bmatrix}1 & 1 & 1 & 1 \\
                            1 & -\I & -1 & \I \\
                            1 & -1 & 1 & -1 \\
                            1 & \I & -1 & -\I
                            \end{bmatrix}
 \Vectt[\E, \E^\I, \E^{-1}, \E^{-\I}] = {1 \over 4} \Vectt[\E + \E^\I + \E^{-1} + \E^{-\I} ,
 \E -\I \E^\I - \E^{-1} + \I \E^{-\I} ,
 \E - \E^\I + \E^{-1} - \E^{-\I} ,
 \E + \I \E^\I - \E^{-1} = \I \E^{-\I}
 ]
\]
\end{example}





\section{Orthogonal Polynomials}
Fourier series are very powerful for approximating periodic functions. If periodicity is lost, however, the Fourier coefficients are no longer in $\ensuremath{\ell}^1$ and uniform convergence is lost. In this chapter we introduce alternative bases, \emph{Orthogonal Polynomials (OPs)} built on polynomials that are applicable in the non-periodic setting. That is we consider expansions of the form
\[
f(x) = \sum_{k=0}^\ensuremath{\infty} c_k p_k(x)
\]
where $p_k(x)$ are special families of polynomials and $c_k$ are expansion coefficients. The approximation of the coefficients $c_k \ensuremath{\approx} c_k^n$ using quadrature will be explored later.

Why not use monomials as in Taylor series? Hidden in the previous sections was that we could effectively compute Taylor coefficients by evaluating on the unit circle in the complex plane, \emph{only} if the radius of convergence was 1. Many functions are smooth on say $[-1,1]$ but have non-convergent Taylor series, e.g.:
\[
{1 \over 25x^2 + 1}
\]
While orthogonal polynomials span the same space as monomials, and therefore we can in theory write an approximation in monomials, orthogonal polynomials are \emph{much} more stable. In particular, where we saw that interpolation by monomials at evenly spaced points performed horribly in practice (and also in theory), we can use orthogonal polynomials with specially chosen points to get reliable interpolation of functions. 

In addition to numerics, OPs play a very important role in many mathematical areas including functional analysis, integrable systems, singular integral equations, complex analysis, and random matrix theory.

\subsection{General properties}
\begin{definition}[graded polynomial basis] A set of polynomials $\{p_0(x), p_1(x), \ensuremath{\ldots} \}$ is \emph{graded} if $p_n$ is precisely degree $n$: i.e.,
\[
p_n(x) = k_n x^n + k_n^{(1)} x^{n-1} + \ensuremath{\cdots} + k_n^{(n-1)} x + k_n^{(n)}
\]
for $k_n \ensuremath{\neq} 0$. \end{definition}

Note that if $p_n$ are graded then $\{p_0(x), \ensuremath{\ldots}, p_n(x) \}$ are a basis of all polynomials of degree $n$.

\begin{definition}[Orthogonal Polynomials] Given an (integrable) \emph{weight} $w(x) > 0$ for $x \ensuremath{\in} (a,b)$, which defines a continuous inner product
\[
\ensuremath{\langle}f,g\ensuremath{\rangle} = \ensuremath{\int}_a^b  f(x) g(x) w(x) {\rm d} x
\]
a graded polynomial basis $\{p_0(x), p_1(x), \ensuremath{\ldots} \}$ are \emph{orthogonal polynomials (OPs)} if
\[
\ensuremath{\langle}p_n,p_m\ensuremath{\rangle} = 0
\]
whenever $n \ensuremath{\neq} m$. We assume through that integrals of polynomials are finite:
\[
\ensuremath{\int}_a^b  x^k w(x) {\rm d} x < \ensuremath{\infty}.
\]
\end{definition}

Note in the above
\[
h_n := \ensuremath{\langle}p_n,p_n\ensuremath{\rangle} = \|p_n\|^2 = \ensuremath{\int}_a^b  p_n(x)^2 w(x) {\rm d} x > 0.
\]
Multiplying any orthogonal polynomial by a nonzero constant necessarily is also an orthogonal polynomial. We have two standard normalisations:

\begin{definition}[Orthonormal Polynomials] A set of orthogonal polynomials $\{q_0(x), q_1(x), \ensuremath{\ldots} \}$ are \emph{orthonormal} if $\|q_n\| = 1$. \end{definition}

\begin{definition}[Monic Orthogonal Polynomials] A set of orthogonal polynomials $\{\ensuremath{\pi}_0(x), \ensuremath{\pi}_1(x), \ensuremath{\ldots} \}$ are \emph{monic} if $k_n = 1$. \end{definition}

\begin{proposition}[existence] Given a weight $w(x)$, monic orthogonal polynomials exist.

\end{proposition}
\textbf{Proof} Existence follows immediately from the Gram\ensuremath{\endash}Schmidt procedure. That is, define $\ensuremath{\pi}_0(x) := 1$ and
\[
\ensuremath{\pi}_n(x) := x^n - \ensuremath{\sum}_{k=0}^{n-1} {\ensuremath{\langle}x^n,\ensuremath{\pi}_k\ensuremath{\rangle} \over \|\ensuremath{\pi}_k\|^2} \ensuremath{\pi}_k(x).
\]
Assume $\ensuremath{\pi}_m$ are monic OPs for all $m < n$. Then we have
\[
\ensuremath{\langle}\ensuremath{\pi}_m, \ensuremath{\pi}_n\ensuremath{\rangle} = \ensuremath{\langle}\ensuremath{\pi}_m, x^n \ensuremath{\rangle} - \ensuremath{\sum}_{k=0}^{n-1} {\ensuremath{\langle}x^n,\ensuremath{\pi}_k\ensuremath{\rangle} \over \|\ensuremath{\pi}_k\|^2} \underbrace{\ensuremath{\langle}\ensuremath{\pi}_m, \ensuremath{\pi}_k\ensuremath{\rangle}}_{= 0 \hbox{if $m \ensuremath{\neq} k$}}  = \ensuremath{\langle}\ensuremath{\pi}_m, x^n \ensuremath{\rangle} - \ensuremath{\langle}x^n,\ensuremath{\pi}_m\ensuremath{\rangle} = 0.
\]
\ensuremath{\QED}

We are primarly concerned with the usage of orthogonal polynomials in approximating functions. First we observe the following:

\begin{proposition}[expansion] If $r(x)$ is a degree $n$ polynomial and $\{p_n\}$ are orthogonal then
\[
r(x) = \ensuremath{\sum}_{k=0}^n {\ensuremath{\langle}p_k,r\ensuremath{\rangle} \over \|p_k\|^2} p_k(x)
\]
Note for $\{q_n\}$ orthonormal we have
\[
r(x) = \ensuremath{\sum}_{k=0}^n \ensuremath{\langle}q_k,r\ensuremath{\rangle} q_k(x).
\]
\end{proposition}
\textbf{Proof} Because $\{p_0,\ensuremath{\ldots},p_n \}$ are a basis of polynomials we can write
\[
r(x) = \ensuremath{\sum}_{k=0}^n r_k p_k(x)
\]
for constants $r_k \ensuremath{\in} \ensuremath{\bbR}$. By linearity we have
\[
\ensuremath{\langle}p_m,r\ensuremath{\rangle} = \ensuremath{\sum}_{k=0}^n r_k \ensuremath{\langle}p_m,p_k\ensuremath{\rangle}= r_m \ensuremath{\langle}p_m,p_m\ensuremath{\rangle}
\]
\ensuremath{\QED}

\begin{corollary}[zero inner product] If a degree $n$ polynomial $r$ satisfies
\[
0 = \ensuremath{\langle}p_0,r\ensuremath{\rangle} = \ensuremath{\ldots} = \ensuremath{\langle}p_n,r\ensuremath{\rangle}
\]
then $r = 0$.

\end{corollary}
\textbf{Proof} If all the inner products are zero the coefficients in the expansion are all zero and $r$ is zero. \ensuremath{\QED}

\begin{corollary}[uniqueness] Monic orthogonal polynomials are unique.

\end{corollary}
\textbf{Proof} If $p_n(x)$ and $\ensuremath{\pi}_n(x)$ are both monic orthogonal polynomials then $r(x) = p_n(x) - \ensuremath{\pi}_n(x)$ is degree $n-1$ but satisfies
\[
\ensuremath{\langle}r, \ensuremath{\pi}_k\ensuremath{\rangle} = \ensuremath{\langle}p_n, \ensuremath{\pi}_k\ensuremath{\rangle} - \ensuremath{\langle}\ensuremath{\pi}_n, \ensuremath{\pi}_k\ensuremath{\rangle} = 0
\]
for $k = 0,\ensuremath{\ldots},{n-1}$. Note $\ensuremath{\langle}p_n, \ensuremath{\pi}_k\ensuremath{\rangle} = 0$ can be seen by expanding
\[
\ensuremath{\pi}_k(x) = \ensuremath{\sum}_{j=0}^k c_j p_j(x).
\]
\ensuremath{\QED}

OPs are uniquely defined (up to a constant) by the property that they are orthogonal to all lower degree polynomials.

\begin{theorem}[orthogonal to lower degree] Given a weight $w(x)$, a polynomial
\[
p(x) = k_n x^n + O(x^{n-1})
\]
with $k_n \ensuremath{\neq} 0$ satisfies
\[
\ensuremath{\langle}p,f_m\ensuremath{\rangle} = 0
\]
for all  polynomials $f_m$ of degree $m < n$ if and only if $p(x) = k_n \ensuremath{\pi}_n(x)$ where $\ensuremath{\pi}_n(x)$ are the monic orthogonal polynomials. Therefore an orthogonal polynomial is uniquely defined by the weight and leading order coefficient $k_n$.

\end{theorem}
\textbf{Proof} We leave this proof to the problem sheets. \ensuremath{\QED}

A consequence of this is that orthonormal polynomials are always a constant multiple of orthogonal polynomials.

\subsection{3-term recurrence}
The most \emph{fundamental} property of orthogonal polynomials is their three-term recurrence.

\begin{theorem}[3-term recurrence, 2nd form] If $\{p_n\}$ are OPs then there exist real constants $a_n b_n, c_{n-1}$ such that
\begin{align*}
x p_0(x) &= a_0 p_0(x) + b_0 p_1(x)  \\
x p_n(x) &= c_{n-1} p_{n-1}(x) + a_n p_n(x) + b_n p_{n+1}(x),
\end{align*}
where $b_n \ensuremath{\neq}0$ and $c_{n-1} \ensuremath{\neq}0$. \end{theorem}
\textbf{Proof} The $n=0$ case is immediate since $\{p_0,p_1\}$ are a basis of degree 1 polynomials. The $n >0$ case follows from
\[
\ensuremath{\langle}x p_n, p_k\ensuremath{\rangle} = \ensuremath{\langle} p_n, xp_k\ensuremath{\rangle} = 0
\]
for $k < n-1$ as $x p_k$ is of degree $k+1 < n$.

Note that
\[
b_n = {\ensuremath{\langle}p_{n+1}, x p_n\ensuremath{\rangle} \over \|p_{n+1} \|^2} \ensuremath{\neq} 0
\]
since $x p_n = k_n x^{n+1} + O(x^n)$ is precisely degree $n$. Further,
\[
c_{n-1} = {\ensuremath{\langle}p_{n-1}, x p_n\ensuremath{\rangle} \over \|p_{n-1}\|^2 } =
{\ensuremath{\langle}p_n, x p_{n-1}\ensuremath{\rangle}  \over \|p_{n-1}\|^2 } =  b_{n-1}{\|p_n\|^2  \over \|p_{n-1}\|^2 } \ensuremath{\neq} 0.
\]
\ensuremath{\QED}

Clearly if $\ensuremath{\pi}_n$ is monic then so is $x \ensuremath{\pi}_n$ which leads to the following:

\begin{corollary}[monic 3-term recurrence] $\{\ensuremath{\pi}_n\}$ are monic if and only if $b_n =  1$. \end{corollary}
\textbf{Proof}

If $b_n = 1$ and $\ensuremath{\pi}_n(x) = x^n + O(x^{n-1})$ then the 3-term recurrence shows us that
\[
\ensuremath{\pi}_{n+1}(x) = x \ensuremath{\pi}_n(x) - c_{n-1} \ensuremath{\pi}_{n-1}(x) - a_n \ensuremath{\pi}_n(x) = x^{n+1} + O(x^n)
\]
and $\ensuremath{\pi}_{n+1}(x)$ is also monic. Similarly, if $\ensuremath{\pi}_n(x)$ is monic and $b_n \ensuremath{\neq} 1$ then $\ensuremath{\pi}_{n+1}(x)$ is not monic, which would be a contradiction. \ensuremath{\QED}

Note this implies that we can define $\ensuremath{\pi}_{n+1}(x)$ in terms of $\ensuremath{\pi}_{n-1}$ and $\ensuremath{\pi}_n$:
\[
\ensuremath{\pi}_{n+1}(x) = x \ensuremath{\pi}_n(x) - a_n \ensuremath{\pi}_n(x) - c_{n-1} \ensuremath{\pi}_{n-1}(x)
\]
where
\[
a_n = {\ensuremath{\langle}x \ensuremath{\pi}_n, \ensuremath{\pi}_n\ensuremath{\rangle} \over \| \ensuremath{\pi}_n\|^2} \qquad \hbox{and} \qquad c_{n-1} = {\ensuremath{\langle}x \ensuremath{\pi}_n, \ensuremath{\pi}_{n-1}\ensuremath{\rangle} \over \| \ensuremath{\pi}_{n-1}\|^2}.
\]
\begin{example}[constructing OPs] What are the  monic OPs $\ensuremath{\pi}_0(x),\ensuremath{\ldots},\ensuremath{\pi}_3(x)$ with respect to $w(x) = 1$ on $[0,1]$? We can construct these using Gram\ensuremath{\endash}Schmidt, but exploiting the 3-term recurrence to reduce the computational cost. We have $\ensuremath{\pi}_0(x) = 1$, which we see is orthogonal:
\[
\|\ensuremath{\pi}_0\|^2 = \ensuremath{\langle}\ensuremath{\pi}_0,\ensuremath{\pi}_0\ensuremath{\rangle} = \ensuremath{\int}_0^1 {\rm d} x = 1.
\]
We know from the 3-term recurrence that
\[
x \ensuremath{\pi}_0(x) = a_0 \ensuremath{\pi}_0(x) +  \ensuremath{\pi}_1(x)
\]
where
\[
a_0 = {\ensuremath{\langle}\ensuremath{\pi}_0,x \ensuremath{\pi}_0\ensuremath{\rangle}  \over \|\ensuremath{\pi}_0\|^2} = \ensuremath{\int}_0^1 x {\rm d} x = 1/2.
\]
Thus
\begin{align*}
\ensuremath{\pi}_1(x) = x \ensuremath{\pi}_0(x) - a_0 \ensuremath{\pi}_0(x) = x-1/2 \qquad  \ensuremath{\Rightarrow} \\
\|\ensuremath{\pi}_1\|^2 = \ensuremath{\int}_0^1 (x^2 - x + 1/4) {\rm d} x = 1/12.    
\end{align*}
From
\[
x \ensuremath{\pi}_1(x) = c_0 \ensuremath{\pi}_0(x) + a_1 \ensuremath{\pi}_1(x) +  \ensuremath{\pi}_2(x)
\]
we have
\begin{align*}
c_0 &= {\ensuremath{\langle}\ensuremath{\pi}_0,x \ensuremath{\pi}_1\ensuremath{\rangle}  \over \|\ensuremath{\pi}_0\|^2} = \ensuremath{\int}_0^1 (x^2 - x/2) {\rm d} x = 1/12, \\
a_1 &= {\ensuremath{\langle}\ensuremath{\pi}_1,x \ensuremath{\pi}_1\ensuremath{\rangle}  \over \|\ensuremath{\pi}_1\|^2} = 12 \ensuremath{\int}_0^1 (x^3 - x^2 + x/4) {\rm d} x = 1/2, \\
\ensuremath{\pi}_2(x) &= x \ensuremath{\pi}_1(x) - c_0 - a_1 \ensuremath{\pi}_1(x) = x^2 - x + 1/6 \qquad \ensuremath{\Rightarrow} \\
\|\ensuremath{\pi}_2\|^2 &= \int_0^1 (x^4 - 2x^3 + 4x^2/3 - x/3 + 1/36) {\rm d} x = {1 \over 180}
\end{align*}
Finally, from
\[
x \ensuremath{\pi}_2(x) = c_1 \ensuremath{\pi}_1(x) + a_2 \ensuremath{\pi}_2(x) +  \ensuremath{\pi}_3(x)
\]
we have
\begin{align*}
c_1 &= {\ensuremath{\langle}\ensuremath{\pi}_1,x \ensuremath{\pi}_2\ensuremath{\rangle}  \over \|\ensuremath{\pi}_1\|^2} = 12 \ensuremath{\int}_0^1 (x^4 - 3x^3/2 +2x^2/3 -x/12)  {\rm d} x = 1/15, \\
a_2 &= {\ensuremath{\langle}\ensuremath{\pi}_2,x \ensuremath{\pi}_2\ensuremath{\rangle}  \over \|\ensuremath{\pi}_2\|^2} = 180 \ensuremath{\int}_0^1 (x^5 - 2x^4 +4x^3/3 - x^2/3 + x/36) {\rm d} x = 1/2, \\
\ensuremath{\pi}_3(x) &= x \ensuremath{\pi}_2(x) - c_1 \ensuremath{\pi}_1(x)- a_2 \ensuremath{\pi}_2(x) \ccr 
= x^3 - x^2 + x/6 - x/15 + 1/30 -x^2/2 + x/2 - 1/12 \\
&= x^3 - 3x^2/2 + 3x/5 -1/20
\end{align*}
\end{example}

\subsection{Jacobi matrices}
The three-term recurrence can also be interpreted as a matrix:

\begin{corollary}[multiplication matrix] For
\[
P(x) := [p_0(x) | p_1(x) | \ensuremath{\cdots}]
\]
then we have
\[
x P(x) = P(x) \underbrace{\begin{bmatrix} a_0 & c_0 \\
                                                        b_0 & a_1 & c_1\\
                                                        & b_1 & a_2 & \ensuremath{\ddots} \\
                                                        && \ensuremath{\ddots} & \ensuremath{\ddots}
                                                        \end{bmatrix}}_X
\]
More generally, for any polynomial $a(x)$ we have
\[
a(x) P(x) = P(x) a(X).
\]
\end{corollary}
\textbf{Proof} The expression follows:
\[
x P(x) = [xp_0(x) | xp_1(x) | \ensuremath{\cdots}] =
[a_0p_0(x) + b_0 p_1(x) | c_0 p_0(x) + a_1 p_1(x) + b_1 p_2(x) | \ensuremath{\cdots}] = P(x) X.
\]
For polynomials, note that
\[
x^k P(x) = x^{k-1} P(x) X = \ensuremath{\cdots} = P(x) X^k.
\]
Thus if $a(x) = \ensuremath{\sum}_{k=0}^n a_k x^k$ we have by linearity
\[
a(x) P(x) = \ensuremath{\sum}_{k=0}^n a_k x^k P(x) = P(x) \ensuremath{\sum}_{k=0}^n a_k X^k = P(x) a(X).
\]
\ensuremath{\QED}

\textbf{Remark} If you are worried about multiplication of infinite matrices/vectors note it is well-defined by the standard definition because it is banded. It can also be defined in terms of functional analysis where one considers these as linear operators (functions of functions) between vector spaces.

For the special cases of orthonormal and monic polynomials we have extra structure, in which case we refer to the matrix as a \emph{Jacobi matrix}:

\begin{corollary}[Jacobi matrix] The multiplication matrix of a family of orthogonal polynomials $p_n(x)$ is symmetric,
\[
X = X^\ensuremath{\top} = \begin{bmatrix} a_0 & b_0 \\
                                                        b_0 & a_1 & b_1\\
                                                        & b_1 & a_2 & \ensuremath{\ddots} \\
                                                        && \ensuremath{\ddots} & \ensuremath{\ddots}
                                                        \end{bmatrix},
\]
if and only if $p_n(x)$ is up-to-sign a fixed constant scaling of orthonormal: for $q_n(x) := \ensuremath{\pi}_n(x)/\|\ensuremath{\pi}_n\|$ we have for a fixed $\ensuremath{\alpha} \ensuremath{\in} \ensuremath{\bbR}$ and $s_n \ensuremath{\in} \{-1,1\}$
\[
p_n(x) = \ensuremath{\alpha} s_n q_n(x).
\]
\end{corollary}
\textbf{Proof} Noting that $\|q_n\|^2 = 1$ and thence $\|p_n\|^2 = \ensuremath{\alpha}^2$, if $p_n(x) = \ensuremath{\alpha} s_n q_n(x)$ we have
\[
b_n = {\ensuremath{\langle}xp_n, p_{n+1}\ensuremath{\rangle} \over \|p_{n+1}\|^2} = s_n s_{n+1} \ensuremath{\langle}x q_n, q_{n+1}\ensuremath{\rangle} =
s_n s_{n+1} \ensuremath{\langle}q_n, x q_{n+1}\ensuremath{\rangle} = {\ensuremath{\langle}p_n, xp_{n+1}\ensuremath{\rangle} \over \|p_n\|^2} = c_{n-1}.
\]
Conversely, suppose $X = X^\ensuremath{\top}$, i.e., $b_n = c_{n-1}$ and write the corresponding orthogonal polynomials as $p_n(x) = \ensuremath{\alpha}_n q_n(x)$. We have
\[
b_n = {\ensuremath{\langle}xp_n, p_{n+1}\ensuremath{\rangle} \over \|p_{n+1}\|^2} =
{\ensuremath{\alpha}_n \over \ensuremath{\alpha}_{n+1}} \ensuremath{\langle}xq_n, q_{n+1}\ensuremath{\rangle} =
{\ensuremath{\alpha}_n \over \ensuremath{\alpha}_{n+1}} \ensuremath{\langle}q_n, x q_{n+1}\ensuremath{\rangle} = {\ensuremath{\alpha}_n^2 \over \ensuremath{\alpha}_{n+1}^2} {\ensuremath{\langle}p_n, xp_{n+1}\ensuremath{\rangle} \over \|p_n\|^2}
= {\ensuremath{\alpha}_n^2 \over \ensuremath{\alpha}_{n+1}^2} c_{n-1} = {\ensuremath{\alpha}_n^2 \over \ensuremath{\alpha}_{n+1}^2} b_n.
\]
Hence $\ensuremath{\alpha}_n^2 = \ensuremath{\alpha}_{n+1}^2$ which implies that $\ensuremath{\alpha}_{n+1} = \ensuremath{\pm} \ensuremath{\alpha}_n$. By induction the result follows, where $\ensuremath{\alpha} := \ensuremath{\alpha}_0$. \ensuremath{\QED}

\textbf{Remark} Every compactly supported integrable weight generates a family of orthonormal polynomials, which in turn generates a symmetric Jacobi matrix. There is a \ensuremath{\ldq}Spectral Theorem for Jacobi matrices" that says one can go the other way: every tridiagonal symmetric matrix with bounded entries is a Jacobi matrix for some integrable weight with compact support. This is an example of what \href{https://en.wikipedia.org/wiki/Barry_Simon}{Barry Simon} calls a \ensuremath{\ldq}Gem of spectral theory", that is.

\begin{example}[uniform weight Jacobi matrix] Consider the monic orthogonal polynomials $\ensuremath{\pi}_0(x),\ensuremath{\pi}_1(x),\ensuremath{\ldots},\ensuremath{\pi}_3(x)$ for $w(x) = 1$ on $[0,1]$ constructed above. We can write the 3-term recurrence coefficients we have computed above as:
\[
x [\ensuremath{\pi}_0(x)| \ensuremath{\pi}_1(x)| \ensuremath{\cdots}] = [\ensuremath{\pi}_0(x)| \ensuremath{\pi}_1(x)| \ensuremath{\cdots}] \underbrace{\begin{bmatrix} 1/2 & 1/12 \\
                                                            1 & 1/2 & 1/15 \\
                                                            & 1 & 1/2 & \ensuremath{\ddots} \\
                                                            & & 1 & \ensuremath{\ddots} & \ensuremath{\ddots} \\
                                                            &&& \ensuremath{\ddots} \end{bmatrix}}_X
\]
We can compute the orthonormal polynomials, using
\[
\|\ensuremath{\pi}_3\|^2 = \int_0^1 (x^6 - 3x^5 + 69x^4/20 -19x^3/10 + 51x^2/100 - 3x/50 + 1/400) {\rm d}x = {1 \over 2800}
\]
since
\begin{align*}
q_0(x) &= \ensuremath{\pi}_0(x), \\
q_1(x) &= \sqrt{12} \ensuremath{\pi}_1(x)= \sqrt{3} (2  x - 1), \\
q_2(x) &= \sqrt{180} \ensuremath{\pi}_2(x) = \sqrt{5} (6x^2 - 6x + 1), \\
q_3(x) &= \sqrt{2800} \ensuremath{\pi}_3(x) = \sqrt{7} (20x^3-30x^2 + 12x - 1),
\end{align*}
which have the Jacobi matrix
\begin{align*}
x [q_0(x)| q_1(x)| \ensuremath{\cdots}] &= x [\ensuremath{\pi}_0(x)| \ensuremath{\pi}_1(x)| \ensuremath{\cdots}] \underbrace{\begin{bmatrix} 1 \\ & 2\sqrt{3} \\ && 6 \sqrt{5} \\ &&& 20 \sqrt{7} \\
&&&& \ensuremath{\ddots}
\end{bmatrix}}_D \\
&= [q_0(x)| q_1(x)| \ensuremath{\cdots}] D^{-1} X D =
     \begin{bmatrix} 1/2 & 1/(2\sqrt{3}) \\
                    1/(2\sqrt{3}) & 1/2 &  1/\sqrt{15} \\
                    & 1/\sqrt{15} & 1/2 & \ensuremath{\ddots} \\
                    && 3/(2 \sqrt{35}) &  \ensuremath{\ddots} \\
                    &&& & \ensuremath{\ddots} \end{bmatrix}
\end{align*}
which is indeed symmetric.  \end{example}

\begin{example}[Jacobi matrix] What are the expansion coefficients of $x^3 - x + 1$ in $\{q_n\}$? We could deduce integrals though its actually simpler to use the multiplication matrix. In particular if we write
\[
Q(x) := [q_0(x) | q_1(x) | q_2(x) | \ensuremath{\cdots}]
\]
Then we have (note: $q_0(x) \ensuremath{\equiv} 1$ only because the weight integrates to 1)
\[
1 = Q(x) \ensuremath{\bm{\e}}_1
\]
Hence we have:
\[
x = x Q(x) \ensuremath{\bm{\e}}_1 = Q(x) X \ensuremath{\bm{\e}}_1 = Q(x) \Vectt[1/2, 1/(2\sqrt{3}), 0, \ensuremath{\vdots}].
\]
Continuing we have
\[
x^2 = Q(x)  X \Vectt[1/2, 1/(2\sqrt{3}), 0, \ensuremath{\vdots}] = Q(x) \Vectt[1/3, 1/(2 \sqrt{3}),  1/(6\sqrt{5}),0,\ensuremath{\vdots} ]
\]
Finally we have
\[
x^3 = Q(x) X  \Vectt[1/3, 1/(2 \sqrt{3}),  1/(6\sqrt{5}),0,\ensuremath{\vdots} ] =
 Q(x) \Vectt[1/4,{3 \sqrt{3} \over 20}, {1 \over 4 \sqrt{5}}, {1 \over 20 \sqrt{7}}, 0, \ensuremath{\vdots}]
\]
Thus by linearity we find that
\meeq{
x^3 - x + 1 = Q(x) \Vectt[3/4, -1/(20\sqrt{3}), {1 \over 4 \sqrt{5}}, {1 \over 20 \sqrt{7}}, 0, \ensuremath{\vdots}] \ccr
= {3 \over 4} q_0(x) - {1 \over 20\sqrt{3}} q_1(x) + {1 \over 4 \sqrt{5}} q_2(x) + {1 \over 20 \sqrt{7}} q_3(x).
}
\end{example}





\section{Classical Orthogonal Polynomials}
Classical orthogonal polynomials are special families of orthogonal polynomials with a number of beautiful properties, for example (1) their derivatives are also OPs and (2) they are eigenfunctions of simple differential operators. As stated above orthogonal polynomials are uniquely defined by the weight $w(x)$ and the constant $k_n$ and hence we can define the classical OPs by specifying their weights and normalisation constants.

The classical orthogonal polynomials are:

\begin{itemize}
\item[1. ] Chebyshev polynomials (1st kind) $T_n(x)$: $w(x) = 1/\sqrt{1-x^2}$ on $[-1,1]$.


\item[2. ] Chebyshev polynomials (2nd kind) $U_n(x)$: $\sqrt{1-x^2}$ on $[-1,1]$.


\item[3. ] Legendre polynomials $P_n(x)$: $w(x) = 1$ on $[-1,1]$.


\item[4. ] Ultrapsherical polynomials (my fav!): $C_n^{(\ensuremath{\lambda})}(x)$: $w(x) = (1-x^2)^{\ensuremath{\lambda}-1/2}$ on $[-1,1]$, $\ensuremath{\lambda} \ensuremath{\neq} 0$, $\ensuremath{\lambda} > -1/2$.


\item[5. ] Jacobi polynomials: $P_n^{(a,b)}(x)$: $w(x) = (1-x)^a (1+x)^b$ on $[-1,1]$, $a,b > -1$.


\item[6. ] Laguerre polynomials: $L_n(x)$: $w(x) = \exp(-x)$ on $[0,\ensuremath{\infty})$.


\item[7. ] Hermite polynomials $H_n(x)$: $w(x) = \exp(-x^2)$  on $(-\ensuremath{\infty},\ensuremath{\infty})$.

\end{itemize}
In the notes we will discuss:

\begin{itemize}
\item[1. ] Chebyshev polynomials: These are closely linked to Fourier series and are one of the most powerful tools in numerics.


\item[2. ] Legendre polynomials: These have no simple closed-form expression but can be defined in terms of a Rodriguez formula, a feature that applies to all other classical families.

\end{itemize}
\subsection{Chebyshev polynomials}
There are four families of Chebyshev polynomials but we will consider the first two:

\begin{definition}[Chebyshev polynomials, 1st kind] $T_n(x)$ are orthogonal with respect to $1/\sqrt{1-x^2}$ and satisfy:
\begin{align*}
T_0(x) &= 1, \\
T_n(x) &= 2^{n-1} x^n + O(x^{n-1})
\end{align*}
\end{definition}

\begin{definition}[Chebyshev polynomials, 2nd kind] $U_n(x)$ are orthogonal with respect to $\sqrt{1-x^2}$.
\[
U_n(x) = 2^n x^n + O(x^{n-1})
\]
\end{definition}

A beautiful fact is that Chebyshev polynomials are really trigonometric polynomials in disguise:

\begin{theorem}[Chebyshev T are cos] For $-1 \ensuremath{\leq} x \ensuremath{\leq} 1$
\[
T_n(x) = \cos n\, {\rm acos}\, x.
\]
In other words
\[
T_n(\cos \ensuremath{\theta}) = \cos n \ensuremath{\theta}.
\]
\end{theorem}
\textbf{Proof}

We need to show that $p_n(x) := \cos n {\rm acos}\, x$ are

\begin{itemize}
\item[1. ] graded polynomials


\item[2. ] orthogonal w.r.t. $1/\sqrt{1-x^2}$ on $[-1,1]$, and


\item[3. ] have the right normalisation constant $k_n = 2^{n-1}$ for $n = 2,\ensuremath{\ldots}$.

\end{itemize}
Property (2) follows under a change of variables:
\[
\int_{-1}^1 {p_n(x) p_m(x) \over \sqrt{1-x^2}} {\rm d} x =
\int_0^\ensuremath{\pi} {\cos(n\ensuremath{\theta}) \cos(m\ensuremath{\theta}) \over \sqrt{1-\cos^2 \ensuremath{\theta}}} \sin \ensuremath{\theta} {\rm d} \ensuremath{\theta} =
\int_0^\ensuremath{\pi} \cos(n\ensuremath{\theta}) \cos(m\ensuremath{\theta}) {\rm d} x = 0
\]
if $n \ensuremath{\neq} m$.

To see that they are graded we use the fact that
\[
x p_n(x) = \cos \ensuremath{\theta} \cos n \ensuremath{\theta} = {\cos(n-1)\ensuremath{\theta} + \cos(n+1)\ensuremath{\theta} \over 2} = {p_{n-1}(x) + p_{n+1}(x) \over 2}.
\]
In other words $p_{n+1}(x) = 2x p_n(x) - p_{n-1}(x)$. Since each time we multiply by $2x$ and $p_0(x) = 1$ we have
\[
p_n(x) = (2x)^n + O(x^{n-1})
\]
which completes the proof.

\ensuremath{\QED}

Recall that the 3-term recurrence is an important property of a family of orthogonal polynomials. We can deduce from the relationship with cosines the following:

\begin{corollary}[Chebyshev 3-term recurrence]
\begin{align*}
x T_0(x) = T_1(x) \\
x T_n(x) = {T_{n-1}(x) + T_{n+1}(x) \over 2}
\end{align*}
\end{corollary}
\textbf{Proof} This is rewriting the expression we used to show that $p_n(x)$ are graded in the previous proof. \ensuremath{\QED}

Chebyshev polynomials are particularly powerful as their expansions are cosine series in disguise: for
\[
f(x) = \ensuremath{\sum}_{k=0}^\ensuremath{\infty} \check f_k T_k(x)
\]
we have
\[
f(\cos \ensuremath{\theta}) = \ensuremath{\sum}_{k=0}^\ensuremath{\infty} \check f_k \cos k \ensuremath{\theta}.
\]
Thus the coefficients can be recovered fast using FFT-based techniques.

We will also see the following:

\begin{theorem}[Chebyshev U are sin] For $x = \cos \ensuremath{\theta}$,
\[
U_n(x) = {\sin(n+1) \ensuremath{\theta} \over \sin \ensuremath{\theta}}
\]
which satisfy:
\begin{align*}
x U_0(x) &= U_1(x)/2 \\
x U_n(x) &= {U_{n-1}(x) \over 2} + {U_{n+1}(x) \over 2}.
\end{align*}
\end{theorem}
\textbf{Proof} Shown in the problem sheet. \ensuremath{\QED}

\subsection{Legendre}
\begin{definition}[Legendre] Legendre polynomials $P_n(x)$ are orthogonal polynomials with respect to $w(x) = 1$ on $[-1,1]$, with
\[
k_n = {1 \over 2^n} \begin{pmatrix} 2n \\ n \end{pmatrix} =
{(2n)! \over 2^n (n!)^2}
\]
\end{definition}

The reason for this complicated normalisation constant is both historical and that it leads to simpler formulae for recurrence relationships.

Classical orthogonal polynomials have \emph{Rodriguez formulae}, defining orthogonal polynomials as high order derivatives of simple functions. In this case we have:

\begin{lemma}[Legendre Rodriguez formula]
\[
P_n(x) = {1 \over (-2)^n n!}{{\rm d}^n \over {\rm d} x^n} (1-x^2)^n
\]
\end{lemma}
\textbf{Proof} We need to verify:

\begin{itemize}
\item[1. ] graded polynomials


\item[2. ] orthogonal to all lower degree polynomials on $[-1,1]$, and


\item[3. ] have the right normalisation constant $k_n = {1 \over 2^n} \begin{pmatrix} 2n \\ n \end{pmatrix}$.

\end{itemize}
(1) follows since its a degree $n$ polynomial (the $n$-th derivative of a degree $2n$ polynomial). (2) follows by integration by parts. Note that $(1-x^2)^n$ and its first $n-1$ derivatives vanish at $\ensuremath{\pm}1$. If $r_m$ is a degree $m < n$ polynomial we have:
\meeq{
\ensuremath{\int}_{-1}^1 {{\rm d}^n \over {\rm d} x^n} (1-x^2)^n r_m(x) {\rm d}x
= -\ensuremath{\int}_{-1}^1 {{\rm d}^{n-1} \over {\rm d} x^{n-1}} (1-x^2)^n r_m'(x) {\rm d}x  
\ccr =
\ensuremath{\cdots} = (-)^n \ensuremath{\int}_{-1}^1 (1-x^2)^n r_m^{(n)}(x) {\rm d}x = 0.
}
(3) follows since:
\begin{align*}
{{\rm d}^n \over {\rm d} x^n}[(-)^n x^{2n} + O(x^{2n-1})] &=
(-)^n 2n {{\rm d}^{n-1} \over {\rm d} x^{n-1}} x^{2n-1}+ O(x^{2n-1})] \\
&=
(-)^n 2n (2n-1) {{\rm d}^{n-2} \over {\rm d} x^{n-2}} x^{2n-2}+ O(x^{2n-2})] = \ensuremath{\cdots} \\
&= (-)^n 2n (2n-1) \ensuremath{\cdots} (n+1) x^n + O(x^{n-1}) \ccr 
=
(-)^n {(2n)! \over n!} x^n + O(x^{n-1})
\end{align*}
\ensuremath{\QED}

This allows us to determine the coefficients $k_n^{(\ensuremath{\lambda})}$ which are useful in proofs. In particular we will use $k_n^{(2)}$:

\begin{lemma}[Legendre monomial coefficients]
\begin{align*}
P_0(x) &= 1 \\
P_1(x) &= x \\
P_n(x) &= \underbrace{{(2n)! \over 2^n (n!)^2}}_{k_n} x^n - \underbrace{(2n-2)! \over 2^n (n-2)! (n-1)!}_{k_n^{(2)}} x^{n-2} + O(x^{n-4}).
\end{align*}
Here the $O(x^{n-4})$ is as $x \ensuremath{\rightarrow} \ensuremath{\infty}$, which implies that the term is a polynomial of degree $\ensuremath{\leq} n-4$. For $n = 2,3$ the $O(x^{n-4})$ term is therefore precisely zero.

\end{lemma}
\textbf{Proof}

The $n=0$ and $1$ case are immediate. For the other case we expand $(1-x^2)^n$ to get:
\begin{align*}
(-)^n {{\rm d}^n \over {\rm d} x^n} (1-x^2)^n &=
{{\rm d}^n \over {\rm d} x^n} [ x^{2n} - n x^{2n-2} + O(x^{2n-4})]\\
&= (2n)\ensuremath{\cdots}(2n-n+1) x^n - n (2n-2)\ensuremath{\cdots}(2n-2-n+1) x^{n-2} + O(x^{n-4}) \\
&= {(2n)! \over n!} x^n - {n (2n-2)! \over (n-2)!} x^{n-2} + O(x^{n-4})
\end{align*}
Multiplying through by ${1 \over 2^n (n!)}$ completes the derivation.

\ensuremath{\QED}

\begin{theorem}[Legendre 3-term recurrence]
\begin{align*}
xP_0(x) &= P_1(x) \\
(2n+1) xP_n(x) &= nP_{n-1}(x) + (n+1)P_{n+1}(x)
\end{align*}
\end{theorem}
\textbf{Proof} The $n = 0$ case is immediate (since $w(x) = w(-x)$ $a_n = 0$, from PS8). For the other cases we match terms:
\begin{align*}
(2n+1)xP_n(x) &- n P_{n-1}(x) - (n+1)P_{n+1}(x) \ccr
 = [(2n+1)k_n - (n+1) k_{n+1}] x^{n+1} \cr
 &\qquad  + [(2n+1) k_n^{(2)} -n k_{n-1} - (n+1) k_{n+1}^{(2)}] x^{n-1} + O(x^{n-3})
\end{align*}
Using the expressions for $k_n$ and $k_n^{(2)}$ above we have (leaving the manipulations as an exercise):
\meeq{
(2n+1)k_n - (n+1) k_{n+1} = {(2n+1)! \over 2^n (n!)^2} - (n+1) {(2n+2)! \over 2^{n+1} ((n+1)!)^2} = 0 \ccr
(2n+1) k_n^{(2)} -n k_{n-1}  - (n+1) k_{n+1}^{(2)} =  -(2n+1) {(2n-2)! \over 2^n (n-2)! (n-1)!} - n {(2n-2)! \over 2^{n-1} ((n-1)!)^2} \cr
&\qquad + (n+1){(2n)! \over 2^{n+1} (n-1)! n!} \ccr 
= 0
}
Thus
\[
(2n+1)xP_n(x) - n P_{n-1}(x) - (n+1)P_{n+1}(x) = O(x^{n-3})
\]
But as it is orthogonal to $P_k(x)$ for $0 \ensuremath{\leq} k \ensuremath{\leq} n-3$ it must be zero. \ensuremath{\QED}





\section{Gaussian Quadrature}
Quadrature is another name for numerical integration. In this section we see that a special quadrature rule can be constructed by using the roots of orthogonal polynomials, leading to a method that is exact for polynomials of twice the expected degree. Importantly, we can use quadrature to compute expansions in orthogonal polynomials that interpolate,  mirroring the link between the Trapezium rule, Fourier series, and interpolation but now for orthogonal polynomials.

\subsection{Interpolatory quadrature rules}
We begin by introducing a type of quadrature rule where one integrates an interpolatory polynomial exactly. This can be viewed as an extension of one-panel rectangle rules (which are degree 0 interpolants at a single point) and Trapezium rules (which are degree 1 interpolants at two points).  Using the Lagrange basis for interpolation we can write general interpolatory quadrature rules as a simple weighted sum:

\begin{definition}[interpolatory quadrature rule] Given a set of points $\ensuremath{\bm{\x}} = [x_1,\ensuremath{\ldots},x_n]$ the interpolatory quadrature rule is:
\[
\ensuremath{\Sigma}_n^{w,\ensuremath{\bm{\x}}}[f] := \ensuremath{\sum}_{j=1}^n w_j f(x_j)
\]
where
\[
w_j := \ensuremath{\int}_a^b \ensuremath{\ell}_j(x) w(x) {\rm d} x
\]
\end{definition}

\begin{proposition}[interpolatory quadrature is exact for polynomials]  Interpolatory quadrature is exact for all degree $n-1$ polynomials $p$:
\[
\ensuremath{\int}_a^b p(x) w(x) {\rm d}x = \ensuremath{\Sigma}_n^{w,\ensuremath{\bm{\x}}}[p]
\]
\end{proposition}
\textbf{Proof} The result follows since, by uniqueness of interpolatory polynomial, if $p$ is a polynomial then
\[
p(x) = \ensuremath{\sum}_{j=1}^n p(x_j) \ensuremath{\ell}_j(x)
\]
Hence
\[
\ensuremath{\int}_a^b p(x) w(x) {\rm d}x = \ensuremath{\sum}_{j=1}^n p(x_j) \int_a^b \ensuremath{\ell}_j(x) w(x) {\rm d}x = \ensuremath{\Sigma}_n^{w,\ensuremath{\bm{\x}}}[p].
\]
\ensuremath{\QED}

\begin{example}[3-point interpolatory quadrature] We find the interpolatory quadrature rule for $w(x) = 1$ on $[0,1]$ with  points $[x_1,x_2,x_3] = [0,1/4,1]$. We have:
\begin{align*}
w_1 = \int_0^1 w(x) \ensuremath{\ell}_1(x) {\rm d}x  = \int_0^1 {(x-1/4)(x-1) \over (-1/4)(-1)}{\rm d}x = -1/6 \\
w_2 = \int_0^1 w(x) \ensuremath{\ell}_2(x) {\rm d}x  = \int_0^1 {x(x-1) \over (1/4)(-3/4)}{\rm d}x = 8/9 \\
w_3 = \int_0^1 w(x) \ensuremath{\ell}_3(x) {\rm d}x  = \int_0^1 {x(x-1/4) \over 3/4}{\rm d}x = 5/18
\end{align*}
That is we have
\[
\ensuremath{\Sigma}_n^{w,\ensuremath{\bm{\x}}}[f]  = -{f(0) \over 6} + {8f(1/4) \over 9} + {5 f(1) \over 18}
\]
This is indeed exact for polynomials up to degree $2$ (and no more):
\[
\ensuremath{\Sigma}_n^{w,\ensuremath{\bm{\x}}}[1] = 1, \ensuremath{\Sigma}_n^{w,\ensuremath{\bm{\x}}}[x] = 1/2, \ensuremath{\Sigma}_n^{w,\ensuremath{\bm{\x}}}[x^2] = 1/3, \ensuremath{\Sigma}_n^{w,\ensuremath{\bm{\x}}}[x^3] = 7/24 \ensuremath{\neq} 1/4.
\]
\end{example}

\begin{example}[Chebyshev roots] We now find the interpolatory quadrature rule for $w(x) = 1/\sqrt{1-x^2}$ on $[-1,1]$ with points equal to the roots of $T_3(x)$. This is a special case of Gaussian quadrature which we will approach in another way below. We use:
\[
\int_{-1}^1 w(x) {\rm d}x = \ensuremath{\pi}, \int_{-1}^1 xw(x) {\rm d}x = 0, \int_{-1}^1 x^2 w(x) {\rm d}x = {\ensuremath{\pi}/2}
\]
Recall from before that $x_1,x_2,x_3 = \sqrt{3}/2,0,-\sqrt{3}/2$. Thus we have:
\begin{align*}
w_1 = \int_{-1}^1 w(x) \ensuremath{\ell}_1(x) {\rm d}x = \int_{-1}^1 {x(x+\sqrt{3}/2) \over (\sqrt{3}/2) \sqrt{3} \sqrt{1-x^2}}{\rm d}x = {\ensuremath{\pi} \over 3} \\
w_2 = \int_{-1}^1 w(x) \ensuremath{\ell}_2(x) {\rm d}x = \int_{-1}^1 {(x-\sqrt{3}/2)(x+\sqrt{3}/2) \over (-3/4)\sqrt{1-x^2}}{\rm d}x = {\ensuremath{\pi} \over 3} \\
w_3 = \int_{-1}^1 w(x) \ensuremath{\ell}_3(x) {\rm d}x = \int_{-1}^1 {(x-\sqrt{3}/2) x \over (-\sqrt{3})(-\sqrt{3}/2) \sqrt{1-x^2}}{\rm d}x = {\ensuremath{\pi} \over 3}
\end{align*}
(It's not a coincidence that they are all the same but this will differ for roots of other OPs.)  That is we have
\[
\ensuremath{\Sigma}_n^{w,\ensuremath{\bm{\x}}}[f]  = {\ensuremath{\pi} \over 3}(f(\sqrt{3}/2) + f(0) + f(-\sqrt{3}/2)
\]
This is indeed exact for polynomials up to degree $n-1=2$, but it goes all the way up to $2n-1 = 5$:
\begin{align*}
\ensuremath{\Sigma}_n^{w,\ensuremath{\bm{\x}}}[1] &= \ensuremath{\pi}, \ensuremath{\Sigma}_n^{w,\ensuremath{\bm{\x}}}[x] = 0, \ensuremath{\Sigma}_n^{w,\ensuremath{\bm{\x}}}[x^2] = {\ensuremath{\pi} \over 2}, \\
\ensuremath{\Sigma}_n^{w,\ensuremath{\bm{\x}}}[x^3] &= 0, \ensuremath{\Sigma}_n^{w,\ensuremath{\bm{\x}}}[x^4] &= {3 \ensuremath{\pi} \over 8}, \ensuremath{\Sigma}_n^{w,\ensuremath{\bm{\x}}}[x^5] = 0 \\
\ensuremath{\Sigma}_n^{w,\ensuremath{\bm{\x}}}[x^6] &= {9 \ensuremath{\pi} \over 32} \ensuremath{\neq} {5 \ensuremath{\pi} \over 16}
\end{align*}
We shall explain this miracle in the next chapter. \end{example}

\subsection{Roots of orthogonal polynomials and truncated Jacobi matrices}
We now consider roots (zeros) of orthogonal polynomials $p_n(x)$. This is important as we shall see they are useful for interpolation and quadrature. For interpolation to be well-defined we first need to guarantee that the roots are distinct.

\begin{lemma}[OP roots] An orthogonal polynomial $p_n(x)$ has exactly $n$ distinct roots.

\end{lemma}
\textbf{Proof}

Suppose $x_1, \ensuremath{\ldots},x_j$ are the roots where $q_n(x)$ changes sign, that is,
\[
p_n(x) = c_k (x-x_k)^{2p+1} + O((x-x_k)^{2p+2})
\]
for $c_k \ensuremath{\neq} 0$ and $k = 1,\ensuremath{\ldots},j$ and $p \ensuremath{\in} \ensuremath{\bbZ}$, as $x \ensuremath{\rightarrow} x_k$. Then
\[
p_n(x) (x-x_1) \ensuremath{\cdots}(x-x_j)
\]
does not change signs: it behaves like $c_k (x-x_k)^{2p+2} + O(x-x_k)^{2p+3}$ as $x \ensuremath{\rightarrow} x_k$. In other words:
\[
\ensuremath{\langle}p_n,(x-x_1) \ensuremath{\cdots}(x-x_j) \ensuremath{\rangle} = \int_a^b p_n(x) (x-x_1) \ensuremath{\cdots}(x-x_j) w(x) {\rm d} x \ensuremath{\neq} 0.
\]
where $w(x)$ is the weight of orthogonality. This is only possible if $j = n$ as $p_n(x)$ is orthogonal w.r.t. all lower degree polynomials.

\ensuremath{\QED}

We will now relate these roots to truncations of Jacobi matrices.

\begin{definition}[truncated Jacobi matrix] Given a Jacobi matrix $X$ associated with a family of orthonormal polynomials,  the \emph{truncated Jacobi matrix} is
\[
J_n := \begin{bmatrix} a_0 & b_0 \\
                         b_0 & \ensuremath{\ddots} & \ensuremath{\ddots} \\
                         & \ensuremath{\ddots} & a_{n-2} & b_{n-2} \\
                         && b_{n-2} & a_{n-1} \end{bmatrix} \ensuremath{\in} \ensuremath{\bbR}^{n \ensuremath{\times} n}
\]
\end{definition}

\begin{lemma}[OP roots and Jacobi matrices] The zeros $x_1, \ensuremath{\ldots},x_n$ of an orthonormal polynomial $q_n(x)$ are the eigenvalues of the truncated Jacobi matrix $J_n$. More precisely,
\[
J_n Q_n = Q_n \begin{bmatrix} x_1 \\ & \ensuremath{\ddots} \\ && x_n \end{bmatrix}
\]
for the orthogonal matrix
\[
Q_n = \underbrace{\begin{bmatrix}
q_0(x_1) & \ensuremath{\cdots} & q_0(x_n) \\
\ensuremath{\vdots}  & \ensuremath{\cdots} & \ensuremath{\vdots}  \\
q_{n-1}(x_1) & \ensuremath{\cdots} & q_{n-1}(x_n)
\end{bmatrix}}_{V_n^\ensuremath{\top}} \begin{bmatrix} \ensuremath{\alpha}_1^{-1} \\ & \ensuremath{\ddots} \\ && \ensuremath{\alpha}_n^{-1} \end{bmatrix}
\]
where $\ensuremath{\alpha}_j = \sqrt{q_0(x_j)^2 + \ensuremath{\cdots} + q_{n-1}(x_j)^2}$.

\end{lemma}
\textbf{Proof}

We construct the eigenvector (noting $b_{n-1} q_n(x_j) = 0$):
\[
J_n \begin{bmatrix} q_0(x_j) \\ \ensuremath{\vdots} \\ q_{n-1}(x_j) \end{bmatrix} =
\begin{bmatrix} a_0 q_0(x_j) + b_0 q_1(x_j) \\
 b_0 q_0(x_j) + a_1 q_1(x_j) + b_1 q_2(x_j) \\
\ensuremath{\vdots} \\
b_{n-3} q_{n-3}(x_j) + a_{n-2} q_{n-2}(x_j) + b_{n-2} q_{n-1}(x_j) \\
b_{n-2} q_{n-2}(x_j) + a_{n-1} q_{n-1}(x_j) + b_{n-1} q_n(x_j)
\end{bmatrix} = x_j \begin{bmatrix} q_0(x_j) \\
 q_1(x_j) \\
\ensuremath{\vdots} \\
q_{n-1}(x_j)
\end{bmatrix}
\]
The result follows from normalising the eigenvectors. Since $J_n$ is symmetric the eigenvector matrix is orthogonal.

\ensuremath{\QED}

\begin{example}[Chebyshev roots] Consider $T_n(x) = \cos n {\rm acos}\, x$. The roots  are $x_j = \cos \ensuremath{\theta}_j$ where $\ensuremath{\theta}_j = (j-1/2)\ensuremath{\pi}/n$ for $j = 1,\ensuremath{\ldots},n$ are the roots of $\cos n \ensuremath{\theta}$ that are inside $[0,\ensuremath{\pi}]$. 

Consider the $n = 3$ case where we have
\[
x_1,x_2,x_3 = \cos(\ensuremath{\pi}/6),\cos(\ensuremath{\pi}/2),\cos(5\ensuremath{\pi}/6) = \sqrt{3}/2,0,-\sqrt{3/2}
\]
We also have from the 3-term recurrence:
\begin{align*}
T_0(x) = 1 \\
T_1(x) = x \\
T_2(x) = 2x T_1(x) - T_0(x) = 2x^2-1 \\
T_3(x) = 2x T_2(x) - T_1(x) = 4x^3-3x
\end{align*}
We orthonormalise by rescaling
\begin{align*}
q_0(x) &= 1/\sqrt{\ensuremath{\pi}} \\
q_k(x) &= T_k(x) \sqrt{2}/\sqrt{\ensuremath{\pi}}
\end{align*}
so that the Jacobi matrix is symmetric:
\[
x [q_0(x)|q_1(x)|\ensuremath{\cdots}] = [q_0(x)|q_1(x)|\ensuremath{\cdots}] \underbrace{\begin{bmatrix} 0 & 1/\sqrt{2} \\
                            1/\sqrt{2} & 0 & 1/2 \\
                            &1/2 & 0 & 1/2 \\
                             &   & 1/2 & 0 & \ensuremath{\ddots} \\
                              &  && \ensuremath{\ddots} & \ensuremath{\ddots}
\end{bmatrix}}_X
\]
We can then confirm that we have constructed an eigenvector/eigenvalue of the $3 \ensuremath{\times} 3$ truncation of the Jacobi matrix, e.g. at $x_2 = 0$:
\[
\begin{bmatrix} 
0 & 1/\sqrt{2} \\
1/\sqrt{2} & 0 & 1/2 \\
    & 1/2 & 0\end{bmatrix} \begin{bmatrix} q_0(0) \\ q_1(0) \\ q_2(0) 
    \end{bmatrix} = {1 \over \sqrt \ensuremath{\pi}} \begin{bmatrix} 
0 & 1/\sqrt{2} \\
1/\sqrt{2} & 0 & 1/2 \\
    & 1/2 & 0\end{bmatrix} \begin{bmatrix} 1 \\ 0 \\ -{\sqrt{2}}
    \end{bmatrix} =\begin{bmatrix} 0 \\ 0 \\ 0
    \end{bmatrix}
\]
\end{example}

\subsection{Gaussian quadrature}
We now introduce Gaussian quadrature, which we shall see is exact for polynomials up to degree $2n-1$, i.e., double the degree of other interpolatory quadrature rules from other grids. We will also prove that it is in fact an interpolatory quadrature rule corresponding to the grid $x_j$ defined as the roots of the orthonormal polynomial $q_n(x)$.

\begin{definition}[Gaussian quadrature] Given a weight $w(x)$, the Gauss quadrature rule is:
\[
\ensuremath{\int}_a^b f(x)w(x) {\rm d}x \ensuremath{\approx} \underbrace{\ensuremath{\sum}_{j=1}^n w_j f(x_j)}_{\ensuremath{\Sigma}_n^w[f]}
\]
where $x_1,\ensuremath{\ldots},x_n$ are the roots of the orthonormal polynomials $q_n(x)$ and 
\[
w_j := {1 \over \ensuremath{\alpha}_j^2} = {1 \over q_0(x_j)^2 + \ensuremath{\cdots} + q_{n-1}(x_j)^2}.
\]
Equivalentally, $x_1,\ensuremath{\ldots},x_n$ are the eigenvalues of $J_n$ and
\[
w_j = \ensuremath{\int}_a^b w(x) {\rm d}x Q_n[1,j]^2.
\]
(Note we have $\ensuremath{\int}_a^b w(x) {\rm d} x q_0(x)^2 = 1$.) \end{definition}

In analogy to how Fourier series are orthogonal with respect to the Trapezium rule, Orthogonal polynomials are orthogonal with respect to Gaussian quadrature:

\begin{lemma}[Discrete orthogonality] For $0 \ensuremath{\leq} \ensuremath{\ell},m \ensuremath{\leq} n-1$, the orthonormal polynomials $q_n(x)$ satisfy
\[
\ensuremath{\Sigma}_n^w[q_\ensuremath{\ell} q_m] = \ensuremath{\delta}_{\ensuremath{\ell}m}
\]
\end{lemma}
\textbf{Proof}
\[
\ensuremath{\Sigma}_n^w[q_\ensuremath{\ell} q_m] = \ensuremath{\sum}_{j=1}^n {q_\ensuremath{\ell}(x_j) q_m(x_j) \over \ensuremath{\alpha}_j^2}
= \left[q_\ensuremath{\ell}(x_1)/ \ensuremath{\alpha}_1 | \ensuremath{\cdots} | {q_\ensuremath{\ell}(x_n)/ \ensuremath{\alpha}_n}\right] 
\begin{bmatrix}
q_m(x_1)/\ensuremath{\alpha}_1 \\
\ensuremath{\vdots} \\
q_m(x_n)/\ensuremath{\alpha}_n \end{bmatrix} = \ensuremath{\bm{\e}}_\ensuremath{\ell} Q_n Q_n^\ensuremath{\top} \ensuremath{\bm{\e}}_m = \ensuremath{\delta}_{\ensuremath{\ell}m}
\]
\ensuremath{\QED}

Just as approximating Fourier coefficients using Trapezium rule gives a way of interpolating at the grid, so does Gaussian quadrature:

\begin{theorem}[interpolation via quadrature] For the orthonormal polynomials $q_n(x)$,
\[
f_n(x) := \ensuremath{\sum}_{k=0}^{n-1} c_k^n q_k(x)\hbox{ for } c_k^n := \ensuremath{\Sigma}_n^w[f q_k]
\]
interpolates $f(x)$ at the Gaussian quadrature points $x_1,\ensuremath{\ldots},x_n$.

\end{theorem}
\textbf{Proof}

Consider the Vandermonde-like matrix from above:
\[
V_n := \begin{bmatrix} q_0(x_1) & \ensuremath{\cdots} & q_{n-1}(x_1) \\
                \ensuremath{\vdots} & \ensuremath{\ddots} & \ensuremath{\vdots} \\
                q_0(x_n) & \ensuremath{\cdots} & q_{n-1}(x_n) \end{bmatrix}
\]
and define
\[
Q_n^w := V_n^\ensuremath{\top} \begin{bmatrix} w_1 \\ &\ensuremath{\ddots} \\&& w_n \end{bmatrix} = \begin{bmatrix} q_0(x_1)w_1 & \ensuremath{\cdots} &  q_0(x_n) w_n \\
                \ensuremath{\vdots} & \ensuremath{\ddots} & \ensuremath{\vdots} \\
                q_{n-1}(x_1) w_1 & \ensuremath{\cdots} & q_{n-1}(x_n)w_n \end{bmatrix}
\]
so that
\[
\begin{bmatrix}
c_0^n \\
\ensuremath{\vdots} \\
c_{n-1}^n \end{bmatrix} = Q_n^w \begin{bmatrix} f(x_1) \\ \ensuremath{\vdots} \\ f(x_n) \end{bmatrix}.
\]
Note that if $p(x) = [q_0(x) | \ensuremath{\cdots} | q_{n-1}(x)] \ensuremath{\bm{\c}}$ then
\[
\begin{bmatrix}
p(x_1) \\
\ensuremath{\vdots} \\
p(x_n)
\end{bmatrix} = V_n \ensuremath{\bm{\c}}
\]
But we see that (similar to the Fourier case)
\[
Q_n^w V_n = \begin{bmatrix} \ensuremath{\Sigma}_n^w[q_0 q_0] & \ensuremath{\cdots} & \ensuremath{\Sigma}_n^w[q_0 q_{n-1}]\\
                \ensuremath{\vdots} & \ensuremath{\ddots} & \ensuremath{\vdots} \\
                \ensuremath{\Sigma}_n^w[q_{n-1} q_0] & \ensuremath{\cdots} & \ensuremath{\Sigma}_n^w[q_{n-1} q_{n-1}]
                \end{bmatrix} = I_n
\]
\ensuremath{\QED}

\begin{example}[Chebyshev expansions]  Consider the construction of Gaussian quadrature associated with the Chebyshev weight for $n = 3$.  To determine the weights we need:
\[
w_j^{-1} = \ensuremath{\alpha}_j^2 = q_0(x_j)^2 + q_1(x_j)^2 + q_2(x_j)^2 = 
{1 \over \ensuremath{\pi}} + {2 \over \ensuremath{\pi}} x_j^2 + {2 \over \ensuremath{\pi}} (2x_j^2-1)^2
\]
We can check each case and deduce that $w_j = \ensuremath{\pi}/3$. Thus we recover the interpolatory quadrature rule. Further, we can construct the transform
\begin{align*}
Q_3^w &= \begin{bmatrix}
w_1 q_0(x_1) & w_2 q_0(x_2) & w_3 q_0(x_3) \\
w_1 q_1(x_1) & w_2 q_1(x_2) & w_3 q_1(x_3) \\
w_1 q_3(x_1) & w_2 q_3(x_2) & w_3 q_3(x_3) 
\end{bmatrix}\\
&= {\ensuremath{\pi} \over 3} \begin{bmatrix} 1/\sqrt{\ensuremath{\pi}} & 1/\sqrt{\ensuremath{\pi}} & 1/\sqrt{\ensuremath{\pi}} \\
                                x_1\sqrt{2/\ensuremath{\pi}} & x_2\sqrt{2/\ensuremath{\pi}} & x_3\sqrt{2/\ensuremath{\pi}} \\
                                (2x_1^2-1)\sqrt{2/\ensuremath{\pi}} &(2x_2^2-1)\sqrt{2/\ensuremath{\pi}} & (2x_3^2-1)\sqrt{2/\ensuremath{\pi}}
                                \end{bmatrix} \\
                                &= 
                                {\sqrt{\ensuremath{\pi}} \over 3} \begin{bmatrix} 1 & 1 & 1 \\
                                \sqrt{6}/2 & 0 & -\sqrt{6}/2 \\
                                1/\sqrt{2} &-\sqrt{2} & 1/\sqrt{2}
                                \end{bmatrix}
\end{align*}
We can use this to expand a polynomial, e.g. $x^2$:
\[
Q_3^w \begin{bmatrix}
x_1^2 \\
x_2^2 \\
x_3^2 
\end{bmatrix} = {\sqrt{\ensuremath{\pi}} \over 3} 
\begin{bmatrix} 1 & 1 & 1 \\
\sqrt{6}/2 & 0 & -\sqrt{6}/2 \\
1/\sqrt{2} &-\sqrt{2} & 1/\sqrt{2}
\end{bmatrix} 
\begin{bmatrix} 3/4 \\ 0 \\ 3/4 \end{bmatrix} =
\begin{bmatrix}
{\sqrt{\ensuremath{\pi}} / 2} \\
0 \\
{\sqrt{\ensuremath{\pi}} / (2\sqrt{2})}
\end{bmatrix}
\]
In other words:
\[
x^2 = {\sqrt \ensuremath{\pi} \over 2} q_0(x) + {\sqrt \ensuremath{\pi} \over 2\sqrt 2} q_2(x) = {1 \over 2} T_0(x) + {1 \over 2} T_2(x)
\]
which can be easily confirmed. \end{example}

\begin{corollary}[Gaussian quadrature is interpolatory] Gaussian quadrature is an interpolatory quadrature rule with the interpolation points equal to the roots of $q_n$:
\[
\ensuremath{\Sigma}_n^w[f] = \ensuremath{\int}_a^b f_n(x) w(x) {\rm d}x 
\]
\end{corollary}
\textbf{Proof} We want to show that its the same as integrating the interpolatory polynomial:
\[
\int_a^b f_n(x) w(x) {\rm d}x = {1 \over q_0(x)} \sum_{k=0}^{n-1} c_k^n \int_a^b q_k(x) q_0(x) w(x) {\rm d}x
= {c_0^n \over q_0} = \ensuremath{\Sigma}_n^w[f].
\]
\ensuremath{\QED}

\begin{example}[Chebyshev quadrature via Lagrange polynomials] The connection with interpolatory quadrature means theres another way to compute the quadrature: integrate the Lagrange polynomials associated with the zeros. For example, with $n = 3$ recall that the roots of $T_3(x)$ are $\ensuremath{\pm}\sqrt{3}/2$ and $0$. Thus we have
\begin{align*}
\ensuremath{\ell}_1(x) &= x (x +\sqrt{3}/2) / (3/2) = {(2x^2+\sqrt{3}x) \over 3} \\
\ensuremath{\ell}_2(x) &= (x - \sqrt{3}/2) (x +\sqrt{3}/2) / (-3/4) = -{4 \over 3} x^2 + 1 \\
\ensuremath{\ell}_3(x) &=(x -\sqrt{3}/2) x / (3/2) = {(2x^2-\sqrt{3}x) \over 3}
\end{align*}
A quick check confirms that
\[
w_j = \ensuremath{\int}_{-1}^1 {\ensuremath{\ell}_j(x) \over \sqrt{1-x^2}} {\rm d}x = {\ensuremath{\pi} \over 3}.
\]
\end{example}

A consequence of being an interpolatory quadrature rule is that it is exact for all polynomials of degree $n-1$. The \emph{miracle} of Gaussian quadrature is it is exact for twice as many!

\begin{theorem}[Exactness of Gauss quadrature] If $p(x)$ is a degree $2n-1$ polynomial then Gauss quadrature is exact:
\[
\ensuremath{\int}_a^b p(x)w(x) {\rm d}x = \ensuremath{\Sigma}_n^w[p].
\]
\end{theorem}
\textbf{Proof} Using polynomial division algorithm (e.g. by matching terms) we can write
\[
p(x) = q_n(x) s(x) + r(x)
\]
where $s$ and $r$ are degree $n-1$ and $q_n(x)$ is the degree $n$ orthonormal polynomial. Then we have:
\begin{align*}
\ensuremath{\Sigma}_n^w[p] &= \underbrace{\ensuremath{\Sigma}_n^w[q_n s]}_{\hbox{$0$ since evaluating $q_n$ at zeros}} + \ensuremath{\Sigma}_n^w[r] = \ensuremath{\int}_a^b r(x) w(x) {\rm d}x\\
&= \underbrace{\ensuremath{\int}_a^b q_n(x)s(x) w(x) {\rm d}x}_{\hbox{$0$ since $s$ is degree $<n$}}  + \ensuremath{\int}_a^b r(x) w(x) {\rm d}x \\
&= \ensuremath{\int}_a^b p(x)w(x) {\rm d}x.
\end{align*}
\ensuremath{\QED}

\begin{example}[Double exactness] We are exact for all polynomials of degree $2n-1$, so for our $n = 3$ rule consider integrating $x^5$. We correctly get:
\[
\ensuremath{\Sigma}_n^w[x^5] = {\ensuremath{\pi} \over 3} \left( {9 \sqrt{3} \over 32}  - {9 \sqrt{3} \over 2} \right) = 0.
\]
We are also correct for $x^4$:
\[
\ensuremath{\Sigma}_n^w[x^4] = {\ensuremath{\pi} \over 3} \left( {9 \over 16} + {9 \over 16} \right) = {3 \ensuremath{\pi} \over 8}.
\]
However,
\[
\ensuremath{\Sigma}_n^w[x^6] = {9 \ensuremath{\pi} \over 64} \ensuremath{\neq} {5 \ensuremath{\pi} \over 16}
\]
hence it is incorrect for larger degree polynomials. \end{example}






\appendix

\chapter{Asymptotics and Computational Cost}

We introduce Big-O, little-o and asymptotic notation and see how they can be used to describe computational cost. 

\section{Asymptotics as $n \ensuremath{\rightarrow} \ensuremath{\infty}$}
Big-O, little-o, and \ensuremath{\ldq}asymptotic to" are used to describe behaviour of functions at infinity. 

\begin{definition}[Big-O] 
\[
f(n) = O(\ensuremath{\phi}(n)) \qquad \hbox{(as $n \ensuremath{\rightarrow} \ensuremath{\infty}$)}
\]
means $\left|{f(n) \over \ensuremath{\phi}(n)}\right|$ is bounded for sufficiently large $n$. That is, there exist constants $C$ and $N_0$ such  that, for all $n \geq N_0$, $|{f(n) \over \ensuremath{\phi}(n)}| \leq C$. \end{definition}

\begin{definition}[little-O] 
\[
f(n) = o(\ensuremath{\phi}(n)) \qquad \hbox{(as $n \ensuremath{\rightarrow} \ensuremath{\infty}$)}
\]
means $\lim_{n \ensuremath{\rightarrow} \ensuremath{\infty}} {f(n) \over \ensuremath{\phi}(n)} = 0.$ \end{definition}

\begin{definition}[asymptotic to] 
\[
f(n) \ensuremath{\sim} \ensuremath{\phi}(n) \qquad \hbox{(as $n \ensuremath{\rightarrow} \ensuremath{\infty}$)}
\]
means $\lim_{n \ensuremath{\rightarrow} \ensuremath{\infty}} {f(n) \over \ensuremath{\phi}(n)} = 1.$ \end{definition}

\begin{example}[asymptotics with $n$]

\begin{itemize}
\item[1. ] \[
{\cos n \over n^2 -1} = O(n^{-2})
\]
as

\end{itemize}
\[
\left|{{\cos n \over n^2 -1} \over n^{-2}} \right| \leq \left| n^2 \over n^2 -1 \right|  \leq 2
\]
for $n \geq N_0 = 2$.

\begin{itemize}
\item[2. ] \[
\log n = o(n)
\]
as $\lim_{n \ensuremath{\rightarrow} \ensuremath{\infty}} {\log n \over n} = 0.$


\item[3. ] \[
n^2 + 1 \ensuremath{\sim} n^2
\]
as ${n^2 +1 \over n^2} \ensuremath{\rightarrow} 1.$

\end{itemize}
\end{example}

Note we sometimes write $f(O(\ensuremath{\phi}(n)))$ for a function of the form $f(g(n))$ such that $g(n) = O(\ensuremath{\phi}(n))$.

We have some simple algebraic rules:

\begin{proposition}[Big-O rules]
\begin{align*}
O(\ensuremath{\phi}(n))O(\ensuremath{\psi}(n)) = O(\ensuremath{\phi}(n)\ensuremath{\psi}(n))  \qquad \hbox{(as $n \ensuremath{\rightarrow} \ensuremath{\infty}$)} \\
O(\ensuremath{\phi}(n)) + O(\ensuremath{\psi}(n)) = O(|\ensuremath{\phi}(n)| + |\ensuremath{\psi}(n)|)  \qquad \hbox{(as $n \ensuremath{\rightarrow} \ensuremath{\infty}$)}.
\end{align*}
\end{proposition}
\textbf{Proof} See any standard book on asymptotics, eg \href{https://www.taylorfrancis.com/books/mono/10.1201/9781439864548/asymptotics-special-functions-frank-olver}{F.W.J. Olver, Asymptotics and Special Functions}. \ensuremath{\QED}

\section{Asymptotics as $x \ensuremath{\rightarrow} x_0$}
We also have Big-O, little-o and "asymptotic to" at a point:

\begin{definition}[Big-O] 
\[
f(x) = O(\ensuremath{\phi}(x)) \qquad \hbox{(as $x \ensuremath{\rightarrow} x_0$)}
\]
means $|{f(x) \over \ensuremath{\phi}(x)}|$ is bounded in a neighbourhood of $x_0$. That is, there exist constants $C$ and $r$ such  that, for all $0 \leq |x - x_0| \leq r$, $|{f(x) \over \ensuremath{\phi}(x)}| \leq C$. \end{definition}

\begin{definition}[little-O] 
\[
f(x) = o(\ensuremath{\phi}(x)) \qquad \hbox{(as $x \ensuremath{\rightarrow} x_0$)}
\]
means $\lim_{x \ensuremath{\rightarrow} x_0} {f(x) \over \ensuremath{\phi}(x)} = 0.$ \end{definition}

\begin{definition}[asymptotic to] 
\[
f(x) \ensuremath{\sim} \ensuremath{\phi}(x) \qquad \hbox{(as $x \ensuremath{\rightarrow} x_0$)}
\]
means $\lim_{x \ensuremath{\rightarrow} x_0} {f(x) \over \ensuremath{\phi}(x)} = 1.$ \end{definition}

\begin{example}[asymptotics with $x$]
\[
\exp x = 1 + x + O(x^2) \qquad \hbox{as $x \ensuremath{\rightarrow} 0$}
\]
since $\exp x = 1 + x + {\exp t \over 2} x^2$ for some $t \in [0,x]$ and
\[
\left|{{\exp t \over 2} x^2 \over x^2}\right| \leq {3 \over 2}
\]
provided $x \leq 1$. \end{example}

\section{Computational cost}
We will use Big-O notation to describe the computational cost of algorithms. Consider the following simple sum
\[
\sum_{k=1}^n x_k^2
\]
which we might implement as:

\begin{verbatim}
function sumsq(x)
    n = length(x)
    ret = 0.0
    for k = 1:n
        ret = ret + x[k]^2
    end
    ret
end
\end{verbatim}
Each step of this algorithm consists of one memory look-up (\texttt{z = x[k]}), one multiplication (\texttt{w = z*z}) and one addition (\texttt{ret = ret + w}). We will ignore the memory look-up in the following discussion. The number of CPU operations per step is therefore 2 (the addition and multiplication). Thus the total number of CPU operations is $2n$. But the constant $2$ here is misleading: we didn't count the memory look-up, thus it is more sensible to just talk about the asymptotic complexity, that is, the \emph{computational cost} is $O(n)$.

Now consider a double sum like:
\[
\sum_{k=1}^n \sum_{j=1}^k x_j^2
\]
which we might implement as:

\begin{verbatim}
function sumsq2(x)
    n = length(x)
    ret = 0.0
    for k = 1:n
        for j = 1:k
            ret = ret + x[j]^2
        end
    end
    ret
end
\end{verbatim}
Now the inner loop is $O(1)$ operations (we don't try to count the precise number), which we do $k$ times for $O(k)$ operations as $k \ensuremath{\rightarrow} \ensuremath{\infty}$. The outer loop therefore takes
\[
\ensuremath{\sum}_{k = 1}^n O(k) = O\left(\ensuremath{\sum}_{k = 1}^n k\right) = O\left( {n (n+1) \over 2} \right) = O(n^2)
\]
operations.





\chapter{Permutation Matrices}

Permutation matrices are matrices that represent the action of permuting the entries of a vector, that is, matrix representations of the symmetric group $S_n$, acting on $\ensuremath{\bbR}^n$. Recall every $\ensuremath{\sigma} \ensuremath{\in} S_n$ is a bijection between $\{1,2,\ensuremath{\ldots},n\}$ and itself. We can write a permutation $\ensuremath{\sigma}$ in \emph{Cauchy notation}:
\[
\begin{pmatrix}
 1 & 2 & 3 & \ensuremath{\cdots} & n \cr
 \ensuremath{\sigma}_1 & \ensuremath{\sigma}_2 & \ensuremath{\sigma}_3 & \ensuremath{\cdots} & \ensuremath{\sigma}_n
 \end{pmatrix}
\]
where $\{\ensuremath{\sigma}_1,\ensuremath{\ldots},\ensuremath{\sigma}_n\} = \{1,2,\ensuremath{\ldots},n\}$ (that is, each integer appears precisely once). We denote the \emph{inverse permutation} by $\ensuremath{\sigma}^{-1}$, which can be constructed by swapping the rows of the Cauchy notation and reordering.

We can encode a permutation in vector $\mathbf \ensuremath{\sigma} = [\ensuremath{\sigma}_1,\ensuremath{\ldots},\ensuremath{\sigma}_n]$.  This induces an action on a vector (using indexing notation)
\[
\ensuremath{\bm{\v}}[\mathbf \ensuremath{\sigma}] = \begin{bmatrix}v_{\ensuremath{\sigma}_1}\\ \vdots \\ v_{\ensuremath{\sigma}_n} \end{bmatrix}
\]
\begin{example}[permutation of a vector]  Consider the permutation $\ensuremath{\sigma}$ given by
\[
\begin{pmatrix}
 1 & 2 & 3 & 4 & 5 \cr
 1 & 4 & 2 & 5 & 3
 \end{pmatrix}
\]
We can apply it to a vector:


\begin{lstlisting}
(*@\HLJLk{using}@*) (*@\HLJLn{LinearAlgebra}@*)
(*@\HLJLn{\ensuremath{\sigma}}@*) (*@\HLJLoB{=}@*) (*@\HLJLp{[}@*)(*@\HLJLni{1}@*)(*@\HLJLp{,}@*) (*@\HLJLni{4}@*)(*@\HLJLp{,}@*) (*@\HLJLni{2}@*)(*@\HLJLp{,}@*) (*@\HLJLni{5}@*)(*@\HLJLp{,}@*) (*@\HLJLni{3}@*)(*@\HLJLp{]}@*)
(*@\HLJLn{v}@*) (*@\HLJLoB{=}@*) (*@\HLJLp{[}@*)(*@\HLJLni{6}@*)(*@\HLJLp{,}@*) (*@\HLJLni{7}@*)(*@\HLJLp{,}@*) (*@\HLJLni{8}@*)(*@\HLJLp{,}@*) (*@\HLJLni{9}@*)(*@\HLJLp{,}@*) (*@\HLJLni{10}@*)(*@\HLJLp{]}@*)
(*@\HLJLn{v}@*)(*@\HLJLp{[}@*)(*@\HLJLn{\ensuremath{\sigma}}@*)(*@\HLJLp{]}@*) (*@\HLJLcs{{\#}}@*) (*@\HLJLcs{we}@*) (*@\HLJLcs{permutate}@*) (*@\HLJLcs{entries}@*) (*@\HLJLcs{of}@*) (*@\HLJLcs{v}@*)
\end{lstlisting}

\begin{lstlisting}
5-element Vector(*@{{\{}}@*)Int64(*@{{\}}}@*):
  6
  9
  7
 10
  8
\end{lstlisting}


Its inverse permutation $\ensuremath{\sigma}^{-1}$ has Cauchy notation coming from swapping the rows of the Cauchy notation of $\ensuremath{\sigma}$ and sorting:
\[
\begin{pmatrix}
 1 & 4 & 2 & 5 & 3 \cr
 1 & 2 & 3 & 4 & 5
 \end{pmatrix} \rightarrow \begin{pmatrix}
 1 & 2 & 4 & 3 & 5 \cr
 1 & 3 & 2 & 5 & 4
 \end{pmatrix} 
\]
\end{example}

Note that the operator
\[
P_\ensuremath{\sigma}(\ensuremath{\bm{\v}}) = \ensuremath{\bm{\v}}[{\mathbf \ensuremath{\sigma}}]
\]
is linear in $\ensuremath{\bm{\v}}$, therefore, we can identify it with a matrix whose action is:
\[
P_\ensuremath{\sigma} \begin{bmatrix} v_1\\ \vdots \\ v_n \end{bmatrix} = \begin{bmatrix}v_{\ensuremath{\sigma}_1} \\ \vdots \\ v_{\ensuremath{\sigma}_n}  \end{bmatrix}.
\]
The entries of this matrix are
\[
P_\ensuremath{\sigma}[k,j] = \ensuremath{\bm{\e}}_k^\ensuremath{\top} P_\ensuremath{\sigma} \ensuremath{\bm{\e}}_j = \ensuremath{\bm{\e}}_k^\ensuremath{\top} \ensuremath{\bm{\e}}_{\ensuremath{\sigma}^{-1}_j} = \ensuremath{\delta}_{k,\ensuremath{\sigma}^{-1}_j} = \ensuremath{\delta}_{\ensuremath{\sigma}_k,j}
\]
where $\ensuremath{\delta}_{k,j}$ is the \emph{Kronecker delta}:
\[
\ensuremath{\delta}_{k,j} := \begin{cases} 1 & k = j \\
                        0 & \hbox{otherwise}
                        \end{cases}.
\]
This construction motivates the following definition:

\begin{definition}[permutation matrix] $P \in \ensuremath{\bbR}^{n \ensuremath{\times} n}$ is a permutation matrix if it is equal to the identity matrix with its rows permuted. \end{definition}

\begin{proposition}[permutation matrix inverse]  Let $P_\ensuremath{\sigma}$ be a permutation matrix corresponding to the permutation $\ensuremath{\sigma}$. Then
\[
P_\ensuremath{\sigma}^\ensuremath{\top} = P_{\ensuremath{\sigma}^{-1}} = P_\ensuremath{\sigma}^{-1}
\]
That is, $P_\ensuremath{\sigma}$ is \emph{orthogonal}:
\[
P_\ensuremath{\sigma}^\ensuremath{\top} P_\ensuremath{\sigma} = P_\ensuremath{\sigma} P_\ensuremath{\sigma}^\ensuremath{\top} = I.
\]
\end{proposition}
\textbf{Proof}

We prove orthogonality via:
\[
\ensuremath{\bm{\e}}_k^\ensuremath{\top} P_\ensuremath{\sigma}^\ensuremath{\top} P_\ensuremath{\sigma} \ensuremath{\bm{\e}}_j = (P_\ensuremath{\sigma} \ensuremath{\bm{\e}}_k)^\ensuremath{\top} P_\ensuremath{\sigma} \ensuremath{\bm{\e}}_j = \ensuremath{\bm{\e}}_{\ensuremath{\sigma}^{-1}_k}^\ensuremath{\top} \ensuremath{\bm{\e}}_{\ensuremath{\sigma}^{-1}_j} = \ensuremath{\delta}_{k,j}
\]
This shows $P_\ensuremath{\sigma}^\ensuremath{\top} P_\ensuremath{\sigma} = I$ and hence $P_\ensuremath{\sigma}^{-1} = P_\ensuremath{\sigma}^\ensuremath{\top}$. 

\ensuremath{\QED}







\end{document}