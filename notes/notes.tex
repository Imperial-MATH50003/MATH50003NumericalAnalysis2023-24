\documentclass[12pt,a4paper]{book}

\usepackage[a4paper,text={16.5cm,25.2cm},centering]{geometry}
\usepackage{lmodern}
\usepackage{amssymb,amsmath}
\usepackage{bm}
\usepackage{graphicx}
\usepackage{microtype}
\usepackage{hyperref}
\setlength{\parindent}{0pt}
\setlength{\parskip}{1.2ex}
\let\QED=\blacksquare

\hypersetup
       {   pdfauthor = { {{Sheehan Olver}} },
           pdftitle={ {{MATH50003 Numerical Analysis}} },
           colorlinks=TRUE,
           linkcolor=black,
           citecolor=blue,
           urlcolor=blue
       }

\title{ MATH50003 Numerical Analysis }


\newtheorem{lemma}{Lemma}
\newtheorem{theorem}{Theorem}


\author{ Sheehan Olver }
\renewcommand{\thechapter}{\Roman{chapter}}


\def\addtab#1={#1\;&=}

\def\meeq#1{\def\ccr{\\\addtab}
%\tabskip=\@centering
 \begin{align*}
 \addtab#1
 \end{align*}
  }  
  
  \def\leqaddtab#1\leq{#1\;&\leq}
  \def\mleeq#1{\def\ccr{\\\addtab}
%\tabskip=\@centering
 \begin{align*}
 \leqaddtab#1
 \end{align*}
  }  


\def\vc#1{\mbox{\boldmath$#1$\unboldmath}}

\def\vcsmall#1{\mbox{\boldmath$\scriptstyle #1$\unboldmath}}

\def\vczero{{\mathbf 0}}


%\def\beginlist{\begin{itemize}}
%
%\def\endlist{\end{itemize}}


\def\pr(#1){\left({#1}\right)}
\def\br[#1]{\left[{#1}\right]}
\def\fbr[#1]{\!\left[{#1}\right]}
\def\set#1{\left\{{#1}\right\}}
\def\ip<#1>{\left\langle{#1}\right\rangle}
\def\iip<#1>{\left\langle\!\langle{#1}\right\rangle\!\rangle}

\def\norm#1{\left\| #1 \right\|}

\def\abs#1{\left|{#1}\right|}
\def\fpr(#1){\!\pr({#1})}

\def\Re{{\rm Re}\,}
\def\Im{{\rm Im}\,}

\def\floor#1{\left\lfloor#1\right\rfloor}
\def\ceil#1{\left\lceil#1\right\rceil}


\def\mapengine#1,#2.{\mapfunction{#1}\ifx\void#2\else\mapengine #2.\fi }

\def\map[#1]{\mapengine #1,\void.}

\def\mapenginesep_#1#2,#3.{\mapfunction{#2}\ifx\void#3\else#1\mapengine #3.\fi }

\def\mapsep_#1[#2]{\mapenginesep_{#1}#2,\void.}


\def\vcbr{\br}


\def\bvect[#1,#2]{
{
\def\dots{\cdots}
\def\mapfunction##1{\ | \  ##1}
\begin{pmatrix}
		 \,#1\map[#2]\,
\end{pmatrix}
}
}

\def\vect[#1]{
{\def\dots{\ldots}
	\vcbr[{#1}]
}}

\def\vectt[#1]{
{\def\dots{\ldots}
	\vect[{#1}]^{\top}
}}

\def\Vectt[#1]{
{
\def\mapfunction##1{##1 \cr} 
\def\dots{\vdots}
	\begin{pmatrix}
		\map[#1]
	\end{pmatrix}
}}



\def\thetaB{\mbox{\boldmath$\theta$}}
\def\zetaB{\mbox{\boldmath$\zeta$}}


\def\newterm#1{{\it #1}\index{#1}}


\def\TT{{\mathbb T}}
\def\C{{\mathbb C}}
\def\R{{\mathbb R}}
\def\II{{\mathbb I}}
\def\F{{\mathcal F}}
\def\E{{\rm e}}
\def\I{{\rm i}}
\def\D{{\rm d}}
\def\dx{\D x}
\def\ds{\D s}
\def\dt{\D t}
\def\CC{{\cal C}}
\def\DD{{\cal D}}
\def\U{{\mathbb U}}
\def\A{{\cal A}}
\def\K{{\cal K}}
\def\DTU{{\cal D}_{{\rm T} \rightarrow {\rm U}}}
\def\LL{{\cal L}}
\def\B{{\cal B}}
\def\T{{\cal T}}
\def\W{{\cal W}}


\def\tF_#1{{\tt F}_{#1}}
\def\Fm{\tF_m}
\def\Fab{\tF_{\alpha,\beta}}
\def\FC{\T}
\def\FCpmz{\FC^{\pm {\rm z}}}
\def\FCz{\FC^{\rm z}}

\def\tFC_#1{{\tt T}_{#1}}
\def\FCn{\tFC_n}

\def\rmz{{\rm z}}

\def\chapref#1{Chapter~\ref{Chapter:#1}}
\def\secref#1{Section~\ref{Section:#1}}
\def\exref#1{Exercise~\ref{Exercise:#1}}
\def\lmref#1{Lemma~\ref{Lemma:#1}}
\def\propref#1{Proposition~\ref{Proposition:#1}}
\def\warnref#1{Warning~\ref{Warning:#1}}
\def\thref#1{Theorem~\ref{Theorem:#1}}
\def\defref#1{Definition~\ref{Definition:#1}}
\def\probref#1{Problem~\ref{Problem:#1}}
\def\corref#1{Corollary~\ref{Corollary:#1}}

\def\sgn{{\rm sgn}\,}
\def\Ai{{\rm Ai}\,}
\def\Bi{{\rm Bi}\,}
\def\wind{{\rm wind}\,}
\def\erf{{\rm erf}\,}
\def\erfc{{\rm erfc}\,}
\def\qqquad{\qquad\quad}
\def\qqqquad{\qquad\qquad}


\def\spand{\hbox{ and }}
\def\spodd{\hbox{ odd}}
\def\speven{\hbox{ even}}
\def\qand{\quad\hbox{and}\quad}
\def\qqand{\qquad\hbox{and}\qquad}
\def\qfor{\quad\hbox{for}\quad}
\def\qqfor{\qquad\hbox{for}\qquad}
\def\qas{\quad\hbox{as}\quad}
\def\qqas{\qquad\hbox{as}\qquad}
\def\qor{\quad\hbox{or}\quad}
\def\qqor{\qquad\hbox{or}\qquad}
\def\qqwhere{\qquad\hbox{where}\qquad}



%%% Words

\def\naive{na\"\i ve\xspace}
\def\Jmap{Joukowsky map\xspace}
\def\Mobius{M\"obius\xspace}
\def\Holder{H\"older\xspace}
\def\Mathematica{{\sc Mathematica}\xspace}
\def\apriori{apriori\xspace}
\def\WHf{Weiner--Hopf factorization\xspace}
\def\WHfs{Weiner--Hopf factorizations\xspace}

\def\Jup{J_\uparrow^{-1}}
\def\Jdown{J_\downarrow^{-1}}
\def\Jin{J_+^{-1}}
\def\Jout{J_-^{-1}}



\def\bD{\D\!\!\!^-}




\def\questionequals{= \!\!\!\!\!\!{\scriptstyle ? \atop }\,\,\,}

\def\elll#1{\ell^{\lambda,#1}}
\def\elllp{\ell^{\lambda,p}}
\def\elllRp{\ell^{(\lambda,R),p}}


\def\elllRpz_#1{\ell_{#1{\rm z}}^{(\lambda,R),p}}


\def\sopmatrix#1{\begin{pmatrix}#1\end{pmatrix}}


\def\bbR{{\mathbb R}}
\def\bbC{{\mathbb C}}


\begin{document}

\maketitle


\chapter{Calculus on a Computer}

In this first chapter we explore the basics of mathematical computing and numerical analysis.
In particular we investigate the following mathematical problems which can not in general be solved exactly:

\begin{enumerate}
\item Integration. General integrals have no closed form expressions. Can we use a computer to approximate the values of definite integrals?
\item Differentiation. Differentiating a formula as in calculus is usually algorithmic, however, it is often needed to compute derivatives without access to an underlying formula, eg, of a function defined only in code. Can we use a computer to approximate derivatives?  A very important application is in Machine Learning, where there is a need to compute gradients to determine the ``right" weights in a neural network. 
\item Root finding. There is no general formula for finding roots (zeros) of arbitrary functions, or even polynomials that are of degree 5 (quintics) or higher. Can we compute roots of general functions using a computer?
\end{enumerate}

In this chapter we discuss:

\begin{enumerate}
\item I.1 Rectangular rule: We review the rectangular rule for integration and deduce the {\it converge rate} of the approximation. In the lab  we investigate its implementation as well as extensions to the Trapezium rule. 
\item I.2 Divided differences: We investigate approximating derivatives by a divided difference and again deduce the convergence rates. In the lab we extend the approach to the central differences formula and computing second derivatives. We also observe a mystery: the approximations may have significant errors in practice, and there is a limit to the accuracy.
\item I.3 Dual numbers:  We introduce the algebraic notion of a {\it dual number} which allows implementing {\it forward-mode automatic differentiation}, a high accuracy alternative to divided differences for computing derivatives.
\item I.4 Newton's method: We review Newton's method for root finding, leveraging dual numbers for computing the derivatives.
\end{enumerate}




\section{Rectangular rule}
One possible definition for an integral is the limit of a Riemann sum, for example:
\[
  \ensuremath{\int}_a^b f(x) {\rm d}x = \lim_{n \ensuremath{\rightarrow} \ensuremath{\infty}} h \ensuremath{\sum}_{j=1}^n f(x_j)
\]
where $x_j = a+jh$ are evenly spaced points dividing up the interval $[a,b]$, that is  with the \emph{step size} $h = (b-a)/n$. This suggests an algorithm known as the \emph{(right-sided) rectangular rule} for approximating an integral: choose $n$ large so that
\[
  \ensuremath{\int}_a^b f(x) {\rm d}x \ensuremath{\approx} h \ensuremath{\sum}_{j=1}^n f(x_j).
\]
In the lab we explore practical implementation of this approximation, and observe that the error in approximation is bounded by $C/n$ for some constant $C$. This can be expressed using "Big-O" notation:
\[
\ensuremath{\int}_a^b f(x) {\rm d}x = h \ensuremath{\sum}_{j=1}^n f(x_j) + O(1/n).
\]
In these notes we consider the $``$Analysis" part of $``$Numerical Analysis": we want to \emph{prove} the convergence rate of the approximation, including finding an explicit expression for the constant $C$.

To tackle this question we consider the error incurred on a single $``$rectangle", then sum up the errors on rectangles.

Now for a secret. There are only so many tools available in analysis (especially at this stage of your career), and one can make a safe bet that the right tool in any analysis proof is either (1) integration-by-parts, (2) geometric series or (3) Taylor series. In this case we use (1):

\begin{lemma}[(Right-sided) Rectangular Rule error on one panel] Assuming $f$ is differentiable we have
\[
\ensuremath{\int}_a^b f(x) {\rm d}x = (b-a) f(b) + \ensuremath{\delta}
\]
where $|\ensuremath{\delta}| \ensuremath{\leq} M (b-a)^2$ for $M = \sup_{a \ensuremath{\leq} x \ensuremath{\leq} b}|f'(x)|$.

\end{lemma}
\textbf{Proof} We write
\meeq{
\ensuremath{\int}_a^b f(x) {\rm d}x = \ensuremath{\int}_a^b (x-a)' f(x)  {\rm d}x = [(x-a) f(x)]_a^b - \ensuremath{\int}_a^b (x-a) f'(x) {\rm d} x \ccr
= (b-a) f(b) + \underbrace{\left(-\ensuremath{\int}_a^b (x-a) f'(x) {\rm d} x \right)}_\ensuremath{\delta}.
}
Recall that we can bound the absolute value of an integral by the sepremum of the integrand times the width of the integration interval:
\[
\abs{\ensuremath{\int}_a^b g(x) {\rm d} x} \ensuremath{\leq} (b-a) \sup_{a \ensuremath{\leq} x \ensuremath{\leq} b}|g(x)|.
\]
The lemma thus follows since
\[
\abs{\ensuremath{\int}_a^b (x-a) f'(x) {\rm d} x} \ensuremath{\leq} (b-a) \sup_{a \ensuremath{\leq} x \ensuremath{\leq} b}|(x-a) f'(x)| \ensuremath{\leq} M (b-a)^2.
\]
\ensuremath{\QED}

Now summing up the errors in each panel gives us the error of using the Rectangular rule:

\begin{theorem}[Rectangular Rule error] Assuming $f$ is differentiable we have
\[
\ensuremath{\int}_a^b f(x) {\rm d}x =  h \ensuremath{\sum}_{j=1}^n f(x_j) +  \ensuremath{\delta}
\]
where $|\ensuremath{\delta}| \ensuremath{\leq} M (b-a) h$ for $M = \sup_{a \ensuremath{\leq} x \ensuremath{\leq} b}|f'(x)|$, $h = (b-a)/n$ and $x_j = a + jh$.

\end{theorem}
\textbf{Proof} We split the integral into a sum of smaller integrals:
\[
\ensuremath{\int}_a^b f(x) {\rm d}x = \ensuremath{\sum}_{j=1}^n  \ensuremath{\int}_{x_{j-1}}^{x_j} f(x) {\rm d}x =
\ensuremath{\sum}_{j=1}^n  \br[(x_j - x_{j-1}) f(x_j) + \ensuremath{\delta}_j] =  h \ensuremath{\sum}_{j=1}^n f(x_j) +  \underbrace{\ensuremath{\sum}_{j=1}^n \ensuremath{\delta}_j}_\ensuremath{\delta}
\]
where $\ensuremath{\delta}_j$, the error on each panel as in the preceding lemma, satisfies
\[
|\ensuremath{\delta}_j| \ensuremath{\leq} (x_j-x_{j-1})^2 \sup_{x_{j-1} \ensuremath{\leq} x \ensuremath{\leq} x_j}|f'(x)| \ensuremath{\leq} M h^2.
\]
Thus using the triangular inequality we have
\[
|\ensuremath{\delta}| = \abs{ \ensuremath{\sum}_{j=1}^n \ensuremath{\delta}_j} \ensuremath{\leq} \ensuremath{\sum}_{j=1}^n |\ensuremath{\delta}_j| \ensuremath{\leq} M n h^2 = M(b-a)h.
\]
\ensuremath{\QED}

Note a consequence of this lemma is that the approximation converges as $n \ensuremath{\rightarrow} \ensuremath{\infty}$ (i.e. $h \ensuremath{\rightarrow} 0$). In the labs and problem sheets we will consider the left-sided rule:
\[
\ensuremath{\int}_a^b f(x) {\rm d}x \ensuremath{\approx}  h \ensuremath{\sum}_{j=0}^{n-1} f(x_j).
\]
We also consider the \emph{Trapezium rule}. Here we approximate an integral  by an affine function:
\[
\ensuremath{\int}_a^b f(x) {\rm d} x \ensuremath{\approx} \ensuremath{\int}_a^b {(b-x)f(a) + (x-a)f(b) \over b-a} \dx
= {b-a \over 2} \br[f(a) + f(b)].
\]
Subdividing an interval $a = x_0 < x_1 < \ensuremath{\ldots} < x_n = b$ and applying this approximation separately on each subinterval $[x_{j-1},x_j]$, where $h = (b-a)/n$ and $x_j = a + jh$, leads to the approximation
\[
\ensuremath{\int}_a^b f(x) {\rm d}x \ensuremath{\approx}  {h \over 2} f(a) + h \ensuremath{\sum}_{j=1}^{n-1} f(x_j) + {h \over 2} f(b)
\]
We shall see both experimentally and provably that this approximation converges faster than the rectangular rule.






\end{document}