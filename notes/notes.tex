\documentclass[12pt,a4paper]{book}

\usepackage[a4paper,text={16.5cm,25.2cm},centering]{geometry}
\usepackage{lmodern}
\usepackage{amssymb,amsmath}
\usepackage{bm}
\usepackage{graphicx}
\usepackage{microtype}
\usepackage{hyperref}
\usepackage{amsthm}
\usepackage{listings}
\usepackage[usenames,dvipsnames]{xcolor}
\setlength{\parindent}{0pt}
\setlength{\parskip}{1.2ex}
\let\QED=\blacksquare
\def\bbD{{\mathbb D}}
\def\bbZ{{\mathbb Z}}
\def\bbN{{\mathbb N}}
\def\emdash{\hbox{---}}
\def\endash{\hbox{--}}
\def\nsubset{\not\subset}
\def\ldq{``}
\def\red#1{{\color{red} #1}}
\def\blue#1{{\color{blue} #1}}
\def\green#1{{\color{ForestGreen} #1}}
\def\euler{\E}

\hypersetup
       {   pdfauthor = { {{Sheehan Olver}} },
           pdftitle={ {{MATH50003 Numerical Analysis}} },
           colorlinks=TRUE,
           linkcolor=black,
           citecolor=blue,
           urlcolor=blue
       }

\title{ MATH50003 Numerical Analysis }


\newtheorem{lemma}{Lemma}
\newtheorem{theorem}{Theorem}
\newtheorem{proposition}{Proposition}

\theoremstyle{definition}
\newtheorem{definition}{Definition}
\newtheorem{example}{Example}

\author{ Sheehan Olver }
\renewcommand{\thechapter}{\Roman{chapter}}


\def\addtab#1={#1\;&=}

\def\meeq#1{\def\ccr{\\\addtab}
%\tabskip=\@centering
 \begin{align*}
 \addtab#1
 \end{align*}
  }  
  
  \def\leqaddtab#1\leq{#1\;&\leq}
  \def\mleeq#1{\def\ccr{\\\addtab}
%\tabskip=\@centering
 \begin{align*}
 \leqaddtab#1
 \end{align*}
  }  


\def\vc#1{\mbox{\boldmath$#1$\unboldmath}}

\def\vcsmall#1{\mbox{\boldmath$\scriptstyle #1$\unboldmath}}

\def\vczero{{\mathbf 0}}


%\def\beginlist{\begin{itemize}}
%
%\def\endlist{\end{itemize}}


\def\pr(#1){\left({#1}\right)}
\def\br[#1]{\left[{#1}\right]}
\def\fbr[#1]{\!\left[{#1}\right]}
\def\set#1{\left\{{#1}\right\}}
\def\ip<#1>{\left\langle{#1}\right\rangle}
\def\iip<#1>{\left\langle\!\langle{#1}\right\rangle\!\rangle}

\def\norm#1{\left\| #1 \right\|}

\def\abs#1{\left|{#1}\right|}
\def\fpr(#1){\!\pr({#1})}

\def\Re{{\rm Re}\,}
\def\Im{{\rm Im}\,}

\def\floor#1{\left\lfloor#1\right\rfloor}
\def\ceil#1{\left\lceil#1\right\rceil}


\def\mapengine#1,#2.{\mapfunction{#1}\ifx\void#2\else\mapengine #2.\fi }

\def\map[#1]{\mapengine #1,\void.}

\def\mapenginesep_#1#2,#3.{\mapfunction{#2}\ifx\void#3\else#1\mapengine #3.\fi }

\def\mapsep_#1[#2]{\mapenginesep_{#1}#2,\void.}


\def\vcbr{\br}


\def\bvect[#1,#2]{
{
\def\dots{\cdots}
\def\mapfunction##1{\ | \  ##1}
\begin{pmatrix}
		 \,#1\map[#2]\,
\end{pmatrix}
}
}

\def\vect[#1]{
{\def\dots{\ldots}
	\vcbr[{#1}]
}}

\def\vectt[#1]{
{\def\dots{\ldots}
	\vect[{#1}]^{\top}
}}

\def\Vectt[#1]{
{
\def\mapfunction##1{##1 \cr} 
\def\dots{\vdots}
	\begin{pmatrix}
		\map[#1]
	\end{pmatrix}
}}



\def\thetaB{\mbox{\boldmath$\theta$}}
\def\zetaB{\mbox{\boldmath$\zeta$}}


\def\newterm#1{{\it #1}\index{#1}}


\def\TT{{\mathbb T}}
\def\C{{\mathbb C}}
\def\R{{\mathbb R}}
\def\II{{\mathbb I}}
\def\F{{\mathcal F}}
\def\E{{\rm e}}
\def\I{{\rm i}}
\def\D{{\rm d}}
\def\dx{\D x}
\def\ds{\D s}
\def\dt{\D t}
\def\CC{{\cal C}}
\def\DD{{\cal D}}
\def\U{{\mathbb U}}
\def\A{{\cal A}}
\def\K{{\cal K}}
\def\DTU{{\cal D}_{{\rm T} \rightarrow {\rm U}}}
\def\LL{{\cal L}}
\def\B{{\cal B}}
\def\T{{\cal T}}
\def\W{{\cal W}}


\def\tF_#1{{\tt F}_{#1}}
\def\Fm{\tF_m}
\def\Fab{\tF_{\alpha,\beta}}
\def\FC{\T}
\def\FCpmz{\FC^{\pm {\rm z}}}
\def\FCz{\FC^{\rm z}}

\def\tFC_#1{{\tt T}_{#1}}
\def\FCn{\tFC_n}

\def\rmz{{\rm z}}

\def\chapref#1{Chapter~\ref{Chapter:#1}}
\def\secref#1{Section~\ref{Section:#1}}
\def\exref#1{Exercise~\ref{Exercise:#1}}
\def\lmref#1{Lemma~\ref{Lemma:#1}}
\def\propref#1{Proposition~\ref{Proposition:#1}}
\def\warnref#1{Warning~\ref{Warning:#1}}
\def\thref#1{Theorem~\ref{Theorem:#1}}
\def\defref#1{Definition~\ref{Definition:#1}}
\def\probref#1{Problem~\ref{Problem:#1}}
\def\corref#1{Corollary~\ref{Corollary:#1}}

\def\sgn{{\rm sgn}\,}
\def\Ai{{\rm Ai}\,}
\def\Bi{{\rm Bi}\,}
\def\wind{{\rm wind}\,}
\def\erf{{\rm erf}\,}
\def\erfc{{\rm erfc}\,}
\def\qqquad{\qquad\quad}
\def\qqqquad{\qquad\qquad}


\def\spand{\hbox{ and }}
\def\spodd{\hbox{ odd}}
\def\speven{\hbox{ even}}
\def\qand{\quad\hbox{and}\quad}
\def\qqand{\qquad\hbox{and}\qquad}
\def\qfor{\quad\hbox{for}\quad}
\def\qqfor{\qquad\hbox{for}\qquad}
\def\qas{\quad\hbox{as}\quad}
\def\qqas{\qquad\hbox{as}\qquad}
\def\qor{\quad\hbox{or}\quad}
\def\qqor{\qquad\hbox{or}\qquad}
\def\qqwhere{\qquad\hbox{where}\qquad}



%%% Words

\def\naive{na\"\i ve\xspace}
\def\Jmap{Joukowsky map\xspace}
\def\Mobius{M\"obius\xspace}
\def\Holder{H\"older\xspace}
\def\Mathematica{{\sc Mathematica}\xspace}
\def\apriori{apriori\xspace}
\def\WHf{Weiner--Hopf factorization\xspace}
\def\WHfs{Weiner--Hopf factorizations\xspace}

\def\Jup{J_\uparrow^{-1}}
\def\Jdown{J_\downarrow^{-1}}
\def\Jin{J_+^{-1}}
\def\Jout{J_-^{-1}}



\def\bD{\D\!\!\!^-}




\def\questionequals{= \!\!\!\!\!\!{\scriptstyle ? \atop }\,\,\,}

\def\elll#1{\ell^{\lambda,#1}}
\def\elllp{\ell^{\lambda,p}}
\def\elllRp{\ell^{(\lambda,R),p}}


\def\elllRpz_#1{\ell_{#1{\rm z}}^{(\lambda,R),p}}


\def\sopmatrix#1{\begin{pmatrix}#1\end{pmatrix}}


\def\bbR{{\mathbb R}}
\def\bbC{{\mathbb C}}


\begin{document}

\maketitle

\tableofcontents

\chapter{Calculus on a Computer}

In this first chapter we explore the basics of mathematical computing and numerical analysis.
In particular we investigate the following mathematical problems which can not in general be solved exactly:

\begin{enumerate}
\item Integration. General integrals have no closed form expressions. Can we use a computer to approximate the values of definite integrals?
\item Differentiation. Differentiating a formula as in calculus is usually algorithmic, however, it is often needed to compute derivatives without access to an underlying formula, eg,  a function defined only in code. Can we use a computer to approximate derivatives?  A very important application is in Machine Learning, where there is a need to compute gradients to determine the ``right" weights in a neural network. 
\item Root finding. There is no general formula for finding roots (zeros) of arbitrary functions, or even polynomials that are of degree 5 (quintics) or higher. Can we compute roots of general functions using a computer?
\end{enumerate}

In this chapter we discuss:

\begin{enumerate}
\item I.1 Rectangular rule: we review the rectangular rule for integration and deduce the {\it converge rate} of the approximation. In the lab/problem sheet  we investigate its implementation as well as extensions to the Trapezium rule. 
\item I.2 Divided differences: we investigate approximating derivatives by a divided difference and again deduce the convergence rates. In the lab/problem sheet we extend the approach to the central differences formula and computing second derivatives. We also observe a mystery: the approximations may have significant errors in practice, and there is a limit to the accuracy.
\item I.3 Dual numbers: we introduce the algebraic notion of a {\it dual number} which allows the implemention of {\it forward-mode automatic differentiation}, a high accuracy alternative to divided differences for computing derivatives.
\item I.4 Newton's method: Newton's method is a basic approach for computing roots/zeros of a function. We use dual numbers to implement this algorithm.
\end{enumerate}




\section{Rectangular rule}
One possible definition for an integral is the limit of a Riemann sum, for example:
\[
  \ensuremath{\int}_a^b f(x) {\rm d}x = \lim_{n \ensuremath{\rightarrow} \ensuremath{\infty}} h \ensuremath{\sum}_{j=1}^n f(x_j)
\]
where $x_j = a+jh$ are evenly spaced points dividing up the interval $[a,b]$, that is  with the \emph{step size} $h = (b-a)/n$. This suggests an algorithm known as the \emph{(right-sided) rectangular rule} for approximating an integral: choose $n$ large so that
\[
  \ensuremath{\int}_a^b f(x) {\rm d}x \ensuremath{\approx} h \ensuremath{\sum}_{j=1}^n f(x_j).
\]
In the lab we explore practical implementation of this approximation, and observe that the error in approximation is bounded by $C/n$ for some constant $C$. This can be expressed using "Big-O" notation:
\[
\ensuremath{\int}_a^b f(x) {\rm d}x = h \ensuremath{\sum}_{j=1}^n f(x_j) + O(1/n).
\]
In these notes we consider the $``$Analysis" part of $``$Numerical Analysis": we want to \emph{prove} the convergence rate of the approximation, including finding an explicit expression for the constant $C$.

To tackle this question we consider the error incurred on a single $``$rectangle", then sum up the errors on rectangles.

Now for a secret. There are only so many tools available in analysis (especially at this stage of your career), and one can make a safe bet that the right tool in any analysis proof is either (1) integration-by-parts, (2) geometric series or (3) Taylor series. In this case we use (1):

\begin{lemma}[(Right-sided) Rectangular Rule error on one panel] Assuming $f$ is differentiable we have
\[
\ensuremath{\int}_a^b f(x) {\rm d}x = (b-a) f(b) + \ensuremath{\delta}
\]
where $|\ensuremath{\delta}| \ensuremath{\leq} M (b-a)^2$ for $M = \sup_{a \ensuremath{\leq} x \ensuremath{\leq} b}|f'(x)|$.

\end{lemma}
\textbf{Proof} We write
\meeq{
\ensuremath{\int}_a^b f(x) {\rm d}x = \ensuremath{\int}_a^b (x-a)' f(x)  {\rm d}x = [(x-a) f(x)]_a^b - \ensuremath{\int}_a^b (x-a) f'(x) {\rm d} x \ccr
= (b-a) f(b) + \underbrace{\left(-\ensuremath{\int}_a^b (x-a) f'(x) {\rm d} x \right)}_\ensuremath{\delta}.
}
Recall that we can bound the absolute value of an integral by the sepremum of the integrand times the width of the integration interval:
\[
\abs{\ensuremath{\int}_a^b g(x) {\rm d} x} \ensuremath{\leq} (b-a) \sup_{a \ensuremath{\leq} x \ensuremath{\leq} b}|g(x)|.
\]
The lemma thus follows since
\[
\abs{\ensuremath{\int}_a^b (x-a) f'(x) {\rm d} x} \ensuremath{\leq} (b-a) \sup_{a \ensuremath{\leq} x \ensuremath{\leq} b}|(x-a) f'(x)| \ensuremath{\leq} M (b-a)^2.
\]
\ensuremath{\QED}

Now summing up the errors in each panel gives us the error of using the Rectangular rule:

\begin{theorem}[Rectangular Rule error] Assuming $f$ is differentiable we have
\[
\ensuremath{\int}_a^b f(x) {\rm d}x =  h \ensuremath{\sum}_{j=1}^n f(x_j) +  \ensuremath{\delta}
\]
where $|\ensuremath{\delta}| \ensuremath{\leq} M (b-a) h$ for $M = \sup_{a \ensuremath{\leq} x \ensuremath{\leq} b}|f'(x)|$, $h = (b-a)/n$ and $x_j = a + jh$.

\end{theorem}
\textbf{Proof} We split the integral into a sum of smaller integrals:
\[
\ensuremath{\int}_a^b f(x) {\rm d}x = \ensuremath{\sum}_{j=1}^n  \ensuremath{\int}_{x_{j-1}}^{x_j} f(x) {\rm d}x =
\ensuremath{\sum}_{j=1}^n  \br[(x_j - x_{j-1}) f(x_j) + \ensuremath{\delta}_j] =  h \ensuremath{\sum}_{j=1}^n f(x_j) +  \underbrace{\ensuremath{\sum}_{j=1}^n \ensuremath{\delta}_j}_\ensuremath{\delta}
\]
where $\ensuremath{\delta}_j$, the error on each panel as in the preceding lemma, satisfies
\[
|\ensuremath{\delta}_j| \ensuremath{\leq} (x_j-x_{j-1})^2 \sup_{x_{j-1} \ensuremath{\leq} x \ensuremath{\leq} x_j}|f'(x)| \ensuremath{\leq} M h^2.
\]
Thus using the triangular inequality we have
\[
|\ensuremath{\delta}| = \abs{ \ensuremath{\sum}_{j=1}^n \ensuremath{\delta}_j} \ensuremath{\leq} \ensuremath{\sum}_{j=1}^n |\ensuremath{\delta}_j| \ensuremath{\leq} M n h^2 = M(b-a)h.
\]
\ensuremath{\QED}

Note a consequence of this lemma is that the approximation converges as $n \ensuremath{\rightarrow} \ensuremath{\infty}$ (i.e. $h \ensuremath{\rightarrow} 0$). In the labs and problem sheets we will consider the left-sided rule:
\[
\ensuremath{\int}_a^b f(x) {\rm d}x \ensuremath{\approx}  h \ensuremath{\sum}_{j=0}^{n-1} f(x_j).
\]
We also consider the \emph{Trapezium rule}. Here we approximate an integral  by an affine function:
\[
\ensuremath{\int}_a^b f(x) {\rm d} x \ensuremath{\approx} \ensuremath{\int}_a^b {(b-x)f(a) + (x-a)f(b) \over b-a} \dx
= {b-a \over 2} \br[f(a) + f(b)].
\]
Subdividing an interval $a = x_0 < x_1 < \ensuremath{\ldots} < x_n = b$ and applying this approximation separately on each subinterval $[x_{j-1},x_j]$, where $h = (b-a)/n$ and $x_j = a + jh$, leads to the approximation
\[
\ensuremath{\int}_a^b f(x) {\rm d}x \ensuremath{\approx}  {h \over 2} f(a) + h \ensuremath{\sum}_{j=1}^{n-1} f(x_j) + {h \over 2} f(b)
\]
We shall see both experimentally and provably that this approximation converges faster than the rectangular rule.





\section{Divided Differences}
Given a function, how can we approximate its derivative at a point? We consider an intuitive approach to this problem using \emph{(Right-sided) Divided Differences}: 
\[
f'(x) \ensuremath{\approx} {f(x+h) - f(x) \over h}
\]
Note by the definition of the derivative we know that this approximation will converge to the true derivative as $h \ensuremath{\rightarrow} 0$. But in numerical approimxations we also need to consider the rate of convergence. 

Now in the previous section I mentioned there are three basic tools in analysis:  (1) integration-by-parts, (2) geometric series or (3) Taylor series. In this case we use (3):

\begin{proposition}[divided differences error] Suppose that $f$ is twice-differentiable on the interval $[x,x+h]$. The error in approximating the derivative using divided differences is
\[
f'(x) = {f(x+h) - f(x) \over h} + \ensuremath{\delta}
\]
where $|\ensuremath{\delta}| \ensuremath{\leq} Mh/2$ for  $M = \sup_{x \ensuremath{\leq} t \ensuremath{\leq} x+h} |f''(t)|$.

\end{proposition}
\textbf{Proof} Follows immediately from Taylor's theorem:
\[
f(x+h) = f(x) + f'(x) h + \underbrace{{f''(t) \over 2} h^2}_{h \ensuremath{\delta}}
\]
for some $x \ensuremath{\leq} t \ensuremath{\leq} x+h$, by bounding:
\[
|\ensuremath{\delta}| \ensuremath{\leq} \abs{{f''(t) \over 2} h} \ensuremath{\leq} {M  h \over 2}.
\]
\ensuremath{\QED}

Unlike the rectangular rule, the computational cost of computing the divided difference is independent of $h$! We only need to evaluate a function $f$ twice and do a single division. Here we are assuming that the computational cost of evaluating $f$ is independent of the point of evaluation. Later we will investigate the details of how computers work with numbers via floating point,  and confirm that this is a sensible assumption.

So why not just set $h$ ridiculously small? In the lab we explore this question and observe that there are significant errors introduced in the numerical realisation of this algorithm. We will return to the question of understanding these errors after learning floating point numbers. 

There are alternative versions of divided differences. Left-side divided differences evaluates to the left of the point where wish to know the derivative:
\[
f'(x) \ensuremath{\approx} {f(x) - f(x-h) \over h}
\]
and central differences:
\[
f'(x) \ensuremath{\approx} {f(x + h) - f(x - h) \over 2h}
\]
We can further arrive at an approximation to the second derivative by composing a left- and right-sided finite difference:
\[
f''(x) \ensuremath{\approx} {f'(x+h) - f'(x) \over h} \ensuremath{\approx} {{f(x+h) - f(x) \over h} - {f(x) - f(x-h) \over h} \over h}
= {f(x+h) - 2f(x)  + f(x-h) \over h^2}
\]
In the lab we investigate the convergence rate of these approximations (in particular, that  central differences is more accurate than standard divided differences) and observe that they too suffer from unexplained (for now) loss of accuracy as $h \ensuremath{\rightarrow} 0$. In the problem sheet we prove the theoretical converge rate, which is never realised because of these errors.





\section{Dual Numbers}
In this section we introduce a mathematically beautiful  alternative to divided differences for computing derivatives: \emph{dual numbers}. These are a commutative ring that \emph{exactly} compute derivatives, which when implemented on a computer gives very high-accuracy approximations to derivatives. They underpin forward-mode \href{https://en.wikipedia.org/wiki/Automatic_differentiation}{automatic differentation}. Automatic differentiation  is a basic tool in Machine Learning for computing gradients necessary for training neural networks.

\begin{definition}[Dual numbers] Dual numbers $\ensuremath{\bbD}$ are a commutative ring (over $\ensuremath{\bbR}$) generated by $1$ and $\ensuremath{\epsilon}$ such that $\ensuremath{\epsilon}^2 = 0$. Dual numbers are typically written as $a + b \ensuremath{\epsilon}$ where $a$ and $b$ are real. \end{definition}

This is very much analoguous to complex numbers, which are a field generated by $1$ and $\I$ such that $\I^2 = -1$. Compare multiplication of each number type:
\meeq{
(a + b \I) (c + d \I) = ac + (bc + ad) \I + bd \I^2 = ac -bd + (bc + ad) \I \ccr
(a + b \ensuremath{\epsilon}) (c + d \ensuremath{\epsilon}) = ac + (bc + ad) \ensuremath{\epsilon} + bd \ensuremath{\epsilon}^2 = ac  + (bc + ad) \ensuremath{\epsilon} 
}
And just as we view $\ensuremath{\bbR} \ensuremath{\subset} \ensuremath{\bbC}$ by equating $a \ensuremath{\in} \ensuremath{\bbR}$ with $a + 0\I \ensuremath{\in} \ensuremath{\bbC}$, we can view $\ensuremath{\bbR} \ensuremath{\subset} \ensuremath{\bbD}$ by equating $a \ensuremath{\in} \ensuremath{\bbR}$ with $a + 0{\rm \ensuremath{\epsilon}} \ensuremath{\in} \ensuremath{\bbD}$.

\subsection{Differentiating polynomials}
Polynomials evaluated on dual numbers are well-defined as they depend only on the operations $+$ and $*$. From the formula for multiplication of dual numbers we deduce that evaluating a polynomial at a dual number $a + b \ensuremath{\epsilon}$ tells us the derivative of the polynomial at $a$:

\begin{theorem}[polynomials on dual numbers] Suppose $p$ is a polynomial. Then
\[
p(a + b \ensuremath{\epsilon}) = p(a) + b p'(a) \ensuremath{\epsilon}
\]
\end{theorem}
\textbf{Proof}

First consider $p(x) = x^n$ for $n \ensuremath{\geq} 0$.  The cases $n = 0$ and $n = 1$ are immediate. For $n > 1$ we have by induction:
\[
(a + b \ensuremath{\epsilon})^n = (a + b \ensuremath{\epsilon}) (a + b \ensuremath{\epsilon})^{n-1} = (a + b \ensuremath{\epsilon}) (a^{n-1} + (n-1) b a^{n-2} \ensuremath{\epsilon}) = a^n + b n a^{n-1} \ensuremath{\epsilon}.
\]
For a more general polynomial
\[
p(x) = \ensuremath{\sum}_{k=0}^n c_k x^k
\]
the result follows from linearity:
\[
p(a + b \ensuremath{\varepsilon}) = \ensuremath{\sum}_{k=0}^n c_k (a+b\ensuremath{\epsilon})^k = c_0 + \ensuremath{\sum}_{k=1}^n c_k (a^k +k b a^{k-1}\ensuremath{\epsilon})
= \ensuremath{\sum}_{k=0}^n c_k a^k + b \ensuremath{\sum}_{k=1}^n c_k k a^{k-1}\ensuremath{\epsilon} = p(a) + b p'(a) \ensuremath{\epsilon}.
\]
\ensuremath{\QED}

\begin{example}[differentiating polynomial] Consider computing $p'(2)$ where
\[
p(x) = (x-1)(x-2) + x^2.
\]
We can use dual numbers to differentiate, avoiding expanding in monomials or applying rules of differentiating:
\[
p(2+\ensuremath{\epsilon}) = (1+\ensuremath{\epsilon})\ensuremath{\epsilon} + (2+\ensuremath{\epsilon})^2 = \ensuremath{\epsilon} + 4 + 4\ensuremath{\epsilon} = 4 + \underbrace{5}_{p'(2)}\ensuremath{\epsilon}
\]
\end{example}

\subsection{Differentiating other functions}
We can extend real-valued differentiable functions to dual numbers in a similar manner. First, consider a standard function with a Taylor series (e.g. ${\rm cos}$, ${\rm sin}$, ${\rm exp}$, etc.)
\[
f(x) = \ensuremath{\sum}_{k=0}^\ensuremath{\infty} f_k x^k
\]
so that $a$ is inside the radius of convergence. This leads naturally to a definition on dual numbers:
\meeq{
f(a + b \ensuremath{\epsilon}) = \ensuremath{\sum}_{k=0}^\ensuremath{\infty} f_k (a + b \ensuremath{\epsilon})^k = f_0 + \ensuremath{\sum}_{k=1}^\ensuremath{\infty} f_k (a^k + k a^{k-1} b \ensuremath{\epsilon}) = \ensuremath{\sum}_{k=0}^\ensuremath{\infty} f_k a^k +  \ensuremath{\sum}_{k=1}^\ensuremath{\infty} f_k k a^{k-1} b \ensuremath{\epsilon}  \ccr
  = f(a) + b f'(a) \ensuremath{\epsilon}
}
More generally, given a differentiable function we can extend it to dual numbers:

\begin{definition}[dual extension] Suppose a real-valued function $f$ is differentiable at $a$. If
\[
f(a + b \ensuremath{\epsilon}) = f(a) + b f'(a) \ensuremath{\epsilon}
\]
then we say that it is a \emph{dual extension at} $a$.

Thus, for basic functions we have natural extensions:


\begin{align*}
\exp(a + b \ensuremath{\epsilon}) &:= \exp(a) + b \exp(a) \ensuremath{\epsilon} \\
\sin(a + b \ensuremath{\epsilon}) &:= \sin(a) + b \cos(a) \ensuremath{\epsilon} \\
\cos(a + b \ensuremath{\epsilon}) &:= \cos(a) - b \sin(a) \ensuremath{\epsilon} \\
\log(a + b \ensuremath{\epsilon}) &:= \log(a) + {b \over a} \ensuremath{\epsilon} \\
\sqrt{a+b \ensuremath{\epsilon}} &:= \sqrt{a} + {b \over 2 \sqrt{a}} \ensuremath{\epsilon} \\
|a + b \ensuremath{\epsilon}| &:= |a| + b\, {\rm sign} a\, \ensuremath{\epsilon}
\end{align*}
provided the function is differentiable at $a$. Note the last example does not have a convergent Taylor series (at 0) but we can still extend it where it is differentiable.

Going further, we can add, multiply, and compose such functions:

\begin{lemma}[product and chain rule] If $f$ is a dual extension at $g(a)$ and $g$ is a dual extension at $a$, then $q(x) := f(g(x))$ is a dual extension at $a$. If $f$ and $g$ are dual extensions at $a$ then  $r(x) := f(x) g(x)$ is also dual extensions at $a$. In other words:
\meeq{
q(a+b \ensuremath{\epsilon}) = q(a) + b q'(a) \ensuremath{\epsilon} \ccr
r(a+b \ensuremath{\epsilon}) = r(a) + b r'(a) \ensuremath{\epsilon}
}
\end{lemma}
\textbf{Proof} For $q$ it follows immediately:
\meeq{
q(a + b \ensuremath{\epsilon}) = f(g(a + b \ensuremath{\epsilon})) = f(g(a) + b g'(a) \ensuremath{\epsilon}) \ccr
= f(g(a)) + b g'(a) f'(g(a))\ensuremath{\epsilon} = q(a) + b q'(a) \ensuremath{\epsilon}.
}
For $r$ we have
\meeq{
r(a + b \ensuremath{\epsilon}) = f(a+b \ensuremath{\epsilon} )g(a+b \ensuremath{\epsilon} )= (f(a) + b f'(a) \ensuremath{\epsilon})(g(a) + b g'(a) \ensuremath{\epsilon}) \ccr
= f(a)g(a) + b (f'(a)g(a) + f(a)g'(a)) \ensuremath{\epsilon} = r(a) +b r'(a) \ensuremath{\epsilon}.
}
\end{definition}

A simple corollary is that any function defined in terms of addition, multiplication, composition, etc. of functions that are dual with differentiation will be differentiable via dual numbers.

\begin{example}[differentiating non-polynomial]

Consider differentiating $f(x) =  \exp(x^2 + \E^x)$ at the point $a = 1$ by evaluating on the duals:
\[
f(1 + \ensuremath{\epsilon}) = \exp(1 + 2\ensuremath{\epsilon} + \E + \E \ensuremath{\epsilon}) =  \exp(1 + \E) + \exp(1 + \E) (2 + \E) \ensuremath{\epsilon}.
\]
Therefore we deduce that
\[
f'(1) = \exp(1 + \E) (2 + \E).
\]
\end{example}





\section{Newton's method}
In school you may recall learning Newton's method: a way of approximating zeros/roots to a function by using a local approximation by an affine function. That is, approximate a function $f(x)$ locally around an initial guess $x_0$ by its first order Taylor series:
\[
f(x) \ensuremath{\approx} f(x_0) + f'(x_0) (x-x_0)
\]
and then find the root of the right-hand side which is
\[
 f(x_0) + f'(x_0) (x-x_0) = 0 \ensuremath{\Leftrightarrow} x = x_0 - {f(x_0) \over f'(x_0)}.
\]
We can then repeat using this root as the new initial guess. In other words we have a sequence of \emph{hopefully} more accurate approximations:
\[
x_{k+1} = x_k - {f(x_k) \over f'(x_k)}.
\]
The convergence theory of Newton's method is rich and beautiful but outside the scope of this module. But provided $f$ is smooth, if $x_0$ is sufficiently close to a root this iteration will converge. 

Thus \emph{if} we can compute derivatives, we can (sometimes) compute roots. The lab will explore using dual numbers to accomplish this task. This is in some sense a baby version of how Machine Learning algorithms train neural networks.






\chapter{Representing Numbers}

In this chapter we aim to answer the question: when can we rely on computations done on a computer?  Why are some computations (differentiation via divided differences), extremely inaccurate whilst others (integration via rectangular rule) accurate up to about 16 digits?  In order to address these questions we need to dig deeper and understand at a basic level what a computer is actually doing when manipulating numbers. 

Before we begin it is important to have a basic model of how a computer works. Our simplified model of a computer will consist of a \href{https://en.wikipedia.org/wiki/Central_processing_unit}{Central Processing Unit (CPU)}\ensuremath{\emdash}the  brains of the computer\ensuremath{\emdash}and \href{https://en.wikipedia.org/wiki/Computer_data_storage#Primary_storage}{Memory}\ensuremath{\emdash}where  data is stored. Inside the CPU there are \href{https://en.wikipedia.org/wiki/Processor_register}{registers}, where data is temporarily stored after being loaded from memory, manipulated by the CPU, then stored back to memory.  Memory is a sequence of bits: \texttt{1}s and \texttt{0}s, essentially ``on/off" switches, and memory is {\it finite}.  Finally, if one has a $p$-bit CPU (eg a 32-bit or 64-bit CPU), each register consists of exactly $p$-bits. Most likely $p = 64$ on your machine. 


Thus representing numbers on a computer must overcome three fundamental limitations:
\begin{enumerate}
\item CPUs can only manipulate data $p$-bits at a time.
\item Memory is finite (in particular at most $2^p$ bytes).
\item There is no such thing as an ``error'': if anything goes wrong in the computation we must use some of the $p$-bits to indicate this.
\end{enumerate}

This is clearly problematic: there are an infinite number of integers and an uncountable number of reals! Each of which we need to store in precisely $p$-bits. Moreover, some operations are simply undefined, like division by 0.  This chapter discusses the solution used to this problem, alongside the mathematical analysis that is needed to understand the implications, in particular, that computations have {\it error}.

In particular we discuss:

\begin{enumerate}
\item II.1 Integers: unsigned (non-negative) and signed integers are representable using exactly $p$-bits by using modular arithmetic in all operations.
\item II.2 Reals:  real numbers are approximated by floating point numbers, which are a computers version of scientific notation.
\item II.3 Floating Point Arithmetic:  arithmetic with floating point numbers is exact up-to-rounding, which introduces small-but-understandable errors in the computations. We explain how these errors can be analysed mathematically to get rigorous bounds. 
\item II.4 Interval Arithmetic: rounding can be controlled in order to implement {\it interval arithmetic}, a way to compute rigorous bounds for computations. In the lab, we use this to compute up to 15 digits of ${\rm e} \equiv \exp 1$ rigorously with precise bounds on the error.
\end{enumerate}



\section{Integers}
In this section we discuss the following:

\begin{itemize}
\item[1. ] Unsigned integers: how computers represent non-negative integers using only $p$-bits, via \href{https://en.wikipedia.org/wiki/Modular_arithmetic}{modular arithmetic}.


\item[2. ] Signed integers: how negative integers are handled using the \href{https://en.wikipedia.org/wiki/Two's_complement}{Two's-complement} format.

\end{itemize}
Mathematically, CPUs only act on $p$-bits at a time, with $2^p$ possible sequences. That is, essentially all functions $f$ are either of the form $f : \ensuremath{\bbZ}_{2^p} \ensuremath{\rightarrow} \ensuremath{\bbZ}_{2^p}$ or  $f : \ensuremath{\bbZ}_{2^p} \ensuremath{\times} \ensuremath{\bbZ}_{2^p} \ensuremath{\rightarrow} \ensuremath{\bbZ}_{2^p}$, where we use the following notation:

\begin{definition}[finite integers] Denote the set of the first $m$ non-negative integers as $\ensuremath{\bbZ}_m := \{0 , 1 , \ensuremath{\ldots}, m-1 \}$. \end{definition}

To translate between integers and bits we will need to write integers in binary format.  That is, as sequence of \texttt{0}s and \texttt{1}s:

\begin{definition}[binary format] For $B_0,\ldots,B_p \in \{0,1\}$ denote an integer in \emph{binary format} by:
\[
\ensuremath{\pm}(B_p\ldots B_1B_0)_2 := \ensuremath{\pm}\sum_{k=0}^p B_k 2^k
\]
\end{definition}

\begin{example}[integers in binary] A simple integer example is $5 = 2^2 + 2^0 = (101)_2$. On the other hand, we write $-5 = -(101)_2$. Another example is $258 = 2^8 + 2 = (100000010)_2$. \end{example}

\subsection{Unsigned Integers}
Computers represent integers by a finite number of $p$-bits, with $2^p$ possible combinations of 0s and 1s. Denote these $p$-bits as $B_{p-1}\ensuremath{\ldots}B_1B_0$ where $B_k \ensuremath{\in} \{0,1\}$. For \emph{unsigned integers} (non-negative integers) these bits dictate the first $p$ binary digits: $(B_{p-1}\ldots B_1B_0)_2$. Integers represented with $p$-bits on a computer are interpreted as  representing elements of ${\mathbb Z}_{2^p}$ and integer arithmetic on a computer is equivalent to arithmetic modulo $2^p$. We denote modular arithmetic with $m = 2^p$ as follows:
\begin{align*}
x \ensuremath{\oplus}_m y &:= (x+y)\ ({\rm mod}\ m) \\
x \ensuremath{\ominus}_m y &:= (x-y)\ ({\rm mod}\ m) \\
x \ensuremath{\otimes}_m y &:= (x*y)\ ({\rm mod}\ m)
\end{align*}
When $m$ is implied by context we just write $\ensuremath{\oplus}, \ensuremath{\ominus}, \ensuremath{\otimes}$. Note that  the $({\rm mod}\ m)$ function simply drops all bits except for the first $p$-bits when writing a number in binary.

\begin{example}[arithmetic with  8-bit unsigned integers] If  the result of an operation lies between $0$ and $m = 2^8 = 256$ then airthmetic works exactly like standard integer arithmetic. For example,
\begin{align*}
17 \ensuremath{\oplus}_{256} 3 = 20\ ({\rm mod}\ 256) = 20 \\
17 \ensuremath{\ominus}_{256} 3 = 14\ ({\rm mod}\ 256) = 14
\end{align*}
\end{example}

\begin{example}[overflow with 8-bit unsigned integers] If we go beyond the range the result \ensuremath{\ldq}wraps around". For example, with true integers we have
\[
255 + 1 = (11111111)_2 + (00000001)_2 = (100000000)_2 = 256
\]
However, the result is impossible to store in just 8-bits! So as mentioned instead it treats the integers as elements of ${\mathbb Z}_{256}$ by dropping any extra digits:
\[
255 \ensuremath{\oplus}_{256} 1 = 255 + 1 \ ({\rm mod}\ 256) = (100000000)_2 \ ({\rm mod}\ 256) = 0.
\]
On the other hand, if we go below $0$ we wrap around from above:
\[
3 \ensuremath{\ominus}_{256} 5 = -2\ ({\rm mod}\ 256) = 254 = (11111110)_2
\]
\end{example}

\begin{example}[multiplication of 8-bit unsigned integers] Multiplication works similarly: for example,
\[
254 \ensuremath{\otimes}_{256} 2 = 254 * 2 \ ({\rm mod}\ 256) = (11111110)_2 * 2  \ ({\rm mod}\ 256)
= (111111100)_2  \ ({\rm mod}\ 256) = 252.
\]
Note that multiplication by $2$ is the same as shifting the binary digits left by one, just as multiplication by $10$ shifts base-10 digits left by 1. \end{example}

\subsection{Signed integer}
Signed integers use the \href{https://epubs.siam.org/doi/abs/10.1137/1.9780898718072.ch3}{Two's complemement} convention. The convention is if the first bit is 1 then the number is negative: in this case if the bits had represented the unsigned integer $2^p - y$ then the represent the signed integer $-y$. Thus for $p = 8$ we are interpreting $2^7$ through $2^8-1$ as negative numbers. More precisely:

\begin{definition}[signed integers] Denote the finite signed integers as
\[
\ensuremath{\bbZ}_{2^p}^{\rm s} := \{-2^{p-1} ,\ensuremath{\ldots}, -1 ,0,1, \ensuremath{\ldots}, 2^{p-1}-1\}.
\]
\end{definition}

\begin{definition}[Shifted mod] Define for $y = x\ ({\rm mod}\ 2^p)$
\[
x\ ({\rm mod}^{\rm s}\ 2^p) := \begin{cases} y & 0 \ensuremath{\leq} y \ensuremath{\leq} 2^{p-1}-1 \\
                             y - 2^p & 2^{p-1} \ensuremath{\leq} y \ensuremath{\leq} 2^p - 1
                             \end{cases}
\]
\end{definition}

Note that if $R_p(x) = x\ ({\rm mod}^{\rm s}\ 2^p)$ then it can be viewed as a map $R_p : \ensuremath{\bbZ} \ensuremath{\rightarrow} \ensuremath{\bbZ}_{2^p}^{\rm s}$ or a one-to-one map $R_p : \ensuremath{\bbZ}_{2^p} \ensuremath{\rightarrow} \ensuremath{\bbZ}_{2^p}^{\rm s}$ whose inverse is $R_p^{-1}(x) = x\ ({\rm mod}\ 2^p)$. It can also be viewed as the identity map on signed integers $R_p : \ensuremath{\bbZ}_{2^p}^{\rm s} \ensuremath{\rightarrow} \ensuremath{\bbZ}_{2^p}^{\rm s}$, that is,  $R_p(x) = x$ if $x \in \ensuremath{\bbZ}_{2^p}^{\rm s}$.

Arithmetic works precisely the same for signed and unsigned integers up to the mapping $R_p$, e.g. we have for $m = 2^p$
\begin{align*}
x \ensuremath{\oplus}_{m}^{\rm s} y &:= (x+y)\ ({\rm mod}^{\rm s}\ m) \\
x \ensuremath{\ominus}_{m}^{\rm s} y &:= (x-y)\ ({\rm mod}^{\rm s}\ m) \\
x \ensuremath{\otimes}_{m}^{\rm s} y &:= (x*y)\ ({\rm mod}^{\rm s}\ m)
\end{align*}
\begin{example}[addition of 8-bit signed integers] Consider \texttt{(-1) + 1} in 8-bit arithmetic:
\[
-1 \ensuremath{\oplus}_{256}^{\rm s} 1 = -1 + 1 \ ({\rm mod}^{\rm s}\ 256) = 0
\]
On the bit level this computation is exactly the same as unsigned integers. We represent the number $-1$ using the same bits as the unsigned integer $2^8 - 1 = 255$, that is  using the bits \texttt{11111111} (i.e., we store it equivalently to  $(11111111)_2 = 255$) and the  number $1$ is stored using the bits \texttt{00000001}. When we add this with true integer arithmetic we have
\begin{align*}
(0 11111111)_2 &\ + \\
(0 00000001)_2 &\ = \\
(1 00000000)_2&
\end{align*}
Modular arithmetic drops the leading $1$ and we are left with all zeros. \end{example}

\begin{example}[signed overflow with 8-bit signed integers] If we go above $2^{p-1}-1 = 2^7 - 1 = 127$  we have perhaps unexpected results:
\[
127 \ensuremath{\oplus}_{256}^{\rm s} 1 = 128\  ({\rm mod}^{\rm s}\ 256) = 128 - 256 = -128.
\]
Again on the bit level this computation is exactly the same as unsigned integers. We represent the number $127$ using the bits \texttt{01111111} and the  number $1$ is stored using the bits \texttt{00000001}. When we add this with true integer arithmetic we have
\begin{align*}
(01111111)_2 &\ + \\
(00000001)_2 &\ = \\
(10000000)_2&
\end{align*}
Because the first bit is \texttt{1} we interpret this as a negative number using the formula:
\[
(10000000)_2\ ({\rm mod}^{\rm s}\ 256) = 128   ({\rm mod}^{\rm s}\ 256) = -128.
\]
\end{example}

\begin{example}[multiplication of 8-bit signed integers] Consider computation of \texttt{(-2) * 2}:
\[
(-2) \ensuremath{\otimes}_{2^p}^{\rm s} 2 = -4 \ ({\rm mod}^{\rm s}\ 2^p) = -4
\]
On the bit level, the bits of $-2$ (which is one less than $-1$) are \texttt{11111110}. Multiplying by 2 is like multiplying by 10 in base-10, that is, we shift the bits. Hence in true arithmetic we have
\begin{align*}
(0 11111110)_2 & * 2 = \\
(1 11111100)_2&
\end{align*}
We drop the leading 1 due to modular arithmetic. We still have a leading $1$ hence the number is viewed as negative. In particular we have
\meeq{
(1 11111100)_2 \ ({\rm mod}^{\rm s}\ 256) = (11111100)_2 \ ({\rm mod}^{\rm s}\ 256) = 
2^7+2^6+2^5+2^4+2^3+2^2 \ ({\rm mod}^{\rm s}\ 256) \ccr
 = 252  \ ({\rm mod}^{\rm s}\ 256) = -4.
}
\end{example}

\subsection{Hexadecimal format}
In coding it is often convenient to use base-16 as it is a power of $2$ but uses less characters than binary. The digits used are $0$ through $9$ followed by $a = 10$, $b = 11$, $c = 12$, $d = 13$, $e = 14$, and $f = 15$. 

\begin{example}[Hexadecimal number] We can interpret a number in format as follows:
\[
(a5f2)_{16} = a*16^3 + 5*16^2 + f*16 + 2 = 
10*16^3 + 5*16^2 + 15*16 + 2 = 42,482
\]
\end{example}

We will see in the labs that unsigned integers are displayed in base-16.





\section{Reals}
In this chapter, we introduce  the  \href{https://en.wikipedia.org/wiki/IEEE_754}{IEEE Standard for Floating-Point Arithmetic}. There are multiplies ways of representing real numbers on a computer, as well as  the precise behaviour of operations such as addition, multiplication, etc.: one can use

\begin{itemize}
\item[1. ] \href{https://en.wikipedia.org/wiki/Fixed-point_arithmetic}{Fixed-point arithmetic}: essentially representing a real number as an integer where a decimal point is inserted at a fixed position. This turns out to be impractical in most applications, e.g., due to loss of relative accuracy for small numbers.


\item[2. ] \href{https://en.wikipedia.org/wiki/Floating-point_arithmetic}{Floating-point arithmetic}: essentially scientific notation where an exponent is stored alongside a fixed number of digits. This is what is used in practice.


\item[3. ] \href{https://en.wikipedia.org/wiki/Symmetric_level-index_arithmetic}{Level-index arithmetic}: stores numbers as iterated exponents. This is the most beautiful mathematically but unfortunately is not as useful for most applications and is not implemented in hardware.

\end{itemize}
Before the 1980s each processor had potentially a different representation for  floating-point numbers, as well as different behaviour for operations.  IEEE introduced in 1985 was a means to standardise this across processors so that algorithms would produce consistent and reliable results.

This chapter may seem very low level for a mathematics course but there are two important reasons to understand the behaviour of floating-point numbers in details:

\begin{itemize}
\item[1. ] Floating-point arithmetic is very precisely defined, and can even be used in rigorous computations as we shall see in the labs. But it is not exact and its important to understand how errors in computations can accumulate.


\item[2. ] Failure to understand floating-point arithmetic can cause catastrophic issues in practice, with the extreme example being the  \href{https://youtu.be/N6PWATvLQCY?t=86}{explosion of the Ariane 5 rocket}.

\end{itemize}
\subsection{Real numbers in binary}
Reals can also be presented in binary format, that is, a sequence of \texttt{0}s and \texttt{1}s alongside a decimal point:

\begin{definition}[real binary format] For $b_1,b_2,\ensuremath{\ldots}\in \{0,1\}$, Denote a non-negative real number in \emph{binary format} by:
\[
(B_p \ensuremath{\ldots}B_0.b_1b_2b_3\ensuremath{\ldots})_2 := (B_p \ensuremath{\ldots}B_0)_2 +  \sum_{k=1}^\ensuremath{\infty} {b_k \over 2^k}.
\]
\end{definition}

\begin{example}[rational in binary] Consider the number \texttt{1/3}.  In decimal recall that:
\[
1/3 = 0.3333\ensuremath{\ldots}=  \sum_{k=1}^\ensuremath{\infty} {3 \over 10^k}
\]
We will see that in binary
\[
1/3 = (0.010101\ensuremath{\ldots})_2 = \sum_{k=1}^\ensuremath{\infty} {1 \over 2^{2k}}
\]
Both results can be proven using the geometric series:
\[
\sum_{k=0}^\ensuremath{\infty} z^k = {1 \over 1 - z}
\]
provided $|z| < 1$. That is, with $z = {1 \over 4}$ we verify the binary expansion:
\[
\sum_{k=1}^\ensuremath{\infty} {1 \over 4^k} = {1 \over 1 - 1/4} - 1 = {1 \over 3}
\]
A similar argument with $z = 1/10$ shows the decimal case. \end{example}

\subsection{Floating-point numbers}
Floating-point numbers are a subset of real numbers that are representable using a fixed number of bits.

\begin{definition}[floating-point numbers] Given integers $\ensuremath{\sigma}$ (the \emph{exponential shift}), $Q$ (the number of \emph{exponent bits}) and  $S$ (the \emph{precision}), define the set of \emph{Floating-point numbers} by dividing into \emph{normal}, \emph{sub-normal}, and \emph{special number} subsets:
\[
F_{\ensuremath{\sigma},Q,S} := F^{\rm normal}_{\ensuremath{\sigma},Q,S} \cup F^{\rm sub}_{\ensuremath{\sigma},Q,S} \cup F^{\rm special}.
\]
The \emph{normal numbers} $F^{\rm normal}_{\ensuremath{\sigma},Q,S} \ensuremath{\subset} \ensuremath{\bbR}$ are
\[
F^{\rm normal}_{\ensuremath{\sigma},Q,S} := \{\ensuremath{\pm} 2^{q-\ensuremath{\sigma}} \ensuremath{\times} (1.b_1b_2b_3\ensuremath{\ldots}b_S)_2 : 1 \ensuremath{\leq} q < 2^Q-1 \}.
\]
The \emph{sub-normal numbers} $F^{\rm sub}_{\ensuremath{\sigma},Q,S} \ensuremath{\subset} \ensuremath{\bbR}$ are
\[
F^{\rm sub}_{\ensuremath{\sigma},Q,S} := \{\ensuremath{\pm} 2^{1-\ensuremath{\sigma}} \ensuremath{\times} (0.b_1b_2b_3\ensuremath{\ldots}b_S)_2\}.
\]
The \emph{special numbers} $F^{\rm special} \ensuremath{\nsubset} \ensuremath{\bbR}$ are 
\[
F^{\rm special} :=  \{\ensuremath{\infty}, -\ensuremath{\infty}, {\rm NaN}\}
\]
where ${\rm NaN}$ is a special symbol representing \ensuremath{\ldq}not a number", essentially an error flag. \end{definition}

Note this set of real numbers has no nice \emph{algebraic structure}: it is not closed under addition, subtraction, etc. On the other hand, we can control errors effectively hence it is extremely useful for analysis.

Floating-point numbers are stored in $1 + Q + S$ total number of bits, in the format
\[
sq_{Q-1}\ensuremath{\ldots}q_0 b_1 \ensuremath{\ldots}b_S
\]
The first bit ($s$) is the \emph{sign bit}: 0 means positive and 1 means negative. The bits $q_{Q-1}\ensuremath{\ldots}q_0$ are the \emph{exponent bits}: they are the binary digits of the unsigned integer $q$: 
\[
q = (q_{Q-1}\ensuremath{\ldots}q_0)_2.
\]
Finally, the bits $b_1\ensuremath{\ldots}b_S$ are the \emph{significand bits}. If $1 \ensuremath{\leq} q < 2^Q-1$ then the bits represent the normal number
\[
x = \ensuremath{\pm} 2^{q-\ensuremath{\sigma}} \ensuremath{\times} (1.b_1b_2b_3\ensuremath{\ldots}b_S)_2.
\]
If $q = 0$ (i.e. all bits are 0) then the bits represent the sub-normal number
\[
x = \ensuremath{\pm} 2^{1-\ensuremath{\sigma}} \ensuremath{\times} (0.b_1b_2b_3\ensuremath{\ldots}b_S)_2.
\]
If $q = 2^Q-1$  (i.e. all bits are 1) then the bits represent a special number, discussed later.

\subsection{IEEE floating-point numbers}
\begin{definition}[IEEE floating-point numbers]  IEEE has 3 standard floating-point formats: 16-bit (half precision), 32-bit (single precision) and 64-bit (double precision) defined by (you \emph{do not} need to memorise these):


\begin{align*}
F_{16} &:= F_{15,5,10} \\
F_{32} &:= F_{127,8,23} \\
F_{64} &:= F_{1023,11,52}
\end{align*}
\end{definition}

\begin{example}[a real number in 16-bits] Consider the number with bits

\begin{verbatim}
0 10000 1010000000
\end{verbatim}
assuming it is a half-prevision float ($F_{16}$).  Since the sign bit is \texttt{0} it is positive. The exponent is $2^4 - Q = 16 - 15 = 1$ Hence this number is:
\[
2^1 (1.1010000000)_2 = 2 (1 + 1/2 + 1/8) = 3+1/4 = 3.25.
\]
\end{example}

\begin{example}[rational in 16-bits] How is the number $1/3$ stored in $F_{16}$? Recall that
\[
1/3 = (0.010101\ensuremath{\ldots})_2 = 2^{-2} (1.0101\ensuremath{\ldots})_2 = 2^{13-15} (1.0101\ensuremath{\ldots})_2
\]
and since $13 = (1101)_2$  the exponent bits are \texttt{01101}. For the significand we round the last bit to the nearest element of $F_{16}$,  (the exact rule for rounding is explained in detail later), so we have
\[
1.010101010101010101010101\ensuremath{\ldots}\approx 1.0101010101 \in F_{16} 
\]
and the significand bits are \texttt{0101010101}. Thus the stored bits for $1/3$ are:

\begin{verbatim}
0 01101 0101010101
\end{verbatim}
\end{example}

\subsubsection{Sub-normal and special numbers}
For sub-normal numbers, the simplest example is zero, which has $q=0$ and all significand bits zero: \texttt{0 00000 0000000000}. Unlike integers, we also have a negative zero, which has bits: \texttt{1 00000 0000000000}. This is treated as identical to positive \texttt{0} (except for degenerate operations as explained in special numbers).

\begin{example}[subnormal in 16-bits] Consider the number with bits

\begin{verbatim}
1 00000 1100000000
\end{verbatim}
assuming it is a half-prevision float ($F_{16}$).  Since all exponent bits are zero it is sub-normal. Since the sign bit is \texttt{1} it is negative.  Hence this number is:
\[
-2^{1-\ensuremath{\sigma}} (0.1100000000)_2 = -2^{-14} (2^{-1} + 2^{-2}) = -3 \ensuremath{\times} 2^{-16}
\]
\end{example}

The special numbers extend the real line by adding $\ensuremath{\pm}\ensuremath{\infty}$ but also a notion of ``not-a-number" ${\rm NaN}$. Whenever the bits of $q$ of a floating-point number are all 1 then they represent an element of $F^{\rm special}$. If all $b_k=0$, then the number represents either $\ensuremath{\pm}\ensuremath{\infty}$. All other special floating-point numbers represent ${\rm NaN}$. 

\begin{example}[special in 16-bits] The number with bits

\begin{verbatim}
1 11111 0000000000
\end{verbatim}
has all exponent bits equal to $1$, and significand bits $0$ and sign bit $1$, hence represents $-\ensuremath{\infty}$. On the other hand, the number with bits

\begin{verbatim}
1 11111 0000000001
\end{verbatim}
has all exponent bits equal to $1$ but does not have all significand bits equal to $0$, hence is one of many representations for  ${\rm NaN}$. \end{example}





\section{Floating Point Arithmetic}
Arithmetic operations on floating-point numbers are  \emph{exact up to rounding}. There are three basic rounding strategies: round up/down/nearest. Mathematically we introduce a function to capture the notion of rounding:

\begin{definition}[rounding] ${\rm fl}^{\rm up}_{\ensuremath{\sigma},Q,S} : \mathbb R \rightarrow F_{\ensuremath{\sigma},Q,S}$ denotes the function that rounds a real number up to the nearest floating-point number that is greater or equal. ${\rm fl}^{\rm down}_{\ensuremath{\sigma},Q,S} : \mathbb R \rightarrow F_{\ensuremath{\sigma},Q,S}$ denotes the function that rounds a real number down to the nearest floating-point number that is greater or equal. ${\rm fl}^{\rm nearest}_{\ensuremath{\sigma},Q,S} : \mathbb R \rightarrow F_{\ensuremath{\sigma},Q,S}$ denotes the function that rounds a real number to the nearest floating-point number. In case of a tie, it returns the floating-point number whose least significant bit is equal to zero. We use the notation ${\rm fl}$ when $\ensuremath{\sigma},Q,S$ and the rounding mode are implied by context, with ${\rm fl}^{\rm nearest}$ being the default rounding mode. \end{definition}

In more detail on the behaviour of nearest mode, if a positive number $x$ is between two normal floats $x_- \ensuremath{\leq} x \ensuremath{\leq} x_+$ we can write its expansion as
\[
x = 2^{\green{q}-\ensuremath{\sigma}} (1.\blue{b_1b_2\ensuremath{\ldots}b_S}\red{b_{S+1}\ensuremath{\ldots}})_2
\]
where
\begin{align*}
x_- &:= {\rm fl}^{\rm down}(x) = 2^{\green{q}-\ensuremath{\sigma}} (1.\blue{b_1b_2\ensuremath{\ldots}b_S})_2 \\
x_+ &:= {\rm fl}^{\rm up}(x) = x_- + 2^{\green{q}-S}
\end{align*}
Write the half-way point as:
\[
x_{\rm h} := {x_+ + x_- \over 2} = x_- + 2^{\green{q}-S-1} = 2^{\green{q}-\ensuremath{\sigma}} (1.\blue{b_1b_2\ensuremath{\ldots}b_S}\red{1})_2
\]
If $x_- \ensuremath{\leq} x < x_{\rm h}$ then ${\rm fl}(x) = x_-$ and if $x_{\rm h} < x \ensuremath{\leq} x_+$ then ${\rm fl}(x) = x_{\rm h}$. If $x = x_{\rm h}$ then it is exactly half-way between $x_-$ and $x_+$. The rule is if $b_S = 0$  then ${\rm fl}(x) = x_-$ and otherwise ${\rm fl}(x) = x_+$.

In IEEE arithmetic, the arithmetic operations \texttt{+}, \texttt{-}, \texttt{*}, \texttt{/} are defined by the property that they are exact up to rounding.  Mathematically we denote these operations as $\ensuremath{\oplus}, \ensuremath{\ominus}, \ensuremath{\otimes}, \ensuremath{\oslash} : F_{\ensuremath{\sigma},Q,S} \ensuremath{\otimes} F_{\ensuremath{\sigma},Q,S} \ensuremath{\rightarrow} F_{\ensuremath{\sigma},Q,S}$ as follows:
\begin{align*}
x \ensuremath{\oplus} y &:= {\rm fl}(x+y) \\
x \ensuremath{\ominus} y &:= {\rm fl}(x - y) \\
x \ensuremath{\otimes} y &:= {\rm fl}(x * y) \\
x \ensuremath{\oslash} y &:= {\rm fl}(x / y)
\end{align*}
Note also that  \texttt{\^{}} and \texttt{sqrt} are similarly exact up to rounding. Also, note that when we convert a Julia command with constants specified by decimal expansions we first round the constants to floats, e.g., \texttt{1.1 + 0.1} is actually reduced to
\[
{\rm fl}(1.1) \ensuremath{\oplus} {\rm fl}(0.1)
\]
This includes the case where the constants are integers (which are normally exactly floats but may be rounded if extremely large).

\begin{example}[decimal is not exact] On a computer \texttt{1.1+0.1} is close to but not exactly the same thing as \texttt{1.2}. This is because ${\rm fl}(1.1) \ensuremath{\neq} 1+1/10$ and ${\rm fl}(0.1) \ensuremath{\neq} 1/10$ since their expansion in \emph{binary} is not finite. For $F_{16}$ we have:
\begin{align*}
{\rm fl}(1.1) &= {\rm fl}((1.0001100110\red{011\ensuremath{\ldots}})_2) =  (1.0001100110)_2 \\
{\rm fl}(0.1) &= {\rm fl}(2^{-4}(1.1001100110\red{011\ensuremath{\ldots}})_2) =  2^{-4} * (1.1001100110)_2 = (0.00011001100110)_2
\end{align*}
Thus when we add them we get
\[
{\rm fl}(1.1) + {\rm fl}(0.1) = (1.0011001100\red{011})_2
\]
where the red digits indicate those beyond the 10 significant digits representable in $F_{16}$. In this case we round down and get
\[
{\rm fl}(1.1) \ensuremath{\oplus} {\rm fl}(0.1) = (1.0011001100)_2
\]
On the other hand,
\[
{\rm fl}(1.2) = {\rm fl}((1.0011001100\red{11001100\ensuremath{\ldots}})_2) = (1.0011001101)_2
\]
which differs by 1 bit. \end{example}

\textbf{WARNING (non-associative)} These operations are not associative! E.g. $(x \ensuremath{\oplus} y) \ensuremath{\oplus} z$ is not necessarily equal to $x \ensuremath{\oplus} (y \ensuremath{\oplus} z)$. Commutativity is preserved, at least.

\subsection{Bounding errors in floating point arithmetic}
When dealing with normal numbers there are some important constants that we will use to bound errors.

\begin{definition}[machine epsilon/smallest positive normal number/largest normal number] \emph{Machine epsilon} is denoted
\[
\ensuremath{\epsilon}_{{\rm m},S} := 2^{-S}.
\]
When $S$ is implied by context we use the notation $\ensuremath{\epsilon}_{\rm m}$. The \emph{smallest positive normal number} is $q = 1$ and $b_k$ all zero:
\[
\min |F_{\ensuremath{\sigma},Q,S}^{\rm normal}| = 2^{1-\ensuremath{\sigma}}
\]
where $|A| := \{|x| : x \in A \}$. The \emph{largest (positive) normal number} is
\[
\max F_{\ensuremath{\sigma},Q,S}^{\rm normal} = 2^{2^Q-2-\ensuremath{\sigma}} (1.11\ensuremath{\ldots})_2 = 2^{2^Q-2-\ensuremath{\sigma}} (2-\ensuremath{\epsilon}_{\rm m})
\]
\end{definition}

We can bound the error of basic arithmetic operations in terms of machine epsilon, provided a real number is close to a normal number:

\begin{definition}[normalised range] The \emph{normalised range} ${\cal N}_{\ensuremath{\sigma},Q,S} \ensuremath{\subset} \ensuremath{\bbR}$ is the subset of real numbers that lies between the smallest and largest normal floating-point number:
\[
{\cal N}_{\ensuremath{\sigma},Q,S} := \{x : \min |F_{\ensuremath{\sigma},Q,S}^{\rm normal}| \ensuremath{\leq} |x| \ensuremath{\leq} \max F_{\ensuremath{\sigma},Q,S}^{\rm normal} \}
\]
When $\ensuremath{\sigma},Q,S$ are implied by context we use the notation ${\cal N}$. \end{definition}

We can use machine epsilon to determine bounds on rounding:

\begin{proposition}[round bound] If $x \in {\cal N}$ then
\[
{\rm fl}^{\rm mode}(x) = x (1 + \ensuremath{\delta}_x^{\rm mode})
\]
where the \emph{relative error} is bounded by:
\begin{align*}
|\ensuremath{\delta}_x^{\rm nearest}| &\ensuremath{\leq} {\ensuremath{\epsilon}_{\rm m} \over 2} \\
|\ensuremath{\delta}_x^{\rm up/down}| &< {\ensuremath{\epsilon}_{\rm m}}.
\end{align*}
\end{proposition}
\textbf{Proof}

We will show this result for the nearest rounding mode. Note first that
\[
{\rm fl}(-x) = -{\rm fl}(x)
\]
and hence it suffices to prove the result for positive $x$. Write
\[
x = 2^{\green{q}-\ensuremath{\sigma}} (1.b_1b_2\ensuremath{\ldots}b_S\red{b_{S+1}\ensuremath{\ldots}})_2.
\]
Define
\begin{align*}
x_- &:= {\rm fl}^{\rm down}(x) = 2^{\green{q}-\ensuremath{\sigma}} (1.b_1b_2\ensuremath{\ldots}b_S)_2 \\
x_+ &:= {\rm fl}^{\rm up}(x) = x_- + 2^{j-S} \\
x_{\rm h} &:= {x_+ + x_- \over 2} = x_- + 2^{j-S-1} = 2^{\green{q}-\ensuremath{\sigma}} (1.b_1b_2\ensuremath{\ldots}b_S\red{1})_2
\end{align*}
so that $x_- \ensuremath{\leq} x \ensuremath{\leq} x_+$. We consider two cases separately.

(\textbf{Round Down}) First consider the case where $x$ is such that we round down: ${\rm fl}(x) = x_-$. Since $2^{\green{q}-\ensuremath{\sigma}} \ensuremath{\leq} x_- \ensuremath{\leq} x \ensuremath{\leq} x_{\rm h}$ we have
\[
|\ensuremath{\delta}_x| = {x - x_- \over x} \ensuremath{\leq} {x_{\rm h} - x_- \over x_-} = {2^{j-S-1} \over 2^{\green{q}-\ensuremath{\sigma}}} = 2^{-S-1} = {\ensuremath{\epsilon}_{\rm m} \over 2}.
\]
(\textbf{Round Up}) If ${\rm fl}(x) = x_+$ then $2^{\green{q}-\ensuremath{\sigma}} \ensuremath{\leq} x_- < x_{\rm h} \ensuremath{\leq} x \ensuremath{\leq} x_+$ and hence
\[
|\ensuremath{\delta}_x| = {x_+ - x \over x} \ensuremath{\leq} {x_+ - x_{\rm h} \over x_-} = {2^{j-S-1} \over 2^{\green{q}-\ensuremath{\sigma}}} = 2^{-S-1} = {\ensuremath{\epsilon}_{\rm m} \over 2}.
\]
\ensuremath{\QED}

This immediately implies relative error bounds on all IEEE arithmetic operations, e.g., if $x+y \in {\cal N}$ then we have
\[
x \ensuremath{\oplus} y = (x+y) (1 + \ensuremath{\delta}_1)
\]
where (assuming the default nearest rounding) $|\ensuremath{\delta}_1| \ensuremath{\leq} {\ensuremath{\epsilon}_{\rm m} \over 2}.$

\subsection{Idealised floating point}
With a complicated formula it is mathematically inelegant to work with normalised ranges: one cannot guarantee apriori that a computation always results in a normal float. Extending the bounds to subnormal numbers is tedious, rarely relevant, and beyond the scope of this module. Thus to avoid this issue we will work with an alternative mathematical model:

\begin{definition}[idealised floating point] An idealised mathematical model of floating point numbers for which the only subnormal number is zero can be defined as:
\[
F_{\ensuremath{\infty},S} := \{\ensuremath{\pm} 2^q \ensuremath{\times} (1.b_1b_2b_3\ensuremath{\ldots}b_S)_2 :  q \ensuremath{\in} \ensuremath{\bbZ} \} \ensuremath{\cup} \{0\}
\]
\end{definition}

Note that $F^{\rm normal}_{\ensuremath{\sigma},Q,S} \ensuremath{\subset} F_{\ensuremath{\infty},S}$ for all $\ensuremath{\sigma},Q \ensuremath{\in} \ensuremath{\bbN}$. The definition of rounding ${\rm fl}_{\ensuremath{\infty},S}^{mode} : \ensuremath{\bbR} \ensuremath{\rightarrow} F_{\ensuremath{\infty},S}$ naturally extend to $F_{\ensuremath{\infty},S}$ and hence we can consider bounds for floating point operations such as $\ensuremath{\oplus}$, $\ensuremath{\ominus}$, etc. And in this model the round bound is valid for all real numbers (including $x = 0$).

\begin{example}[bounding a simple computation] We show how to bound the error in computing $(1.1 + 1.2) * 1.3 = 2.99$ and we may assume idealised floating-point arithmetic $F_{\ensuremath{\infty},S}$. First note that \texttt{1.1} on a computer is in fact ${\rm fl}(1.1)$, and we will always assume nearest rounding unless otherwise stated. Thus this computation becomes
\[
({\rm fl}(1.1) \ensuremath{\oplus} {\rm fl}(1.2)) \ensuremath{\otimes} {\rm fl}(1.3)
\]
We will show the \emph{absolute error} is given by
\[
({\rm fl}(1.1) \ensuremath{\oplus} {\rm fl}(1.2)) \ensuremath{\otimes} {\rm fl}(1.3) = 2.99 + \ensuremath{\delta}
\]
where $|\ensuremath{\delta}| \ensuremath{\leq}  11 \ensuremath{\epsilon}_{\rm m}.$ First we find
\meeq{
{\rm fl}(1.1) \ensuremath{\oplus} {\rm fl}(1.2) = (1.1(1 + \ensuremath{\delta}_1) + 1.2 (1+\ensuremath{\delta}_2))(1 + \ensuremath{\delta}_3) \ccr
 = 2.3 + \underbrace{1.1 \ensuremath{\delta}_1 + 1.2 \ensuremath{\delta}_2 + 2.3 \ensuremath{\delta}_3 + 1.1 \ensuremath{\delta}_1 \ensuremath{\delta}_3 + 1.2 \ensuremath{\delta}_2 \ensuremath{\delta}_3}_{\ensuremath{\delta}_4}.
}
While $\ensuremath{\delta}_1 \ensuremath{\delta}_3$ and $\ensuremath{\delta}_2 \ensuremath{\delta}_3$ are absolutely tiny in practice we will bound them rather naïvely by eg.
\[
|\ensuremath{\delta}_1 \ensuremath{\delta}_3| \ensuremath{\leq} |\ensuremath{\epsilon}_{\rm m}^2/4| \ensuremath{\leq} |\ensuremath{\epsilon}_{\rm m}/4|.
\]
Further we round up constants to integers in the bounds for simplicity. We thus have the bound
\[
|\ensuremath{\delta}_4| \ensuremath{\leq} (1+1+2+1+1) \ensuremath{\epsilon}_{\rm m} = 6\ensuremath{\epsilon}_{\rm m}
\]
Thus the computation becomes
\[
(2.3 + \ensuremath{\delta}_4) 1.3 (1 + \ensuremath{\delta}_5) (1 + \ensuremath{\delta}_6) = 2.99 + \underbrace{1.3( \ensuremath{\delta}_4 + 2.3\ensuremath{\delta}_5 + 2.3\ensuremath{\delta}_6 + 
\ensuremath{\delta}_4 \ensuremath{\delta}_5 + \ensuremath{\delta}_4 \ensuremath{\delta}_6 + 2.3 \ensuremath{\delta}_5 \ensuremath{\delta}_6 + \ensuremath{\delta}_4\ensuremath{\delta}_5\ensuremath{\delta}_6)}_{\ensuremath{\delta}_7}
\]
where the \emph{absolute error} is bounded by
\[
|\ensuremath{\delta}_7| \ensuremath{\leq} 2 (6 +  2 + 2 + 1/4 + 1/4 + 3/4 + 1/8) \ensuremath{\epsilon}_{\rm m} \ensuremath{\leq} 24 \ensuremath{\epsilon}_{\rm m}
\]
\end{example}

\subsection{Divided differences floating point error bound}
We can use the bound on floating point arithmetic to deduce a bound on divided differences that captures the phenomena we observed where the error of divided differences became large as $h \ensuremath{\rightarrow} 0$. We assume that the function we are attempting to differentiate is computed using floating point arithmetic in a way that has a small absolute error.

\begin{theorem}[divided difference error bound] Assume we are working in idealised floating-point arithmetic $F_{\ensuremath{\infty},S}$. Let $f$ be twice-differentiable in a neighbourhood of $x \ensuremath{\in} F_{\ensuremath{\infty},S}$ and assume that
\[
 f(x) = f^{\rm FP}(x) + \ensuremath{\delta}_x^f
\]
where $f^{\rm FP} : F_{S,\ensuremath{\infty}} \ensuremath{\rightarrow} F_{S,\ensuremath{\infty}}$ has uniform absolute accuracy in that neighbourhood, that is:
\[
|\ensuremath{\delta}_x^f| \ensuremath{\leq} c \ensuremath{\epsilon}_{\rm m}
\]
for a fixed constant $c \ensuremath{\geq} 0$. For simplicity assume that $h = 2^{-n}$ where $n \ensuremath{\leq} S$ and $|x| \ensuremath{\leq} 1$. The divided difference approximation satisfies
\[
(f^{\rm FP}(x + h) \ensuremath{\ominus} f^{\rm FP}(x)) \ensuremath{\oslash} h = f'(x) + \ensuremath{\delta}_{x,h}^{\rm FD}
\]
where
\[
|\ensuremath{\delta}_{x,h}^{\rm FD}| \ensuremath{\leq} {|f'(x)| \over 2} \ensuremath{\epsilon}_{\rm m} + M h +  {4c \ensuremath{\epsilon}_{\rm m} \over h}
\]
for $M = \sup_{x \ensuremath{\leq} t \ensuremath{\leq} x+h} |f''(t)|$.

\end{theorem}
\textbf{Proof}

We have (noting by our assumptions $x \ensuremath{\oplus} h = x + h$ and that dividing by $h$ will only change the exponent so is exact)
\begin{align*}
(f^{\rm FP}(x + h) \ensuremath{\ominus} f^{\rm FP}(x)) \ensuremath{\oslash} h &= {f(x + h) +  \ensuremath{\delta}^f_{x+h} - f(x) - \ensuremath{\delta}^f_x \over h} (1 + \ensuremath{\delta}_1) \\
&= {f(x+h) - f(x) \over h} (1 + \ensuremath{\delta}_1) + {\ensuremath{\delta}^f_{x+h}- \ensuremath{\delta}^f_x \over h} (1 + \ensuremath{\delta}_1)
\end{align*}
where $|\ensuremath{\delta}_1| \ensuremath{\leq} {\ensuremath{\epsilon}_{\rm m} / 2}$. Applying Taylor's theorem we get
\[
(f^{\rm FP}(x + h) \ensuremath{\ominus} f^{\rm FP}(x)) \ensuremath{\oslash} h = f'(x) + \underbrace{f'(x) \ensuremath{\delta}_1 + {f''(t) \over 2} h (1 + \delta_1) + {\ensuremath{\delta}^f_{x+h}- \ensuremath{\delta}^f_x \over h} (1 + \ensuremath{\delta}_1)}_{\ensuremath{\delta}_{x,h}^{\rm FD}}
\]
The bound then follows, using the very pessimistic bound $|1 + \ensuremath{\delta}_1| \ensuremath{\leq} 2$.

\ensuremath{\QED}

The three-terms of this bound tell us a story: the first term is a fixed (small) error, the second term tends to zero as $h \rightarrow 0$, while the last term grows like $\ensuremath{\epsilon}_{\rm m}/h$ as $h \rightarrow 0$.  Thus we observe convergence while the second term dominates, until the last term takes over. Of course, a bad upper bound is not the same as a proof that something grows, but it is a good indication of what happens \emph{in general} and suffices to choose $h$ so that these errors are balanced (and thus minimised). Since in general we do not have access to the constants $c$ and $M$   we employ the following heuristic to balance the two sources of errors:

\textbf{Heuristic (divided difference with floating-point step)} Choose $h$ proportional to $\sqrt{\ensuremath{\epsilon}_{\rm m}}$ in divided differences  so that $M h$ and ${4c \ensuremath{\epsilon}_{\rm m} \over h}$ are (roughly) the same magnitude.

In the case of double precision $\sqrt{\ensuremath{\epsilon}_{\rm m}} \ensuremath{\approx} 1.5\ensuremath{\times} 10^{-8}$, which is close to when the observed error begins to increase in the examples we saw before.

\textbf{Remark} While divided differences is of debatable utility for computing derivatives, it is extremely effective in building methods for solving differential equations, as we shall see later. It is also very useful as a \ensuremath{\ldq}sanity check" if one wants something to compare with other numerical methods for differentiation.

\textbf{Remark} It is also possible to deduce an error bound for the rectangular rule showing that the error caused by round-off is on the order of $n \ensuremath{\epsilon}_{\rm m}$, that is it does in fact grow but the error without round-off which was bounded by $M/n$ will be substantially greater for all reasonable values of $n$.





\section{Interval Arithmetic}
It is possible to use rounding modes (up/down)  to do rigorous computation to compute bounds on the error in, for example, the digits of $\E$. To do this we will use set/interval arithmetic. For sets $X,Y \ensuremath{\subseteq} \ensuremath{\bbR}$, the set arithmetic operations are defined as
\begin{align*}
X + Y &:= \{x + y : x \ensuremath{\in} X, y \ensuremath{\in} Y\}, \\
XY &:= \{xy : x \ensuremath{\in} X, y \ensuremath{\in} Y\}, \\
X/Y &:= \{x/y : x \ensuremath{\in} X, y \ensuremath{\in} Y\}
\end{align*}
We will use floating point arithmetic to construct approximate set operations $\ensuremath{\oplus}$, $\ensuremath{\otimes}$ so that
\begin{align*}
  X + Y &\ensuremath{\subseteq} X \ensuremath{\oplus} Y, \\
   XY &\ensuremath{\subseteq} X \ensuremath{\otimes} Y,\\
    X/Y &\ensuremath{\subseteq} X \ensuremath{\oslash} Y
    \end{align*}
thereby a complicated algorithm can be run on sets and the true result is guaranteed to be a subset of the output.

When our sets are intervals we can deduce simple formulas for basic arithmetic operations. For simplicity we only consider the case where all values are positive.

\begin{proposition}[interval bounds] For intervals  $X = [a,b]$ and $Y = [c,d]$ satisfying $0 < a \ensuremath{\leq} b$ and $0 < c \ensuremath{\leq} d$, and $n > 0$, we have:
\meeq{
X + Y = [a+c, b+d] \ccr
X/n = [a/n,b/n] \ccr
XY = [ac, bd]
}
\end{proposition}
\textbf{Proof} We first show $X+Y \ensuremath{\subseteq} [a+c,b+d]$. If $z \ensuremath{\in} X + Y$ then $z = x+y$ such that $a \ensuremath{\leq} x \ensuremath{\leq} b$ and $c \ensuremath{\leq} y \ensuremath{\leq} d$ and therefore $a + c \ensuremath{\leq} z \ensuremath{\leq} c + d$ and $z \ensuremath{\in} [a+c,b+d]$. Equality follows from convexity. First note that $a+c, b+d \ensuremath{\in} X+Y$. Any point $z \ensuremath{\in}  [a+b,c+d]$ can be written  as a convex combination of the two endpoints: there exists $0 \ensuremath{\leq} t \ensuremath{\leq} 1$ such that
\[
z = (1-t) (a+c) + t (b+d) =  \underbrace{(1-t) a + t b}_x + \underbrace{(1-t) c + t d}_y
\]
Because intervals are convex we have $x \ensuremath{\in} X$ and $y \ensuremath{\in} Y$ and hence $z \ensuremath{\in} X+Y$. 

The remaining two proofs are left for the problem sheet. 

\ensuremath{\QED}

We want to  implement floating point variants of these operations that are guaranteed to contain the true set arithmetic operations. We do so as follows:

\begin{definition}[floating point interval arithmetic] For intervals  $A = [a,b]$ and $B = [c,d]$ satisfying $0 < a \ensuremath{\leq} b$ and $0 < c \ensuremath{\leq} d$, and $n > 0$, define:
\begin{align*}
[a,b] \ensuremath{\oplus} [c,d] &:= [{\rm fl}^{\rm down}(a+c), {\rm fl}^{\rm up}(b+d)] \\
[a,b] \ensuremath{\ominus} [c,d] &:= [{\rm fl}^{\rm down}(a-d), {\rm fl}^{\rm up}(b-c)] \\
[a,b] \ensuremath{\oslash} n &:= [{\rm fl}^{\rm down}(a/n), {\rm fl}^{\rm up}(b/n)] \\
[a,b] \ensuremath{\otimes} [c,d] &:= [{\rm fl}^{\rm down}(ac), {\rm fl}^{\rm up}(bd)]
\end{align*}
\end{definition}

\begin{example}[small sum] consider evaluating the first few terms in the Taylor series of the exponential at $x = 1$ using interval arithmetic with half-precision $F_{16}$ arithmetic.  The first three terms are exact since all numbers involved are exactly floats, in particular if we evaluate $1 + x + x^2/2$ with $x = 1$ we get
\[
1 + 1 + 1/2 \ensuremath{\in} 1 \ensuremath{\oplus} [1,1] \ensuremath{\oplus} ([1,1] \ensuremath{\otimes} [1,1]) \ensuremath{\oslash} 2 = [5/2, 5/2]
\]
Noting that 
\[
1/6 = (1/3)/2 = 2^{-3} (1.01010101\ensuremath{\ldots})_2
\]
we can extend the computation to another term:
\begin{align*}
1 + 1 + 1/2 + 1/6 &\ensuremath{\in} [5/2,5/2] \ensuremath{\oplus} ([1,1] \ensuremath{\oslash} 6) \ccr
= [2 (1.01)_2, 2 (1.01)_2] \ensuremath{\oplus} 2^{-3}[(1.0101010101)_2, (1.0101010110)_2] \ccr
= [{\rm fl}^{\rm down}(2 (1.0101010101\red{0101})_2), {\rm fl}^{\rm up}(2 (1.0101010101\red{011})_2)] \ccr
= [2(1.0101010101)_2, 2(1.0101010110)_2] \ccr 
= [2.666015625, 2.66796875]
\end{align*}
\end{example}

\begin{example}[exponential with intervals] Consider computing $\exp(x)$ for $0 \ensuremath{\leq} x \ensuremath{\leq} 1$ from the Taylor series approximation:
\[
\exp(x) = \sum_{k=0}^n {x^k \over k!} + \underbrace{\exp(t){x^{n+1} \over (n+1)!}}_{\ensuremath{\delta}_{x,n}}
\]
where we can bound the error by (using the fact that $\ensuremath{\euler} = 2.718\ensuremath{\ldots} \ensuremath{\leq} 3$)
\[
|\ensuremath{\delta}_{x,n}| \ensuremath{\leq} {\exp(1) \over (n+1)!} \ensuremath{\leq} {3 \over (n+1)!}.
\]
Put another way: $\ensuremath{\delta}_{x,n} \ensuremath{\in} \left[-{3 \over (n+1)!}, {3 \over (n+1)!}\right]$. We can use this to adjust the bounds derived from interval arithmetic for the interval arithmetic expression:
\[
\exp(X) \ensuremath{\subseteq} \left(\ensuremath{\bigoplus}_{k=0}^n {X^k \ensuremath{\oslash} k!}\right) \ensuremath{\oplus} \left[-{3 \over (n+1)!}, {3 \over (n+1)!}\right]
\]
For example, with $n = 3$ we have $|\ensuremath{\delta}_{1,2}| \ensuremath{\leq} 3/4! = 1/2^3$. Thus we can prove that:
\meeq{
\ensuremath{\euler} = 1 + 1 + 1/2 + 1/6 + \ensuremath{\delta}_x \ensuremath{\in} [2(1.0101010101)_2, 2(1.0101010110)_2] \ensuremath{\oplus} [-1/2^3, 1/2^3] \ccr
= [2(1.0100010101)_2, 2(1.0110010110)_2] = [2.541015625,2.79296875]
}
In the lab we get many more digits by using a computer to compute the bounds. \end{example}






\chapter{Numerical Linear Algebra}

Many problems in mathematics are linear: for example, polynomial regression and
differential equations. Numerical methods for such applications invariably result
in (finite-dimensional) linear systems that must be solved numerically on a computer: 
the dimensions of the problems are often in the 1000s, millions, or even billion.
One would certainly not want to tackle that with Gaussian elimination by hand!
In this chapter we discuss algorithms, and in particular matrix factorisations, that are
computed using floating point operations. We also introduce some basic applications.



In particular we discuss:

\begin{enumerate}
\item III.1 Polynomial Regression: often if statistics one needs to approximate data by a polynomial.
We discuss how to set up and solve the resulting rectangular systems.
\item III.2 Least Squares: 
\item III.3 Structured Matrices: 
\item III.4 QR Factorisation: 
\item III.5 Differential Equations:
\item III.6 Cholesky Factorisation:
\end{enumerate}

\end{document}