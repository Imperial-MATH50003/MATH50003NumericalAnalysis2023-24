
\section{Differential Equations via Finite Differences}
Linear algebra is a powerful tool for solving linear equations, including \ensuremath{\infty}-dimensional ones like differential equations. In this section we discuss \emph{finite differences}: an algorithmic way of reducing ODEs to linear systems by replacing derivatives with divided differences approximations. 

We will focus on the following differential equations:

\begin{itemize}
\item[1. ] Indefinite integration for $a \ensuremath{\leq} x \ensuremath{\leq} b$:

\end{itemize}

\begin{align*}
u(a) = c, u'(x) = f(x)
\end{align*}
\begin{itemize}
\item[2. ] First-order linear ODEs:
\end{itemize}

\begin{align*}
u(a) = c, u'(x) - \ensuremath{\omega}(x) u(x) = f(x)
\end{align*}
\begin{itemize}
\item[3. ] Poisson equation with Dirichlet conditions:

\end{itemize}

\begin{align*}
u(a) &= c_0, u(b) = c_1, \\
u''(x) &= f(x)
\end{align*}
Briefly, the basic idea of finite differences is as follows:

\begin{itemize}
\item[1. ] Discretise $[a,b]$ by a grid of points $x_0,\ensuremath{\ldots},x_n$.


\item[2. ] Replace derivatives with divided difference approximations.


\item[3. ] Use this to deduce a linear system whose solution approximates $u(x_j)$. 

\end{itemize}
\subsection{Indefinite integration}
We begin with the simplest differential equation on an interval $[a,b]$:
\begin{align*}
u(a) &= c \\
u'(x) &= f(x)
\end{align*}
As in integration we will use an evenly spaced grid $a = x_0 , x_1 , \ensuremath{\ldots}, x_n = b$ defined by $x_j :=  a + h j$ where $h := (b-a)/n$. The solution is of course $u(x) = c + \ensuremath{\int}_a^x f(x) {\rm d}x$. We could use Rectangular or Trapezium rules to to obtain approximations to $u(x_j)$ for each $j$, however, we shall take another (equivalent) approach that will generalise to other differential equations. 

Finite differences consists of three stages to reduce the problem to a linear system. Consider a divided difference approximation like right-sided divided differences: 
\[
u'(x) \ensuremath{\approx} {u(x+h) - u(x)\over h}.
\]
When applied to a grid point $x_0,\ensuremath{\ldots},x_{n-1}$ this becomes:
\[
u'(x_j) \ensuremath{\approx} {u(x_j+h) - u(x_j)\over h} = {u(x_{j+1}) - u(x_j)\over h}
\]
Note that $x_n$ is not permitted since that would go past the interval. We use this approximation as follows:

\begin{itemize}
\item[1. ] Write the ODE and initial conditions on the grid. Since right-sided differences will depend on $x_j$ and $x_{j+1}$ we stop at $x_{n-1}$ to avoid going past our grid: 

\end{itemize}
\[
\Vectt[u(x_0) , 
     u'(x_0) ,
u'(x_1) ,
\ensuremath{\vdots} ,
u'(x_{n-1})] = \underbrace{\Vectt[c, f(x_0), f(x_1), \ensuremath{\vdots} , f(x_{n-1})]}_{\ensuremath{\bm{\b}}}
\]
\begin{itemize}
\item[2. ] Replace derivatives with divided differences:

\end{itemize}
\[
\Vectt[u(x_0) \\ 
(u(x_1) - u(x_0))/h \\
(u(x_2) - u(x_1))/h \\
\ensuremath{\vdots} \\
(u(x_n) - u(x_{n-1})/h] \ensuremath{\approx} \ensuremath{\bm{\b}}
\]
\begin{itemize}
\item[3. ] We do not know $u(x_j)$ hence we replace it with other unknowns $u_j$,

\end{itemize}
but where the approximation becomes equality:
\[
\Vectt[u_0 \\ 
(u_1 - u_0)/h \\
(u_2 - u_1)/h \\
\ensuremath{\vdots} \\
(u_n - u_{n-1})/h] = \ensuremath{\bm{\b}}
\]
\begin{itemize}
\item[4. ] This is actually a lower bidiagonal linear system:

\end{itemize}
\[
\underbrace{\begin{bmatrix}
    1 \\ 
    -1/h & 1/h \\
    & \ensuremath{\ddots} & \ensuremath{\ddots} \\
    && -1/h & 1/h \end{bmatrix}}_L \underbrace{\Vectt[u_0,u_1,\ensuremath{\vdots},u_n]}_{\ensuremath{\bm{\u}}} = \ensuremath{\bm{\b}}
\]
We can solve $L \ensuremath{\bm{\u}} = \ensuremath{\bm{\b}}$ using forward-substitution.

\subsection{Forward Euler}
We can extend this to more general first-order linear differential equations with a variable coefficient:
\begin{align*}
u(a) &= c \\
u'(x) - \ensuremath{\omega}(x) u(x) &= f(x)
\end{align*}
The steps proceed very similar to before:

\begin{itemize}
\item[1. ] Write the ODE and initial conditions on the grid:

\end{itemize}
\[
\Vectt[u(x_0) \\ 
u'(x_0) + \ensuremath{\omega}(x_0) u(x_0) \\
u'(x_1) + \ensuremath{\omega}(x_1) u(x_1) \\
\ensuremath{\vdots} \\
u'(x_{n-1})+ \ensuremath{\omega}(x_{n-1}) u(x_{n-1})] = \underbrace{\Vectt[c, f(x_0), f(x_1), \ensuremath{\vdots} , f(x_{n-1})]}_{\ensuremath{\bm{\b}}}
\]
\begin{itemize}
\item[2. ] Replace derivatives with divided differences:

\end{itemize}
\[
\Vectt[u(x_0) \\ 
(u(x_1) - u(x_0))/h + \ensuremath{\omega}(x_0)u(x_0) \\
(u(x_2) - u(x_1))/h + \ensuremath{\omega}(x_1)u(x_1) \\
\ensuremath{\vdots} \\
(u(x_n) - u(x_{n-1})/h + \ensuremath{\omega}(x_{n-1})u(x_{n-1})] \ensuremath{\approx} \ensuremath{\bm{\b}}
\]
\begin{itemize}
\item[3. ] Replace $u(x_j)$  by its approximation $u_j$:

\end{itemize}
\[
\Vectt[u_0 \\ 
(u_1 - u_0)/h + \ensuremath{\omega}(x_0) u_0 \\
(u_2 - u_1)/h + \ensuremath{\omega}(x_1) u_1 \\
\ensuremath{\vdots} \\
(u_n - u_{n-1})/h  + \ensuremath{\omega}(x_{n-1}) u_{n-1}] = \ensuremath{\bm{\b}}
\]
\begin{itemize}
\item[4. ] We now get the linear system:

\end{itemize}
\[
\underbrace{\begin{bmatrix}
    1 \\ 
    \ensuremath{\omega}(x_0)-1/h & 1/h \\
    & \ensuremath{\ddots} & \ensuremath{\ddots} \\
    && \ensuremath{\omega}(x_{n-1})-1/h & 1/h \end{bmatrix}}_L \underbrace{\Vectt[u_0,u_1,\ensuremath{\vdots},u_n]}_{\ensuremath{\bm{\u}}} = \ensuremath{\bm{\b}}
\]
We can solve $L \ensuremath{\bm{\u}} = \ensuremath{\bm{\b}}$ using forward-substitution.

\subsection{Poisson equation}
Consider the Poisson equation with Dirichlet conditions (a two-point boundary value problem):
\begin{align*}
u(0) &= c \\
u''(x) &= f(x) \\
u(1) &= d
\end{align*}
We shall adapt the procedure using the second-order divided difference approximation from the first probem sheet:
\[
u''(x) \ensuremath{\approx} {u(x-h) - 2u(x) + u(x+h)\over h^2}
\]
When applied to a grid point $x_1,\ensuremath{\ldots},x_{n-1}$ this becomes:
\[
u'(x_j) \ensuremath{\approx} {u(x_j-h) - 2u(x_j) + u(x_j+h)\over h^2} = {u(x_{j-1}) - 2u(x_j) + u(x_{j+1})\over h^2}
\]
Note that $x_0$ and $x_n$ is not permitted since that would go past the interval. We use this approximation as follows:

\begin{itemize}
\item[1. ] Write the ODE and boundary conditions on the grid:

\end{itemize}
\[
\Vectt[u(x_0) \\ 
u''(x_1) \\
u''(x_1) \\
\ensuremath{\vdots} \\
u''(x_{n-1}) \\
u(x_n)] = \underbrace{\Vectt[c, f(x_0), f(x_1), \ensuremath{\vdots} , f(x_{n-1}), d]}_{\ensuremath{\bm{\b}}}
\]
\begin{itemize}
\item[2. ] Replace derivatives with divided differences:

\end{itemize}
\[
\Vectt[u(x_0) \\ 
{u(x_0) - 2u(x_1) + u(x_2)\over h^2} \\
{u(x_1) - 2u(x_2) + u(x_3)\over h^2} \\
\ensuremath{\vdots} \\
{u(x_{n-2}) - 2u(x_{n-1}) + u(x_n)\over h^2} \\
u(x_n)] \ensuremath{\approx} \ensuremath{\bm{\b}}
\]
\begin{itemize}
\item[3. ] Replace $u(x_j)$  by its approximation $u_j$:

\end{itemize}
\[
\Vectt[u_0 \\ 
{u_0 - 2u_1 + u_2\over h^2} \\
{u_1 - 2u_2 + u_3\over h^2} \\
\ensuremath{\vdots} \\
{u_{n-2} - 2u_{n-1} + u_n\over h^2} \\
u_n] = \ensuremath{\bm{\b}}
\]
\begin{itemize}
\item[4. ] We now get a tridiagonal linear system:

\end{itemize}
\[
\underbrace{\begin{bmatrix}
    1 \\ 
    1/h^2 & -2/h^2 & 1/h \\
    & \ensuremath{\ddots} & \ensuremath{\ddots} & \ensuremath{\ddots} \\
   && 1/h^2 & -2/h^2 & 1/h \\ 
   &&&& 1 \end{bmatrix}}_A \underbrace{\Vectt[u_0,u_1,\ensuremath{\vdots},u_n]}_{\ensuremath{\bm{\u}}} = \ensuremath{\bm{\b}}
\]
But how do we solve a tridiagonal linear system $A \ensuremath{\bm{\u}} = \ensuremath{\bm{\b}}$?



