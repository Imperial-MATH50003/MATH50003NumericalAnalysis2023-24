\documentclass[12pt,a4paper]{article}

\usepackage[a4paper,text={16.5cm,25.2cm},centering]{geometry}
\usepackage{lmodern}
\usepackage{amssymb,amsmath}
\usepackage{bm}
\usepackage{graphicx}
\usepackage{microtype}
\usepackage{hyperref}
\setlength{\parindent}{0pt}
\setlength{\parskip}{1.2ex}




\hypersetup
       {   pdfauthor = {  },
           pdftitle={  },
           colorlinks=TRUE,
           linkcolor=black,
           citecolor=blue,
           urlcolor=blue
       }




\usepackage{upquote}
\usepackage{listings}
\usepackage{xcolor}
\lstset{
    basicstyle=\ttfamily\footnotesize,
    upquote=true,
    breaklines=true,
    breakindent=0pt,
    keepspaces=true,
    showspaces=false,
    columns=fullflexible,
    showtabs=false,
    showstringspaces=false,
    escapeinside={(*@}{@*)},
    extendedchars=true,
}
\newcommand{\HLJLt}[1]{#1}
\newcommand{\HLJLw}[1]{#1}
\newcommand{\HLJLe}[1]{#1}
\newcommand{\HLJLeB}[1]{#1}
\newcommand{\HLJLo}[1]{#1}
\newcommand{\HLJLk}[1]{\textcolor[RGB]{148,91,176}{\textbf{#1}}}
\newcommand{\HLJLkc}[1]{\textcolor[RGB]{59,151,46}{\textit{#1}}}
\newcommand{\HLJLkd}[1]{\textcolor[RGB]{214,102,97}{\textit{#1}}}
\newcommand{\HLJLkn}[1]{\textcolor[RGB]{148,91,176}{\textbf{#1}}}
\newcommand{\HLJLkp}[1]{\textcolor[RGB]{148,91,176}{\textbf{#1}}}
\newcommand{\HLJLkr}[1]{\textcolor[RGB]{148,91,176}{\textbf{#1}}}
\newcommand{\HLJLkt}[1]{\textcolor[RGB]{148,91,176}{\textbf{#1}}}
\newcommand{\HLJLn}[1]{#1}
\newcommand{\HLJLna}[1]{#1}
\newcommand{\HLJLnb}[1]{#1}
\newcommand{\HLJLnbp}[1]{#1}
\newcommand{\HLJLnc}[1]{#1}
\newcommand{\HLJLncB}[1]{#1}
\newcommand{\HLJLnd}[1]{\textcolor[RGB]{214,102,97}{#1}}
\newcommand{\HLJLne}[1]{#1}
\newcommand{\HLJLneB}[1]{#1}
\newcommand{\HLJLnf}[1]{\textcolor[RGB]{66,102,213}{#1}}
\newcommand{\HLJLnfm}[1]{\textcolor[RGB]{66,102,213}{#1}}
\newcommand{\HLJLnp}[1]{#1}
\newcommand{\HLJLnl}[1]{#1}
\newcommand{\HLJLnn}[1]{#1}
\newcommand{\HLJLno}[1]{#1}
\newcommand{\HLJLnt}[1]{#1}
\newcommand{\HLJLnv}[1]{#1}
\newcommand{\HLJLnvc}[1]{#1}
\newcommand{\HLJLnvg}[1]{#1}
\newcommand{\HLJLnvi}[1]{#1}
\newcommand{\HLJLnvm}[1]{#1}
\newcommand{\HLJLl}[1]{#1}
\newcommand{\HLJLld}[1]{\textcolor[RGB]{148,91,176}{\textit{#1}}}
\newcommand{\HLJLs}[1]{\textcolor[RGB]{201,61,57}{#1}}
\newcommand{\HLJLsa}[1]{\textcolor[RGB]{201,61,57}{#1}}
\newcommand{\HLJLsb}[1]{\textcolor[RGB]{201,61,57}{#1}}
\newcommand{\HLJLsc}[1]{\textcolor[RGB]{201,61,57}{#1}}
\newcommand{\HLJLsd}[1]{\textcolor[RGB]{201,61,57}{#1}}
\newcommand{\HLJLsdB}[1]{\textcolor[RGB]{201,61,57}{#1}}
\newcommand{\HLJLsdC}[1]{\textcolor[RGB]{201,61,57}{#1}}
\newcommand{\HLJLse}[1]{\textcolor[RGB]{59,151,46}{#1}}
\newcommand{\HLJLsh}[1]{\textcolor[RGB]{201,61,57}{#1}}
\newcommand{\HLJLsi}[1]{#1}
\newcommand{\HLJLso}[1]{\textcolor[RGB]{201,61,57}{#1}}
\newcommand{\HLJLsr}[1]{\textcolor[RGB]{201,61,57}{#1}}
\newcommand{\HLJLss}[1]{\textcolor[RGB]{201,61,57}{#1}}
\newcommand{\HLJLssB}[1]{\textcolor[RGB]{201,61,57}{#1}}
\newcommand{\HLJLnB}[1]{\textcolor[RGB]{59,151,46}{#1}}
\newcommand{\HLJLnbB}[1]{\textcolor[RGB]{59,151,46}{#1}}
\newcommand{\HLJLnfB}[1]{\textcolor[RGB]{59,151,46}{#1}}
\newcommand{\HLJLnh}[1]{\textcolor[RGB]{59,151,46}{#1}}
\newcommand{\HLJLni}[1]{\textcolor[RGB]{59,151,46}{#1}}
\newcommand{\HLJLnil}[1]{\textcolor[RGB]{59,151,46}{#1}}
\newcommand{\HLJLnoB}[1]{\textcolor[RGB]{59,151,46}{#1}}
\newcommand{\HLJLoB}[1]{\textcolor[RGB]{102,102,102}{\textbf{#1}}}
\newcommand{\HLJLow}[1]{\textcolor[RGB]{102,102,102}{\textbf{#1}}}
\newcommand{\HLJLp}[1]{#1}
\newcommand{\HLJLc}[1]{\textcolor[RGB]{153,153,119}{\textit{#1}}}
\newcommand{\HLJLch}[1]{\textcolor[RGB]{153,153,119}{\textit{#1}}}
\newcommand{\HLJLcm}[1]{\textcolor[RGB]{153,153,119}{\textit{#1}}}
\newcommand{\HLJLcp}[1]{\textcolor[RGB]{153,153,119}{\textit{#1}}}
\newcommand{\HLJLcpB}[1]{\textcolor[RGB]{153,153,119}{\textit{#1}}}
\newcommand{\HLJLcs}[1]{\textcolor[RGB]{153,153,119}{\textit{#1}}}
\newcommand{\HLJLcsB}[1]{\textcolor[RGB]{153,153,119}{\textit{#1}}}
\newcommand{\HLJLg}[1]{#1}
\newcommand{\HLJLgd}[1]{#1}
\newcommand{\HLJLge}[1]{#1}
\newcommand{\HLJLgeB}[1]{#1}
\newcommand{\HLJLgh}[1]{#1}
\newcommand{\HLJLgi}[1]{#1}
\newcommand{\HLJLgo}[1]{#1}
\newcommand{\HLJLgp}[1]{#1}
\newcommand{\HLJLgs}[1]{#1}
\newcommand{\HLJLgsB}[1]{#1}
\newcommand{\HLJLgt}[1]{#1}


\def\endash{–}
\def\bbD{ {\mathbb D} }
\def\bbZ{ {\mathbb Z} }

\def\x{ {\vc x} }
\def\a{ {\vc a} }
\def\b{ {\vc b} }
\def\e{ {\vc e} }
\def\f{ {\vc f} }
\def\u{ {\vc u} }

\input{somacros}

\begin{document}



\textbf{Numerical Analysis MATH50003 (2023\ensuremath{\endash}24) Problem Sheet 5}

\textbf{Problem 1(a)}  Suppose $|\ensuremath{\epsilon}_k| \ensuremath{\leq} \ensuremath{\epsilon}$ and $n \ensuremath{\epsilon} < 1$. Use induction to show that
\[
\ensuremath{\prod}_{k=1}^n (1+\ensuremath{\epsilon}_k) = 1+\ensuremath{\theta}_n
\]
for some constant $\ensuremath{\theta}_n$ satisfying
\[
|\ensuremath{\theta}_n| \ensuremath{\leq} \underbrace{n \ensuremath{\epsilon} \over 1-n\ensuremath{\epsilon}}_{E_{n,\ensuremath{\epsilon}}}
\]
\textbf{Problem 1(b)} Show for an idealised floating point vector $\ensuremath{\bm{\x}} \ensuremath{\in} F_{\ensuremath{\infty},S}^n$  that
\[
x_1 \ensuremath{\oplus} \ensuremath{\cdots} \ensuremath{\oplus} x_n = x_1 +  \ensuremath{\cdots} + x_n + \ensuremath{\sigma}_n
\]
where
\[
|\ensuremath{\sigma}_n| \ensuremath{\leq} \| \ensuremath{\bm{\x}} \|_\ensuremath{\infty} E_{n-1,\ensuremath{\epsilon}_{\rm m}/2},
\]
assuming $n \ensuremath{\epsilon}_{\rm m} < 2$ and where $\|\ensuremath{\bm{\x}}\|_\ensuremath{\infty} := \max_k |x_k|$. Hint: use the previous part to first write
\[
x_1 \ensuremath{\oplus} \ensuremath{\cdots} \ensuremath{\oplus} x_n = x_1(1+\ensuremath{\theta}_{n-1}) + \ensuremath{\sum}_{j=2}^n x_j (1 + \ensuremath{\theta}_{n-j+1}).
\]
\textbf{Problem 1(c)} For $A \ensuremath{\in} F_{\ensuremath{\infty},S}^{n \ensuremath{\times} n}$ and $\ensuremath{\bm{\x}} \ensuremath{\in} F_{\ensuremath{\infty},S}^n$ consider the error in approximating matrix multiplication with idealised floating point: for
\[
A \ensuremath{\bm{\x}} =  \begin{pmatrix}
\ensuremath{\bigoplus}_{j=1}^n A_{1,j} \ensuremath{\otimes} x_j\\
\ensuremath{\vdots} \\
\ensuremath{\bigoplus}_{j=1}^n A_{1,j} \ensuremath{\otimes} x_j
\end{pmatrix} + \ensuremath{\delta}
\]
show that
\[
\| \ensuremath{\delta} \|_\ensuremath{\infty} \ensuremath{\leq} 2 \|A\|_\ensuremath{\infty} \| \ensuremath{\bm{\x}} \|_\ensuremath{\infty} E_{n,\ensuremath{\epsilon}_{\rm m}/2}
\]
where  $n \ensuremath{\epsilon}_{\rm m} < 2$ and the matrix norm is $\|A \|_\ensuremath{\infty} := \max_k \ensuremath{\sum}_{j=1}^n |a_{kj}|$.

\rule{\textwidth}{1pt}
\textbf{Problem 2} Derive  Backward Euler: use the left-sided divided difference approximation
\[
u'(x) \ensuremath{\approx} {u(x) - u(x-h)  \over h}
\]
to reduce the first order ODE
\meeq{
u(a) =  c, \qquad u'(x) + \ensuremath{\omega}(x) u(x) = f(x)
}
to a lower triangular system by discretising on the grid $x_j = a + j h$ for $h = (b-a)/n$. Hint: only impose the ODE on the gridpoints $x_1,\ensuremath{\ldots},x_n$ so that the divided difference does not depend on behaviour at $x_{-1}$.

\#\# SOLUTION

We have, {\textbackslash}begin\{align\emph{\} {\textbackslash}frac\{u\emph{\{k+1\} - u}k\}\{h\} {\textbackslash}approx {\textbackslash}frac\{u'(x\emph{\{k+1\}) + u'(x}k)\}\{2\} = {\textbackslash}frac\{a(x\emph{\{k+1\})u}\{k+1\} + a(x\emph{\{k\})u}\{k\}\}\{2\} + {\textbackslash}frac\{1\}\{2\}(f(x\emph{\{k+1\}) + f(x}k)), {\textbackslash}end\{align}\} so we can write, {\textbackslash}begin\{align\emph{\} {\textbackslash}left({\textbackslash}frac\{1\}\{h\} - {\textbackslash}frac\{a(x\emph{\{k+1\})\}\{2\}{\textbackslash}right)u}\{k+1\} + {\textbackslash}left(-{\textbackslash}frac\{1\}\{h\} - {\textbackslash}frac\{a(x\emph{\{k\})\}\{2\}{\textbackslash}right)u}k = {\textbackslash}frac\{1\}\{2\}(f(x\emph{\{k+1\}) + f(x}k)). {\textbackslash}end\{align}\} With the initial condition $u(0) = c$, we can write the whole system as,
\[
\left[\begin{matrix}
1 \\
-\frac{1}{h} - \frac{a(x_1)}{2} && \frac{1}{h} - \frac{a(x_2)}{2} \\
& \ddots && \ddots \\
 & & -\frac{1}{h} - \frac{a(x_{n-1})}{2} && \frac{1}{h} - \frac{a(x_n)}{2}
\end{matrix}\right]\mathbf{u} =  \left[\begin{matrix} 
c \\
\frac{1}{2}\left(f(x_1) + f(x_2)\right) \\
\vdots \\
\frac{1}{2}\left(f(x_{n-1}) + f(x_n)\right)
\end{matrix}\right],
\]
which is lower bidiagonal.

Now if we wish to use forward substitution in a vector linear problem, we can derive in much the same way as above:
\[
\left(\frac{1}{h}I - \frac{A(x_{k+1})}{2}\right)\mathbf{u}_{k+1} + \left(-\frac{1}{h}I - \frac{A(x_{k})}{2}\right)\mathbf{u}_k = \frac{1}{2}(\mathbf{f}(x_{k+1}) + \mathbf{f}(x_k)),
\]
to make the update equation,
\[
\mathbf{u}_{k+1} = \left(I - \frac{h}{2}A(x_{k+1})\right)^{-1}\left(\left(I + \frac{h}{2}A(x_{k})\right)\mathbf{u}_k + \frac{h}{2}(\mathbf{f}(x_{k+1}) + \mathbf{f}(x_k)) \right),
\]
with initial value,
\[
\mathbf{u}_1 = \mathbf{c}.
\]
\textbf{Problem 3} Reduce a Schrödinger equation to a tridiagonal linear system by discretising on the grid $x_j = a + j h$ for $h = (b-a)/n$:
\meeq{
u(a) =  c,\qquad u''(x) + V(x) u(x) = f(x), \qquad u(b) = d.
}


\end{document}