\documentclass[12pt,a4paper]{article}

\usepackage[a4paper,text={16.5cm,25.2cm},centering]{geometry}
\usepackage{lmodern}
\usepackage{amssymb,amsmath}
\usepackage{bm}
\usepackage{graphicx}
\usepackage{microtype}
\usepackage{hyperref}
\setlength{\parindent}{0pt}
\setlength{\parskip}{1.2ex}

\hypersetup
       {   pdfauthor = {  },
           pdftitle={  },
           colorlinks=TRUE,
           linkcolor=black,
           citecolor=blue,
           urlcolor=blue
       }




\usepackage{upquote}
\usepackage{listings}
\usepackage{xcolor}
\lstset{
    basicstyle=\ttfamily\footnotesize,
    upquote=true,
    breaklines=true,
    breakindent=0pt,
    keepspaces=true,
    showspaces=false,
    columns=fullflexible,
    showtabs=false,
    showstringspaces=false,
    escapeinside={(*@}{@*)},
    extendedchars=true,
}
\newcommand{\HLJLt}[1]{#1}
\newcommand{\HLJLw}[1]{#1}
\newcommand{\HLJLe}[1]{#1}
\newcommand{\HLJLeB}[1]{#1}
\newcommand{\HLJLo}[1]{#1}
\newcommand{\HLJLk}[1]{\textcolor[RGB]{148,91,176}{\textbf{#1}}}
\newcommand{\HLJLkc}[1]{\textcolor[RGB]{59,151,46}{\textit{#1}}}
\newcommand{\HLJLkd}[1]{\textcolor[RGB]{214,102,97}{\textit{#1}}}
\newcommand{\HLJLkn}[1]{\textcolor[RGB]{148,91,176}{\textbf{#1}}}
\newcommand{\HLJLkp}[1]{\textcolor[RGB]{148,91,176}{\textbf{#1}}}
\newcommand{\HLJLkr}[1]{\textcolor[RGB]{148,91,176}{\textbf{#1}}}
\newcommand{\HLJLkt}[1]{\textcolor[RGB]{148,91,176}{\textbf{#1}}}
\newcommand{\HLJLn}[1]{#1}
\newcommand{\HLJLna}[1]{#1}
\newcommand{\HLJLnb}[1]{#1}
\newcommand{\HLJLnbp}[1]{#1}
\newcommand{\HLJLnc}[1]{#1}
\newcommand{\HLJLncB}[1]{#1}
\newcommand{\HLJLnd}[1]{\textcolor[RGB]{214,102,97}{#1}}
\newcommand{\HLJLne}[1]{#1}
\newcommand{\HLJLneB}[1]{#1}
\newcommand{\HLJLnf}[1]{\textcolor[RGB]{66,102,213}{#1}}
\newcommand{\HLJLnfm}[1]{\textcolor[RGB]{66,102,213}{#1}}
\newcommand{\HLJLnp}[1]{#1}
\newcommand{\HLJLnl}[1]{#1}
\newcommand{\HLJLnn}[1]{#1}
\newcommand{\HLJLno}[1]{#1}
\newcommand{\HLJLnt}[1]{#1}
\newcommand{\HLJLnv}[1]{#1}
\newcommand{\HLJLnvc}[1]{#1}
\newcommand{\HLJLnvg}[1]{#1}
\newcommand{\HLJLnvi}[1]{#1}
\newcommand{\HLJLnvm}[1]{#1}
\newcommand{\HLJLl}[1]{#1}
\newcommand{\HLJLld}[1]{\textcolor[RGB]{148,91,176}{\textit{#1}}}
\newcommand{\HLJLs}[1]{\textcolor[RGB]{201,61,57}{#1}}
\newcommand{\HLJLsa}[1]{\textcolor[RGB]{201,61,57}{#1}}
\newcommand{\HLJLsb}[1]{\textcolor[RGB]{201,61,57}{#1}}
\newcommand{\HLJLsc}[1]{\textcolor[RGB]{201,61,57}{#1}}
\newcommand{\HLJLsd}[1]{\textcolor[RGB]{201,61,57}{#1}}
\newcommand{\HLJLsdB}[1]{\textcolor[RGB]{201,61,57}{#1}}
\newcommand{\HLJLsdC}[1]{\textcolor[RGB]{201,61,57}{#1}}
\newcommand{\HLJLse}[1]{\textcolor[RGB]{59,151,46}{#1}}
\newcommand{\HLJLsh}[1]{\textcolor[RGB]{201,61,57}{#1}}
\newcommand{\HLJLsi}[1]{#1}
\newcommand{\HLJLso}[1]{\textcolor[RGB]{201,61,57}{#1}}
\newcommand{\HLJLsr}[1]{\textcolor[RGB]{201,61,57}{#1}}
\newcommand{\HLJLss}[1]{\textcolor[RGB]{201,61,57}{#1}}
\newcommand{\HLJLssB}[1]{\textcolor[RGB]{201,61,57}{#1}}
\newcommand{\HLJLnB}[1]{\textcolor[RGB]{59,151,46}{#1}}
\newcommand{\HLJLnbB}[1]{\textcolor[RGB]{59,151,46}{#1}}
\newcommand{\HLJLnfB}[1]{\textcolor[RGB]{59,151,46}{#1}}
\newcommand{\HLJLnh}[1]{\textcolor[RGB]{59,151,46}{#1}}
\newcommand{\HLJLni}[1]{\textcolor[RGB]{59,151,46}{#1}}
\newcommand{\HLJLnil}[1]{\textcolor[RGB]{59,151,46}{#1}}
\newcommand{\HLJLnoB}[1]{\textcolor[RGB]{59,151,46}{#1}}
\newcommand{\HLJLoB}[1]{\textcolor[RGB]{102,102,102}{\textbf{#1}}}
\newcommand{\HLJLow}[1]{\textcolor[RGB]{102,102,102}{\textbf{#1}}}
\newcommand{\HLJLp}[1]{#1}
\newcommand{\HLJLc}[1]{\textcolor[RGB]{153,153,119}{\textit{#1}}}
\newcommand{\HLJLch}[1]{\textcolor[RGB]{153,153,119}{\textit{#1}}}
\newcommand{\HLJLcm}[1]{\textcolor[RGB]{153,153,119}{\textit{#1}}}
\newcommand{\HLJLcp}[1]{\textcolor[RGB]{153,153,119}{\textit{#1}}}
\newcommand{\HLJLcpB}[1]{\textcolor[RGB]{153,153,119}{\textit{#1}}}
\newcommand{\HLJLcs}[1]{\textcolor[RGB]{153,153,119}{\textit{#1}}}
\newcommand{\HLJLcsB}[1]{\textcolor[RGB]{153,153,119}{\textit{#1}}}
\newcommand{\HLJLg}[1]{#1}
\newcommand{\HLJLgd}[1]{#1}
\newcommand{\HLJLge}[1]{#1}
\newcommand{\HLJLgeB}[1]{#1}
\newcommand{\HLJLgh}[1]{#1}
\newcommand{\HLJLgi}[1]{#1}
\newcommand{\HLJLgo}[1]{#1}
\newcommand{\HLJLgp}[1]{#1}
\newcommand{\HLJLgs}[1]{#1}
\newcommand{\HLJLgsB}[1]{#1}
\newcommand{\HLJLgt}[1]{#1}


\def\endash{–}
\def\bbD{ {\mathbb D} }
\def\bbZ{ {\mathbb Z} }


\def\addtab#1={#1\;&=}

\def\meeq#1{\def\ccr{\\\addtab}
%\tabskip=\@centering
 \begin{align*}
 \addtab#1
 \end{align*}
  }  
  
  \def\leqaddtab#1\leq{#1\;&\leq}
  \def\mleeq#1{\def\ccr{\\\addtab}
%\tabskip=\@centering
 \begin{align*}
 \leqaddtab#1
 \end{align*}
  }  


\def\vc#1{\mbox{\boldmath$#1$\unboldmath}}

\def\vcsmall#1{\mbox{\boldmath$\scriptstyle #1$\unboldmath}}

\def\vczero{{\mathbf 0}}


%\def\beginlist{\begin{itemize}}
%
%\def\endlist{\end{itemize}}


\def\pr(#1){\left({#1}\right)}
\def\br[#1]{\left[{#1}\right]}
\def\fbr[#1]{\!\left[{#1}\right]}
\def\set#1{\left\{{#1}\right\}}
\def\ip<#1>{\left\langle{#1}\right\rangle}
\def\iip<#1>{\left\langle\!\langle{#1}\right\rangle\!\rangle}

\def\norm#1{\left\| #1 \right\|}

\def\abs#1{\left|{#1}\right|}
\def\fpr(#1){\!\pr({#1})}

\def\Re{{\rm Re}\,}
\def\Im{{\rm Im}\,}

\def\floor#1{\left\lfloor#1\right\rfloor}
\def\ceil#1{\left\lceil#1\right\rceil}


\def\mapengine#1,#2.{\mapfunction{#1}\ifx\void#2\else\mapengine #2.\fi }

\def\map[#1]{\mapengine #1,\void.}

\def\mapenginesep_#1#2,#3.{\mapfunction{#2}\ifx\void#3\else#1\mapengine #3.\fi }

\def\mapsep_#1[#2]{\mapenginesep_{#1}#2,\void.}


\def\vcbr{\br}


\def\bvect[#1,#2]{
{
\def\dots{\cdots}
\def\mapfunction##1{\ | \  ##1}
\begin{pmatrix}
		 \,#1\map[#2]\,
\end{pmatrix}
}
}

\def\vect[#1]{
{\def\dots{\ldots}
	\vcbr[{#1}]
}}

\def\vectt[#1]{
{\def\dots{\ldots}
	\vect[{#1}]^{\top}
}}

\def\Vectt[#1]{
{
\def\mapfunction##1{##1 \cr} 
\def\dots{\vdots}
	\begin{pmatrix}
		\map[#1]
	\end{pmatrix}
}}



\def\thetaB{\mbox{\boldmath$\theta$}}
\def\zetaB{\mbox{\boldmath$\zeta$}}


\def\newterm#1{{\it #1}\index{#1}}


\def\TT{{\mathbb T}}
\def\C{{\mathbb C}}
\def\R{{\mathbb R}}
\def\II{{\mathbb I}}
\def\F{{\mathcal F}}
\def\E{{\rm e}}
\def\I{{\rm i}}
\def\D{{\rm d}}
\def\dx{\D x}
\def\ds{\D s}
\def\dt{\D t}
\def\CC{{\cal C}}
\def\DD{{\cal D}}
\def\U{{\mathbb U}}
\def\A{{\cal A}}
\def\K{{\cal K}}
\def\DTU{{\cal D}_{{\rm T} \rightarrow {\rm U}}}
\def\LL{{\cal L}}
\def\B{{\cal B}}
\def\T{{\cal T}}
\def\W{{\cal W}}


\def\tF_#1{{\tt F}_{#1}}
\def\Fm{\tF_m}
\def\Fab{\tF_{\alpha,\beta}}
\def\FC{\T}
\def\FCpmz{\FC^{\pm {\rm z}}}
\def\FCz{\FC^{\rm z}}

\def\tFC_#1{{\tt T}_{#1}}
\def\FCn{\tFC_n}

\def\rmz{{\rm z}}

\def\chapref#1{Chapter~\ref{Chapter:#1}}
\def\secref#1{Section~\ref{Section:#1}}
\def\exref#1{Exercise~\ref{Exercise:#1}}
\def\lmref#1{Lemma~\ref{Lemma:#1}}
\def\propref#1{Proposition~\ref{Proposition:#1}}
\def\warnref#1{Warning~\ref{Warning:#1}}
\def\thref#1{Theorem~\ref{Theorem:#1}}
\def\defref#1{Definition~\ref{Definition:#1}}
\def\probref#1{Problem~\ref{Problem:#1}}
\def\corref#1{Corollary~\ref{Corollary:#1}}

\def\sgn{{\rm sgn}\,}
\def\Ai{{\rm Ai}\,}
\def\Bi{{\rm Bi}\,}
\def\wind{{\rm wind}\,}
\def\erf{{\rm erf}\,}
\def\erfc{{\rm erfc}\,}
\def\qqquad{\qquad\quad}
\def\qqqquad{\qquad\qquad}


\def\spand{\hbox{ and }}
\def\spodd{\hbox{ odd}}
\def\speven{\hbox{ even}}
\def\qand{\quad\hbox{and}\quad}
\def\qqand{\qquad\hbox{and}\qquad}
\def\qfor{\quad\hbox{for}\quad}
\def\qqfor{\qquad\hbox{for}\qquad}
\def\qas{\quad\hbox{as}\quad}
\def\qqas{\qquad\hbox{as}\qquad}
\def\qor{\quad\hbox{or}\quad}
\def\qqor{\qquad\hbox{or}\qquad}
\def\qqwhere{\qquad\hbox{where}\qquad}



%%% Words

\def\naive{na\"\i ve\xspace}
\def\Jmap{Joukowsky map\xspace}
\def\Mobius{M\"obius\xspace}
\def\Holder{H\"older\xspace}
\def\Mathematica{{\sc Mathematica}\xspace}
\def\apriori{apriori\xspace}
\def\WHf{Weiner--Hopf factorization\xspace}
\def\WHfs{Weiner--Hopf factorizations\xspace}

\def\Jup{J_\uparrow^{-1}}
\def\Jdown{J_\downarrow^{-1}}
\def\Jin{J_+^{-1}}
\def\Jout{J_-^{-1}}



\def\bD{\D\!\!\!^-}




\def\questionequals{= \!\!\!\!\!\!{\scriptstyle ? \atop }\,\,\,}

\def\elll#1{\ell^{\lambda,#1}}
\def\elllp{\ell^{\lambda,p}}
\def\elllRp{\ell^{(\lambda,R),p}}


\def\elllRpz_#1{\ell_{#1{\rm z}}^{(\lambda,R),p}}


\def\sopmatrix#1{\begin{pmatrix}#1\end{pmatrix}}


\def\bbR{{\mathbb R}}
\def\bbC{{\mathbb C}}


\begin{document}



\textbf{Numerical Analysis MATH50003 (2023\ensuremath{\endash}24) Problem Sheet 2}

\textbf{Problem 1} Using dual number arithmetic, compute the following polynomials evaluated at the dual number $2+\ensuremath{\epsilon}$ and use this to deduce their derivative at $2$:
\[
2x^2 + 3x + 4, (x+1)(x+2)(x+3), (2x+1)x^3
\]
\textbf{SOLUTION} (a)
\[
2(2+\ensuremath{\epsilon})^2 + 3(2+\ensuremath{\epsilon}) + 4 = 2(4+4\ensuremath{\epsilon}) + 6+3\ensuremath{\epsilon} + 4 = 18 + 11\ensuremath{\epsilon}
\]
so the derivative is 11.

(b) 
\[
(3+\ensuremath{\epsilon})(4+\ensuremath{\epsilon})(5+\ensuremath{\epsilon}) = (12+7\ensuremath{\epsilon})(5+\ensuremath{\epsilon}) = 60+47\ensuremath{\epsilon}
\]
so the derivative is 47.

(c)
\[
(2(2+\ensuremath{\epsilon})+1)(2+\ensuremath{\epsilon})^3 = 
(5+2\ensuremath{\epsilon})(4+4\ensuremath{\epsilon})(2+\ensuremath{\epsilon}) = 
(20 + 28\ensuremath{\epsilon})(2+\ensuremath{\epsilon}) = 40 + 76\ensuremath{\epsilon}
\]
so the derivative is 76.

\textbf{END}

\textbf{Problem 2} What should the following functions applied to dual numbers return for $x = a+b \ensuremath{\epsilon}$:
\[
f(x) = x^{100}    + 1, g(x) = 1/x, h(x) = \tan x
\]
State  the domain where these definitions are valid.

\textbf{SOLUTION}
\[
f(a+ b\ensuremath{\epsilon}) = f(a) + b f'(a) \ensuremath{\epsilon} = a^{100} + 100ba^{99} \ensuremath{\epsilon}
\]
valid everywhere.
\[
g(a+ b\ensuremath{\epsilon}) =  {1 \over a} - {b \over a^2} \ensuremath{\epsilon}
\]
valid for $a \ensuremath{\neq} 0$.
\[
h(a+b\ensuremath{\epsilon}) = \tan a + b \sec^2 a \ensuremath{\epsilon}
\]
valid for $a \ensuremath{\notin} \{ k\ensuremath{\pi}+\ensuremath{\pi}/2 : k \ensuremath{\in} \ensuremath{\bbZ}\}$.

\textbf{END}

\textbf{Problem 3(a)} What is the correct definition of division on dual numbers, i.e.,
\[
(a + b \ensuremath{\epsilon} )/(c + d \ensuremath{\epsilon} ) = s + t \ensuremath{\epsilon}
\]
for what choice of $s$ and $t$? 

\textbf{SOLUTION}

As with complex numbers, division is easiest to understand by first multiplying with the conjugate, that is:
\[
\frac{a+b\ensuremath{\epsilon}}{c+d\ensuremath{\epsilon}} = \frac{(a+b\ensuremath{\epsilon})(c-d\ensuremath{\epsilon})}{(c+d\ensuremath{\epsilon})(c-d\ensuremath{\epsilon})}.
\]
Expanding the products and dropping terms with $\ensuremath{\epsilon}^2$ then leaves us with the definition of division for dual numbers (where the denominator must have non-zero real part):
\[
\frac{a}{c} + \frac{bc - ad}{c^2}\ensuremath{\epsilon}.
\]
Thus we have $s = \frac{a}{c}$ and $t = \frac{bc - ad}{c^2}$.

\textbf{END}

\textbf{Problem 3(b)} A \emph{field} is a commutative ring such that $0 \ensuremath{\neq} 1$ and all nonzero elements have a multiplicative inverse, i.e., there exists $a^{-1}$ such that $a a^{-1} = 1$. Can we use the previous part to define $a^{-1} := 1/a$ to make $\ensuremath{\bbD}$ a field? Why or why not?

\textbf{SOLUTION}

Fields require that all nonzero elements have a unique multiplicative inverse. However, this is not the case for dual numbers. To give an explicit counter example, we show that there is no dual number $z$ which is the inverse of $0+\ensuremath{\epsilon}$, i.e. a dual number $z$ such that
\[
\frac{(0+\ensuremath{\epsilon})}{(z_r+z_d \ensuremath{\epsilon})} = 1 + 0 \ensuremath{\epsilon}.
\]
By appropriate multiplication with the conjugate we show that
\[
\frac{(0+\ensuremath{\epsilon})(z_r-z_d \ensuremath{\epsilon})}{(z_r+z_d \ensuremath{\epsilon})(z_r-z_d \ensuremath{\epsilon})} = \frac{z_r \ensuremath{\epsilon}}{z_r^2} = \frac{\ensuremath{\epsilon}}{z_r}.
\]
This proves that no choice of real part $z_r$ can reach the multiplicative identity $1+0 \ensuremath{\epsilon}$ when starting from the number $0+\ensuremath{\epsilon}$. More general results for zero real part dual numbers can also be proved.

\textbf{END}

\textbf{Problem 4} Use dual numbers to compute the derivative of the following functions at $x = 0.1$:
\[
\exp(\exp x \cos x + \sin x), \prod_{k=1}^3 \left({x \over k}-1\right),\hbox{ and } f^{\rm s}_2(x) = {1 + {x - 1 \over 2 + {x-1 \over 2}}}
\]
\textbf{SOLUTION}

We now compute the derivatives of the three functions by evaluating for $x = 0.1 + \ensuremath{\epsilon}$. For the first function we have:
\begin{align*}
&\exp(\exp(0.1 + \ensuremath{\epsilon})\cos(0.1+\ensuremath{\epsilon}) + \sin(0.1+\ensuremath{\epsilon})) \\
&=\exp((\exp(0.1) + \ensuremath{\epsilon}\exp(0.1))(\cos(0.1)-\sin(0.1)\ensuremath{\epsilon}) + \sin(0.1)+\cos(0.1)\ensuremath{\epsilon}) \\
&= \exp(\exp(0.1)\cos(0.1)+ \sin(0.1) + (\exp(0.1)(\cos(0.1)-\sin(0.1))+\cos(0.1))\ensuremath{\epsilon}) \\
&= \exp(\exp(0.1)\cos(0.1)+ \sin(0.1))\\
&\qquad + \exp(\exp(0.1)\cos(0.1)+ \sin(0.1))\exp(0.1)(\cos(0.1)-\sin(0.1))+\cos(0.1))\ensuremath{\epsilon} 
\end{align*}
therefore the derivative is the dual part
\[
\exp(\exp(0.1)\cos(0.1)+ \sin(0.1))(\exp(0.1)(\cos(0.1)-\sin(0.1))+\cos(0.1))
\]
For the second function we have:
\begin{align*}
 \left(0.1+\ensuremath{\epsilon}-1\right) \left({0.1 + \ensuremath{\epsilon} \over 2}-1\right)\left({0.1 + \ensuremath{\epsilon} \over 3}-1\right)
 &=\left(-0.9+\ensuremath{\epsilon}\right) \left(-0.95 + \ensuremath{\epsilon}/2\right)\left(-29/30 + \ensuremath{\epsilon}/3\right) \\
&=\left(171/200 -1.4\ensuremath{\epsilon}\right)\left(-29/30 + \ensuremath{\epsilon}/3\right)  \\
&= -1653/2000 + 983\ensuremath{\epsilon}/600
\end{align*}
Thus the derivative is $983/600$.

For the third function we have:
\begin{align*}
{1 + {0.1+\ensuremath{\epsilon} - 1 \over 2 + {0.1+\ensuremath{\epsilon}-1 \over 2}}} &=  {1 + {-0.9+\ensuremath{\epsilon} \over 1.55 + \ensuremath{\epsilon}/2}}\\
&= 1 -18/31 + 2\ensuremath{\epsilon}/1.55^2
\end{align*}
Thus the derivative is $2/1.55^2$.

\textbf{END}



\end{document}