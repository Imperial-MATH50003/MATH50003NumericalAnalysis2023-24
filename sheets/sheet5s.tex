\documentclass[12pt,a4paper]{article}

\usepackage[a4paper,text={16.5cm,25.2cm},centering]{geometry}
\usepackage{lmodern}
\usepackage{amssymb,amsmath}
\usepackage{bm}
\usepackage{graphicx}
\usepackage{microtype}
\usepackage{hyperref}
\usepackage[usenames,dvipsnames]{xcolor}
\setlength{\parindent}{0pt}
\setlength{\parskip}{1.2ex}




\hypersetup
       {   pdfauthor = {  },
           pdftitle={  },
           colorlinks=TRUE,
           linkcolor=black,
           citecolor=blue,
           urlcolor=blue
       }




\usepackage{upquote}
\usepackage{listings}
\usepackage{xcolor}
\lstset{
    basicstyle=\ttfamily\footnotesize,
    upquote=true,
    breaklines=true,
    breakindent=0pt,
    keepspaces=true,
    showspaces=false,
    columns=fullflexible,
    showtabs=false,
    showstringspaces=false,
    escapeinside={(*@}{@*)},
    extendedchars=true,
}
\newcommand{\HLJLt}[1]{#1}
\newcommand{\HLJLw}[1]{#1}
\newcommand{\HLJLe}[1]{#1}
\newcommand{\HLJLeB}[1]{#1}
\newcommand{\HLJLo}[1]{#1}
\newcommand{\HLJLk}[1]{\textcolor[RGB]{148,91,176}{\textbf{#1}}}
\newcommand{\HLJLkc}[1]{\textcolor[RGB]{59,151,46}{\textit{#1}}}
\newcommand{\HLJLkd}[1]{\textcolor[RGB]{214,102,97}{\textit{#1}}}
\newcommand{\HLJLkn}[1]{\textcolor[RGB]{148,91,176}{\textbf{#1}}}
\newcommand{\HLJLkp}[1]{\textcolor[RGB]{148,91,176}{\textbf{#1}}}
\newcommand{\HLJLkr}[1]{\textcolor[RGB]{148,91,176}{\textbf{#1}}}
\newcommand{\HLJLkt}[1]{\textcolor[RGB]{148,91,176}{\textbf{#1}}}
\newcommand{\HLJLn}[1]{#1}
\newcommand{\HLJLna}[1]{#1}
\newcommand{\HLJLnb}[1]{#1}
\newcommand{\HLJLnbp}[1]{#1}
\newcommand{\HLJLnc}[1]{#1}
\newcommand{\HLJLncB}[1]{#1}
\newcommand{\HLJLnd}[1]{\textcolor[RGB]{214,102,97}{#1}}
\newcommand{\HLJLne}[1]{#1}
\newcommand{\HLJLneB}[1]{#1}
\newcommand{\HLJLnf}[1]{\textcolor[RGB]{66,102,213}{#1}}
\newcommand{\HLJLnfm}[1]{\textcolor[RGB]{66,102,213}{#1}}
\newcommand{\HLJLnp}[1]{#1}
\newcommand{\HLJLnl}[1]{#1}
\newcommand{\HLJLnn}[1]{#1}
\newcommand{\HLJLno}[1]{#1}
\newcommand{\HLJLnt}[1]{#1}
\newcommand{\HLJLnv}[1]{#1}
\newcommand{\HLJLnvc}[1]{#1}
\newcommand{\HLJLnvg}[1]{#1}
\newcommand{\HLJLnvi}[1]{#1}
\newcommand{\HLJLnvm}[1]{#1}
\newcommand{\HLJLl}[1]{#1}
\newcommand{\HLJLld}[1]{\textcolor[RGB]{148,91,176}{\textit{#1}}}
\newcommand{\HLJLs}[1]{\textcolor[RGB]{201,61,57}{#1}}
\newcommand{\HLJLsa}[1]{\textcolor[RGB]{201,61,57}{#1}}
\newcommand{\HLJLsb}[1]{\textcolor[RGB]{201,61,57}{#1}}
\newcommand{\HLJLsc}[1]{\textcolor[RGB]{201,61,57}{#1}}
\newcommand{\HLJLsd}[1]{\textcolor[RGB]{201,61,57}{#1}}
\newcommand{\HLJLsdB}[1]{\textcolor[RGB]{201,61,57}{#1}}
\newcommand{\HLJLsdC}[1]{\textcolor[RGB]{201,61,57}{#1}}
\newcommand{\HLJLse}[1]{\textcolor[RGB]{59,151,46}{#1}}
\newcommand{\HLJLsh}[1]{\textcolor[RGB]{201,61,57}{#1}}
\newcommand{\HLJLsi}[1]{#1}
\newcommand{\HLJLso}[1]{\textcolor[RGB]{201,61,57}{#1}}
\newcommand{\HLJLsr}[1]{\textcolor[RGB]{201,61,57}{#1}}
\newcommand{\HLJLss}[1]{\textcolor[RGB]{201,61,57}{#1}}
\newcommand{\HLJLssB}[1]{\textcolor[RGB]{201,61,57}{#1}}
\newcommand{\HLJLnB}[1]{\textcolor[RGB]{59,151,46}{#1}}
\newcommand{\HLJLnbB}[1]{\textcolor[RGB]{59,151,46}{#1}}
\newcommand{\HLJLnfB}[1]{\textcolor[RGB]{59,151,46}{#1}}
\newcommand{\HLJLnh}[1]{\textcolor[RGB]{59,151,46}{#1}}
\newcommand{\HLJLni}[1]{\textcolor[RGB]{59,151,46}{#1}}
\newcommand{\HLJLnil}[1]{\textcolor[RGB]{59,151,46}{#1}}
\newcommand{\HLJLnoB}[1]{\textcolor[RGB]{59,151,46}{#1}}
\newcommand{\HLJLoB}[1]{\textcolor[RGB]{102,102,102}{\textbf{#1}}}
\newcommand{\HLJLow}[1]{\textcolor[RGB]{102,102,102}{\textbf{#1}}}
\newcommand{\HLJLp}[1]{#1}
\newcommand{\HLJLc}[1]{\textcolor[RGB]{153,153,119}{\textit{#1}}}
\newcommand{\HLJLch}[1]{\textcolor[RGB]{153,153,119}{\textit{#1}}}
\newcommand{\HLJLcm}[1]{\textcolor[RGB]{153,153,119}{\textit{#1}}}
\newcommand{\HLJLcp}[1]{\textcolor[RGB]{153,153,119}{\textit{#1}}}
\newcommand{\HLJLcpB}[1]{\textcolor[RGB]{153,153,119}{\textit{#1}}}
\newcommand{\HLJLcs}[1]{\textcolor[RGB]{153,153,119}{\textit{#1}}}
\newcommand{\HLJLcsB}[1]{\textcolor[RGB]{153,153,119}{\textit{#1}}}
\newcommand{\HLJLg}[1]{#1}
\newcommand{\HLJLgd}[1]{#1}
\newcommand{\HLJLge}[1]{#1}
\newcommand{\HLJLgeB}[1]{#1}
\newcommand{\HLJLgh}[1]{#1}
\newcommand{\HLJLgi}[1]{#1}
\newcommand{\HLJLgo}[1]{#1}
\newcommand{\HLJLgp}[1]{#1}
\newcommand{\HLJLgs}[1]{#1}
\newcommand{\HLJLgsB}[1]{#1}
\newcommand{\HLJLgt}[1]{#1}


\def\endash{–}
\def\bbD{ {\mathbb D} }
\def\bbZ{ {\mathbb Z} }
\def\bbR{ {\mathbb R} }
\def\bbC{ {\mathbb C} }

\def\x{ {\vc x} }
\def\a{ {\vc a} }
\def\b{ {\vc b} }
\def\e{ {\vc e} }
\def\f{ {\vc f} }
\def\u{ {\vc u} }
\def\v{ {\vc v} }
\def\y{ {\vc y} }
\def\z{ {\vc z} }

\def\red#1{ {\color{red} #1} }
\def\blue#1{ {\color{blue} #1} }
\def\green#1{ {\color{ForestGreen} #1} }
\def\magenta#1{ {\color{magenta} #1} }


\def\addtab#1={#1\;&=}

\def\meeq#1{\def\ccr{\\\addtab}
%\tabskip=\@centering
 \begin{align*}
 \addtab#1
 \end{align*}
  }  
  
  \def\leqaddtab#1\leq{#1\;&\leq}
  \def\mleeq#1{\def\ccr{\\\addtab}
%\tabskip=\@centering
 \begin{align*}
 \leqaddtab#1
 \end{align*}
  }  


\def\vc#1{\mbox{\boldmath$#1$\unboldmath}}

\def\vcsmall#1{\mbox{\boldmath$\scriptstyle #1$\unboldmath}}

\def\vczero{{\mathbf 0}}


%\def\beginlist{\begin{itemize}}
%
%\def\endlist{\end{itemize}}


\def\pr(#1){\left({#1}\right)}
\def\br[#1]{\left[{#1}\right]}
\def\fbr[#1]{\!\left[{#1}\right]}
\def\set#1{\left\{{#1}\right\}}
\def\ip<#1>{\left\langle{#1}\right\rangle}
\def\iip<#1>{\left\langle\!\langle{#1}\right\rangle\!\rangle}

\def\norm#1{\left\| #1 \right\|}

\def\abs#1{\left|{#1}\right|}
\def\fpr(#1){\!\pr({#1})}

\def\Re{{\rm Re}\,}
\def\Im{{\rm Im}\,}

\def\floor#1{\left\lfloor#1\right\rfloor}
\def\ceil#1{\left\lceil#1\right\rceil}


\def\mapengine#1,#2.{\mapfunction{#1}\ifx\void#2\else\mapengine #2.\fi }

\def\map[#1]{\mapengine #1,\void.}

\def\mapenginesep_#1#2,#3.{\mapfunction{#2}\ifx\void#3\else#1\mapengine #3.\fi }

\def\mapsep_#1[#2]{\mapenginesep_{#1}#2,\void.}


\def\vcbr{\br}


\def\bvect[#1,#2]{
{
\def\dots{\cdots}
\def\mapfunction##1{\ | \  ##1}
\begin{pmatrix}
		 \,#1\map[#2]\,
\end{pmatrix}
}
}

\def\vect[#1]{
{\def\dots{\ldots}
	\vcbr[{#1}]
}}

\def\vectt[#1]{
{\def\dots{\ldots}
	\vect[{#1}]^{\top}
}}

\def\Vectt[#1]{
{
\def\mapfunction##1{##1 \cr} 
\def\dots{\vdots}
	\begin{pmatrix}
		\map[#1]
	\end{pmatrix}
}}



\def\thetaB{\mbox{\boldmath$\theta$}}
\def\zetaB{\mbox{\boldmath$\zeta$}}


\def\newterm#1{{\it #1}\index{#1}}


\def\TT{{\mathbb T}}
\def\C{{\mathbb C}}
\def\R{{\mathbb R}}
\def\II{{\mathbb I}}
\def\F{{\mathcal F}}
\def\E{{\rm e}}
\def\I{{\rm i}}
\def\D{{\rm d}}
\def\dx{\D x}
\def\ds{\D s}
\def\dt{\D t}
\def\CC{{\cal C}}
\def\DD{{\cal D}}
\def\U{{\mathbb U}}
\def\A{{\cal A}}
\def\K{{\cal K}}
\def\DTU{{\cal D}_{{\rm T} \rightarrow {\rm U}}}
\def\LL{{\cal L}}
\def\B{{\cal B}}
\def\T{{\cal T}}
\def\W{{\cal W}}


\def\tF_#1{{\tt F}_{#1}}
\def\Fm{\tF_m}
\def\Fab{\tF_{\alpha,\beta}}
\def\FC{\T}
\def\FCpmz{\FC^{\pm {\rm z}}}
\def\FCz{\FC^{\rm z}}

\def\tFC_#1{{\tt T}_{#1}}
\def\FCn{\tFC_n}

\def\rmz{{\rm z}}

\def\chapref#1{Chapter~\ref{Chapter:#1}}
\def\secref#1{Section~\ref{Section:#1}}
\def\exref#1{Exercise~\ref{Exercise:#1}}
\def\lmref#1{Lemma~\ref{Lemma:#1}}
\def\propref#1{Proposition~\ref{Proposition:#1}}
\def\warnref#1{Warning~\ref{Warning:#1}}
\def\thref#1{Theorem~\ref{Theorem:#1}}
\def\defref#1{Definition~\ref{Definition:#1}}
\def\probref#1{Problem~\ref{Problem:#1}}
\def\corref#1{Corollary~\ref{Corollary:#1}}

\def\sgn{{\rm sgn}\,}
\def\Ai{{\rm Ai}\,}
\def\Bi{{\rm Bi}\,}
\def\wind{{\rm wind}\,}
\def\erf{{\rm erf}\,}
\def\erfc{{\rm erfc}\,}
\def\qqquad{\qquad\quad}
\def\qqqquad{\qquad\qquad}


\def\spand{\hbox{ and }}
\def\spodd{\hbox{ odd}}
\def\speven{\hbox{ even}}
\def\qand{\quad\hbox{and}\quad}
\def\qqand{\qquad\hbox{and}\qquad}
\def\qfor{\quad\hbox{for}\quad}
\def\qqfor{\qquad\hbox{for}\qquad}
\def\qas{\quad\hbox{as}\quad}
\def\qqas{\qquad\hbox{as}\qquad}
\def\qor{\quad\hbox{or}\quad}
\def\qqor{\qquad\hbox{or}\qquad}
\def\qqwhere{\qquad\hbox{where}\qquad}



%%% Words

\def\naive{na\"\i ve\xspace}
\def\Jmap{Joukowsky map\xspace}
\def\Mobius{M\"obius\xspace}
\def\Holder{H\"older\xspace}
\def\Mathematica{{\sc Mathematica}\xspace}
\def\apriori{apriori\xspace}
\def\WHf{Weiner--Hopf factorization\xspace}
\def\WHfs{Weiner--Hopf factorizations\xspace}

\def\Jup{J_\uparrow^{-1}}
\def\Jdown{J_\downarrow^{-1}}
\def\Jin{J_+^{-1}}
\def\Jout{J_-^{-1}}



\def\bD{\D\!\!\!^-}




\def\questionequals{= \!\!\!\!\!\!{\scriptstyle ? \atop }\,\,\,}

\def\elll#1{\ell^{\lambda,#1}}
\def\elllp{\ell^{\lambda,p}}
\def\elllRp{\ell^{(\lambda,R),p}}


\def\elllRpz_#1{\ell_{#1{\rm z}}^{(\lambda,R),p}}


\def\sopmatrix#1{\begin{pmatrix}#1\end{pmatrix}}


\def\bbR{{\mathbb R}}
\def\bbC{{\mathbb C}}


\begin{document}



\textbf{Numerical Analysis MATH50003 (2023\ensuremath{\endash}24) Problem Sheet 5}

\textbf{Problem 1(a)}  Suppose $|\ensuremath{\epsilon}_k| \ensuremath{\leq} \ensuremath{\epsilon}$ and $n \ensuremath{\epsilon} < 1$. Use induction to show that
\[
\ensuremath{\prod}_{k=1}^n (1+\ensuremath{\epsilon}_k) = 1+\ensuremath{\theta}_n
\]
for some constant $\ensuremath{\theta}_n$ satisfying
\[
|\ensuremath{\theta}_n| \ensuremath{\leq} \underbrace{n \ensuremath{\epsilon} \over 1-n\ensuremath{\epsilon}}_{E_{n,\ensuremath{\epsilon}}}
\]
\textbf{SOLUTION}
\[
\ensuremath{\prod}_{k=1}^{n+1} (1+\ensuremath{\epsilon}_k) = \ensuremath{\prod}_{k=1}^n (1+\ensuremath{\epsilon}_k) (1+\ensuremath{\epsilon}_{n+1}) = (1+\ensuremath{\theta}_n)(1+\ensuremath{\epsilon}_{n+1}) = 1 + \underbrace{\ensuremath{\theta}_n + \ensuremath{\epsilon}_{n+1} + \ensuremath{\theta}_n\ensuremath{\epsilon}_{n+1}}_{\ensuremath{\theta}_{n+1}}
\]
where
\begin{align*}
|\ensuremath{\theta}_{n+1}| &\ensuremath{\leq} {n \ensuremath{\epsilon} \over 1-n\ensuremath{\epsilon}}(1+\ensuremath{\epsilon}) + \ensuremath{\epsilon} \\
&={n \ensuremath{\epsilon} + n \ensuremath{\epsilon}^2 \over 1-(n+1)\ensuremath{\epsilon}} \underbrace{{1-(n+1)\ensuremath{\epsilon} \over 1-n\ensuremath{\epsilon}}}_{\ensuremath{\leq} 1}  + {\ensuremath{\epsilon}-(n+1)\ensuremath{\epsilon}^2 \over 1-(n+1)\ensuremath{\epsilon}} \\
&\ensuremath{\leq} {(n+1)  -  \ensuremath{\epsilon} \over 1-(n+1)\ensuremath{\epsilon}} \ensuremath{\epsilon} \ensuremath{\leq} {(n+1) \ensuremath{\epsilon} \over 1-(n+1)\ensuremath{\epsilon}} = E_{n+1,\ensuremath{\epsilon}}.
\end{align*}
\textbf{END}

\textbf{Problem 1(b)} Show for an idealised floating point vector $\ensuremath{\bm{\x}} \ensuremath{\in} F_{\ensuremath{\infty},S}^n$  that
\[
x_1 \ensuremath{\oplus} \ensuremath{\cdots} \ensuremath{\oplus} x_n = x_1 +  \ensuremath{\cdots} + x_n + \ensuremath{\sigma}_n
\]
where
\[
|\ensuremath{\sigma}_n| \ensuremath{\leq} \| \ensuremath{\bm{\x}} \|_1 E_{n-1,\ensuremath{\epsilon}_{\rm m}/2},
\]
assuming $n \ensuremath{\epsilon}_{\rm m} < 2$ and where
\[
\|\ensuremath{\bm{\x}}\|_1 := \ensuremath{\sum}_{k=1}^n |x_k|.
\]
Hint: use the previous part to first write
\[
x_1 \ensuremath{\oplus} \ensuremath{\cdots} \ensuremath{\oplus} x_n = x_1(1+\ensuremath{\theta}_{n-1}) + \ensuremath{\sum}_{j=2}^n x_j (1 + \ensuremath{\theta}_{n-j+1}).
\]
\textbf{SOLUTION}

Using Problem 2.1 we write:
\meeq{
(\ensuremath{\cdots}((x_1 + x_2)(1+\ensuremath{\delta}_1) + x_3)(1+\ensuremath{\delta}_2)\ensuremath{\cdots}+x_n)(1+\ensuremath{\delta}_{n-1})
= x_1 \ensuremath{\prod}_{k=1}^{n-1} (1+\ensuremath{\delta}_k) +  \ensuremath{\sum}_{j=2}^n x_j \ensuremath{\prod}_{k=j-1}^{n-1} (1+\ensuremath{\delta}_j) \ccr
= x_1(1+\ensuremath{\theta}_{n-1}) + \ensuremath{\sum}_{j=2}^n x_j (1 + \ensuremath{\theta}_{n-j+1})
}
where we have for $j = 2,\ensuremath{\ldots},n$
\[
|\ensuremath{\theta}_{n-j+1}| \ensuremath{\leq} E_{n-j+1,\ensuremath{\epsilon}_{\rm m}/2} \ensuremath{\leq} E_{n-1,\ensuremath{\epsilon}_{\rm m}/2}.
\]
Thus we have 
\[
\ensuremath{\sum}_{j=1}^n x_j (1 + \ensuremath{\theta}_{n-j+1})= \ensuremath{\sum}_{j=1}^n x_j + \underbrace{\ensuremath{\sum}_{j=1}^n x_j \ensuremath{\theta}_{n-j+1}}_{\ensuremath{\sigma}_n}
\]
where
\[
|\ensuremath{\sigma}_n| \ensuremath{\leq}  \ensuremath{\sum}_{j=1}^n |x_j \ensuremath{\theta}_{n-j+1}| \ensuremath{\leq} \sup_j |\ensuremath{\theta}_{n-j+1}| \ensuremath{\sum}_{j=1}^n |x_j|  \ensuremath{\leq} \| \ensuremath{\bm{\x}}\|_1 E_{n-1,\ensuremath{\epsilon}_{\rm m}/2}.
\]
\textbf{END}

\textbf{Problem 1(c)} For $A \ensuremath{\in} F_{\ensuremath{\infty},S}^{n \ensuremath{\times} n}$ and $\ensuremath{\bm{\x}} \ensuremath{\in} F_{\ensuremath{\infty},S}^n$ consider the error in approximating matrix multiplication with idealised floating point: for
\[
A \ensuremath{\bm{\x}} =  \begin{pmatrix}
\ensuremath{\bigoplus}_{j=1}^n A_{1,j} \ensuremath{\otimes} x_j\\
\ensuremath{\vdots} \\
\ensuremath{\bigoplus}_{j=1}^n A_{1,j} \ensuremath{\otimes} x_j
\end{pmatrix} + \ensuremath{\delta}
\]
show that
\[
\| \ensuremath{\delta} \|_\ensuremath{\infty} \ensuremath{\leq} 2 \|A\|_\ensuremath{\infty} \| \ensuremath{\bm{\x}} \|_\ensuremath{\infty} E_{n,\ensuremath{\epsilon}_{\rm m}/2}
\]
where  $n \ensuremath{\epsilon}_{\rm m} < 2$ and the matrix norm is $\|A \|_\ensuremath{\infty} := \max_k \ensuremath{\sum}_{j=1}^n |a_{kj}|$.

\textbf{SOLUTION} We have for the $k$=th row
\[
\ensuremath{\bigoplus}_{j=1}^n A_{k,j} \ensuremath{\otimes} x_j =  \ensuremath{\bigoplus}_{j=1}^n A_{k,j} x_j (1+\ensuremath{\delta}_j) = 
\ensuremath{\sum}_{j=1}^n A_{k,j} x_j (1+\ensuremath{\delta}_j) + \ensuremath{\sigma}_{k,n}
\]
where we know $|\ensuremath{\sigma}_n| \ensuremath{\leq} M_k E_{n-1,\ensuremath{\epsilon}_{\rm m}/2}$, where from 1(b) we have
\[
M_k = \ensuremath{\Sigma}_{j=1}^n |A_{k,j}x_j (1+\ensuremath{\delta}_j)| = \ensuremath{\Sigma}_{j=1}^n |A_{k,j}| |x_j| (1+|\ensuremath{\delta}_j|)  \ensuremath{\leq} 2 \max |x_j| \ensuremath{\Sigma}_{j=1}^n |A_{k,j}|
\ensuremath{\leq} 2 \|\ensuremath{\bm{\x}}\|_\ensuremath{\infty} \|A \|_\ensuremath{\infty}
\]
Similarly, we also have 
\[
|\ensuremath{\sum}_{j=1}^n A_{k,j} x_j \ensuremath{\delta}_j| \ensuremath{\leq} \|\ensuremath{\bm{\x}}\|_\ensuremath{\infty} \|A \|_\ensuremath{\infty} \ensuremath{\epsilon}_{\rm m}/2
\]
and so the result follows from
\[
\ensuremath{\epsilon}_{\rm m}/2 + 2E_{n-1,\ensuremath{\epsilon}_{\rm m}/2} \ensuremath{\leq} {\ensuremath{\epsilon}_{\rm m}/2 + \ensuremath{\epsilon}_{\rm m} (n-1) \over 1-(n-1)\ensuremath{\epsilon}_{\rm m}/2} \ensuremath{\leq} {\ensuremath{\epsilon}_{\rm m} n \over 1-n\ensuremath{\epsilon}_{\rm m}/2} = 2E_{n,\ensuremath{\epsilon}_{\rm m}/2}.
\]
\textbf{END}

\rule{\textwidth}{1pt}
\textbf{Problem 2} Derive  Backward Euler: use the left-sided divided difference approximation
\[
u'(x) \ensuremath{\approx} {u(x) - u(x-h)  \over h}
\]
to reduce the first order ODE
\meeq{
u(a) =  c, \qquad u'(x) + \ensuremath{\omega}(x) u(x) = f(x)
}
to a lower triangular system by discretising on the grid $x_j = a + j h$ for $h = (b-a)/n$. Hint: only impose the ODE on the gridpoints $x_1,\ensuremath{\ldots},x_n$ so that the divided difference does not depend on behaviour at $x_{-1}$.

\textbf{SOLUTION}

We go through all 4 steps (this is to help you understand what to do. In an exam I will still give full credit if you get the right result, even if you don't write down all 4 steps):

(Step 1) Since we need to avoid going off the left in step 2 we start the ODE discretisation at $x_1$:
\[
\Vectt[u(x_0), u'(x_1), \ensuremath{\vdots}, u'(x_n)] = \underbrace{\Vectt[c, f(x_1),\ensuremath{\vdots},f(x_n)]}_{\ensuremath{\bm{\b}}}
\]
(Step 2) Replace with divided differences:
\[
\Vectt[u(x_0), (u(x_1)-u(x_0))/h, \ensuremath{\vdots}, (u(x_n)-u(x_{n-1}))/h] \ensuremath{\approx} \ensuremath{\bm{\b}}
\]
(Step 3) Replace with discrete system with equality:
\[
\Vectt[u_0, (u_1-u_0)/h, \ensuremath{\vdots}, (u_n-u_{n-1}))/h] = \ensuremath{\bm{\b}}
\]
(Step 4) Write as linear system:
\[
\begin{bmatrix}
1 \\
-1/h & 1/h \\
& \ensuremath{\ddots} & \ensuremath{\ddots} \\
&& -1/h & 1/h 
\end{bmatrix} \Vectt[u_0,\ensuremath{\vdots},u_n] = \ensuremath{\bm{\b}}
\]
This is the exact same matrix as Forward Euler but with a different right-hand side.

\textbf{END}

\textbf{Problem 3} Reduce a Schrödinger equation to a tridiagonal linear system by discretising on the grid $x_j = a + j h$ for $h = (b-a)/n$:
\meeq{
u(a) =  c,\qquad u''(x) + V(x) u(x) = f(x), \qquad u(b) = d.
}
\textbf{SOLUTION}

(Step 1) 
\[
\Vectt[u(x_0), u''(x_1) + V(x_1) u(x_1), \ensuremath{\vdots}, u'(x_{n-1}) + V(x_{n-1}) u(x_{n-1}), u(x_n)] = \underbrace{\Vectt[c, f(x_1),\ensuremath{\vdots},f(x_{n-1}), d]}_{\ensuremath{\bm{\b}}}
\]
(Step 2) Replace with divided differences:
\[
\Vectt[u(x_0), (u(x_0)-2u(x_1)+u(x_2))/h^2 + V(x_1)u(x_1), \ensuremath{\vdots}, (u(x_{n-2} - 2u(x_{n-1})+u(x_n))/h^2 + V(x_{n-1})u(x_{n-1}), u(x_n)] \ensuremath{\approx} \ensuremath{\bm{\b}}
\]
(Step 3) Replace with discrete system with equality:
\[
\Vectt[u_0, (u_0-2u_1+u_2)/h^2 + V(x_1)u_1, \ensuremath{\vdots}, (u_{n-2}-2u_{n-1}+u_n))/h^2+ V(x_{n-1})u_{n-1},u_n] = \ensuremath{\bm{\b}}
\]
(Step 4) Write as a tridiagonal linear system:
\[
\begin{bmatrix}
1 \\
1/h^2 & V(x_1)-2/h^2 & 1/h^2 \\
& \ensuremath{\ddots} & \ensuremath{\ddots} & \ensuremath{\ddots} \\
&&1/h^2 & V(x_{n-1})-2/h^2 & 1/h^2 \\
&& &&1
\end{bmatrix} \Vectt[u_0,\ensuremath{\vdots},u_n] = \ensuremath{\bm{\b}}
\]
\textbf{END}



\end{document}